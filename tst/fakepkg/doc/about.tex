%%%%%%%%%%%%%%%%%%%%%%%%%%%%%%%%%%%%%%%%%%%%%%%%%%%%%%%%%%%%%%%%%%%%%%%%

\Chapter{About this package}

%%%%%%%%%%%%%%%%%%%%%%%%%%%%%%%%%%%%%%%%%%%%%%%%%%%%%%%%%%%%%%%%%%%%%%%%

\Section{Testing general text}

This is a mock package to be used to test {\GAP} library code
related to {\GAP} packages, for example to validate `PackageInfo.g`
files. Clearly, it's not available anywhere, and it has it's `ArchiveURL' set
to `Concatenation( ~.PackageWWWHome, "/", ~.PackageName, "-", ~.Version )'
only for testing purposes.

\medskip

This manual is used to construct a test for the code from
`lib/helpt2t.g\{d,i\}' which converts {\TeX} documentation 
written in `gapmacro.tex' format into text for the ``screen''
online help viewer.

\bigskip
\hrule

Note that `gapmacro.tex' format is obsolete. If you are planning to
write new documentation for a {\GAP} package, don{\pif}t use it.
Instead, we recommend to use the \package{GAPDoc} package by Frank
L{\accent127u}beck and Max Neunh{\accent127o}ffer,
see \URL{http://www.math.rwth-aachen.de/~Frank.Luebeck/GAPDoc/}.

If you have any questions, please contact {\bsf GAP Support}
by email: \Mailto{support@gap-system.org}.

\par
*Examples:*
The following text demonstrates various features of
`gapmacro.tex' format.

This is an example of a {\GAP} session:

\beginexample
gap> a:=42;
42
\endexample

This assigns $42$ to the variable `a'
(see Section~"ref:Variables" in the {\GAP} Reference Manual).

This is another example which is excluded from automated testing:

\testexamplefalse
\beginexample
gap> Exec("date");
Sun Oct 7 16:23:45 CEST 2001
\endexample

This is an example of using the matrix environment:
$$
b_N = \left\{
\matrix{
\frac{1}{2}(-1+\sqrt{N}) &{\rm if} &N \equiv 1 &\pmod 4\cr
\frac{1}{2}(-1+i \sqrt{N}) &{\rm if} &N \equiv -1 &\pmod 4\cr
}
\right.
$$

This is an example of using the begintt environment:

\begintt
A.~X || B.~Y || C.~Z
--------------------
1    || 2    || 3
X    || Y    || Z
--------------------
\endtt

These common domains are defined by special macros:
\beginitems
natural numbers &
  \N

integers &
  \Z
\enditems
\hfill Similarly, there are macros for \Q, \R, \C, \F, \calR.

This is a collection of symbols to exercise various special
cases in the core to render them as a text:
\beginlist%unordered
\item{(a)} \c a \ss \aa \`a \'a \lq a \rq \accent18 a \accent19 a \accent20 \copyright
\item{(b)} \accent21 a \accent22 a \accent94 a \accent95 a \accent125 a \accent126 a
\item{(c)} \H a \u a \v a
\item{(d)} \hbox{more} mathematical symbols:
\itemitem{--} $ 1 \over 2 $ \quad $=$ \qquad $ \dot a \cdot a \dots a \pif $
\itemitem{--} $ \langle \rangle \ne 1 \bmod 2 \le \ge \setminus% 
                \bullet \circ \mapsto $
\itemitem{--} $ \longmapsto \to \Rightarrow \tilde a \vdash \iff \times \in \gamma $
\itemitem{--} $ \forall \exists \mid \colon \ast \lfloor \rfloor \prime \cup $
\endlist

Furthermore, $\Sigma_{k}$ is given by

$$
\Sigma_{k} = \sum\limits_{i\in N}(A_{i,k})
$$

Finally,
%
\){\fmark ...}

produces no label and index entry, and 

\){\kernttindent ...}

is useful for producing a line in typewriter type.

%%%%%%%%%%%%%%%%%%%%%%%%%%%%%%%%%%%%%%%%%%%%%%%%%%%%%%%%%%%%%%%%%%%%%%%%

\Section{Testing various mansection formats}

The following examples are taken from the gapmacro documentation.

\>Size(<obj>) A

is an attribute of an object.

\>Size(<obj>)!{for permutation groups} A

is special form of the previous command for permutation groups.

\>`<a> + <b>'{addition}

is used to display {\it command} as a header.

\>`Size( <set> )'{size of a set} A

is another example of the previous command.

\>`Size( <list> )'{size!of a list} A

is an example with a sub-entry.

\>`Size( <obj> )'{size}@{`Size'} A

is an example equivalent to the first one in this section.

\>`Size( GL( <n>, <q> ) )'%
% a percent sign at the end of line indicates a continuation
{Size!GL( n, q )}%
@{`Size'! `GL'( \noexpand<n>, \noexpand<q> )} A

is a more complex example.

\>`SomeGlobalVariable' V

is an example of a global variable.

%%%%%%%%%%%%%%%%%%%%%%%%%%%%%%%%%%%%%%%%%%%%%%%%%%%%%%%%%%%%%%%%%%%%%%%%
