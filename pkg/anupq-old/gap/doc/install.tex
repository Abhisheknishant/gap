%%%%%%%%%%%%%%%%%%%%%%%%%%%%%%%%%%%%%%%%%%%%%%%%%%%%%%%%%%%%%%%%%%%%%%%%%%%%%
\Chapter{Installing the ANU PQ Package}

This  chapter  describes  the  installation  of the  ANU  PQ  package.  A
description of the functions available in  the ANU PQ package is given in
Chapter "The ANU pq in GAP 4".

The ANU pq is written in *C*  and the package can only be installed under
UNIX.  It has been tested on DECstation running Ultrix, a HP 9000/700 and
HP 9000/800 running HP-UX, a MIPS running RISC/os Berkeley, a NeXTstation
running NeXTSTEP 3.0,  a SUN running SunOS and an  Intel Pentium based PC
running Linux.

In the example  installation below we will show how  a user `me' installs
the ANU  pq package on a PC  called `mypc' running Linux.   It is assumed
that  {\GAP}  4 is  already  installed on  this  machine  and the  {\GAP}
programme is in the directory `/usr/local/bin/gap4'.

Note that  certain parts  of the  output in the  examples should  only be
taken as rough  outline, especially file sizes and file  dates are not to
be  taken literally.   For the  sake  of brevity  directory listings  are
usually shown shorter than they really are.

First of all you have to get the file 'anupq.zoo' from the {\GAP} 4 share
packages page on the GAP server in St Andrews or one of its mirrors.

\begintt
        www-gap.dcs.st-and.ac.uk/~gap/Info4/share.html
\endtt

Then locate  the {\GAP}  directory of the  {\GAP} 4 installation  on your
system.   This  directory  contains,  among others,  'lib/',  'doc/'  and
'pkg/'.   The {\GAP} directory  itself is  usually called  'gap4r2'.  The
last digit is the revision number which may be higher on your system.

\begintt
    me@mypc > ls -l
    drwxr-xr-x   15 me   mygroup         1024 Mar 23 17:25 gap4r2
    -rw-r--r--   1  me   mygroup       560891 Dec 27 15:16 anupq.zoo
    me@mypc > ls -l gap4r2
    -rw-r--r--    1 me   mygroup        36930 Feb 29  2000 INSTALL
    drwxr-xr-x    3 me   mygroup         1024 Mar 23 17:25 bin
    -rwxr-xr-x    1 me   mygroup        36562 Jul  9  1999 configure
    drwxr-xr-x    8 me   mygroup         1024 Mar 23 17:19 doc
    drwxr-xr-x    2 me   mygroup         1024 Mar 23 17:19 grp
    drwxr-xr-x    2 me   mygroup         8192 Mar 23 17:19 lib
    drwxr-xr-x    3 me   mygroup         1024 Mar 23 17:19 pkg
    -rw-r--r--    1 me   mygroup           30 Mar 23 17:25 sysinfo.gap
    -rw-r--r--    1 me   mygroup           22 Jul 16  1998 sysinfo.in
    [...]
\endtt

Unpack the package using 'unzoo'.  Note that you must be in the directory
'gap4r2/pkg' to unpack the files.  After you have unpacked the source you
may remove the .zoo file.  A version of unzoo is available from

\begintt
        http://www-gap.dcs.st-and.ac.uk/~gap/Info4/distrib.html
\endtt

The result should look like this:

\begintt
    me@mypc > cd gap4r2/pkg
    me@mypc > mv ../../anupq.zoo .
    me@mypc > unzoo x anupq
    [..unpack messages..]
    me@mypc > cd anupq
    me@mypc > ls -l
    -rw-r--r--    1 me   mygroup  10167 Aug 12  1998 README
    drwxr-sr-x    2 me   mygroup   4096 Mar 24 13:52 TEST
    drwxr-sr-x    3 me   mygroup   4096 Mar 24 13:52 bin
    drwxr-sr-x    2 me   mygroup   4096 Mar 24 13:52 doc
    drwxr-sr-x    2 me   mygroup   4096 Mar 24 13:52 examples
    drwxr-sr-x    6 me   mygroup   4096 Mar 24 13:58 gap
    drwxr-sr-x    2 me   mygroup   4096 Mar 24 13:52 include
    lrwxrwxrwx    1 me   mygroup     10 Feb  5 09:31 init.g -> gap/init.g
    drwxr-sr-x    2 me   mygroup   4096 Mar 24 13:52 isom
    drwxr-sr-x    2 me   mygroup   4096 Mar 24 13:52 magma
    lrwxrwxrwx    1 me   mygroup     10 Feb  5 09:31 read.g -> gap/read.g
    drwxr-sr-x    2 me   mygroup   8192 Mar 24 13:57 src
\endtt

Change into the subdirectory 'gap'.  Typing 'make' will produce a list of
possible targets.

\begintt
me@mypc > cd gap
me@mypc > make
usage: 'make <target> EXT=<ext>'  where <target> is one of
'dec-mips-ultrix-gcc2-gmp' for DECstations under Ultrix with gcc/gmp
'dec-mips-ultrix-cc-gmp'   for DECstations under Ultrix with cc/gmp

 [..messages deleted..]

'unix-gmp'                 for a generic unix system with cc/gmp
'unix'                     for a generic unix system with cc
'clean'                    remove all created files

   where <ext> should be a sensible extension, i.e.,
   'EXT=-sun-sparc-sunos' for SUN 4 or 'EXT=' if the PQ only
   runs on a single architecture

   targets are listed according to preference,
   i.e., 'sun-sparc-sunos-gcc2' is better than 'sun-sparc-sunos-cc'.
   additional C compiler and linker flags can be passed with
   'make <target> COPTS=<compiler-opts> LOPTS=<linker-opts>',
   i.e., 'make sun-sparc-sunos-cc COPTS=-g LOPTS=-g'.

   set GAP if gap is not started with the command 'gap',
   i.e., 'make sun-sparc-sunos-cc GAP=/home/gap/bin/gap-4.2'.

   in order to use the GNU multiple precision (gmp) set
   'GNUINC' (default '/usr/local/include') and 
   'GNULIB' (default '/usr/local/lib')
\endtt

Select the appropriate target.  If  you have the *GNU* multiple precision
arithmetic (gmp)  installed on your  system, select the target  ending in
'-gmp'.  Note that  the gmp  is  *not required*.

\begintt
    me@mypc > make unix GAP=/usr/local/bin/gap4
    [..compiler messages..]
\endtt

The executable pq programm is put into the directory 'anupq/bin/<system>'
where '<system>'  is a string  encoding the architecture of  your system,
the compiler used, etc.  It is obtained from the file gap4r2/sysinfo.gap.
In our case it is 'i686-pc-linux-gnu-gcc'.  The reason for putting the pq
executable into  a directory  with this  name is that  this is  the place
where  the {\GAP}~4  mechanism  for loading  share  packages expects  the
executable to be.

\begintt
    me@mypc > ls -l ../bin/                      
    drwxr-sr-x    2 me   mygroup   4096 Mar 24 13:47 i686-pc-linux-gnu-gcc
    me@mypc > ls -l ../bin/i686-pc-linux-gnu-gcc/
    -rwxr-xr-x    1 me   mygroup 221875 Mar 24 13:48 pq
\endtt

The  {\GAP} documentation  is  compiled  by running  {\TeX}  on the  file
manual.tex:

\begintt
    me@mypc > cd doc
    me@mypc > tex manual
    This is TeX, Version 3.14159 (Web2C 7.3.1)
    [..TeX compilation messages..]
    Output written on manual.dvi (19 pages, 43516 bytes).
    Transcript written on manual.log.
\endtt


Now it is time to test  the installation.  The  first test will only test
the ANU pq.

\begintt
    me@mypc > cd ../..
    me@mypc > bin/i686-pc-linux-gnu-gcc/pq < gap/tst/test1.pga
    # a lot of messages ending in
    **************************************************
    Starting group: c3c3 # 2;2 # 4;3
    Order: 3^7
    Nuclear rank: 3
    3-multiplicator rank: 4
    # of immediate descendants of order 3^8 is 7
    # of capable immediate descendants is 5

    **************************************************
    34 capable groups saved on file c3c3_class4
    Construction of descendants took 1.92 seconds

    Select option: 0 
    Exiting from p-group generation

    Select option: 0 
    Exiting from ANU p-Quotient Program
    Total user time in seconds is 1.97
    me@mypc > ls -l c3c3*
    total 89
    -rw-r--r--    1 gap    3320 Jun 24 11:24 c3c3_class2
    -rw-r--r--    1 gap    5912 Jun 24 11:24 c3c3_class3
    -rw-r--r--    1 gap   56184 Jun 24 11:24 c3c3_class4
    gap:../anupq > rm c3c3_class*
\endtt

The second test will test the stacksize. If it is too small you will  get
a memory fault, try to raise the stacksize as described above.

\begintt
    me@mypc > bin/i686-pc-linux-gnu-gcc/pq < gap/tst/test2.pga
    # a lot of messages ending in
    **************************************************
    Starting group: c2c2 # 1;1 # 1;1 # 1;1
    Order: 2^5
    Nuclear rank: 1
    2-multiplicator rank: 3
    Group c2c2 # 1;1 # 1;1 # 1;1 is an invalid starting group

    **************************************************
    Starting group: c2c2 # 2;1 # 1;1 # 1;1
    Order: 2^5
    Nuclear rank: 1
    2-multiplicator rank: 3
    Group c2c2 # 2;1 # 1;1 # 1;1 is an invalid starting group
    Construction of descendants took 0.47 seconds

    Select option: 0 
    Exiting from p-group generation

    Select option: 0 
    Exiting from ANU p-Quotient Program
    Total user time in seconds is 0.50
    me@mypc > ls -l c2c2*
    total 45
    -rw-r--r--    1 gap   6228 Jun 24 11:25 c2c2_class2
    -rw-r--r--    1 gap  11156 Jun 24 11:25 c2c2_class3
    -rw-r--r--    1 gap   2248 Jun 24 11:25 c2c2_class4
    -rw-r--r--    1 gap      0 Jun 24 11:25 c2c2_class5
    me@mypc > rm c2c2_class*
\endtt

The third example tests the link between the ANU pq and {\GAP}.  If there
is a problem you will get a error message saying
'Error in  system  call  to GAP';  if this happens, check the environment
variable ANUPQ\_GAP\_EXEC.

\begintt
    me@mypc > bin/i686-pc-linux-gnu-gcc/pq < gap/tst/test3.pga
    [..messages from the ANU pq..]
    **************************************************
    Starting group: c5c5 # 1;1 # 1;1
    Order: 5^4
    Nuclear rank: 1
    5-multiplicator rank: 2
    # of immediate descendants of order 5^5 is 2

    **************************************************
    Starting group: c5c5 # 1;1 # 2;2
    Order: 5^5
    Nuclear rank: 3
    5-multiplicator rank: 3
    # of immediate descendants of order 5^6 is 3
    # of immediate descendants of order 5^7 is 3
    # of capable immediate descendants is 1
    # of immediate descendants of order 5^8 is 1
    # of capable immediate descendants is 1

    **************************************************
    2 capable groups saved on file c5c5_class4

    **************************************************
    Starting group: c5c5 # 1;1 # 2;2 # 4;2
    Order: 5^7
    Nuclear rank: 1
    5-multiplicator rank: 2
    # of immediate descendants of order 5^8 is 2
    # of capable immediate descendants is 2

    **************************************************
    Starting group: c5c5 # 1;1 # 2;2 # 7;3
    Order: 5^8
    Nuclear rank: 2
    # of immediate descendants of order 5^9 is 1
    # of capable immediate descendants is 1
    # of immediate descendants of order 5^10 is 1
    # of capable immediate descendants is 1

    **************************************************
    4 capable groups saved on file c5c5_class5
    Construction of descendants took 0.62 seconds

    Select option: 0 
    Exiting from p-group generation

    Select option: 0 
    Exiting from ANU p-Quotient Program
    Total user time in seconds is 0.68
    me@mypc > ls -l c5c5*
    total 41
    -rw-r--r--    1 gap     924 Jun 24 11:27 c5c5_class2
    -rw-r--r--    1 gap    2220 Jun 24 11:28 c5c5_class3
    -rw-r--r--    1 gap    3192 Jun 24 11:30 c5c5_class4
    -rw-r--r--    1 gap    7476 Jun 24 11:32 c5c5_class5
    me@mypc > rm c5c5_class*
\endtt

The fourth test will test the standard presentation part of the pq.

\begintt
    me@mypc > bin/pq -i -k < gap/tst/test4.sp
    [..messages from the ANU pq..]
    Computing standard presentation for class 5 took 0.03 seconds

    Select option: 0 
    Exiting from ANU p-Quotient Program
    Total user time in seconds is 6.97
    me@mypc > ls -l SPRES
    -rw-r--r--    1 me   mygroup        488 Mar 27 11:40 SPRES
    me@mypc > diff SPRES gap/out4.sp
    # there should be no difference if compiled with '-gmp'
    156250000
    me@mypc > rm SPRES
\endtt

The last  test  will test the  link  between {\GAP} and   the ANU pq.  If
everything goes well you should not see any message.

\begintt
    me@mypc > gap -b
    gap> RequirePackage( "anupq" );
    gap> ReadTest( "gap/tst/anupga.tst" );
    gap>
\endtt


