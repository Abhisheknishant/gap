%%%%%%%%%%%%%%%%%%%%%%%%%%%%%%%%%%%%%%%%%%%%%%%%%%%%%%%%%%%%%%%%%%%%%%%%%
%%
%W  norm.tex              FORMAT documentation    B. Eick and C.R.B. Wright
%%
%%  10-30-00

%%%%%%%%%%%%%%%%%%%%%%%%%%%%%%%%%%%%%%%%%%%%%%%%%%%%%%%%%%%%%%%%%%%%%%%%%
\Chapter{FNormalizers}

%\index{FNormalizers}

Let $\F$ be an integrated locally defined formation, and let $G$ be 
a finite solvable group with Sylow complement basis $\Sigma := 
\{ S^p \mid p$ divides  $ \|G\| \}$.  Let $\pi$ be the set of prime
divisors of the order of $G$ that are in the support of $\F$ and 
$\overline{\pi}$ the remaining prime divisors of the order  of $G$. 
Then the *$\F$-normalizer* of $G$ with respect to $\Sigma$ is defined 
to be  
$\bigcap_{p \in \overline{\pi}} S^p \cap 
 \bigcap_{p \in \pi} N_G( G^{\F(p)} \cap S^p )$. 
The special case $\F(p) = \{ 1 \}$ for all $p$ defines the formation 
of nilpotent groups, whose $\F$-normalizers $ \bigcap_{p} N_G( S^p )$ 
are the *system normalizers* of $G$. The $\F$-normalizers of a group 
$G$ for a given $\F$ are all conjugate. They cover $\F$-central chief 
factors and avoid $\F$-hypereccentric ones.

\> FNormalizerWrtFormation( <G>, <F> ) O
\> SystemNormalizer( <G> ) A

If <F> is a locally defined integrated formation in {\GAP} and 
<G> is a finite solvable group, then the function `FNormalizerWrtFormation'
returns an <F>-normalizer of <G>. The function `SystemNormalizer' yields a 
system normalizer of <G>.

The underlying algorithm here requires <G> to have a special pcgs (see SpecialPcgs), so the algorithm's first step is
 to compute such a pcgs for <G> if one is not known. The complement basis
$\Sigma$ associated with this pcgs is then used to compute the
<F>-normalizer of <G> with respect to $\Sigma$. This process means that 
in the case of a finite solvable group <G> that does not have a special pcgs, 
the first call of `FNormalizerWrtFormation' (or similarly of `FormationCoveringGroup') 
will  take longer than subsequent calls, since it will include the
computation  of a special pcgs.

The `FNormalizerWrtFormation' algorithm next computes an <F>-system for <G>, a
complicated record that includes a pcgs corresponding to a normal series 
of <G> whose factors are either <F>-central or <F>-hypereccentric. A subset 
of this pcgs then exhibits the <F>-normalizer of <G> determined by
$\Sigma$. The list `ComputedFNormalizerWrtFormations( <G> )' stores the <F>-normalizers
of <G> that have been found for various formations <F>.  

The `FNormalizerWrtFormation' function can be used to study the subgroups of a 
single group <G>, as illustrated in an example in Section "Other Applications". In that case it is sufficient to have a function
`ScreenOfFormation' that  returns a normal subgroup of <G> on each call.

