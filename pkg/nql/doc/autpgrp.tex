%%%%%%%%%%%%%%%%%%%%%%%%%%%%%%%%%%%%%%%%%%%%%%%%%%%%%%%%%%%%%%%%%%%%%%%%%%%%
%%
%W  autpgrp.tex		NQL Doc				René Hartung
%%
%H  $Id: autpgrp.tex,v 1.1 2009/07/02 12:26:02 gap Exp $
%%

%%%%%%%%%%%%%%%%%%%%%%%%%%%%%%%%%%%%%%%%%%%%%%%%%%%%%%%%%%%%%%%%%%%%%%%%%%%%
\Chapter{Investigating the automorphism group}

Let $G$ be a finitely generated group. Then the term $\gamma_cG$
of the lower central series is fully invariant subgroup of $G$.
Thus every automorphism $\alpha\in{\rm Aut}(G)$ induces an automorphism
$\varphi_c\in{\rm Aut}(G/\gamma_cG)$. We obtain a homomorphism 
$\nu_c\colon{\rm Aut}(G)\to{\rm Aut}(G/\gamma_cG)$, $\alpha\mapsto\alpha_c$.
This homomorphism map the inner automorphism ${\rm Inn}(G)$ onto
${\rm Inn}(G/\gamma_cG)$ and thus we obtain a homomorphism
$$
  \nu_c\colon{\rm Out}(G)\to {\rm Out}(G/\gamma_cG).
$$
Similar, for every $d\leq c$, we obtain a homomorphism $\mu_{c,d}\colon
{\rm Out}(G/\gamma_cG)\to{\rm Out}(G/\gamma_dG)$. Since $\nu_d = \nu_c\circ
\mu_{c,d}$, this yields that
$$
  {\rm im}(\nu_d)\leq \ldots\leq {\rm im}(\mu_{c,d})\leq {\rm im}(\mu_{c-1,d})
  \leq \ldots \leq {\rm im}(\mu_{d,d}) = {\rm Out}(G/\gamma_dG).
$$
This sequence can be used to guess the shape of ${\rm im}(\nu_d)$ and 
therefore to guess the shape of ${\rm Out}(G)/\ker\nu_d$. The \AutPGrp-Package
can be used to compute the images ${\rm im}(\mu_{c,d})$ if the abelian
quotient of $G$ is elementary abelian. For further details we refer to
\cite{EH09}.

\> AutomorphismGroupSequence( <PcpGroup> ) F

if the abelianization of <PcpGroup> is elementary abelian, this
method computes a list of the images of the outer automorphism group
of $G/\gamma_cG$ in ${\rm Out}(G/\gamma_dG)$ for any $d\leq c$ with
${\rm Out}(G/\gamma_dG)$ being still solvable. More precisely, the
entry {\bf a[i][j]} denotes the image of ${\rm Out}(G/\gamma_{j+1}G)$ in
${\rm Out}(G/\gamma_{i+1}G)$.

\vskip 3ex

In the following example we consider the nilpotent quotients of the
Grigorchuk group and compute its outer automorphism group sequence.
\beginexample
gap> G := ExamplesOfLPresentations( 1 );;
gap> A := AutomorphismGroupSequence( G, 5 );;
[1,2]: ab [ [ 2, 1 ] ]
[1,3]: ab [ [ 2, 1 ] ]
[1,4]: ab [ [ 2, 1 ] ]
[1,5]: ab [ [ 2, 1 ] ]

[2,3]: ab [ [ 2, 1 ] ]
[2,4]: ab [ [ 2, 1 ] ]
[2,5]: ab [ [ 2, 1 ] ]

[3,4]: id [ 16, 11 ]
[3,5]: ab [ [ 2, 2 ] ]

[4,5]: ab [ [ 2, 2 ] ]
\endexample
