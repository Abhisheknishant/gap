%%%%%%%%%%%%%%%%%%%%%%%%%%%%%%%%%%%%%%%%%%%%%%%%%%%%%%%%%%%%%%%%%%%%%%%%%
%%
%W  messages.tex        ACE appendix - messages          Alexander Hulpke
%W                                                      Joachim Neub"user
%W                                                            Greg Gamble
%%
%H  $Id: messages.tex,v 1.12 2006/01/26 16:15:05 gap Exp $
%%
%Y  Copyright (C) 2000  Centre for Discrete Mathematics and Computing
%Y                      Department of Information Tech. & Electrical Eng.
%Y                      University of Queensland, Australia.
%%

%%%%%%%%%%%%%%%%%%%%%%%%%%%%%%%%%%%%%%%%%%%%%%%%%%%%%%%%%%%%%%%%%%%%%%%
\Chapter{The Meanings of ACE's output messages}

In this chapter, we discuss the meanings of the messages  that  appear
in output from the {\ACE} binary, the verbosity of which is determined
by the `messages' option (see~"option  messages").  Actually  our  aim
here is to concentrate  on  describing  those  ``messages''  that  are
controlled  by  the  `messages'  option,  namely  the  *progress*  and
*results   messages*;   other   output   from   {\ACE}    is    fairly
self-explanatory. (We describe some other output  to  give  points  of
reference.)

Both interactively and non-interactively, {\ACE} output  is  `Info'-ed
at `InfoACE' level 3. To see the  `Info'-ed  {\ACE}  output,  set  the
`InfoLevel' of `InfoACE' to at least 3, e.g.

\beginexample
gap> SetInfoACELevel(3);
\endexample

This  causes  each  line  of  {\ACE}  output  to  be  prepended   with
```\#I  '''.  Below,  we describe the `Info'-ed output  observed  when
each of `ACECosetTableFromGensAndRels', `ACECosetTable', `ACEStats' or
`ACEStart' is called with three arguments, and presume that users will
be able to extend the description  to  explain  output  observed  from
other  {\ACE}  interface  functions.  The  first-observed  (`Info'-ed)
output from {\ACE}, is {\ACE}'s banner, e.g.

\begintt
#I  ACE 3.001        Sun Sep 30 21:45:39 2001
#I  =========================================
#I  Host information:
#I    name = rigel
\endtt

If there were any errors in the directives put to  {\ACE},  or  output
from the options described in Appendix~"Other ACE Options", they  will
appear next. Then,  the  next  observed  output  is  a  row  of  three
asterisks:

\begintt
#I  ***
\endtt

*Guru note:*
The three asterisks are generated  by  a  ```text:***'''  (see~"option
text") directive, emitted to {\ACE}, so that {\ACE}'s response can  be
used as a sentinel to flush out any user errors, or any output from  a
user's use of Appendix~"Other ACE Options" options.

Next, if the `messages' option (see~"option messages")  is  set  to  a
non-zero value, what is observed is a heading like:

\begintt
#I    #--- ACE 3.001: Run Parameters ---
\endtt

(where `3.001' may be replaced be some later  version  number  of  the
{\ACE} binary) followed by the ``input parameters'' developed from the
arguments  and  options  passed   to   `ACECosetTableFromGensAndRels',
`ACEStats' or `ACEStart'. After these appears a separator:

\begintt
#I    #---------------------------------
\endtt

followed by any *progress messages* (progress messages only  occur  if
`messages' is non-zero;  recall  that  by  default  `messages'  =  0),
followed by a *results message*.

In the case of `ACECosetTableFromGensAndRels', the  *results  message*
is followed by a compaction (`CO' or  `co')  progress  message  and  a
coset table.

{\ACE}'s exit banner which should look something like:

\begintt
=========================================
ACE 3.001        Sun Sep 30 22:03:36 2001
\endtt

is only seen when running {\ACE} as a standalone.

Both *progress messages* and  the  *results  message*  consist  of  an
initial tag followed by a list of statistics. All messages have values
for the statistics `a', `r', `h', `n', `h',  `l'  and  `c'  (excepting
that the second `h', the one following  the  `n'  statistic,  is  only
given if hole monitoring has been turned on by setting `messages'  $\<
0$, which as noted above is expensive and  should  be  avoided  unless
really needed). Additionally, there may appear the statistics: `m' and
`t' (as for the results message); `d'; or `s', `d' and `c' (as for the
`DS' progress message). The meanings of  the  various  statistics  and
tags will follow later. The following is a sample progress message:

\begintt
#I  AD: a=2 r=1 h=1 n=3; h=66.67% l=1 c=+0.00; m=2 t=2
\endtt

with tag `AD' and values for the statistics `a', `r',  `h',  `n',  `h'
(appears because `messages' $\<  0$),  `l',  `c',  `m'  and  `t'.  The
following is a sample results message:

\begintt
#I  INDEX = 12 (a=12 r=16 h=1 n=16; l=3 c=0.01; m=14 t=15)
\endtt

which, in this case, declares a successful enumeration  of  the  coset
numbers of a subgroup of index 12 within a group,  and,  as  it  turns
out, values for the same statistics as the sample progress message.

You should see all the above (and a little  more),  except  that  your
dates and host information will no doubt differ, by running:

\beginexample
gap> ACEExample("A5-C5" : echo:=0, messages:=-1);
\endexample

In the following table we  list  the  statistics  that  can  follow  a
progress or results message tag, in order:

\begintt
--------------------------------------------------------------------
statistic   meaning
--------------------------------------------------------------------
a           number of active coset numbers
r           number of applied coset numbers
h           first (potentially) incomplete row
n           next coset number definition
h           percentage of holes in the table (if `messages'$ \< 0$) 
l           number of main loop passes
c           total CPU time
m           maximum number of active coset numbers
t           total number of coset numbers defined
s           new deduction stack size (with DS tag)
d           current deduction stack size, or
              no. of non-redundant deductions retained (with DS tag)
c           no. of redundant deductions discarded (with DS tag)
--------------------------------------------------------------------
\endtt

Now that we have discussed the various  meanings  of  the  statistics,
it's time to  discuss  the  various  types  of  progress  and  results
messages possible. First we describe progress messages.

%%%%%%%%%%%%%%%%%%%%%%%%%%%%%%%%%%%%%%%%%%%%%%%%%%%%%%%%%%%%%%%%%%%%%%%%
\Section{Progress Messages}

A progress message (and its tag) indicates the function just completed
by the enumerator. In the following table, the possible message `tag's
appear in the first column. In the `action' column,  a  `y'  indicates
the function is aggregated and counted. Every time this count  reaches
the value of `messages', a message line is printed and  the  count  is
zeroed. Those tags flagged  with  a  `y*'  are  only  present  if  the
appropriate option was included when the {\ACE} binary was compiled (a
default compilation includes the appropriate options; so normally read
`y*' as `y'). Tags with an `n' in the  `action'  column  indicate  the
function is not counted, and cause a message line to be  output  every
time they occur. They also cause the action count to be reset.

\begintt
------------------------------------------------------------------
tag   action      meaning
------------------------------------------------------------------
AD         y      coset 1 application definition (`SG'/`RS' phase)
RD         y      R-style definition
RF         y      row-filling definition
CG         y      immediate gap-filling definition
CC         y*     coincidence processed
DD         y*     deduction processed
CP         y      preferred list gap-filling definition
CD         y      C-style definition
Lx         n      lookahead performed (type `x')
CO         n      table compacted
co         n      compaction routine called, but nothing recovered
CL         n      complete lookahead (table as deduction stack)
UH         n      updated completed-row counter
RA         n      remaining coset numbers applied to relators
SG         n      subgroup generator phase
RS         n      relators in subgroup phase
DS         n      stack overflowed (compacted and doubled)
------------------------------------------------------------------
\endtt

% \begin{table}
% \hrule
% \caption{Possible progress messages}
% \label{tab:prog}
% \smallskip
% \renewcommand{\arraystretch}{0.875}
% \begin{tabular*}{\textwidth}{@{\extracolsep{\fill}}lll} 
% \hline\hline
% message & action & meaning \\
% \hline
% \ttt{AD} & y  & coset \#1 application definition 
% 			(\ttt{SG}/\ttt{RS} phase) \\
% \ttt{RD} & y  & R-style definition \\
% \ttt{RF} & y  & row-filling definition \\
% \ttt{CG} & y  & immediate gap-filling definition \\
% \ttt{CC} & y* & coincidence processed \\
% \ttt{DD} & y* & deduction processed \\
% \ttt{CP} & y  & preferred list gap-filling definition \\
% \ttt{CD} & y  & C-style definition \\
% \ttt{Lx} & n  & lookahead performed (type \ttt{x}) \\
% \ttt{CO} & n  & table compacted \\
% \ttt{CL} & n  & complete lookahead (table as deduction stack) \\
% \ttt{UH} & n  & updated completed-row counter \\
% \ttt{RA} & n  & remaining cosets applied to relators \\
% \ttt{SG} & n  & subgroup generator phase \\
% \ttt{RS} & n  & relators in subgroup phase \\
% \ttt{DS} & n  & stack overflowed (compacted and doubled) \\
% \hline\hline
% \end{tabular*}
% \end{table}

%%%%%%%%%%%%%%%%%%%%%%%%%%%%%%%%%%%%%%%%%%%%%%%%%%%%%%%%%%%%%%%%%%%%%%%
\Section{Results Messages}

The possible results are given in the following table; any result  not
listed represents an internal error and  should  be  reported  to  the
{\ACE} authors.

% The level column is omitted ... since it won't mean anything to a
% GAP user
\begintt
result tag           meaning 
-------------------------------------------------------------------
INDEX = x            finite index of `x' obtained
OVERFLOW             out of table space
SG PHASE OVERFLOW    out of space (processing subgroup generators)
ITERATION LIMIT      `loop' limit triggered
TIME LIMT            `ti' limit triggered
HOLE LIMIT           `ho' limit triggered
INCOMPLETE TABLE     all coset numbers applied, but table has holes
MEMORY PROBLEM       out of memory (building data structures)
-------------------------------------------------------------------
\endtt

% \begin{table}
% \hrule
% \caption{Possible enumeration results}
% \label{tab:rslts}
% \smallskip
% \renewcommand{\arraystretch}{0.875}
% \begin{tabular*}{\textwidth}{@{\extracolsep{\fill}}lll} 
% \hline\hline
% result & level & meaning \\
% \hline
% \ttt{INDEX = x}         & 0 & finite index of \ttt{x} obtained \\
% \ttt{OVERFLOW}          & 0 & out of table space \\
% \ttt{SG PHASE OVERFLOW} & 0 & out of space (processing subgroup
% 				generators) \\
% \ttt{ITERATION LIMIT}   & 0 & \ttt{loop} limit triggered \\
% \ttt{TIME LIMT}         & 0 & \ttt{ti} limit triggered \\
% \ttt{HOLE LIMIT}        & 0 & \ttt{ho} limit triggered \\
% \ttt{INCOMPLETE TABLE}  & 0 & all cosets applied, but table has holes \\
% \ttt{MEMORY PROBLEM}    & 1 & out of memory (building data structures) \\
% \hline\hline
% \end{tabular*}
% \end{table}

*Notes:*
Recall that hole monitoring is switched on by setting a negative value
for the `messages' (see~"option messages") option, but note that  hole
monitoring is expensive, so don't turn it on unless  you  really  need
it. If you wish to print out the presentation and the options, but not
the progress messages, then set `messages' non-zero, but  very  large.
(You'll still get the `SG', `DS', etc. messages,  but  not  the  `RD',
`DD', etc. ones.) You can  set  `messages'  to  $1$,  to  monitor  all
enumerator actions, but be warned  that  this  can  yield  very  large
output files.

%%%%%%%%%%%%%%%%%%%%%%%%%%%%%%%%%%%%%%%%%%%%%%%%%%%%%%%%%%%%%%%%%%%%%%%%%
%%
%E
