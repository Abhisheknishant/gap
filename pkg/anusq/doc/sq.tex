\Chapter{ANU Sq Package}

\>Sq( <G>, <L> )

The function  `Sq' is  the  interface to the  Soluble Quotient standalone
program.

Let <G> be  a finitely presented group  and let <L> be  a list  of lists.
Each of these lists is a list of integer pairs $[p_i,c_i]$, where $p_i$
is  a prime and  $c_i$ is a  non-negative integer and $p_i \not= p_{i+1}$
and $c_i$ positive for  $i\le k+1$. 
`Sq' computes a  consistent
power conjugate   presentation  for a  finite  soluble group  given  as a
quotient of the finitely presented group <G> which is described by <L> as
follows.

Let  $H$ be a  group and  $p$ a  prime.  The  series  $$H = {P}^p_0(H)\ge
{P}^p_1(H)\ge\cdots\,\,\hbox  to  1.5   cm{\hfil  with\ }    {P}^p_i(H) =
[{P}^p_{i-1}(H),H] \left({P}^p_{i-1} (H)\right)^p$$ for $i  \ge 1$ is the
{\it lower exponent-$p$ central series} of $H.$

For $1 \le i \le k$ and $0 \le j \le  c_i$ define the list ${\rm L}_{i,j}
=   [(p_1,c_1),\ldots,  (p_{i-1},c_{i-1}),(p_{i}, j)    ].$ Define  ${\rm
L}_{1,0}(G) = G.$ For $1  \le i \le  k$ and $1  \le j \le c_i$ define the
subgroups  $${\rm L}_{i,j}(G) =  {\rm  P}_{j}^{p_i}( {\rm L}_{i,0}(G) )$$
and for $1  \le i \le  k+1$ define the  subgroups $${\rm L}_{i+1,0}(G) = {\rm
L}_{i,c_i}(G)$$ and ${\rm L}(G)  = {\rm  L}_{k,c_k}(G).$ Note that  ${\rm
L}_{i,j}(G) \ge {\rm L}_{i,j+1}(G)$ holds for $j \le c_i+1.$

The chain of subgroups $$G  = {\rm L}_{1,0}(G)  \ge {\rm L}_{1,1}(G)  \ge
\cdots  \ge {\rm L}_{1,c_1}(G)    = {\rm L}_{2,0}(G)\ge \cdots   \ge {\rm
L}_{k,c_k}(G) = {\rm  L}(G)  $$   is   called  the  {\it  soluble   ${\rm
L}$-series} of $G.$

'Sq' computes  a  consistent  power  conjugate  presentation for  $G/{\rm
L}(G),$ where  the   presentation exhibits a  composition  series  of the
quotient group which is a refinement of the soluble ${\rm L}$-series.  An
epimorphism from $G$ onto $G/{\rm L}(G)$ is listed in comments.

The algorithm proceeds by computing power conjugate presentations for the
quotients  $G/{\rm L}_{i,j}(G)$ in   turn.   Without loss of   generality
assume that a  power conjugate presentation  for $G/{\rm L}_{i,j}(G)$ has
been computed  for $j \le c_i+1.$ The  basic step computes a  power conjugate
presentation for $G/{\rm L}_{i,j+1}(G).$ The group ${\rm L}_{i,j}(G)/{\rm
L}_{i,j+1}(G)$  is  a  $p_{i}$-group.   If  during  the  basic step it is
discovered that ${\rm  L}_{i,j}(G)   = {\rm L}_{i,j+1}(G),$  then   ${\rm
L}_{i+1,0}(G)$ is set to ${\rm L}_{i,j}(G).$

Note that during the basic step the vector enumerator is called.

\beginexample
     f := FreeGroup( "a", "b" );
     f := f/[ (f.1*f.2)^2*f.2^-6, f.1^4*f.2^-1*f.1*f.2^-9*f.1^-1*f.2 ];
     gap> g := Sq( f, [[2,1],[3,1],[2,2],[3,2]] );
     rec(
      generators := [ a.1, a.2, a.3, a.4, a.5, a.6, a.7, a.8 ],
      relators := [ a.1^2*a.3^-1, a.1^-1*a.2*a.1*a.4^-1*a.2^-2, 
          a.2^3*a.5^-1, a.1^-1*a.3*a.1*a.3^-1, 
          a.2^-1*a.3*a.2*a.6^-1*a.5^-1*a.4^-1*a.3^-1, 
          a.3^2*a.7^-1*a.5^-1, a.1^-1*a.4*a.1*a.7^-1*a.4^-1*a.3^-1, 
          a.2^-1*a.4*a.2*a.8^-1*a.7^-1*a.6^-2*a.3^-1, 
          a.3^-1*a.4*a.3*a.8^-2*a.7^-2*a.5^-1*a.4^-1, 
          a.4^2*a.8^-2*a.7^-2*a.6^-2*a.5^-1, 
          a.1^-1*a.5*a.1*a.8^-1*a.7^-1*a.6^-1*a.5^-1, a.2^-1*a.5*a.2*a.5^-1, 
          a.3^-1*a.5*a.3*a.8^-2*a.6^-1*a.5^-1, a.4^-1*a.5*a.4*a.7^-1*a.5^-1, 
          a.5^2, a.1^-1*a.6*a.1*a.8^-1*a.7^-2*a.6^-1, 
          a.2^-1*a.6*a.2*a.8^-2*a.6^-2, a.3^-1*a.6*a.3*a.8^-2*a.7^-2*a.6^-2, 
          a.4^-1*a.6*a.4*a.8^-1*a.7^-2, a.5^-1*a.6*a.5*a.8^-2*a.6^-2, a.6^3, 
          a.1^-1*a.7*a.1*a.6^-2, a.2^-1*a.7*a.2*a.7^-2*a.6^-2, 
          a.3^-1*a.7*a.3*a.8^-1*a.7^-1*a.6^-2, a.4^-1*a.7*a.4*a.6^-1, 
          a.5^-1*a.7*a.5*a.7^-2, a.6^-1*a.7*a.6*a.8^-1*a.7^-1, a.7^3, 
          a.1^-1*a.8*a.1*a.8^-2, a.2^-1*a.8*a.2*a.8^-1,
          a.3^-1*a.8*a.3*a.8^-1, a.4^-1*a.8*a.4*a.8^-1,
          a.5^-1*a.8*a.5*a.8^-1, a.6^-1*a.8*a.6*a.8^-1, 
          a.7^-1*a.8*a.7*a.8^-1, a.8^3 ] )
\endexample

This implementation was developed in C by

Alice C. Niemeyer
Department of Mathematics
University of Western Australia
Nedlands, WA 6009
Australia

email: alice@maths.uwa.edu.au 

%%%%%%%%%%%%%%%%%%%%%%%%%%%%%%%%%%%%%%%%%%%%%%%%%%%%%%%%%%%%%%%%%%%%%%%%%
\Section{Installing the ANU Sq Package}

The ANU Sq  requires Steve Linton\'s vector enumerator (either as standalone
or as GAP share  library). Make sure  that   it is installed before   trying
to install the ANU Sq.

Go to the `pkg/anusq' directory.

Call `./configure ../..'  

Edit `Makefile' to have the variable `ME' point to the location of the
vector enumerator binary.

Call `make bsd-gcc'  (replace `bsd-gcc' by the architecture you use, `make'
without argument gives a list)
