%%%%%%%%%%%%%%%%%%%%%%%%%%%%%%%%%%%%%%%%%%%%%%%%%%%%%%%%%%%%%%%%%%%%%%%%%%%%%%%%%%%%%%%%%%%%%%%%%%%%
%%
%%  simplegp.tex                                                                         Stefan Kohl
%%
%%%%%%%%%%%%%%%%%%%%%%%%%%%%%%%%%%%%%%%%%%%%%%%%%%%%%%%%%%%%%%%%%%%%%%%%%%%%%%%%%%%%%%%%%%%%%%%%%%%%

%\documentclass{tran-l}
\documentclass{amsart}

\usepackage{enumerate}
\usepackage{amssymb}
\usepackage{latexsym}
\usepackage{times}

\copyrightinfo{2006}{}

%%%%%%%%%%%%%%%%%%%%%%%%%%%%%%%%%%%%%%%%%%%%%%%%%%%%%%%%%%%%%%%%%%%%%%%%%%%%%%%%%%%%%%%%%%%%%%%%%%%%
%%
%%  macros.tex                                                                           Stefan Kohl
%%
%%%%%%%%%%%%%%%%%%%%%%%%%%%%%%%%%%%%%%%%%%%%%%%%%%%%%%%%%%%%%%%%%%%%%%%%%%%%%%%%%%%%%%%%%%%%%%%%%%%%

\makeatletter

\newcommand{\NN}{\mathbb{N}}
\newcommand{\PP}{\mathbb{P}}
\newcommand{\ZZ}{\mathbb{Z}}
\newcommand{\QQ}{\mathbb{Q}}

\renewcommand{\leq}{\leqslant}
\renewcommand{\geq}{\geqslant}
\newcommand{\ti}[1]{\tilde{#1}}

\newcommand{\lcm}{\mathop{\operator@font lcm}}
\newcommand{\Units}[1]{#1^\times}
\newcommand{\im}{\mathop{\operator@font im}}
\newcommand{\ord}[1]{\mathop{\operator@font ord}(#1)}
\newcommand{\supp}[1]{\mathop{\operator@font supp}(#1)}

\newcommand{\C}[1]{\mathop{\operator@font C \kern 0pt}_{#1}}
\newcommand{\Sym}[1]{\mathop{\operator@font S \kern 0pt}_{#1}}
\newcommand{\SYM}[1]{\mathop{\operator@font Sym \kern 0pt}(#1)}
\newcommand{\PSL}[2]{\mathop{\operator@font PSL}(#1,#2)}

\newcommand{\RCWA}[1]{\mathop{\operator@font RCWA}(#1)}
\newcommand{\RCWAp}[1]{\mathop{\operator@font RCWA^+}(#1)}
\newcommand{\CT}[1]{\mathop{\operator@font CT}(#1)}
\newcommand{\CTP}[1]{\mathop{\operator@font CT}_{\mathbb{P}}(#1)}
\newcommand{\CTPk}[2]{\mathop{\operator@font CT}_{\mathbb{P}_#1}(#2)}
\newcommand{\CTint}[1]{\mathop{\operator@font CT}_{\rm int}(#1)}
\newcommand{\Mod}[1]{\mathop{\operator@font Mod}(#1)}
\newcommand{\Pcal}{\mathcal{P}}

\def\GAP{{\sf GAP}\hspace{1mm}}
\def\RCWAPackage{{\sf RCWA}\hspace{1.5mm}}

\makeatother

%%%%%%%%%%%%%%%%%%%%%%%%%%%%%%%%%%%%%%%%%%%%%%%%%%%%%%%%%%%%%%%%%%%%%%%%%%%%%%%%%%%%%%%%%%%%%%%%%%%%

\allowdisplaybreaks[1]

\theoremstyle{definition} \newtheorem{CTZDefinition}{Definition}[section]
\theoremstyle{plain}      \newtheorem{CTZPropertiesTheorem}[CTZDefinition]{Theorem}
\theoremstyle{plain}      \newtheorem{CTZSubgroupsTheorem}[CTZDefinition]{Theorem}

\theoremstyle{definition} \newtheorem{RcwaMappingDefinition}{Definition}[section]
\theoremstyle{definition} \newtheorem{RCWADefinition}[RcwaMappingDefinition]{Definition}
\theoremstyle{plain}      \newtheorem{CTZNotFinitelyGeneratedTheorem}
                                     [RcwaMappingDefinition]{Theorem}
\theoremstyle{definition} \newtheorem{CTZSmEmbeddingDefinition}[RcwaMappingDefinition]{Definition}
\theoremstyle{plain}      \newtheorem{CTZHighlyTransitiveTheorem}[RcwaMappingDefinition]{Theorem}
\theoremstyle{plain}      \newtheorem{CTZTorsionElementsDivisibleTheorem}
                                     [RcwaMappingDefinition]{Theorem}

\theoremstyle{plain}      \newtheorem{CTLemma}{Lemma}[section]
\theoremstyle{plain}      \newtheorem{IntegralCommutatorLemma}[CTLemma]{Lemma}
\theoremstyle{plain}      \newtheorem{NormalSubgroupContainsIntegralElementLemma}[CTLemma]{Lemma}
\theoremstyle{plain}      \newtheorem{CTZSimpleTheorem}[CTLemma]{Theorem}
\theoremstyle{remark}     \newtheorem{CTZSimpleRemark}[CTLemma]{Remark}
\theoremstyle{definition} \newtheorem{CTPZDefinition}[CTLemma]{Definition}
\theoremstyle{plain}      \newtheorem{CTPZSimpleCorollary}[CTLemma]{Corollary}
\theoremstyle{plain}      \newtheorem{CTPZSimpleProblem}[CTLemma]{Problem}

\theoremstyle{plain}      \newtheorem{FnPSL2ZEmbeddingTheorem}{Theorem}[section]
\theoremstyle{plain}      \newtheorem{FreeProductEmbeddingTheorem}[FnPSL2ZEmbeddingTheorem]{Theorem}
\theoremstyle{definition} \newtheorem{RestrictionMonomorphismDefinition}
                                     [FnPSL2ZEmbeddingTheorem]{Definition}
\theoremstyle{plain}      \newtheorem{DirectAndWreathProductsEmbeddingTheorem}
                                     [FnPSL2ZEmbeddingTheorem]{Theorem}
\theoremstyle{plain}      \newtheorem{DirectAndWreathProductsEmbeddingCorollary}
                                     [FnPSL2ZEmbeddingTheorem]{Corollary}
\theoremstyle{definition} \newtheorem{CTintZDefinition}[FnPSL2ZEmbeddingTheorem]{Definition}
\theoremstyle{plain}      \newtheorem{CTintZSimpleTheorem}[FnPSL2ZEmbeddingTheorem]{Theorem}

\theoremstyle{definition} \newtheorem{KernelDefinition}{Definition}[section]
\theoremstyle{definition} \newtheorem{TameWildDefinition}[KernelDefinition]{Definition}
\theoremstyle{definition} \newtheorem{SimpleSupergroupsDefinition}[KernelDefinition]{Definition}
\theoremstyle{definition} \newtheorem{CSCRDefinition}[KernelDefinition]{Definition}
\theoremstyle{plain}      \newtheorem{SimpleSupergroupsGeneratorsTheorem}[KernelDefinition]{Theorem}
\theoremstyle{plain}      \newtheorem{SimpleSupergroupsTheorem}[KernelDefinition]{Theorem}
\theoremstyle{plain}      \newtheorem{SimpleSupergroupsTransitivityTheorem}
                                     [KernelDefinition]{Theorem}
\theoremstyle{plain}      \newtheorem{TameGenerationConjecture}[KernelDefinition]{Conjecture}
\theoremstyle{remark}     \newtheorem{TameGenerationRemark}[KernelDefinition]{Remark}

\begin{document}

\title[A Simple Group Generated by Involutions Interchanging Residue Classes]
{A Simple Group Generated by Involutions Interchanging Residue Classes of the Integers}

\author{Stefan Kohl}

\address{Institut f\"ur Geometrie und Topologie \\
         Pfaffenwaldring 57 \\
         Universit\"at Stuttgart \newline
         70550 Stuttgart \\
         Germany}

\email{kohl@mathematik.uni-stuttgart.de}

\subjclass[2000]{Primary 20E32, secondary 20B40, 20B22, 20-04.}

\date{}

\begin{abstract}
  We present a countable simple group which arises in a natural way from the arithmetical
  structure of the ring of integers.
\end{abstract}

\maketitle

\section{Introduction} \label{Introduction}

Various types of infinite simple groups are treated in the literature so far:
We refer to Carter~\cite{Carter72} for the simple groups of Lie type, to Higman~\cite{Higman74}
and Stein~\cite{Stein92} for finitely presented simple groups, to Kegel,
Wehrfritz~\cite{KegelWehrfritz73} for locally finite simple groups, to Baer~\cite{Baer34} for
composition factors of infinite symmetric groups, and to Ol'shanskii~\cite{Olshanskii79} and
Chehata~\cite{Chehata52} for constructions of simple groups with certain given special properties.

Here we present and investigate an infinite simple group, which emerges in a natural way from
the arithmetical structure of the ring of integers:

\begin{CTZDefinition} \label{CTZDefinition}
  Let \(\CT{\ZZ}\) be the group generated by the set of all \emph{class transpositions}:
  Given disjoint residue classes \(r_1(m_1)\) and \(r_2(m_2)\) of~\(\ZZ\), we define the
  \emph{class transposition} \(\tau_{r_1(m_1),r_2(m_2)} \in \SYM{\ZZ}\) as the permutation which
  interchanges \(r_1 + km_1\) and \(r_2 + km_2\) for each integer~\(k\) and which fixes all other
  points. Here we assume that \(0 \leq r_1 < m_1\) and that \(0 \leq r_2 < m_2\).
  For convenience, we set \(\tau := \tau_{0(2),1(2)}: n \mapsto n + (-1)^n\).
\end{CTZDefinition}

The theorems given below list various properties of the group \(\CT{\ZZ}\) and provide information
on the class of groups which embed into it, respectively. Their proof is the main subject of this
article.

\begin{CTZPropertiesTheorem}[Properties of \(\CT{\ZZ}\)] \label{CTZPropertiesTheorem}
  \verb| |
  \begin{enumerate}

    \item The group \(\CT{\ZZ}\) is simple.

    \item The group \(\CT{\ZZ}\) is countable, but it has an uncountable series of simple subgroups
          which is parametrized by the sets of odd primes.

    \item The group \(\CT{\ZZ}\) is not finitely generated.

    \item The torsion elements of \(\CT{\ZZ}\) are divisible.

    \item The group \(\CT{\ZZ}\) acts highly transitively on~\(\NN_0\), and it has a locally
          finite simple subgroup which does so as well.

  \end{enumerate}
\end{CTZPropertiesTheorem}

\begin{CTZSubgroupsTheorem}[Richness of the class of subgroups of \(\CT{\ZZ}\)]
\label{CTZSubgroupsTheorem}
  \verb| |
  \begin{enumerate}

    \item Every finite group embeds into \(\CT{\ZZ}\).

    \item Every free group of finite rank embeds into \(\CT{\ZZ}\).

    \item The modular group \(\PSL{2}{\ZZ}\) embeds into \(\CT{\ZZ}\).

    \item Every free product of finitely many finite groups embeds into \(\CT{\ZZ}\).

    \item The class of subgroups of \(\CT{\ZZ}\) is closed under taking
          \begin{enumerate}

            \item direct products,

            \item wreath products with finite groups, and

            \item restricted wreath products with \((\ZZ,+)\).

          \end{enumerate}

    \item The group \(\CT{\ZZ}\) has
          \begin{enumerate}

            \item finitely generated subgroups which do not have finite presentations, and

            \item finitely generated subgroups with unsolvable membership problem.

          \end{enumerate}

  \end{enumerate}
\end{CTZSubgroupsTheorem}

So far, research in computational group theory focussed mainly on finite
permutation groups, matrix groups, finitely presented groups, polycyclically presented groups
and automata groups. For details, we refer to~\cite{HoltEickOBrien05}.

This article describes another large class of groups which are accessi\-ble to computational
methods. This class includes the subgroups of \(\CT{\ZZ}\). Algorithms to compute with such
groups are described in~\cite{Kohl07b} and implemented in the package \RCWAPackage~\cite{Kohl05a}
for the computer algebra system \GAP~\cite{GAP4}. Many of the results proved in this article
have first been discovered during extensive experiments with the \RCWAPackage package.

As a little example of how to compute in the group \(\CT{\ZZ}\), we factor the permutation
\[
  \alpha \in \SYM{\ZZ}: \ \ n \ \longmapsto \
  \begin{cases}
    2n/3       & \text{if} \ \ n \in 0(3), \\
    (4n - 1)/3 & \text{if} \ \ n \in 1(3), \\
    (4n + 1)/3 & \text{if} \ \ n \in 2(3)
  \end{cases}
\]
into class transpositions, which shows that \(\alpha \in \CT{\ZZ}\). This permutation has
already been investigated by Lothar Collatz in 1932, and its cycle structure is unknown so far
(cf. Keller~\cite{Keller99}, Wirsching~\cite{Wirsching96}).

In addition to the results given in Theorem~\ref{CTZPropertiesTheorem}
and~\ref{CTZSubgroupsTheorem}, in the last section we determine two simple supergroups
of \(\CT{\ZZ}\) which are in a certain sense `canonical'.

\section{Basic Properties of \(\CT{\ZZ}\)} \label{BasicPropertiesSection}

In this section we prove that the group \(\CT{\ZZ}\) is not finitely generated, that every
finite group embeds into it, that its torsion elements are divisible and that it acts
highly transitively on~\(\NN_0\). This covers Theorem~\ref{CTZPropertiesTheorem},
\mbox{Assertion (3), (4)} and the first part of~(5), as well as Theorem~\ref{CTZSubgroupsTheorem},
Assertion~(1). However, first we need to introduce some basic terms:

\begin{RcwaMappingDefinition} \label{RcwaMappingDefinition}
  We call a mapping \(f: \ZZ \rightarrow \ZZ\) \emph{residue-class-wise affine} if there
  is a positive integer~\(m\) such that the restrictions of \(f\) to the residue classes
  \(r(m) \in \ZZ/m\ZZ\) are all affine, i.e. given by
  \(f|_{r(m)}: \ r(m) \rightarrow \ZZ, \ n \mapsto (a_{r(m)} \cdot n + b_{r(m)})/c_{r(m)}\)
  for certain coefficients \(a_{r(m)}\), \(b_{r(m)}\), \(c_{r(m)} \in \ZZ\) depending on~\(r(m)\).
  We call the least possible \(m\) the \emph{modulus} of \(f\), written \(\Mod{f}\).
  For reasons of uniqueness, we assume that \(\gcd(a_{r(m)},b_{r(m)},c_{r(m)}) = 1\)
  and that \(c_{r(m)} > 0\).
  We define the \emph{multiplier} of~\(f\) by \(\lcm_{r(m) \in \ZZ/m\ZZ} a_{r(m)}\),
  and the \emph{divisor} of~\(f\) by \(\lcm_{r(m) \in \ZZ/m\ZZ} c_{r(m)}\).
  We call the mapping~\(f\) \emph{integral} if its divisor is~1. We call \(f\) \emph{class-wise}
  \emph{order-preserving} if all \(a_{r(m)}\) are positive.
\end{RcwaMappingDefinition}

It is easy to see that the permutations of this kind form a countable supergroup of~\(\CT{\ZZ}\).

\begin{RCWADefinition} \label{RCWADefinition}
  We denote the group which is formed by all residue-class-wise affine permutations of~\(\ZZ\)
  by \(\RCWA{\ZZ}\), and call its subgroups \emph{residue-class-wise affine} groups.
\end{RCWADefinition}

The notation `\(\CT{\ZZ}\)' respectively `\(\RCWA{\ZZ}\)' reflects that generalizations to
suitable rings other than~\(\ZZ\) make perfect sense. For the sake of simplicity and to keep the
article easy to read, we refrain from following this possibly fruitful direction of research here.

\goodbreak

\begin{CTZNotFinitelyGeneratedTheorem} \label{CTZNotFinitelyGeneratedTheorem}
  The group \(\CT{\ZZ}\) is not finitely generated.
\end{CTZNotFinitelyGeneratedTheorem}
\begin{proof}
  It is easy to see that the multiplier of a product of residue-class-wise affine permutations
  divides the product of the multipliers of the factors, and that the corresponding assertion about
  divisors holds as well. Further, inversion obviously interchanges multiplier and divisor.
  Therefore as there are infinitely many primes and as for any~\(n \in \NN\) there is a class
  transposition \(\tau_{1(2),0(2n)}\) with multiplier and divisor~\(n\), the assertion follows.
\end{proof}

\noindent Finite symmetric groups embed into \(\CT{\ZZ}\):

\begin{CTZSmEmbeddingDefinition} \label{CTZSmEmbeddingDefinition}
  Let \(m \in \NN\), and let \(\Sym{m}\) be the symmetric group of degree~\(m\).
  We define the monomorphism \(\varphi_m: {\rm S}_m \hookrightarrow \CT{\ZZ}\) by
  \(\sigma \mapsto \left(\sigma^{\varphi_m}: n \mapsto n + (n \hspace{-1.6mm} \mod
  \hspace{-0.4mm} m)^\sigma - n \hspace{-1.6mm} \mod \hspace{-0.4mm} m \right)\),
  where we assume that \(\Sym{m}\) acts naturally on the set \(\{0,1, \dots, m-1\}\).
\end{CTZSmEmbeddingDefinition}

\begin{CTZHighlyTransitiveTheorem} \label{CTZHighlyTransitiveTheorem}
  Any finite group embeds into \(\CT{\ZZ}\), and the group \(\CT{\ZZ}\) acts highly
  transitively on~\(\NN_0\).
\end{CTZHighlyTransitiveTheorem}
\begin{proof}
  The first assertion is immediate.

  Let \(m \in \NN\). Just like the group~\(\Sym{m}\) itself, its image under \(\varphi_m\) acts
  \(m\)-transitively on the set \(\{0,1, \dots, m-1\}\). The second assertion follows since \(m\)
  can be chosen arbitrary large and since class transpositions map nonnegative integers to
  nonnegative integers.
\end{proof}

\begin{CTZTorsionElementsDivisibleTheorem} \label{CTZTorsionElementsDivisibleTheorem}
  The torsion elements of \(\CT{\ZZ}\) are divisible.
\end{CTZTorsionElementsDivisibleTheorem}
\begin{proof}
  We show that given an element \(g \in \CT{\ZZ}\) of finite order and a positive
  integer~\(k\), there is always an \(h \in \CT{\ZZ}\) such that \(h^k = g\):
  Since \(g\) has finite order, it permutes a partition~\(\Pcal\)
  of~\(\ZZ\) into finitely many residue classes on all of which it is affine.
  A \(k\)-th root~\(h\) can be constructed from \(g\) by `slicing' cycles
  \(\prod_{i=2}^l \tau_{r_1(m_1),r_i(m_i)}\) on~\(\Pcal\) into cycles
  \(\prod_{i=1}^l \prod_{j=\max(2-i,0)}^{k-1} \tau_{r_1(km_1),r_i+jm_i(km_i)}\) of the
  \(k\)-fold length on the refined partition obtained from \(\Pcal\) by decomposing
  any \(r_i(m_i) \in \Pcal\) into residue classes (mod~\(km_i\)).
\end{proof}

\section{The Simplicity of \(\CT{\ZZ}\)} \label{CTZSimpleSection}

The aim of this section is to show that the group \(\CT{\ZZ}\) is simple, and that it has
an uncountable series of simple subgroups which is parametrized by the sets of odd primes.
This covers Theorem~\ref{CTZPropertiesTheorem}, Assertion (1) and~(2).
First we need some lemmata:

\begin{CTLemma} \label{CTLemma}
  Given two class transpositions \(\tau_{r_1(m_1),r_2(m_2)}, \tau_{r_3(m_3),r_4(m_4)}\)
  not equal to \(\tau\), there is a product \(\pi\) of 6 class transpositions such that
  \(\tau_{r_1(m_1),r_2(m_2)}^\pi = \tau_{r_3(m_3),r_4(m_4)}\).
\end{CTLemma}
\begin{proof}
  Let \(r_5(m_5), r_6(m_6) \subset \ZZ \setminus (r_1(m_1) \cup r_2(m_2))\) be disjoint residue
  classes such that \(\cup_{i=3}^6 r_i(m_i) \neq \ZZ\), and let
  \(r_7(m_7), r_8(m_8) \subset \ZZ \setminus \cup_{i=3}^6 r_i(m_i)\) be disjoint
  residue classes. Then the following hold:
  \begin{enumerate}

    \item \({\tau_{r_1(m_1),r_2(m_2)}}^{\displaystyle{       \tau_{r_1(m_1),r_5(m_5)}
                                                       \cdot \tau_{r_2(m_2),r_6(m_6)}}}
             \ = \ \tau_{r_5(m_5),r_6(m_6)}\).

    \item \({\tau_{r_5(m_5),r_6(m_6)}}^{\displaystyle{       \tau_{r_5(m_5),r_7(m_7)}
                                                       \cdot \tau_{r_6(m_6),r_8(m_8)}}}
             \ = \ \tau_{r_7(m_7),r_8(m_8)}\).

    \item \({\tau_{r_7(m_7),r_8(m_8)}}^{\displaystyle{       \tau_{r_3(m_3),r_7(m_7)}
                                                       \cdot \tau_{r_4(m_4),r_8(m_8)}}}
             \ = \ \tau_{r_3(m_3),r_4(m_4)}\).

  \end{enumerate}
  The assertion follows.
\end{proof}

\begin{IntegralCommutatorLemma} \label{IntegralCommutatorLemma}
  Let \(\sigma, \upsilon \in \RCWA{\ZZ}\), and put \(m := \Mod{\sigma}\).
  If \(\upsilon\) is integral and fixes all residue classes (mod~\(m\)) setwise,
  then the commutator \([\sigma,\upsilon]\) is integral as well.
\end{IntegralCommutatorLemma}
\begin{proof}
  Since \(\upsilon\) fixes all residue classes (mod~\(m\)), an affine partial mapping \(\alpha\)
  of \([\sigma,\upsilon]\) is given by \(\alpha_{\upsilon^{-1}}^{\alpha_{\sigma}} \cdot
  \alpha_{\upsilon}\) for certain affine partial mappings \(\alpha_{\sigma}\), \(\alpha_{\upsilon}\)
  and \(\alpha_{\upsilon^{-1}}\) of \(\sigma\), \(\upsilon\) and \(\upsilon^{-1}\), respectively.
  The assertion follows, since the translations and reflections generate a normal subgroup of the
  affine group of the rationals.
\end{proof}

\begin{NormalSubgroupContainsIntegralElementLemma}
\label{NormalSubgroupContainsIntegralElementLemma}
  Let \(G\) be a subgroup of \(\RCWA{\ZZ}\) which contains \(\CT{\ZZ}\).
  Then any nontrivial normal subgroup \(N \unlhd G\) has an integral element \(\iota \neq 1\).
\end{NormalSubgroupContainsIntegralElementLemma}
\begin{proof}
  Let \(\sigma \in N \setminus \{1\}\), and let \(m := \Mod{\sigma}\).
  Without loss of generality we can assume that there is a residue class \(r(m)\)
  such that \(r(m)^\sigma \neq r(m)\). By Lemma~\ref{IntegralCommutatorLemma},
  the mapping \(\iota := [\sigma,\tau_{r(2m),r+m(2m)}] \in N \setminus \{1\}\) is integral.
\end{proof}

\noindent Now we can prove our theorem:

\begin{CTZSimpleTheorem} \label{CTZSimpleTheorem}
  The group \(\CT{\ZZ}\) is simple.
\end{CTZSimpleTheorem}
\begin{proof}
  Let \(N\) be a nontrivial normal subgroup of \(\CT{\ZZ}\).
  We have to show that \(N\) contains all class transpositions.

  By Lemma~\ref{CTLemma}, all class transpositions except of \(\tau\) are conjugate in \(\CT{\ZZ}\).
  Furthermore we have \(\tau = \tau_{0(4),1(4)} \cdot \tau_{2(4),3(4)}\). Therefore it is already
  sufficient to show that \(N\) contains one class transposition which is not equal to~\(\tau\).

  By Lemma~\ref{NormalSubgroupContainsIntegralElementLemma}, the normal subgroup \(N\) has
  an integral element \(\iota_1 \neq 1\). Let \(m \geq 3\) be a multiple of the modulus
  of~\(\iota_1\), and choose a residue class \(r(m)\) which is moved by~\(\iota_1\). Then
  put \(\iota_2 := \tau_{r(2m),r+m(2m)} \cdot \tau_{r(2m)^{\iota_1},(r+m(2m))^{\iota_1}} =
  [\tau_{r(2m),r+m(2m)},\iota_1] \in N\).

  By the choice of \(m\), we can now choose two distinct residue classes \(r_1(2m)\) and
  \(r_2(2m)\) in the complement of the support of~\(\iota_2\). Then we have
  \begin{align*}
    \tau_{r_1(2m),r_2(2m)} \ = \ \ \
              &\iota_2^{\tau_{r(2m),r_1(4m)}    \cdot \tau_{r+m(2m),r_2(4m)}} \\
      \cdot \ &\iota_2^{\tau_{r(2m),r_1+2m(4m)} \cdot \tau_{r+m(2m),r_2+2m(4m)}} \ \in \ N,
  \end{align*}
  which completes the proof of the theorem.
\end{proof}

\begin{CTZSimpleRemark} \label{CTZSimpleRemark}
  Assume \(\CT{\ZZ} \leq G \leq \RCWA{\ZZ}\), and let \(N\) be a nontrivial normal subgroup
  of~\(G\). Then the proof of Theorem~\ref{CTZSimpleTheorem} shows in fact that \(N\)
  contains~\(\CT{\ZZ}\), if we additionally take care that in the third paragraph we
  choose \(m\) sufficiently large such that there is indeed a residue class \(r(m)\) which
  is not mapped to itself under~\(\iota_1\).
\end{CTZSimpleRemark}

\begin{CTPZDefinition} \label{CTPZDefinition}
  Given a set \(\PP\) of odd primes, let \(\CTP{\ZZ} \leq \CT{\ZZ}\) denote the subgroup which
  is generated by all class transpositions \(\tau_{r_1(m_1),r_2(m_2)}\) for which all odd prime
  factors of \(m_1\) and \(m_2\) lie in~\(\PP\).
\end{CTPZDefinition}

\begin{CTPZSimpleCorollary} \label{CTPZSimpleCorollary}
  The groups \(\CTP{\ZZ}\) are simple. Therefore the group \(\CT{\ZZ}\) has an uncountable
  series of simple subgroups, which is parametrized by the sets of odd primes.
\end{CTPZSimpleCorollary}
\begin{proof}
  All of our arguments in this section apply to the groups~\(\CTP{\ZZ}\) as well:
  In the proof of Lemma~\ref{CTLemma}, we can choose the four residue classes
  \(r_5(m_5), \dots, r_8(m_8)\) in such a way that all prime factors of their moduli
  already divide \(m_1 m_2 m_3 m_4\). The proofs of Lemma~\ref{IntegralCommutatorLemma},
  Lemma~\ref{NormalSubgroupContainsIntegralElementLemma} and Theorem~\ref{CTZSimpleTheorem}
  likewise do not require the presence of class transpositions whose moduli have certain
  odd factors.
\end{proof}

\begin{CTPZSimpleProblem}[Isomorphism Problem] \label{CTPZSimpleProblem}
  Are there are distinct sets \(\PP_1\) and \(\PP_2\) of odd primes such that
  \(\CTPk{1}{\ZZ} \cong \CTPk{2}{\ZZ}\)?
\end{CTPZSimpleProblem}

\goodbreak

\section{Richness of the Class of Subgroups of \(\CT{\ZZ}\)} \label{CTZSubgroupsSection}

In this section we prove Theorem~\ref{CTZSubgroupsTheorem}, Assertion (2) -- (6),
as well as the second part of Theorem~\ref{CTZPropertiesTheorem}, Assertion~(5).

\goodbreak

\begin{FnPSL2ZEmbeddingTheorem} \label{FnPSL2ZEmbeddingTheorem}
  Free groups of finite rank and \(\PSL{2}{\ZZ}\) embed into~\(\CT{\ZZ}\).
\end{FnPSL2ZEmbeddingTheorem}
\begin{proof}
  To prove the assertion concerning free groups, it suffices to show that the free group
  of rank~2 embeds. An example of an embedding is
  \[
    \varphi_{{\rm F}_2}: \ \ {\textrm F}_2 = \langle a, b \rangle \
    \hookrightarrow \ \CT{\ZZ}, \ \
    a \ \mapsto \ (\tau \cdot \tau_{0(2),1(4)})^2, \ \ 
    b \ \mapsto \ (\tau \cdot \tau_{0(2),3(4)})^2.
  \]
  This can be seen by applying the Table-Tennis Lemma (see for example
  la~Harpe~\cite{LaHarpe00}, Section~II.B.) to the cyclic groups generated by the
  images of~\(a\) and~\(b\) under~\(\varphi_{{\rm F}_2}\) and the sets \(0(4) \cup 1(4)\) and
  \(2(4) \cup 3(4)\). 
  Likewise it follows from the Table-Tennis Lemma that
  \begin{align*}
    \varphi_{\PSL{2}{\ZZ}}: \ \ &\PSL{2}{\ZZ} \ \cong \ {\rm C}_2 \star {\rm C}_3 \ \cong \
    \langle a,b \ | \ a^2 = b^3 = 1 \rangle \ \hookrightarrow \ \CT{\ZZ}, \\
    &a \ \mapsto \ \tau, \ \ b \ \mapsto \ \tau_{0(4),2(4)} \cdot \tau_{1(2),0(4)}
  \end{align*}
  is an embedding of the modular group \(\PSL{2}{\ZZ}\). This time one can use the
  sets \(1(2)\) and \(0(2)\) in place of \(0(4) \cup 1(4)\) and \(2(4) \cup 3(4)\).
\end{proof}

\begin{FreeProductEmbeddingTheorem} \label{FreeProductEmbeddingTheorem}
  Every free product of finitely many finite groups embeds into~\(\CT{\ZZ}\).
\end{FreeProductEmbeddingTheorem}
\begin{proof}
  Let \(G_0, \dots, G_{m-1}\) be finite groups. To see that their free product embeds
  into \(\CT{\ZZ}\), proceed as follows: First consider regular permutation representations
  \(\varphi_r\) of the groups \(G_r\) on the residue classes (mod~\(|G_r|\)).
  Then take conjugates \(H_r := (\im \varphi_r)^{\sigma_r}\) of the images of these
  representations under mappings \(\sigma_r \in \CT{\ZZ}\) which map \(0(|G_r|)\) to
  \(\ZZ \setminus r(m)\). Finally use the fact that point stabilizers in regular permutation
  groups are trivial and apply the Table-Tennis Lemma to the groups \(H_r\) and the residue
  classes \(r(m)\) to see that the group generated by the \(H_r\) is isomorphic to their
  free product.
\end{proof}

\noindent The group \(\RCWA{\ZZ}\) is not co-Hopfian.
We need the following monomorphisms:

\begin{RestrictionMonomorphismDefinition} \label{RestrictionMonomorphismDefinition}
  Let \(f\) be an injective residue-class-wise affine mapping, and let further
  \(\pi_f: \RCWA{\ZZ} \hookrightarrow \RCWA{\ZZ}, \sigma \mapsto \sigma_f\) be the
  monomorphism defined by the properties \(\forall \sigma \in \RCWA{\ZZ} \ f \sigma_f = \sigma f\)
  and \(\supp{\im \pi_f} \subseteq \im f\).
  Then we call \(\pi_f\) the \emph{restriction monomorphism} associated with~\(f\).
\end{RestrictionMonomorphismDefinition}

\begin{DirectAndWreathProductsEmbeddingTheorem} \label{DirectAndWreathProductsEmbeddingTheorem}
  The class of subgroups of \(\CT{\ZZ}\) is closed under taking \mbox{direct} pro\-ducts, under
  taking wreath products with finite groups and \mbox{under} taking restricted wreath products
  with \((\ZZ,+)\). It is also closed under taking upwards extensions by finite groups.
\end{DirectAndWreathProductsEmbeddingTheorem}
\begin{proof}
  Given any two subgroups \(G, H \leq \RCWA{\ZZ}\), the group generated by
  \(\pi_{n \mapsto 2n}(G)\) and \(\pi_{n \mapsto 2n+1}(H)\) is clearly isomorphic
  to \(G \times H\). This argument applies to subgroups of our group \(\CT{\ZZ}\) as well,
  since the image of a class transposition \(\tau_{r_1(m_1),r_2(m_2)}\) under
  a restriction monomorphism \(\pi_{n \mapsto mn+r}\) is \(\tau_{mr_1+r(mm_1),mr_2+r(mm_2)}\).

  Looking at the monomorphisms \(\pi_{n \mapsto mn+r}\) and \(\varphi_m\), it is immediate to see
  that the classes of subgroups of \(\RCWA{\ZZ}\) and \(\CT{\ZZ}\) are also closed under
  taking wreath pro\-ducts with finite groups. The assertion on upwards extensions by finite
  groups follows now from the Universal Embedding Theorem (see e.g. Theorem~2.6A in Dixon,
  Mortimer~\cite{DixonMortimer96}).

  Given a subgroup \(G \leq \CT{\ZZ}\), the group generated by \(\pi_{n \mapsto 4n+3}(G)\) and
  \(\tau \cdot \tau_{0(2),1(4)}\) is isomorphic to the restricted wreath product \(G \wr (\ZZ,+)\).
  This holds since the orbit of the residue class 3(4) under the action of the cyclic group
  \(\langle \tau \cdot \tau_{0(2),1(4)} \rangle\) consists of pairwise disjoint residue classes,
  which means that the conjugates of \(\pi_{n \mapsto 4n+3}(G)\) under powers
  of~\(\tau \cdot \tau_{0(2),1(4)}\) have pairwise disjoint supports.
\end{proof}

\begin{DirectAndWreathProductsEmbeddingCorollary} \label{DirectAndWreathProductsEmbeddingCorollary}
  The group \(\CT{\ZZ}\) has
  \begin{enumerate}

    \item finitely generated subgroups which do not have finite presentations, and

    \item finitely generated subgroups with unsolvable membership problem.

  \end{enumerate}
\end{DirectAndWreathProductsEmbeddingCorollary}
\begin{proof}
  \verb| |
  \begin{enumerate}

    \item By Theorem~\ref{DirectAndWreathProductsEmbeddingTheorem}, the group \(\CT{\ZZ}\)
          contains nontrivial restricted wreath products \(G \wr (\ZZ,+)\).
          By Baumslag~\cite{Baumslag61}, these do not have finite presentations.

    \item Let \({\rm F}_2 = \langle a, b \rangle\) be the free group of rank~2. Further let
          \mbox{\(r_1, \dots, r_k \in {\rm F}_2\)} be the relators of a finitely presented group
          with unsolvable word problem -- by Novikov~\cite{Novikov55} and Boone~\cite{Boone59},
          such groups exist. Then the membership problem for the group
          \(\langle (a,a), (b,b), (1,r_1), \dots, (1,r_k) \rangle < {\rm F}_2 \times {\rm F}_2\)
          is unsolvable (cf. Mihailova~\cite{Mihailova66};
          see also Lyndon, Schupp~\cite{LyndonSchupp77}, Chapter~IV.4).
          Therefore as by Theorem~\ref{FnPSL2ZEmbeddingTheorem} and
          Theorem~\ref{DirectAndWreathProductsEmbeddingTheorem} the group
          \({\rm F}_2 \times {\rm F}_2\) embeds into \(\CT{\ZZ}\), there exist finitely
          generated subgroups \(G < \CT{\ZZ}\) with unsolvable membership problem.

  \end{enumerate}
  \vspace{-5mm}
\end{proof}

The class transpositions which interchange two residue classes with the same modulus generate
a proper subgroup of~\(\CT{\ZZ}\), which acts highly transitively on \(\NN_0\) as well:

\begin{CTintZDefinition} \label{CTintZDefinition}
  Let \(\CTint{\ZZ}\) denote the subgroup of \(\CT{\ZZ}\) which is gene\-rated by all integral
  class transpositions.
\end{CTintZDefinition}

\begin{CTintZSimpleTheorem} \label{CTintZSimpleTheorem}
  The group \(\CTint{\ZZ}\) is locally finite and simple.
\end{CTintZSimpleTheorem}
\begin{proof}
  Finitely generated subgroups of \(\CTint{\ZZ}\) act faithfully on the set of residue
  classes modulo the lcm of the moduli of the generators. Therefore they are finite.
  Hence the group \(\CTint{\ZZ}\) is locally finite.

  Let \(N\) be a nontrivial normal subgroup of \(\CTint{\ZZ}\).
  In order to prove that \(\CTint{\ZZ}\) is simple, we have to show that \(N\) contains
  any class transposition of the form \(\tau_{r_1(m),r_2(m)}\).

  Let \(\iota \in N\hspace{-0.2mm}\setminus\hspace{-0.2mm}\{1\}\), let \(m\!>\!2\) be
  a multiple of \(\Mod{\iota}\) and choose a residue class \(r(m)\) in the support of~\(\iota\).
  Then we have \(\tau_{r(2m),r+m(2m)} \cdot \tau_{r(2m)^{\iota},(r+m(2m))^{\iota}} \! =
  [\iota,\tau_{r(2m),r+m(2m)}] \in \! N\).

  All such products of two integral class transpositions with modulus~\(2m\) and disjoint
  supports are conjugate in \(\CTint{\ZZ}\). The reason for this is that their preimages under
  the monomorphism \(\varphi_{2m}\) have the same cycle structure, and are therefore conjugate
  in~\(\Sym{2m}\).

  Let \(\tau_1 = \tau_{r_1(m),r_2(m)}\) be an integral class transposition with modulus~\(m\).
  Then for any integral class transposition \(\tau_2\) with modulus~\(2m\) whose support intersects
  trivially with the one of~\(\tau_1\), we have \(\tau_1 = (\tau_2 \cdot \tau_{r_1(2m),r_2(2m)})
  \cdot (\tau_2 \cdot \tau_{r_1+m(2m),r_2+m(2m)}) \in N\).

  Given a divisor \(d\) of \(m\) and an integral class transposition \(\tau_{r_1(d),r_2(d)}\) with
  modulus~\(d\), we have \(\tau_{r_1(d),r_2(d)} = \prod_{k=0}^{m/d-1} \tau_{r_1+kd(m),r_2+kd(m)}
  \in N\).

  In the second paragraph of the proof, we can choose \(m\) to be a multiple of any given
  positive integer. Therefore we can conclude that \(N\) contains indeed any integral class
  transposition. Hence the group \(\CTint{\ZZ}\) is simple, as claimed. 
\end{proof}

\goodbreak

\section{Collatz' Permutation Lies in \(\CT{\ZZ}\)} \label{CollatzPermSection}

\noindent In this section, as a small but illustrative example we show that Collatz' permutation
\[
  \alpha \in \RCWA{\ZZ}: \ \ n \ \longmapsto \
  \begin{cases}
    2n/3       & \text{if} \ \ n \in 0(3), \\
    (4n - 1)/3 & \text{if} \ \ n \in 1(3), \\
    (4n + 1)/3 & \text{if} \ \ n \in 2(3)
  \end{cases}
\]
lies in \(\CT{\ZZ}\).

\goodbreak

In Keller~\cite{Keller99} it is shown that \(\alpha\) has at most finitely many cycles of any
given finite length. However according to Wirsching~\cite{Wirsching96}, it is for example not
yet known whether the cycle
\(( \ \dots \ 34 \ 45 \ 30 \ 20 \ 27 \ 18 \ 12 \
               8 \ 11 \ 15 \ 10 \ 13 \ 17 \ 23 \ 31 \ \dots \ )\)
of \(\alpha\) is finite or infinite.

In spite of this we can show that the permutation~\(\alpha\) lies in \(\CT{\ZZ}\) by determining
an explicit factorization into generators.

The major obstacle we are confronted with when trying to obtain such a factorization is the fact
that multiplier and divisor of \(\alpha\) are coprime, whereas multiplier and divisor of a class
transposition are always the same. We even need to form a product of class transpositions in such
a way that one prime divisor gets eliminated from the multiplier of the product, but appears in
the denominators of \emph{all} of its affine partial mappings.

As a first step towards a solution of the factorization problem, we hence attempt to
determine some product of class transpositions which has coprime multiplier and divisor.
We find that given an odd prime~\(p\), the permutation
\begin{align*}
  \sigma_p \ := \ \ &\tau_{0(8),1(2p)} \cdot \tau_{4(8),2p-1(2p)} \\
            \cdot \ &\tau_{0(4),1(2p)} \cdot \tau_{2(4),2p-1(2p)} \\
            \cdot \ &\tau_{2(2p),1(4p)} \cdot \tau_{4(2p),2p+1(4p)} \ \in \ \CT{\ZZ}
\end{align*}
has multiplier \(p\) and divisor~2. Indeed, evaluating this product yields
\[
  \sigma_p: \ \ n \ \longmapsto \
  \begin{cases}
    (pn + 2p - 2)/2 & \text{if} \ \ n \in 2(4), \\
    n/2             & \text{if} \ \ n \in 0(4) \setminus (4(4p) \cup 8(4p)), \\
    n + 2p - 7      & \text{if} \ \ n \in 8(4p), \\
    n - 2p + 5      & \text{if} \ \ n \in 2p-1(2p), \\
    n + 1           & \text{if} \ \ n \in 1(2p), \\
    n - 3           & \text{if} \ \ n \in 4(4p), \\
    n               & \text{if} \ \ n \in 1(2) \setminus (1(2p) \cup 2p-1(2p)).
  \end{cases}
\]
The \GAP~\cite{GAP4} package \RCWAPackage~\cite{Kohl05a} provides a factorization routine for
residue-class-wise affine permutations, which uses certain elaborate heuristics. The permutations
\(\sigma_p\) and their images under restriction monomorphisms \(\pi_{n \mapsto mn+r}\) play a key
role in this routine. It has been used to obtain the following factorization of~\(\alpha\):
\begin{align*}
  \alpha \ = \ \ &\tau_{2(3),3(6)} \cdot \tau_{1(3),0(6)}
           \cdot \tau_{0(3),1(3)} \cdot \tau
           \cdot \tau_{0(36),1(36)} \\
           \cdot \ &\tau_{0(36),35(36)} \cdot \tau_{0(36),31(36)}
           \cdot \tau_{0(36),23(36)} \cdot \tau_{0(36),18(36)}
           \cdot \tau_{0(36),19(36)} \\
           \cdot \ &\tau_{0(36),17(36)} \cdot \tau_{0(36),13(36)}
           \cdot \tau_{0(36),5(36)} \cdot \tau_{2(36),10(36)}
           \cdot \tau_{2(36),11(36)} \\
           \cdot \ &\tau_{2(36),15(36)} \cdot \tau_{2(36),20(36)}
           \cdot \tau_{2(36),28(36)} \cdot \tau_{2(36),26(36)}
           \cdot \tau_{2(36),25(36)} \\
           \cdot \ &\tau_{2(36),21(36)} \cdot \tau_{2(36),4(36)}
           \cdot \tau_{3(36),8(36)} \cdot \tau_{3(36),7(36)}
           \cdot \tau_{9(36),16(36)} \\
           \cdot \ &\tau_{9(36),14(36)} \cdot \tau_{9(36),12(36)}
           \cdot \tau_{22(36),34(36)} \cdot \tau_{27(36),32(36)}
           \cdot \tau_{27(36),30(36)} \\
           \cdot \ &\tau_{29(36),33(36)} \cdot \tau_{10(18),35(36)}
           \cdot \tau_{5(18),35(36)} \cdot \tau_{10(18),17(36)}
           \cdot \tau_{5(18),17(36)} \\
           \cdot \ &\tau_{8(12),14(24)} \cdot \tau_{6(9),17(18)}
           \cdot \tau_{3(9),17(18)} \cdot \tau_{0(9),17(18)}
           \cdot \tau_{6(9),16(18)} \\
           \cdot \ &\tau_{3(9),16(18)} \cdot \tau_{0(9),16(18)}
           \cdot \tau_{6(9),11(18)} \cdot \tau_{3(9),11(18)}
           \cdot \tau_{0(9),11(18)} \\
           \cdot \ &\tau_{6(9),4(18)} \cdot \tau_{3(9),4(18)}
           \cdot \tau_{0(9),4(18)} \cdot \tau_{0(6),14(24)}
           \cdot \tau_{0(6),2(24)} \\
           \cdot \ &\tau_{8(12),17(18)} \cdot \tau_{7(12),17(18)}
           \cdot \tau_{8(12),11(18)} \cdot \tau_{7(12),11(18)}
           \cdot \sigma_{3}^{-1} \\
           \cdot \ &\tau_{7(12),17(18)} \cdot \tau_{2(6),17(18)}
           \cdot \tau_{0(3),17(18)} \cdot \sigma_{3}^{-3}.
\end{align*}
This shows constructively that \(\alpha \in \CT{\ZZ}\).

\section{Two Simple Supergroups of \(\CT{\ZZ}\)} \label{SimpleSupergroupsSection}

In this section we present two simple subgroups of \(\RCWA{\ZZ}\) which properly contain
\(\CT{\ZZ}\) and which act highly transitively on~\(\ZZ\).

We find them in the kernels of certain epimorphisms \(\pi^+: \RCWAp{\ZZ} \rightarrow (\ZZ,+)\)
and \(\pi^-: \RCWA{\ZZ} \rightarrow \Units{\ZZ} \cong \C{2}\), where \(\RCWAp{\ZZ} < \RCWA{\ZZ}\)
denotes the subgroup consisting of all class-wise order-preserving elements.

Using the notation \(\sigma|_{r(m)}: n \mapsto (a_{r(m)} \cdot n + b_{r(m)})/c_{r(m)}\)
for the affine partial mappings of an rcwa permutation~\(\sigma\) with modulus~\(m\), these
epimorphisms are given by
\begin{align*}
  &\pi^+: \ \ \sigma \ \mapsto \
  \displaystyle{\frac{1}{m} \sum_{r(m) \in \ZZ/m\ZZ} \frac{b_{r(m)}}{|a_{r(m)}|}} \\[-2mm]
\intertext{and} \\[-9mm]
  &\pi^-: \ \ \sigma \ \mapsto \
  (-1)^{\displaystyle{\pi^+(\sigma) + \sum_{r(m): \ a_{r(m)} < 0} \frac{m - 2r}{m}}},
\end{align*}
respectively (see Sections~2.11 and~2.12 in Kohl~\cite{Kohl05b}).

\begin{KernelDefinition} \label{KernelDefinition}
  We denote the kernels of \(\pi^+\) and \(\pi^-\) by \(K^+\) and \(K^-\), respectively.
\end{KernelDefinition}

\noindent It is easy to see that \(\CT{\ZZ} < K^+ < K^- < \RCWA{\ZZ}\).

Our simple groups will be the subgroups of \(K^+\) and \(K^-\), respectively, which are generated
by the elements which are \emph{tame} in the following sense:

\begin{TameWildDefinition} \label{TameWildDefinition}
  We call an element \(\sigma \in \RCWA{\ZZ}\) \emph{tame} if it permutes a partition of \(\ZZ\)
  into finitely many residue classes on each of which it is affine, and \emph{wild} otherwise.
  We call a group \(G < \RCWA{\ZZ}\) \emph{tame} if there is a common such partition for all
  elements of~\(G\), and \emph{wild} otherwise. We call the specified partitions \emph{respected
  partitions} of~\(\sigma\) respectively~\(G\).
\end{TameWildDefinition}

\noindent For an alternative characterization of this notion of tameness and a generalization
of it to not necessarily bijective residue-class-wise affine mappings, see Kohl~\cite{Kohl05b},
\cite{Kohl07a}.

Obviously, finite residue-class-wise affine groups and integral residue-class-wise
affine permutations are tame.

Tameness is invariant under conjugation: If \(\alpha \in \RCWA{\ZZ}\) respects a partition
\(\Pcal\), then a conjugate \(\alpha^\beta\) respects the partition consisting of the images of
the intersections of the residue classes in~\(\Pcal\) with the sources of the affine partial
mappings of \(\beta\) under~\(\beta\).

The product of two tame permutations is in general not tame.
Tameness of products also does not induce an equivalence relation on the set of tame
permutations: Let for example \(a := \tau_{1(6),4(6)}\), \(b := \tau_{0(5),2(5)}\) and
\(c := \tau_{3(4),4(6)}\). Then \(ab\) and \(bc\) are tame, but \(ac\) is not.

If a tame group does not act faithfully on a respected partition, then the kernel of the action
clearly does not act on~\(\NN_0\). Thus as the group \(\CT{\ZZ}\) acts on~\(\NN_0\), its tame
subgroups are finite.

\begin{SimpleSupergroupsDefinition} \label{SimpleSupergroupsDefinition}
  We denote the normal subgroups of \(K^+\) and \(K^-\) which are generated by the tame elements
  by~\(\ti{K}^+\) and~\(\ti{K}^-\), respectively.
\end{SimpleSupergroupsDefinition}

\goodbreak

It is easy to see that all tame elements of \(\RCWA{\ZZ}\) can be factored into
class transpositions and members of the following two series:

\begin{CSCRDefinition} \label{CSCRDefinition}
  Let \(r(m) \subseteq \ZZ\) be a residue class.
  \begin{enumerate}

    \item We define the \emph{class shift} \(\nu_{r(m)} \in \RCWA{\ZZ}\) by
          \[
            \nu_{r(m)}: \ \ n \ \mapsto \
            \begin{cases}
              n + m & \text{if} \ \ n \in r(m), \\
              n     & \text{otherwise}.
            \end{cases}
          \]

    \item We define the \emph{class reflection} \(\varsigma_{r(m)} \in \RCWA{\ZZ}\) by
          \[
            \varsigma_{r(m)}: \ \ n \ \mapsto \
            \begin{cases}
              -n + 2r & \text{if} \ \ n \in r(m), \\
              n       & \text{otherwise},
            \end{cases}
          \]
          where we assume that \(0 \leq r < m\).

  \end{enumerate}
  For convenience, we set \(\nu := \nu_{\ZZ}: n \mapsto n + 1\)
  and \(\varsigma := \varsigma_{\ZZ}: n \mapsto -n\).
\end{CSCRDefinition}

Obviously, class shifts and class reflections do not lie in~\(K^-\). 

\goodbreak

\begin{SimpleSupergroupsGeneratorsTheorem} \label{SimpleSupergroupsGeneratorsTheorem}
  We have
  \begin{enumerate}

    \item \(\ti{K}^+ = \langle \CT{\ZZ}, \nu_{1(3)} \cdot \nu_{2(3)}^{-1} \rangle\), and

    \item \(\ti{K}^- = \langle \CT{\ZZ}, \nu_{1(3)} \cdot \nu_{2(3)},
                                         \varsigma_{0(2)} \cdot \nu_{0(2)} \rangle\).
  \end{enumerate}
\end{SimpleSupergroupsGeneratorsTheorem}
\begin{proof}
  We determine series of generators:
  \begin{enumerate}

    \item Considering respected partitions, we check that \(\ti{K}^+\) is generated by
          \begin{enumerate}

            \item all class transpositions and

            \item all quotients of two class shifts with disjoint supports whose union has
                  a nontrivial complement in~\(\ZZ\).

          \end{enumerate}
          For this we look at the process of factoring a given tame \(\vartheta \in K^+\)
          into these elements:
          \begin{enumerate}[{ad} (a)]

            \item Let \(\Pcal\) be a respected partition of~\(\vartheta\).
                  Divide \(\vartheta\) by a product of class transpositions which respects \(\Pcal\)
                  as well and which induces on \(\Pcal\) the same permutation as \(\vartheta\) does.
                  Now \(\vartheta\) is integral and fixes~\(\Pcal\).

            \item Finally factor \(\vartheta\) into quotients of two class shifts whose supports
                  are distinct residue classes in~\(\Pcal\). This is possible since the lattice in
                  \(\ZZ^{|\Pcal|}\) which consists of all vectors with zero coordinate sum is
                  spanned by the differences of two distinct canonical basis vectors.

          \end{enumerate}

    \goodbreak

    \item Considering respected partitions, we check that \(\ti{K}^-\) is generated by
          \begin{enumerate}

            \item all products of a class reflection and a class shift with the same support which
                  has a nontrivial complement in~\(\ZZ\),

            \item all class transpositions and

            \item all products of two class shifts with disjoint supports whose union has
                  a non\-trivial complement in~\(\ZZ\).

          \end{enumerate}
          For this we look at the process of factoring a given tame \(\vartheta \in K^-\)
          into these elements:
          \begin{enumerate}[{ad} (a)]

            \item Let \(\Pcal\) be a respected partition of \(\vartheta\) of length at least~3.
                  Divide \(\vartheta\) from the left by products \(\varsigma_{r(m)} \cdot
                  \nu_{r(m)}\), where \(r(m)\) runs over all residue classes in~\(\Pcal\) on
                  which \(\vartheta\) is order-reversing. Now \(\vartheta\) is class-wise
                  order-preserving.

            \item Divide \(\vartheta\) by a product of class transpositions which respects the
                  partition~\(\Pcal\) as well, and which also induces the same permutation on it.
                  Now \(\vartheta\) is integral and fixes~\(\Pcal\).

            \item Finally factor \(\vartheta\) into products of two class shifts whose supports
                  are distinct residue classes in~\(\Pcal\) and inverses of such products.
                  This is possible since the lattice in \(\ZZ^{|\Pcal|}\) which consists of
                  all vectors with even coordinate sum is spanned by the sums of two distinct
                  canonical basis vectors.

          \end{enumerate}

  \end{enumerate}
  Now we collapse series 1.(b), 2.(a) and 2.(c) by taking orbit representatives under
  the conjugation action of the group \(\CT{\ZZ}\) to obtain the indicated single generators.
\end{proof}

\goodbreak

\begin{SimpleSupergroupsTheorem} \label{SimpleSupergroupsTheorem}
  The groups \(\ti{K}^+\) and \(\ti{K}^-\) are simple.
\end{SimpleSupergroupsTheorem}
\begin{proof}
  By Remark~\ref{CTZSimpleRemark}, nontrivial normal subgroups of~\(\ti{K}^+\) and~\(\ti{K}^-\)
  contain~\(\CT{\ZZ}\).
  \begin{enumerate}

    \item Let \(N\) be a nontrivial normal subgroup of~\(\ti{K}^+\). Given disjoint residue classes
          \(r_1(m_1)\) and \(r_2(m_2)\) whose union has a nontrivial complement in~\(\ZZ\), for an
          arbitrary residue class \(r_3(m_3) \subseteq \ZZ \setminus (r_1(m_1) \cup r_2(m_2))\)
          we have
          \[
            \nu_{r_1(m_1)} \cdot \nu_{r_2(m_2)}^{-1} \ = \
            [\tau_{r_1(m_1),r_2(m_2)},\nu_{r_3(m_3)} \cdot \nu_{r_2(m_2)}^{-1}] \ \in \ N.
          \]
          Putting \(r_1(m_1) := 1(3)\) and \(r_2(m_2) := 2(3)\), the simplicity of \(\ti{K}^+\)
          follows from Theorem~\ref{SimpleSupergroupsGeneratorsTheorem}, Assertion~(1).

    \item Let \(N\) be a nontrivial normal subgroup of~\(\ti{K}^-\). Given disjoint residue classes
          \(r_1(m_1)\) and \(r_2(m_2)\) whose union has a nontrivial complement in~\(\ZZ\), for an
          arbitrary residue class \(r_3(m_3) \subseteq \ZZ \setminus (r_1(m_1) \cup r_2(m_2))\)
          we have
          \begin{align*}
            \nu_{r_1(m_1)} \cdot \nu_{r_2(m_2)} \ = \ \
            &[\tau_{r_1(m_1),r_2(m_2)},\nu_{r_3(m_3)} \cdot \varsigma_{r_1(m_1)}] \\
            \cdot \ &[\tau_{r_1(m_1),r_2(m_2)},\varsigma_{r_1(m_1)} \cdot \nu_{r_1(m_1)}] \ \in \ N.
          \end{align*}
          This shows in particular that \(N\) contains \(\nu_{1(3)} \cdot \nu_{2(3)}\).
          Let \(r(m) \subset \ZZ\) be a residue class.
          Then for any residue class \(\ti{r}(\ti{m}) \subseteq \ZZ \setminus r(m)\) we have
          \begin{align*}
            \hspace{12mm} \varsigma_{r(m)} \cdot \nu_{r(m)} \ = \ \
            &[\tau_{r(m),\ti{r}(\ti{m})},\varsigma_{r(m)} \cdot \nu_{\ti{r}(\ti{m})}] \
            \cdot \ [\tau_{\ti{r}(2\ti{m}),\ti{r}+\ti{m}(2\ti{m})},
            \varsigma_{\ti{r}(2\ti{m})} \cdot \nu_{\ti{r}(2\ti{m})}] \\
            \cdot \ &(\nu_{\ti{r}(2\ti{m})} \cdot \nu_{\ti{r}+\ti{m}(2\ti{m})} \cdot
            \tau_{\ti{r}(2\ti{m}),\ti{r}+\ti{m}(2\ti{m})})^{-1} \ \in \ N.
          \end{align*}
          This shows in particular that \(N\) contains also \(\varsigma_{0(2)} \cdot \nu_{0(2)}\),
          and the simplicity of the group \(\ti{K}^-\) follows from
          Theorem~\ref{SimpleSupergroupsGeneratorsTheorem}, Assertion~(2).

  \end{enumerate}
  \vspace{-5mm}
\end{proof}

\begin{SimpleSupergroupsTransitivityTheorem} \label{SimpleSupergroupsTransitivityTheorem}
  The groups \(\ti{K}^+\) and \(\ti{K}^-\) act highly transitively on~\(\ZZ\).
\end{SimpleSupergroupsTransitivityTheorem}
\begin{proof}
  Since \(\ti{K}^+ < \ti{K}^-\) it is sufficient to prove the assertion for~\(\ti{K}^+\).
  Let \(k\) be a positive integer, and let \((n_1, \dots, n_k)\) and \((\ti{n}_1, \dots, \ti{n}_k)\)
  be two \(k\)-tuples of pairwise distinct integers.
  We have to show that there is an element \(\sigma \in \ti{K}^+\) such that
  \((n_1^\sigma, \dots, n_k^\sigma) = (\ti{n}_1, \dots, \ti{n}_k)\).

  Let \(m := 2k+1\), and choose a residue class \(r(m)\) which does not contain one of the
  points \(n_i\) or \(\ti{n}_i\). Define \(\sigma_1, \ti{\sigma}_1 \in \ti{K}^+\) by
  \[
    \sigma_1 \ := \!
    \prod_{i: n_i < 0} (\nu_{r(m)} \cdot \nu_{n_i(m)}^{-1})^{\lfloor \frac{n_i}{m} \rfloor}
    \ \ \ \text{and} \ \ \
    \ti{\sigma}_1 \ := \!
    \prod_{i: \ti{n}_i < 0}
    (\nu_{r(m)} \cdot \nu_{\ti{n}_i(m)}^{-1})^{\lfloor \frac{\ti{n}_i}{m} \rfloor},
  \]
  respectively.
  Then the images of all points \(n_i\) under \(\sigma_1\) are nonnegative,
  and the same holds for the images of the points \(\ti{n}_i\) under \(\ti{\sigma}_1\).
  Since \(\CT{\ZZ}\) acts highly transitively on \(\NN_0\), we can choose
  a \(\sigma_2 \in \CT{\ZZ} < \ti{K}^+\) which maps the images of the \(n_i\) under
  \(\sigma_1\) to the images of the \(\ti{n}_i\) under \(\ti{\sigma}_1\). Now the
  permutation \(\sigma := \sigma_1 \cdot \sigma_2 \cdot \ti{\sigma}_1^{-1}\) serves
  our purposes.
\end{proof}

\begin{TameGenerationConjecture} \label{TameGenerationConjecture}
  The group \(\RCWA{\ZZ}\) is generated by its tame elements.
\end{TameGenerationConjecture}

\begin{TameGenerationRemark} \label{TameGenerationRemark}
  Conjecture~\ref{TameGenerationConjecture} is equivalent to the assertion that
  \(\ti{K}^+ = K^+\) and \(\ti{K}^- = K^-\). If it holds, we have
  \(\RCWA{\ZZ} = \langle \CT{\ZZ}, \varsigma_{0(2)} \rangle\):
  \begin{enumerate}

    \item It is \(\nu = \varsigma_{0(2)} \cdot \tau \cdot
          (\varsigma_{0(2)}^{\tau_{1(4),2(4)}} \cdot
           \varsigma_{0(2)}^{\tau_{1(2),0(4)}})^{\tau_{0(2),1(4)}}
          \in \langle \CT{\ZZ}, \varsigma_{0(2)} \rangle\).

    \item It is \(\nu_{0(2)} = \tau \nu\), \(\nu_{1(2)} = \nu_{0(2)}^\tau\),
          \(\varsigma_{1(2)} = \varsigma_{0(2)}^\tau\) and 
          \(\varsigma = \varsigma_{0(2)} \cdot \nu_{1(2)} \cdot \varsigma_{1(2)}\).
          Therefore we know that \(\{\nu_{0(2)}, \nu_{1(2)}, \varsigma_{1(2)}, \varsigma\}
          \subset \langle \CT{\ZZ}, \varsigma_{0(2)} \rangle\).

    \item Let \(r(m) \subset \ZZ\) be a residue class \(\neq 1(2)\). We choose an arbitrary
          residue class \(\ti{r}(\ti{m}) \subseteq \ZZ \setminus (0(2) \cup r(m))\), and put
          \(\vartheta := \tau_{0(2),\ti{r}(\ti{m})} \cdot \tau_{\ti{r}(\ti{m}),r(m)} \in \CT{\ZZ}\).
          Then we have \(\{\nu_{r(m)}, \varsigma_{r(m)}\} = \{\nu_{0(2)}^\vartheta,
          \varsigma_{0(2)}^\vartheta\} \subset \langle \CT{\ZZ}, \varsigma_{0(2)} \rangle\).

  \end{enumerate}
  The factorization routine in \RCWAPackage~\cite{Kohl05a} provides some evidence
  for Conjecture~\ref{TameGenerationConjecture}.
\end{TameGenerationRemark}

\section*{Acknowledgements}

I thank Bettina Eick for her numerous and valuable hints regarding the layout of this article.
Likewise I thank Laurent Bartholdi for pointing out that the classes of subgroups
of \(\CT{\ZZ}\) and \(\RCWA{\ZZ}\) are closed under taking restricted wreath products
with~\((\ZZ,+)\).

\bibliographystyle{amsplain}
\bibliography{simplegp}

\end{document}

%%%%%%%%%%%%%%%%%%%%%%%%%%%%%%%%%%%%%%%%%%%%%%%%%%%%%%%%%%%%%%%%%%%%%%%%%%%%%%%%%%%%%%%%%%%%%%%%%%%%