\documentclass[12pt]{article}

\usepackage{amsmath}
\usepackage{theorem}
\usepackage{amssymb}
\usepackage{latexsym}

\usepackage{a4wide}

\title{SLAC manual}
\date{}

\begin{document}

\maketitle

This document gives a description of the SLAC functions, that
allow one to work with the classification of all solvable Lie
algebras of dimension up to $4$, and nilpotent Lie algebras of
dimension up to $6$.

\section{The classification}

The following classification of solvable Lie algebras is used. It is taken
from \cite{gra11}.\par
In dimension $2$ there are just two classes of solvable Lie algebras:
\begin{itemize}
\item[$L_2^1$] The Abelian Lie algebra.
\item[$L_2^2$] $[x_2,x_1]=x_1$.
\end{itemize}

We have the following solvable Lie algebras of dimension $3$:
\begin{itemize}
\item[$L_3^1$] The Abelian Lie algebra.
\item[$L_3^2$] $[x_3,x_1]=x_1$, $[x_3,x_2]=x_2$.
\item[$L_3^3(a)$] $[x_3,x_1]=x_2$, $[x_3,x_2]=ax_1+x_2$.
\item[$L_3^4(a)$] $[x_3,x_1]=x_2$, $[x_3,x_2]=ax_1$. Condition of isomorphism:
$L_3^4(a)\cong L_3^4(b)$ if and only if there is an $\alpha\in F^*$ with 
$a=\alpha^2 b$.
\end{itemize}

And the following solvable Lie algebras of dimension $4$:

\begin{itemize}
\item[$L_4^1$] The Abelian Lie algebra.
\item[$L_4^2$] $[x_4,x_1]=x_1$, $[x_4,x_2]=x_2$, $[x_4,x_3]=x_3$.
\item[$L_4^3(a)$] $[x_4,x_1]=x_1$, $[x_4,x_2]=x_3$, $[x_4,x_3]=-ax_2+(a+1)x_3$.
\item[$L_4^4$] $[x_4,x_2]=x_3$, $[x_4,x_3]= x_3$.
\item[$L_4^5$] $[x_4,x_2]=x_3$.
\item[$L_4^6(a,b)$] $[x_4,x_1] = x_2$, $[x_4,x_2]=x_3$, $[x_4,x_3] = 
ax_1+bx_2+x_3$.
\item[$L_4^7(a,b)$] $[x_4,x_1] = x_2$, $[x_4,x_2]=x_3$, $[x_4,x_3] = 
ax_1+bx_2$. Isomorphism condition: $L_4^7(a,b)\cong L_4^7(c,d)$ if and only
if there is an $\alpha\in F^*$ with $a=\alpha^3c$ and $b=\alpha^2d$.
\item[$L_4^8$] $[x_1,x_2]=x_2$, $[x_3,x_4]=x_4$.
\item[$L_4^9(a)$] $[x_4,x_1] = x_1+ax_2$, $[x_4,x_2]=x_1$, $[x_3,x_1]=x_1$, 
$[x_3,x_2]=x_2$. Condition on the parameter $a$: $T^2-T-a$ has no roots
in $F$. Isomorphism condition: $L_4^9(a)\cong L_4^9(b)$ if and only if
the characteristic of $F$ is not $2$ and there is an $\alpha\in F^*$ with 
$a+\frac{1}{4} = \alpha^2(b+\frac{1}{4})$, or the characteristic of $F$ 
is $2$ and $X^2+X+a+b$ has roots in $F$.
\item[$L_4^{10}(a)$] $[x_4,x_1] = x_2$, $[x_4,x_2]=ax_1$, $[x_3,x_1]=x_1$,
$[x_3,x_2]=x_2$. Condition on $F$: the characteristic of $F$ is $2$.
Condition on the parameter $a$: $a\not\in F^2$. Isomorphism condition:
$L_4^{10}(a)\cong L_4^{10}(b)$ if and only if $Y^2+X^2b+a$ has a solution
$(X,Y)\in F\times F$ with $X\neq 0$. 
\item[$L_4^{11}(a,b)$] $[x_4,x_1] = x_1$, $[x_4,x_2] = bx_2$, 
$[x_4,x_3]=(1+b)x_3$, $[x_3,x_1]=x_2$, $[x_3,x_2]=ax_1$.
Condition on $F$: the characteristic of $F$ is $2$. Condition on the
parameters $a,b$: $a\neq 0$, $b\neq 1$.
Isomorphism condition: 
$L_4^{11}(a,b)\cong L_4^{11}(c,d)$ if and only if $\frac{a}{c}$ and
$(\delta^2+(b+1)\delta+b)/c$ are squares in $F$, where $\delta = 
(b+1)/(d+1)$.
\item[$L_4^{12}$] $[x_4,x_1] = x_1$, $[x_4,x_2]=2x_2$, $[x_4,x_3] = x_3$,
$[x_3,x_1]=x_2$. 
\item[$L_4^{13}(a)$] $[x_4,x_1] = x_1+ax_3$, $[x_4,x_2]=x_2$, 
$[x_4,x_3] = x_1$, $[x_3,x_1]=x_2$.
\item[$L_4^{14}(a)$] $[x_4,x_1] = ax_3$, $[x_4,x_3]=x_1$, $[x_3,x_1]=x_2$.
Condition on parameter $a$: $a\neq 0$. Isomorphism condition:
$L_4^{14}(a)\cong L_4^{14}(b)$ if and only if there is an $\alpha\in F^*$ 
with $a=\alpha^2 b$. 
\end{itemize} 

Nilpotent of dimension 5:

\begin{enumerate}
\item[$N_{5,1}$] Abelian.
\item[$N_{5,2}$] $[x_1,x_2]=x_3$.
\item[$N_{5,3}$] $[x_1,x_2]=x_3, [x_1,x_3]=x_4.$
\item[$N_{5,4}$] $[x_1,x_2] = x_5, [x_3,x_4]=x_5.$
\item[$N_{5,5}$] $[x_1,x_2]=x_3, [x_1,x_3]= x_5, [x_2,x_4] = x_5.$
\item[$N_{5,6}$] $[x_1,x_2]=x_3, [x_1,x_3]=x_4, [x_1,x_4]=x_5, [x_2,x_3]=x_5.$
\item[$N_{5,7}$] $[x_1,x_2]=x_3, [x_1,x_3]=x_4, [x_1,x_4]=x_5.$
\item[$N_{5,8}$] $[x_1,x_2]=x_4, [x_1,x_3]=x_5.$
\item[$N_{5,9}$] $ [x_1,x_2]=x_3, [x_1,x_3]=x_4, [x_2,x_3]=x_5.$
\end{enumerate}


We get nine $6$-dimensional nilpotent Lie algebras denoted 
$N_{6,k}$ for $k=1,\ldots,9$ that are the 
direct sum of $N_{5,k}$ and a $1$-dimensional abelian ideal. Subsequently
we get the following Lie algebras.

\begin{enumerate}
\item[$N_{6,10}$] $[x_1,x_2]=x_3, [x_1,x_3]=x_6, [x_4,x_5]=x_6.$
\item[$N_{6,11}$] $[x_1,x_2]=x_3, [x_1,x_3]=x_4, [x_1,x_4]=x_6, [x_2,x_3]=x_6,
[x_2,x_5]=x_6.$
\item[$N_{6,12}$] $[x_1,x_2]=x_3, [x_1,x_3]=x_4, [x_1,x_4]=x_6, [x_2,x_5]=x_6.$
\item[$N_{6,13}$] $[x_1,x_2]=x_3, [x_1,x_3]=x_5, [x_2,x_4]=x_5, [x_1,x_5]=x_6,
[x_3,x_4]=x_6.$
\item[$N_{6,14}$] $[x_1,x_2]=x_3, [x_1,x_3]=x_4, [x_1,x_4]=x_5, [x_2,x_3]=x_5, 
[x_2,x_5]=x_6,[x_3,x_4]=-x_6.$
\item[$N_{6,15}$] $[x_1,x_2]=x_3, [x_1,x_3]=x_4, [x_1,x_4]=x_5, [x_2,x_3]=x_5, 
[x_1,x_5]=x_6,[x_2,x_4]=x_6.$
\item[$N_{6,16}$] $[x_1,x_2]=x_3, [x_1,x_3]=x_4, [x_1,x_4]=x_5, [x_2,x_5]=x_6,
[x_3,x_4]=-x_6.$
\item[$N_{6,17}$]  $[x_1,x_2]=x_3, [x_1,x_3]=x_4, [x_1,x_4]=x_5, [x_1,x_5]=x_6,
[x_2,x_3]= x_6.$
\item[$N_{6,18}$]  $[x_1,x_2]=x_3, [x_1,x_3]=x_4, [x_1,x_4]=x_5, 
[x_1,x_5]=x_6.$
\item[$N_{6,19}(\epsilon)$] $[x_1,x_2]=x_4, [x_1,x_3]=x_5, [x_2,x_4]=x_6, 
[x_3,x_5]=\epsilon x_6.$
Isomorphism: $N_{6,19}(\epsilon)\cong N_{6,19}(\delta)$ if and only if 
there is an 
$\alpha\in F^*$ such that $\delta=\alpha^2\epsilon$. 
\item[$N_{6,20}$] $[x_1,x_2]=x_4, [x_1,x_3]=x_5, [x_1,x_5]=x_6, 
[x_2,x_4]=x_6. $
\item[$N_{6,21}(\epsilon)$] $[x_1,x_2]=x_3, [x_1,x_3]=x_4, [x_2,x_3]=x_5, 
[x_1,x_4]=x_6, [x_2,x_5]= \epsilon x_6.$
Isomorphism: $N_{6,21}(\epsilon)\cong N_{6,21}(\delta)$ if and only if there 
is an $\alpha\in F^*$ such that $\delta=\alpha^2\epsilon$. 
\item[$N_{6,22}(\epsilon)$] $[x_1,x_2]=x_5, [x_1,x_3]=x_6, [x_2,x_4]=
\epsilon x_6, [x_3,x_4]=x_5.$ 
Isomorphism: $N_{6,22}(\epsilon)\cong N_{6,22}(\delta)$ if and only if there 
is an $\alpha\in F^*$ such that $\delta=\alpha^2\epsilon$.
\item[$N_{6,23}$] $[x_1,x_2]=x_3, [x_1,x_3]=x_5, [x_1,x_4]=x_6, 
[x_2,x_4]= x_5.$
\item[$N_{6,24}(\epsilon)$] $[x_1,x_2]=x_3, [x_1,x_3]=x_5, [x_1,x_4]=\epsilon 
x_6, [x_2,x_3]=x_6, [x_2,x_4]= x_5.$
Isomorphism: $N_{6,24}(\epsilon)\cong N_{6,24}(\delta)$ if and only if there 
is an $\alpha\in F^*$ such that $\delta=\alpha^2\epsilon$.
\item[$N_{6,25}$] $[x_1,x_2]=x_3, [x_1,x_3]=x_5, [x_1,x_4]=x_6.$
\item[$N_{6,26}$] $[x_1,x_2]=x_4, [x_1,x_3]=x_5, [x_2,x_3]=x_6.$
\end{enumerate}




\section{Comments on the classification over finite fields}

Over general infinite fields the lists are not ``precise'' in the sense
that there are too many algebras. However, for finite fields we 
make a precise list, by restricting the parameter values in some cases.
In this section we describe how we do this. For this $F$ will be a finite
field of size $q$ with primitive root $\gamma$. 

\begin{itemize}
\item If the characteristic of $F$ is $2$, then there are two algebras
of type $L_3^4(a)$, namely $L_3^4(0)$ and $L_3^4(1)$. If the characteristic
is not $2$, then there are three algebras of this type, $L_3^4(0)$, $L_3^4(1)$,
$L_3^4(\gamma)$.
\item The class $L_4^7(a,b)$ falls apart into three classes:
$L_4^7(a,a)$ ($a\in F$), $L_4^7(a,0)$ ($a\neq 0$), $L_4^7(0,b)$ ($b\neq 0$).
Among the elements of the first class there are no isomorphisms. However,
for the other two classes we have the following.
\begin{itemize}
\item $L_4^7(a,0)\cong L_4^7(b,0)$ if and only if there is an $\alpha\in F^*$
such that $a=\alpha^3 b$. If $q\equiv 1 \bmod 3$, then exactly a third of
the elements of $F^*$ are cubes, naley the $\gamma^i$ with $i$ divisible 
by $3$. So in this case we get three algebras, $L_4^7(1,0)$, 
$L_4^7(\gamma,0)$, $L_4^7(\gamma^2,0)$. If $q\not\equiv 1 \bmod 3$
then $F^3=F$, and hence there is only one algebra, namely $L_4^7(1,0)$.
\item $L_4^7(0,a)\cong L_4^7(0,b)$ if and only if there is an $\alpha\in F^*$
such that $a=\alpha2 b$. So if $q$ is even then we get one algebra,
$L_4^7(0,1)$. If $q$ is odd we get two algebras, $L_4^7(0,1)$, 
$L_4^7(0,\gamma)$.
\end{itemize}
\item In \cite{gra11} it is shown that there is only one Lie algebra
in the class $L_4^9(a)$. We let $e$ be the smallest positive integer
such that $T^2-T-\gamma^e$ has no roots in $F$. Then we use the Lie
algebra  $L_4^9(\gamma^e)$.
\item Over a finite field of characteristic $2$ there are no Lie
algebras of type $L_4^{10}(a)$, as $F^2=F$ in that case.
\item There is only one Lie algebra of type $L_4^{11}(a,b)$ over a field
of characteristic $2$, namely $L_4^{11}(1,0)$.
\item If $q$ is even then there is only one algebra of type $L_4^{14}(a)$,
namely $L_4^{14}(1)$. If $q$ is odd, then there are two algebras,
$L_4^{14}(1)$, $L_4^{14}(\gamma)$.
\end{itemize}

\section{The SLAC functions}

\subsection{Ln\_k( F, pp )}

This is a function for creating a solvable Lie algebra. Here $n$ is the
dimension, which can be $2,3,4$. Secondly, $k$ is the number of the Lie 
algebra. Finally, $F$ is the field, and $pp$ are zero or more parameters.
So if a Lie algebra occurs in the list with $s$ parameters then exactly
$s$ extra arguments are expected. For example {\sf L4\_4( Rationals )}
returns the Lie algebra $L_4^4$ over the rational numbers, and
{\sf L4\_6( Rationals, 1, 1/2 )} returns the Lie algebra
$L_4^6(1,1/2)$ over the rational numbers.

\begin{verbatim}
gap> L:= L4_7( Rationals, 1/2, 1/2 );
<Lie algebra of dimension 4 over Rationals>
\end{verbatim}

\subsection{Nn\_k( F, pp )}

Same as above, but here $n$ is 5 or 6, and the corresponding
nilpotent Lie algebra is returned.

\subsection{IdDataSLAC( L )}

Here $L$ is a solvable Lie algebra of dimension $2$, $3$ or $4$.
This function returns a record with components {\sf name} and
{\sf isom}. The component {\sf name } is a list of a string and
a list of field elements. The string is contains the name of the Lie 
algebra from the list to which $L$ is isomorphic. If one enters this
string as input to �{\sf GAP}, then this Lie algebra is returned.
The list of field elements contains the parameters of the Lie algebra.
The component {\sf isom} contains an isomorphism of the algebra from
the list to the given algebra $L$.\par
For a Lie algebra over a finite field, the {\sf name} component is
uniquely determined (see the previous section). For Lie algebras 
over infinite fields this is not the case. However, the description
in the first section gives precise conditions on the various ways in
which the parameters can differ.

\begin{verbatim}
gap> K:= SimpleLieAlgebra("B",2,Rationals);
<Lie algebra of dimension 10 over Rationals>
gap> c:= ChevalleyBasis( K );
[ [ v.1, v.2, v.3, v.4 ], [ v.5, v.6, v.7, v.8 ], [ v.9, v.10 ] ]
gap> L:= Subalgebra( K, c[1] );
<Lie algebra over Rationals, with 4 generators>
gap> IdDataSLAC( L );
rec( name := [ "L4_7( Rationals, 0, 0 )", [ 0, 0 ] ],
  isom := CanonicalBasis( <Lie algebra of dimension 4 over Rationals> ) ->
    [ (5)*v.1+(2)*v.3+(5)*v.4, (-5)*v.3+(-4)*v.4, (10)*v.4, v.2 ] )
\end{verbatim}

\subsection{NameSLAC( L )}

This is the first entry of the component {\sf name} of 
{\sf IdDataSLAC( L )}.

\subsection{ParametersSLAC( L )}

This is the second entry of the component {\sf name} of 
{\sf IdDataSLAC( L )}.

\subsection{IsomorphismSLAC( L )}

This is the component {\sf isom} of {\sf IdDataSLAC( L )}.

\subsection{AllSolvableLieAlgebrasSLAC( F, n )}

Here $F$ is a finite field, and $n$ an integer between $2$ and $4$.
This function returns the list of all solvable Lie algebras of
dimension $n$ over $F$.

\begin{verbatim}
gap> AllSolvableLieAlgebrasSLAC( GF(4), 3 );
[ <Lie algebra of dimension 3 over GF(2^2)>,
  <Lie algebra of dimension 3 over GF(2^2)>,
  <Lie algebra of dimension 3 over GF(2^2)>,
  <Lie algebra of dimension 3 over GF(2^2)>,
  <Lie algebra of dimension 3 over GF(2^2)>,
  <Lie algebra of dimension 3 over GF(2^2)>,
  <Lie algebra of dimension 3 over GF(2^2)>,
  <Lie algebra of dimension 3 over GF(2^2)> ]
\end{verbatim}


\begin{thebibliography}{1}

\bibitem{gra11}
W.~A. de~Graaf.
\newblock Classification of solvable {L}ie algebras.
\newblock {\em Experimental Mathematics}, to appear, 2004.

\end{thebibliography}

\end{document}
