\documentclass{article}
\usepackage{a4wide}
\usepackage{times}
\parindent=0pt
\parskip=\medskipamount
\def\GAP{\textsf{GAP}}
\let\code\texttt
\def\var#1{{\rmfamily\slshape#1}}
\def\bs{$\backslash$}
\title{Errors, exceptions, and break loops}
\author{Steve Linton}
\begin{document}
\maketitle

This is working notes towards a new scheme for handling
``exceptions'', which I define as events the response to which cannot
sensibly be decided within the currently executing function. These
include:
\begin{enumerate}
\item Unexpected external events like Control-C, running out of disk
space, or running close to the limit of memory
\item Assertion failures, and other ``this should never happen''
events
\item Forbidden input to a routine
\end{enumerate}

There are two kinds of event which can be understood locally, but for
which the `Error' mechanism is still used at present:
\begin{enumerate}
\item A need to interact with the user, for example to allow them to
reset the coset enumeration limit
\item Mathematical exceptions, where the question was well-posed but
no proper answer exists -- we tend, with time, to move to using `fail' 
for this case, which is fine, but, if not tested, usually results in
``forbidden input'' a little later
\end{enumerate}

The 'Error' mechanism (followed by `quit;' is also the only way at
present to ``jump out of'' the currently executing routines and all
the routines that called it back to some place that feels able to
``pick up the pieces'' and continue -- currently always a
read-eval-view loop.

Only two parts of this: the interactive ``subshell'' and the ``jump
back'' capability require support from the kernel. So the first part
of the proposal is to separate them out, and provide

\begin{itemize}
\item A general ``longjump'' or ``throw -- catch'' mechanism allowing
a routine to say that it is willing to accept tranfers of control from 
the routines it call, and the routines they call and so on, and
allowing those routines to make such a tranfer.
\item A flexible way to start a ``subshell'' allowing control of the
prompt, which variables are seen, whether this loop catches longjumps, 
what input and output streams are used, etc. The subshell would
return some values indicating whether it was exited by `return;',
`return <obj>;' or `quit;'.
\end{itemize}

The standard behaviour (the function `Error') is then to
print various information, enter a subshell and do a longjump if the
user quits from the loop. The kernel functions `ErrorQuit',
`ErrorReturnObj', and `ErrorReturnVoid' do similar sorts of things.

Situations where GAP library functions simply need to interact with
the user, but are capable of making decisions locally about the result 
of this interaction can then avoid using `Error' at all, and instead
just use a Subshell and process the results themselves.

For other situations, I believe that we can implement a more general
and flexible notion than Error, purely in \GAP\ and then reimplement
`Error' on top of it.

The new structure is based around an Operation, perhaps called
`Exception', which takes one argument, an object in the new Category
IsException.  Existing error situations could be mapped onto
subCategories, such as IsErrorCallException, with an Attribute
ArgumentsOfErrorCallException and IsKernelErrorException, with an
Attribute Message1OfKernelException, and subsubCategories
IsKernelErrorQuitException, IsKernelErrorReturnVoidException,
IsKernelErrorReturnObjException (supporting an Attribute to get at the 
second message) and so on. 

Methods installed for this Operation are Exception handlers (actually
one might want an extra level of indirection here, to allow handlers
to be more easily uninstalled or disabled). In the
default situation, Methods installed for the various
Is..Error...Exception  subCategories could (using the two new kernel
primitives above) emulate the existing error behaviour, while Error,
ErrorQuit, etc. are replaced by functions that create the appropriate
objects and then call Exception.

The basic benefit of this is that control of what happens in the event 
of error is now entirely at the GAP level. Furthermore, other types of 
Exception can be sensibly added, and calls to Error in the library can 
gradually be replaced by direct calls to Exception, passing specially
constructed Exception objects that convey more information about the
problem, allowing new handler methods to deal more sensibly with the
situation. It might even prove sensible to use Exceptions
rather than returning `fail' for some mathematical errors, since you
can then install one handler rather than having to check for a 	`fail' 
return in lots of places. Using something like `return Exception(
DivideByZeroException( zero ));' would allow a default handler to
return 'fail', but also allow for a non-default handler.

There is actually not a great deal of programming needed here. It's
mainly a matter of repackaging what is already present.

Some subtlety is needed in a couple of areas:
\begin{itemize}
\item Errors occuring early in \GAP\ start-up have to do something,
even if we don't yet have enough of the system to do method selection
for Exception
\item Exception handlers that generate exceptions could cause infinite 
regress rather easily. It might be worthwhile to have something like
the POSIX signal behaviour, where a handler is disabled when it is
called, and must reenable itself, if it is to get called again. This
might be another argument for an indirection layer between the actual
Methods and the handlers.
\end{itemize}

\end{document}

