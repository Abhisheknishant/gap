\documentclass[10pt]{article}
\usepackage{a4wide}
\usepackage{times}
\parindent=0pt
\parskip=\medskipamount
\def\GAP{\textsf{GAP}}
\let\code\texttt
\def\var#1{{\rmfamily\slshape#1}}
\def\bs{$\backslash$}
\def\replydate{May 1st, 2011}
\def\pkgdeadline{July 1st, 2011}
\def\releasemonth{August 2011}
\def\betaversion{4.5.1}

% use comments to switch between draft/final mode
%\newcommand{\todo}[1]{\marginpar{\footnotesize #1}} % for the draft mode
\newcommand{\todo}[1]{}% for the final mode

\title{Beta release of GAP 4.5 for package authors}
\author{The GAP Group}
\begin{document}
\maketitle

Dear authors of GAP packages,

There have been many changes in the development version of GAP during the more
than two years since the release of GAP 4.4.12. Some
major developments that were started earlier are now ready for release and it's
now high time for the GAP 4.5 release, which we plan to have in \releasemonth.

To allow you to test your packages with the new features and prepare them for the release, we have prepared a first beta version of GAP 4.5, named GAP \betaversion, which
is available to package authors in the following directory:

\begin{center}
\verb|ftp://ftp.gap-system.org/pub/gap/gap45/beta/|
\end{center}

This is especially important since some of the changes involve improvements to the GAP package interface to deal with problems which have arisen and 
cope more smoothly as the number of packages approaches~100 and the complexity of the functionality supplied by packages and the relationships between packages increases. 
A summary of these changes with links to the full documentation appears below.

\todo{Moving to a single archive, we wrap GAP 4.5 with packages
from the start. Need more thought about version numbers - may need
to reflect package binaries if statically compiled into the kernel.}

\subsection*{Archive Structure}

The archive you will find to download at the above URL is not quite the same as the GAP 4.4 archives. For the general release of GAP 4.5 (and subsequent updates) 
we plan the main form of distribution of GAP to be a single dated archive file including current versions of all packages 
at that date and binaries for the most common 
environments (Windows, MacOS, and Linux on 32 and 64 bit intel processors). For this beta release, we do not quite have the setup for this fully in place. 
The beta archives will include packages, but not binaries. 

We plan that these archives will be rebuilt and marked with a new timestamp
whenever a new version of a package is released, which we expect to be quite frequently 
and we will make no special distinction between 
updates to packages and to the core system.

The beta release however will only be updated 
if we deem there to be significant changes to the core system or to packages on which 
other packages depend, so that you have a reasonably stable platform for your testing. 

Some of the packages that are loaded at the beginning of the GAP session 
currently produce errors; these may be ignored at the moment unless you 
need those particular packages, and should be eventually eliminated after 
relevant package updates.



\subsection*{What we'd like you to do}



Please download and install GAP \betaversion \, and test it with your package(s). 
Since we have not yet included binaries, you will need to compile the GAP kernel as usual. To run GAP under
Windows, you may either compile it yourself with Cygwin or write to Alexander 
Konovalov (\verb|alexk at mcs.st-andrews.ac.uk|) and ask him to send you the  GAP 
executable file for Windows.

% Roadmap and policy to discuss. 

If you have a new version of your package that you would like to  release with the first
public release of GAP 4.5, please prepare the new version of your package 
before \pkgdeadline. When you've done that, please make the new version available to us for testing.

If the new version is intended only to work with GAP 4.5, please 
contact Alexander Konovalov writing to \verb|alexk at mcs.st-andrews.ac.uk|
to arrange for the GAP Group to obtain your package for further tests and to include it later versions of the beta. If the
new version is compatible both with with GAP 4.4 and GAP 4.5, you
may follow the standard procedure to publish it, but please
inform Alexander Konovalov about this as well.

If a package is released to us later than \pkgdeadline \, and our testing shows any problems 
(most likely in interactions with other packages) we may not be able to 
sort them out in time to include it in the first released archive of GAP 4.5. 

Please also respond to this announcement of the beta release before \replydate \,
to Alexander Konovalov (\verb|alexk at mcs.st-andrews.ac.uk|) 
telling us whether or not you should be able to update your package before the 
first public release of GAP 4.5 -- we use email addresses of package authors 
given in the packages and would like to make sure that this announcement
was delivered to you.

% Now a reassuring note to explain the amount of work required

We have already tested most of the current versions of GAP packages with the
beta-release of GAP 4.5, and did not find any serious problems. Thus, we hope
that for most packages the change will go smoothly - a version of the
package with basic compatibility with GAP 4.5 can be produced with little or no
effort (for most packages simply recompiling the documentation would be good
enough). We would encourage you though, to do a little more work and adjust your package to use 
some of the new features of the GAP 4.5 package interface. 

As well as testing your package, please let us know of any problems you find 
with GAP 4.5 itself. We encourage you to test GAP 4.5 on as many configurations
as possible. We welcome all feedback to \verb|support at gap-system.org| 
including comments, suggestions, bug reports and manual corrections.

Below you will find a brief overview of main changes in GAP 4.5 from the
package author's perspective, descriptions of known problems that we observed while
testing current packages with GAP 4.5, and some further guidelines.


\section{What will be changed in GAP 4.5 for packages ?}

Here we list primarily those changes which may have some implications for the 
packages. A more comprehensive list of other changes will be available in the 
public release. 

\begin{itemize}

\item
Changing the distribution format providing one archive with the core system
and all currently redistributed packages.

\item
The GAP kernel is now compiled by default to use the GMP large integer arithmetic library, speeding up arithmetic by a factor of 4 or more in many cases.
 This slightly changes the build process, affecting mainly packages with dynamically loaded modules (see installation instructions for further details).

\todo{Document build options --with-gmp, --with-abi, --with-readline in the manual and README.}

\item
The GAP documentation has been converted to the GAPDoc format, and extensively reviewed, and now has only
two books: the Tutorial and the Reference Manual. Two other books, ``Extending 
GAP'' and ``Programming Tutorial'' became parts of the Reference Manual. Packages
that refer to parts of the GAP documentation may need to rebuild their manuals
to update references. Some packages still use the old ``gapmacro'' format for
their manuals, for which support may be discontinued in the future. There is no urgent need to convert such manuals into the GAPDoc format before GAP 4.5 release, but we 
encourage you very much to consider doing this at some later point. 

\item
The old concept of an autoloaded package has been integrated with the ``needed'' 
and ``suggested'' mechanism that already exists between packages. GAP itself 
now ``needs'' certain packages 
(for instance GAPDoc) and ``suggests'' others (typically the packages that were 
autoloaded). The decisions 
which packages GAP should need or suggest are made by developers based on 
technical criteria. They can be easily overridden by a user 
using the new \verb|gap.ini| file (see section 4.2 of the reference manual). 
The default file ensures that all previously autoloaded packages are still loaded if present.

\item
Optional \verb|~/.gap| directory for user's customisations which may contain e.g.
locally installed packages (see section 10.2 of the reference manual). If package installation instructions explain how
to install the package in a non-standard location, they may need an update. 
This is intended to replace \verb|.gaprc| files, but those are still supported 
for backwards compatibility (see section 79.5 of the reference manual).

\item
Various improvements in the packages loading mechanism to make it more
informative, while avoiding confusing the user with warnings and error 
messages for packages they didn't
know they were loading. For example, many messages are stored but not 
displayed \, and there is \, a function {\tt DisplayPackageLoadingLog} to show 
log messages that 
occur during package loading (see subsection 78.2-4 of the reference manual for further details).  
Packages are encouraged 
to use these mechanisms to report problems in loading (e.g. binaries not compiled), rather than printing messages directly.
\end{itemize}


\section{Suggestions for package authors}

Here we list known problems that we observed while testing current packages 
with GAP 4.5. Shortly we will start to contact authors of individual 
packages with their further details.

\begin{itemize}

\item
Several package break some tests from the GAP standard test suite 
(including \verb|tst/testall.g| and \verb|tst/testinstall.g|).

\item
Some of the standard tests run 
significantly slower if all available packages are loaded (although not with 
just autoloaded packages). 
We are working on identifying  the responsible packages in all such cases.

\item
Some packages have test files displaying the progress of tests. If the 
author wishes to have such tests output, it is recommended to produce it only
in extended packages tests, while the standard test should only compare the 
output with the test file and report any discrepancies. This is important for
the automated continuous build system which runs all tests from the GAP test
suite nightly/weekly on a number of configurations and reports about any new
discrepancies. 

\item
Moreover, some packages have test files listed in {\tt PackageInfo.g}
which are actually GAP input files, so they can not be automatically tested
with {\tt ReadTest}.

\item
Some packages may need to update manual examples and test files
 because the order in which record components are 
printed has changed (but is now more consistent and less dependent on how the record was created).


\item
Other problems may include missing declarations of local variables in 
functions, not reported before because of existing of a global variable with
the same name in GAP; creating global variables not following the recommended 
naming conventions; failure to load packages in a different order.
\end{itemize}

Additionally, we encourage the package authors to revise their copyright
information, autoloading status and the list of needed and suggested
packages:

\begin{itemize}

\item
If your packages use the text ``We adopt the copyright regulations of GAP as 
detailed in the copyright notice in the GAP manual'' or a similar statement, 
it may not present enough evidence for the package to be included in the 
installation in some cases when GAP is distributed together with another software. 
We advise to consider making the exact reference to the GPL license, for example: 

{\small
"{\sf package-name} is free software; you can redistribute it and/or modify it under the terms of the GNU General Public License as published by the Free Software Foundation; either version 2 of the License, or (at your option) any later version. For details, see the FSF's own site {\tt http://www.gnu.org/licenses/gpl.html}."
}


\item
If your package is currently autoloaded, please revise its status against 
the following technical criteria and consult with us in case of any doubts:
\begin{itemize}
\item
is there any code in the GAP library that calls functions from these 
packages if they are available?
\item
does the package install superior methods for operations defined in the GAP 
library?
\end{itemize}

\item
Please revise the list of needed and suggested packages and document 
explicitly all dependencies in the {\tt PackageInfo.g} file:

\begin{itemize}

\item
In some packages, explicit {\tt LoadPackage} statements are
executed \emph{when the package is loaded}. This may distort the order
of package loading and result in warning messages. It is recommended
to turn such dependencies into needed or suggested packages.
For example, a package can be designed in such a way that it can be loaded 
with restricted functionality if another package (or standalone program) 
is missing, and in this case the missing package (or binary) is \emph{suggested}.
Alternatively, if the package author decides that loading the package in
this situation makes no sense, then the missing component is \emph{needed}.

\item 
On the other hand, if {\tt LoadPackage} is called inside functions of the
package then there is no such problem -- provided that these functions are 
called only after the package has been loaded, so it is not necessary to
specify the other package as suggested. Same applies to test files
and manual examples, which may be simply extended by calls to {\tt LoadPackage}.

\item
It may happen that a package B that is listed as a suggested package
of package A is actually needed by A. This can be detected only
by deinstalling the package B and trying to load A -- one may run into an
error or may see warnings about undefined global variables.
For example, if package A calls {\tt LoadPackage} for B at loading time
(see above) then there is no other chance to find out whether B is in
fact needed.

If no explicit {\tt LoadPackage} call occurs in A at loading time then one
might try the new possibility to load a package without loading its
suggested packages using the global option {\tt OnlyNeeded} 
(see subsection 78.2-1 of the reference manual for further details); 
this way, one can check whether errors or warnings appear when B is not 
available. The consequence can then be either to turn B into a needed package 
or (since apparently B was not intended to become a needed package)
to change the code accordingly.

\item
Finally, if the package manual is in the GAPDoc format, 
then GAPDoc is needed for this package.
\end{itemize}

\end{itemize}

For your convenience, the text of this announcement is duplicated in the
temporary preface of the reference manual for GAP 4.5 beta, from where you 
may follow hyperlinks to appropriate manual sections.

Best wishes,\\
the GAP development team

April 2011

\end{document}
In the initial version of the beta release archive
the only package requiring GAP 4.5 is the Example package that contains an
updated guidelines for the package development. 


