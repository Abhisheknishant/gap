\documentclass{acmconf}
\usepackage{amsmath,amssymb,algorithmic,algorithm}

\title{Effective Identification of Commutative Semigroup Algebras which are 
Principal Ideal Rings with Identity}

\author{I.~M.~Ara\'{u}jo{*}, A.~V.~Kelarev{\dag} and  A.~Solomon{\ddag}}
\affiliation{
{*}Department of Mathematics\\
University of St. Andrews,
Fife KY16 9SS, United Kingdom\\ \vspace{0.5cm} 
{\dag}School of Mathematics and Physics\\ 
University of Tasmania, GPO Box 252-37, Hobart  \\
Tasmania 7001, Australia \\ \vspace{0.5cm}
{\ddag}Department of Computer Science\\ \vspace{0.15cm}
University of St. Andrews, 
Fife KY16 9SS, United Kingdom}




\def\gap{\sf GAP}
\def\N{{\bf N}}
\def\char{{\mbox{char}}}

\newtheorem{theorem}{Theorem}
\newtheorem{lemma}{Lemma}
\newtheorem{proposition}{Proposition}
\newtheorem{definition}{Definition}
\newtheorem{corollary}{Corollary}
\newtheorem{notation}{Notation}


\begin{document}
\maketitle
\renewcommand{\algorithmicrequire}{\textbf{Input:}}
\renewcommand{\algorithmicensure}{\textbf{Output:}}

\section*{Abstract} 

Associative and commutative algebras with identity have various 
well-known applications. In particular, many classical codes 
are ideals in commutative algebras (see 
\cite{CaKe98:gwpc}, \cite{LaMa92:cciga} for references). 
Computer storage, encoding and decoding algorithms simplify if 
all these codes have single generator polynomials. Thus it
is of interest to determine when all ideals of an algebra are
principal, i.e., are generated by one element. An algebra $R$
is said to be a \textit{principal ideal ring} if all ideals
of $R$ are principal.  

In the paper \cite{pDJW91} of Decruyenaere, 
Jespers and Wauters,
commutative semigroup algebras with identity which are 
principal ideal rings are completely described. 
In the more general case of noncommutative
algebras all principal ideal semigroup algebras have been described
by Jespers and Okninski \cite{JO96}. 

In this paper we develop an algorithm which, 
given a presentation for a commutative 
semigroup $S$ and the characteristic of a field $k$, 
decides whether the semigroup algebra $k[S]$ is a 
principal ideal ring with identity.
This builds upon the work of Rosales, Garc\'{\i}a-S\'{a}nchez and
Garc\'{\i}a-Garc\'{\i}a, (\cite{pRGG99}, \cite{bRG99})
who have developed a number of useful algorithms for computing
with finitely presented commutative semigroups: most importantly
an algorithm to compute the (necessarily finite) set of idempotents.
Much of the remaining work depends on finding presentations
for ideals, subgroups and extensions of finitely presented 
commutative semigroups. This draws on techniques developed 
by Ru\v{s}kuc and others 
in \cite{CaRoRuTh:Reidmeister} and \cite{Ru99}
for finding presentations of subsemigroups.

In practical terms, this work has evolved from an 
ongoing project involving two of the authors
\cite{braga99} at the University of St.~Andrews to 
develop functionality within {\gap} \cite{gap} to compute with 
semigroups. 



\begin{thebibliography}{9}

\bibitem{braga99} I. Ara\'ujo and A.~Solomon, 
{\em Computing with Semigroups in GAP},
to appear in the proceedings of {\em The International Conference on
Semigroups, Braga, Portugal}  (1999).

\bibitem{CaRoRuTh:Reidmeister}
C.M.~Campbell, E.F.~Robertson, N.~Ru\v{s}kuc and  R.M. Thomas,
{\it Reidemeister-Schreier type rewriting for semigroups},
Semigroup Forum 51 (1995), 47-62.


\bibitem{CaKe98:gwpc}
J. Cazaran and A.V. Kelarev,
{\it Generators and weights of polynomial codes},
Arch. Math. (Basel) {\bf 69} (1997), 479--486.

\bibitem{CaKe:gwpc}
J. Cazaran and A. V. Kelarev,
{\it On generators and weights of polynomial codes},
Acta Math. Univ. Comeniae 68 (1999)(1), 77--84.


\bibitem{pDJW91}
   F.~Decruyenaere, E.~Jespers and P.~Wauters, 
	\emph{On commutative principal ideal semigroup rings},
   Semigroup Forum, Vol.43 (1991) 367--377.

\bibitem{gap} The GAP Group, {\gap} -- Groups, Algorithms and Programming,
Version 4.1; http://www-gap.dcs.st-and.ac.uk/~gap (Aachen, St Andrews, 1999).

\bibitem{JO96} E. Jespers, J. Okni\'{n}ski, 
\emph{Semigroup algebras that are principal ideal rings}, 
J. Algebra vol.183 (1996), 837--863.

\bibitem{LaMa92:cciga}
P. Landrock and  O. Manz,
{\it Classical codes as ideals in group algebras},
Des. Codes Cryptogr. {\bf 2} (1992)(3), 273--285.


\bibitem{poniz87} J.~S.~Ponizovskii, 
\emph{Semigroup rings}, Semigroup Forum vol.36 (1987), 1--46.


\bibitem{pRGG99}
   J.~C.~Rosales, P.~A.~Garc\'{\i}a-S\'{a}nchez and 
	J.~I.~Garc\'{\i}a-Garc\'{\i}a,
  \emph{Commutative ideal extensions of Abelian groups},
   Semigroup Forum, (to appear).


\bibitem{bRG99}
   J.~C.~Rosales and  P.~A.~Garc\'{\i}a-S\'{a}nchez,
  \emph{Finitely generated commutative monoids}, Nova Science Publ., NY, 1999.

\bibitem{Ru99}
N.~Ru\v{s}kuc,
{\it Presentations for subgroups of monoids},
J. Algebra 220 (1999)(1), 365-380.



 
\end{thebibliography}
\end{document}

