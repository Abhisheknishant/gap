\documentclass{article}
\usepackage{a4wide}
\usepackage{times}
\parindent=0pt
\parskip=\medskipamount
\def\GAP{\textsf{GAP}}
\let\code\texttt
\def\var#1{{\rmfamily\slshape#1}}
\def\bs{$\backslash$}
\title{Namespaces}
\author{Steve Linton}
\begin{document}
\maketitle

This is working notes towards a scheme for reducing the pressure on
the global namespace, in the light of the increasing number of
packages, and also, perhaps, the partial modularisation of the
library.

\section{Changes to GAP Syntax}

The general form of a  global variable name in GAP would become
\textit{namespace\/}\verb|$|\textit{name}. Every global variable would
be uniquely identified by its \textit{(namespace,name)} pair.

The forms \verb|$|\textit{name} (which would always denote a global
variable) and \textit{name} (which might be a local or higher
variable) would also be allowed. We call these ``unqualified global
variable names''.  To resolve such names, one or more namespaces would
be open for access to existing variables (open for access), and just one namespace would
be open for creation of new variables (open for creation). If an
unqalified variable name was referred to, and more than one of the
namespaces open for access contained a variable with the given name,
then this would be an error (``ambiguous variable reference'' or some
such).

This can be achieved by kernel functions
\verb|OpenNamespaceForAccess|, \verb|CloseNamespaceForAccess|,
\verb|OpenNamespaceForCreation|, etc. Note that a namespace is
\emph{not} a GAP object. These functions would take a string as
argument. 

There might be a separate \verb|CreateNamespace| or it might be a side
effect of \verb|OpenNamespaceForCreation|. Similarly, assignment to a
variable in a non-existant namespace may, or may not, create that namespace.

Importantly, the resolution of global variable names would take place,
as now, at the time code is parsed, not at the time it is executed
(except via calls to functions like \verb|BindGlobal|). This is
essential for performance, and will do what people expect with regard
to library code, but may confuse people interactively.

\begin{verbatim}
gap> foo := function() ......  var .... end;
gap> foo();

<error because var defaults to incorrect namespace bar>

gap> CloseNamespaceForAccess(``bar'');
gap> OpenNamespaceForAccess(``therightone'');
gap> foo();

<same error, var was resolved when foo was parsed>
\end{verbatim}
\section{Use of Namespaces}

\subsection{In the library}

Clearly, there will be a default global namespace, which I propose to
call GLOBAL. Initially this will be open for access and for creation
and nothing else will.

Library authors, especially perhaps library module authors, may wish
to create private namespaces in parts of the library for functions not
intended to be documented or used interactively. This might be
something like:


In a .gd file:
\begin{verbatim}
CreateNamespace(``XyzInternals'');
OpenNamespaceForCreation(``XyzInternals'');

DeclareOperation(``DoTechnicalStuff''...);

DeclareGlobalVariable(``TechnicalData''...);

OpenNameSpaceForCreation(``GLOBAL'');
\end{verbatim}

In a .gi file
\begin{verbatim}
OpenNamespaceForAccess(``XyzInternals'');

InstallMethod(DoTechnicalStuff,...)

InstallMethod(Size,.... function()......TechnicalData.....end);

CloseNamespaceForAccess(``XyzInternals'');
\end{verbatim}

\subsection{In Packages}

I suggest that each package should get one namespace automatically
with the same name as the package. 

When packages files are read, that namespace should be open for
Creation and Access, so that all variables created in a package,
unless either they are refered to as \verb|GLOBAL$|\textit{name}, or
the package calls the namespace management functions to open some
other namespace for creation. The package may define further private
namespaces just like library code can. Packages can refer to variables
from other packages using \verb|$| or by opening the namespace (most
likely via \verb|RequirePackage|).

I suggest that explicit \verb|RequirePackage| should read in the
package files and open the package namespace for access. I am not sure
whether autoloading of a package should do this, however. If it didn't
then autoloading would still give access to new methods installed in
the package for existing operations, but would not see any new
operations.





\end{document}

