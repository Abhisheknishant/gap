%%%%%%%%%%%%%%%%%%%%%%%%%%%%%%%%%%%%%%%%%%%%%%%%%%%%%%%%%%%%%%%%%%%%%%%%%
%%
%W  algvspc.tex            GAP documentation                Thomas Breuer
%%
%H  @(#)$Id$
%%
%Y  Copyright (C) 1997, Lehrstuhl D fuer Mathematik, RWTH Aachen, Germany
%%
\Chapter{Vector Spaces and Algebras}

In this chapter we show some computations with vector spaces and
algebras.

%%%%%%%%%%%%%%%%%%%%%%%%%%%%%%%%%%%%%%%%%%%%%%%%%%%%%%%%%%%%%%%%%%%%%%%%%
\Section{Vector Spaces}

A *vector space* over the field $F$ is an additive group that is closed
under scalar multiplication with elements in $F$.
In {\GAP}, only those domains are regarded as vector spaces that are
constructed as vector spaces.
In particular, an additive group that does not know about an
acting domain of scalars is not regarded as a vector space in {\GAP}.

Every field is itself a vector space, the set of matrices of the same
shape over a given field forms a vector space, and the set of linear
mappings between two vector spaces forms a vector space.

\beginexample
    gap> f:= GF(8); IsVectorSpace( f );
    # the field with 8 elements
    GF(2^3)
    true
    gap> v:= GF(3)^[2,3]; IsVectorSpace( v );
    # 2 by 3 matrices over the field with 3 elements
    ( GF(3)^[ 2, 3 ] )
    true
    gap> w:= GF(3)^2; IsVectorSpace( w );
    # row vectors of length 2 over `GF(3)'
    ( GF(3)^2 )
    true
    gap> hom:= Hom( GF(3), v, w ); IsVectorSpace( hom );
    # `GF(3)'-linear maps from `v' to `w'
    Hom( GF(3), ( GF(3)^[ 2, 3 ] ), ( GF(3)^2 ) )
    true                                         
\endexample

Given a vector space, the linear closure of any subset is again a vector
space.

\beginexample
    gap> Subspace( f, [ Z(2), Z(8)^3 ] );
    VectorSpace( GF(2), [ Z(2)^0, Z(2^3)^3 ] )
    gap> m:= [ [ 1, 0, 0 ], [ 0, 1, 1 ] ] * Z(3)^0;
    [ [ Z(3)^0, 0*Z(3), 0*Z(3) ], [ 0*Z(3), Z(3)^0, Z(3)^0 ] ]
    gap> Subspace( v, [ m ] );
    VectorSpace( GF(3), 
    [ [ [ Z(3)^0, 0*Z(3), 0*Z(3) ], [ 0*Z(3), Z(3)^0, Z(3)^0 ] ] ] )
    gap> z:= Subspace( hom, [ ZeroMapping( v, w ) ] );
    VectorSpace( GF(3), [ ZeroMapping( ( GF(3)^[ 2, 3 ] ), ( GF(3)^2 ) )
     ] )
\endexample

Probably the most common $F$-vector spaces in {\GAP} are so-called
*row spaces*.
They consist of row vectors, that is, lists whose elements lie in $F$
or an extension field of $F$.

\beginexample
    gap> rowsp:= VectorSpace( GF(9), m );
    VectorSpace( GF(3^2), 
    [ [ Z(3)^0, 0*Z(3), 0*Z(3) ], [ 0*Z(3), Z(3)^0, Z(3)^0 ] ] )
    gap> Size( rowsp );
    81
    gap> [ 1, 1, 1 ] * Z(3) in rowsp;
    true
    gap> [ 0, 0, 1 ] * Z(3) in rowsp;
    false
\endexample

Note that the same set of vectors can be a vector space over different
fields,
one can construct spaces over suitable other fields using
`AsVectorSpace'.

\beginexample
    gap> as:= AsVectorSpace( GF(3), rowsp );
    VectorSpace( GF(3), 
    [ [ Z(3)^0, 0*Z(3), 0*Z(3) ], [ 0*Z(3), Z(3)^0, Z(3)^0 ], 
      [ Z(3^2), 0*Z(3), 0*Z(3) ], [ 0*Z(3), Z(3^2), Z(3^2) ] ] )
    gap> as = rowsp;
    true
    gap> AsVectorSpace( GF(27), rowsp );
    fail
\endexample

The coefficient field of a vector space in {\GAP} can be accessed by
`LeftActingDomain'.

\beginexample
    gap> LeftActingDomain( rowsp );
    GF(3^2)
    gap> LeftActingDomain( as );
    GF(3)
    gap> LeftActingDomain( f );
    GF(2)
\endexample

We see that the field `f' is a vector space over its prime field,
which is the field with 2 elements.
Note that it might also be natural to view any field as a vector space
over itself.
But in {\GAP}, `GF(p^n)' is a vector space over `GF(p)' for each prime `p'.
Of course one can construct a field as a vector space over any subfield
using `AsVectorSpace'.

The defining invariant of a vector space $V$ is its *dimension*,
the maximal cardinality of a tuple of vectors in $V$ that are linearly
independent over the coefficient field $F$ of $V$.

\beginexample
    gap> Dimension( rowsp );
    2
    gap> Dimension( as );
    4
    gap> Dimension( z );
    0
    gap> Dimension( f );
    3
    gap> Dimension( hom );
    12
\endexample

Each such maximal tuple of vectors is called a *basis* of the vector
space.
For a given basis $B$ of $V$, every vector $v$ in $V$ can be expressed
uniquely as $v = \sum_{b \in B} c_b b$, with coefficients $c_b \in F$.

In {\GAP}, bases are special lists of vectors.
They are used mainly for the computation of coefficients and linear
combinations.

\beginexample
    gap> bas:= Basis( f );
    CanonicalBasis( GF(2^3) )
    gap> bas[1]; bas[2]; bas[3];
    Z(2)^0
    Z(2^3)
    Z(2^3)^2
    gap> coeff:= Coefficients( bas, Z(8)^3 );
    [ Z(2)^0, Z(2)^0, 0*Z(2) ]
    gap> LinearCombination( bas, coeff );
    Z(2^3)^3
    gap> v:= Rationals^3;
    ( Rationals^3 )
    gap> bas:= Basis( v, [ [ 1, 1, 1 ], [ 0, 1, 1 ], [ 0, 0, 1 ] ] );;
    gap> Coefficients( bas, [ 1, 2, 3 ] );
    [ 1, 1, 1 ]
    gap> Coefficients( bas, [ 3, 2, 1 ] );
    [ 3, -1, -1 ]
    gap> LinearCombination( bas, [ 1, 2, 3 ] );
    [ 1, 3, 6 ]
\endexample

The factor space of a vector space by a subspace is again a vector
space.

\beginexample
    gap> s:= Subspace( v, [ [ 1, 1, 1 ] ] );;
    gap> hom:= NaturalHomomorphismBySubspace( v, s );;
    gap> KernelOfAdditiveGeneralMapping( hom );
    VectorSpace( Rationals, [ [ 1, 1, 1 ] ] )
    gap> Image( hom );
    ( Rationals^2 )
    gap> List( bas, vec -> Image( hom, vec ) );
    [ [ 0, 0 ], [ -1, 0 ], [ -1, -1 ] ]
\endexample

%%%%%%%%%%%%%%%%%%%%%%%%%%%%%%%%%%%%%%%%%%%%%%%%%%%%%%%%%%%%%%%%%%%%%%%%%
\Section{Algebras}

If a multiplication is defined for the elements of a vector space,
and if the vector space is closed under this multiplication then it is
called an *algebra*.
For example, every field is an algebra, the vector space of all
square matrices of a given dimension over a given field is an algebra,
and the endomorphisms of a vector space form an algebra.
{\GAP} knows these facts.
Note that an algebra in {\GAP} need not be associative.

\beginexample
    gap> f:= GF(8); IsAlgebra( f );
    GF(2^3)
    true
    gap> a:= GF(2)^[2,2]; IsAlgebra( a );
    <algebra-with-one over GF(2), with 2 generators>
    true
    gap> hom:= End( GF(3), GF(3)^2 );
    End( GF(3), ( GF(3)^2 ) )
    gap> IsAlgebra( hom );
    true
    gap> l:= SimpleLieAlgebra( "A", 1, Rationals );
    <Lie algebra of dimension 3 over Rationals>
    gap> IsAssociative( l );
    false
\endexample

But if a domain was not constructed as an algebra,
for example if it knows only about its additive or multiplicative
structure, then {\GAP} does not regard it as algebra.
One can use `AsAlgebra' to get the domain consisting of the same elements
but viewed as algebra over an appropriate coefficient domain.

\beginexample
    gap> bas:= Basis( a );
    CanonicalBasis( <algebra-with-one over GF(2), with 2 generators> )
    gap> v:= VectorSpace( GF(2), bas );
    VectorSpace( GF(2), [ [ [ Z(2)^0, 0*Z(2) ], [ 0*Z(2), 0*Z(2) ] ], 
      [ [ 0*Z(2), Z(2)^0 ], [ 0*Z(2), 0*Z(2) ] ], 
      [ [ 0*Z(2), 0*Z(2) ], [ Z(2)^0, 0*Z(2) ] ], 
      [ [ 0*Z(2), 0*Z(2) ], [ 0*Z(2), Z(2)^0 ] ] ] )
    gap> IsAlgebra( v );
    false
    gap> AsAlgebra( GF(4), v );
    fail
    gap> va:= AsAlgebra( GF(2), v );
    <algebra of dimension 4 over GF(2)>
    gap> va = v; va = a;
    true
    true
\endexample

In the above example, note that the operator `=' tests equality as sets.

An algebra may or may not contain a multiplicatively neutral element.
If yes then the algebra may or may not contain the zeroth power
of its elements.
If this zeroth power is contained in the algebra then it is accessible
by `One'.
The latter distinction constitutes two notions of generation,
namely of an algebra and an algebra-with-one.

\beginexample
    gap> m:= [ [ 1, 0 ], [ 0, 0 ] ];;
    gap> one:= m^0;
    [ [ 1, 0 ], [ 0, 1 ] ]
    gap> a:= Algebra( Rationals, [ m ] );
    <algebra over Rationals, with 1 generators>
    gap> One( a );
    fail
    gap> one in a;
    false
    gap> MultiplicativeNeutralElement( a );
    [ [ 1, 0 ], [ 0, 0 ] ]
    gap> Dimension( a );
    1
    gap> a1:= AlgebraWithOne( Rationals, [ m ] );
    <algebra-with-one over Rationals, with 1 generators>
    gap> One( a1 );
    [ [ 1, 0 ], [ 0, 1 ] ]
    gap> one in a1;
    true
    gap> MultiplicativeNeutralElement( a1 );
    [ [ 1, 0 ], [ 0, 1 ] ]
    gap> Dimension( a1 );
    2
    gap> GeneratorsOfAlgebra( a1 );
    [ [ [ 1, 0 ], [ 0, 0 ] ], [ [ 1, 0 ], [ 0, 1 ] ] ]
    gap> GeneratorsOfAlgebraWithOne( a1 );
    [ [ [ 1, 0 ], [ 0, 0 ] ] ]
\endexample

The factor of an algebra by a subspace is of course a vector space,
and the factor by an ideal is again an algebra.

\beginexample
    gap> m1:= [ [ 0, 1, 0 ], [ 1, 0, 0 ], [ 0, 0, 0 ] ];;
    gap> m2:= [ [ 0, 0, 0 ], [ 0, 0, 0 ], [ 0, 0, 1 ] ];;
    gap> a:= Algebra( Rationals, [ m1, m2 ] );
    <algebra over Rationals, with 2 generators>
    gap> Dimension( a );
    3
    gap> s:= Subspace( a, [ m1 ] );
    VectorSpace( Rationals, [ [ [ 0, 1, 0 ], [ 1, 0, 0 ], [ 0, 0, 0 ] ]
     ] )
    gap> Dimension( s );
    1
    gap> hom1:= NaturalHomomorphismBySubspace( a, s );;
    gap> Dimension( Image( hom1 ) );
    2
    gap> i:= Subalgebra( a, [ m1 ] );
    <algebra over Rationals, with 1 generators>
    gap> Dimension( i );
    2
    gap> hom2:= NaturalHomomorphismByIdeal( a, i );;
    gap> Dimension( Image( hom2 ) );
    1
\endexample

Besides the more or less ``natural'' occurrences of the algebras
shown above, an algebra can also be given by a presentation.
A special kind of presentation is given by so-called
*structure constants*, that is, for a fixed basis $B$ of the algebra,
the $B$-coefficients of the products of elements in $B$.
Any finite dimensional algebra can be described by finitely many
structure constants. (Here `CF(7)' is the cyclotomic field
generated by the complex 7th roots of unity.)

\beginexample
    gap> bas:= Basis( a );;
    gap> T:= StructureConstantsTable( bas );
    [ [ [ [ 3 ], [ 1 ] ], [ [  ], [  ] ], [ [ 1 ], [ 1 ] ] ], 
      [ [ [  ], [  ] ], [ [ 2 ], [ 1 ] ], [ [  ], [  ] ] ], 
      [ [ [ 1 ], [ 1 ] ], [ [  ], [  ] ], [ [ 3 ], [ 1 ] ] ], 0, 0 ]
    gap> sc:= AlgebraByStructureConstants( Rationals, T );
    <algebra of dimension 3 over Rationals>
    gap> AlgebraByStructureConstants( CF(7), T );
    <algebra of dimension 3 over CF(7)>
\endexample

An interesting class of algebras for which many special algorithms
are implemented is the class of *Lie algebras*.
They arise for example as algebras of matrices, with product defined
by the Lie bracket $[ A, B ] = A \* B - B \* A$,
where $\*$ denotes the ordinary matrix product.

Since the multiplication in a domain in {\GAP} is always assumed to be
the operation `\\\*' (resp. the infix operator `\*'), 
and since there is already the ``ordinary'' matrix product defined for
matrices that are formed as lists of their rows,
we must take other matrices to generate Lie matrix algebras.
Such matrices can be constructed by `LieObject' from ordinary matrices.

\beginexample
    gap> a:= GF(2)^[2,2];
    <algebra-with-one over GF(2), with 2 generators>
    gap> IsAssociative( a ); IsLieAlgebra( a );
    true
    false
    gap> Dimension( a );
    4
    gap> gens:= GeneratorsOfAlgebra( a );
    [ [ [ 0*Z(2), Z(2)^0 ], [ Z(2)^0, 0*Z(2) ] ], 
      [ [ Z(2)^0, 0*Z(2) ], [ 0*Z(2), 0*Z(2) ] ], 
      [ [ Z(2)^0, 0*Z(2) ], [ 0*Z(2), Z(2)^0 ] ] ]
    gap> liegens:= List( gens, LieObject );
    [ LieObject( [ [ 0*Z(2), Z(2)^0 ], [ Z(2)^0, 0*Z(2) ] ] ), 
      LieObject( [ [ Z(2)^0, 0*Z(2) ], [ 0*Z(2), 0*Z(2) ] ] ), 
      LieObject( [ [ Z(2)^0, 0*Z(2) ], [ 0*Z(2), Z(2)^0 ] ] ) ]
    gap> ForAll( liegens, IsMatrix );
    true
    gap> liegens[1] = gens[1];
    false
    gap> liegens[1][1]; Length( liegens[1] );
    [ 0*Z(2), Z(2)^0 ]
    2
    gap> liegens[1][1] = gens[1][1];
    true
    gap> l:= Algebra( GF(2), liegens );
    <Lie algebra over GF(2), with 3 generators>
    gap> IsAssociative( l ); IsLieAlgebra( l );
    false
    true
    gap> Dimension( l );
    3
\endexample

So the algebras `a' and `l' behave differently.
Also the generators of the Lie matrix algebra are regarded as
different from the generators of the associative matrix algebra.
But the elements of both algebras are matrices, so one can access
their rows, and the rows of a matrix and its Lie object are equal.

Since a Lie algebra is a domain that is closed under multiplication,
operations such as centre and centralizer are defined in {\GAP}.
But usually the ``centre of a Lie algebra'' and the ``centralizer of a
subset in a Lie algebra'' are defined as annihilators.
In {\GAP}, one can compute these algebras with the operations `LieCentre'
and `LieCentralizer'.

\beginexample
    gap> LieCentre( l );
    <Lie algebra of dimension 1 over GF(2)>
    gap> Centre(l);
    <Lie algebra of dimension 3 over GF(2)>
\endexample

Many algorithms for Lie algebras use structure constants tables,
so it is reasonable to compute with s.c. algebras.
In order to avoid the check that a s.c. algebra is a Lie algebra,
one can use `LieAlgebraByStructureConstants'.
For several series of simple Lie algebras (see "SimpleLieAlgebra")
there are implementations via s.c. tables.

\beginexample
    gap> T:= StructureConstantsTable( BasisOfDomain( l ) );;
    gap> sc:= LieAlgebraByStructureConstants( GF(2), T );
    <Lie algebra of dimension 3 over GF(2)>
    gap> SimpleLieAlgebra( "A", 1, Rationals );
    <Lie algebra of dimension 3 over Rationals>
\endexample

Another example of algebras is provided by *quaternion algebras*.
We define a quaternion algebra over an extension field of the
rationals, namely the field generated by $\sqrt{5}$.
(The number `EB(5)' is equal to $1/2 (-1+\sqrt{5})$.
The field is printed as `NF(5,[ 1, 4 ])'.)

\beginexample
    gap> b5:= EB(5);
    E(5)+E(5)^4
    gap> q:= QuaternionAlgebra( FieldByGenerators( [ b5 ] ) );
    <algebra-with-one of dimension 4 over NF(5,[ 1, 4 ])>
    gap> gens:= GeneratorsOfAlgebra( q );
    [ e, i, j, k ]
    gap> e:= gens[1];; i:= gens[2];; j:= gens[3];; k:= gens[4];;
    gap> IsAssociative( q );
    true
    gap> IsCommutative( q );
    false
    gap> i*j; j*i;
    k
    (-1)*k
    gap> One( q );
    e
\endexample

If the coefficient field is a real subfield of the complex numbers
then the quaternion algebra is in fact a division ring.

\beginexample
    gap> IsDivisionRing( q );
    true
    gap> Inverse( e+i+j );
    (1/3)*e+(-1/3)*i+(-1/3)*j
\endexample

So {\GAP} knows about this fact.
As in any ring, we can look at groups of units.
(The function `StarCyc' used below computes the unique algebraic
conjugate of an element in a quadratic subfield of a cyclotomic field.)

\beginexample
    gap> c5:= StarCyc( b5 );
    E(5)^2+E(5)^3
    gap> g1:= 1/2*( b5*e + i - c5*j );
    (1/2*E(5)+1/2*E(5)^4)*e+(1/2)*i+(-1/2*E(5)^2-1/2*E(5)^3)*j
    gap> Order( g1 );
    5
    gap> g2:= 1/2*( -c5*e + i + b5*k );
    (-1/2*E(5)^2-1/2*E(5)^3)*e+(1/2)*i+(1/2*E(5)+1/2*E(5)^4)*k
    gap> Order( g2 );
    10
    gap> g:= Group( g1, g2 );;
    gap> Size( g );
    120
    gap> IsPerfect( g );
    true
\endexample

Since there is only one perfect group of order 120, up to isomorphism,
we see that the group `g' is isomorphic to $SL_2(5)$.
As usual, a permutation representation of the group can be constructed
using a suitable action of the group.

\beginexample
    gap> cos:= RightCosets( g, Subgroup( g, [ g1 ] ) );;
    gap> Length( cos );
    24
    gap> hom:= OperationHomomorphism( g, cos, OnRight );;
    gap> im:= Image( hom );
    Group( 
    [ ( 2, 3, 5, 9,15)( 4, 7,12, 8,14)(10,17,23,20,24)(11,19,22,16,13),
      ( 1, 2, 4, 8, 3, 6,11,20,17,19)( 5,10,18, 7,13,22,12,21,24,15)
        ( 9,16)(14,23) ], ... )
    gap> Size( im );
    120
\endexample

To get a matrix representation of `g' or of the whole algebra `q',
we must specify a basis of the vector space on that the algebra acts,
and compute the linear action of elements w.r.t. this basis.

\beginexample
    gap> bas:= CanonicalBasis( q );;
    gap> BasisVectors( bas );
    [ e, i, j, k ]
    gap> op:= OperationAlgebraHomomorphism( q, bas, OnRight );
    <op. hom. <algebra-with-one of dimension 4 over NF(5,
    [ 1, 4 ])> -> matrices of dim. 4>
    gap> ImagesRepresentative( op, e );
    [ [ 1, 0, 0, 0 ], [ 0, 1, 0, 0 ], [ 0, 0, 1, 0 ], [ 0, 0, 0, 1 ] ]
    gap> ImagesRepresentative( op, i );
    [ [ 0, 1, 0, 0 ], [ -1, 0, 0, 0 ], [ 0, 0, 0, -1 ], [ 0, 0, 1, 0 ] ]
    gap> ImagesRepresentative( op, g1 );
    [ [ 1/2*E(5)+1/2*E(5)^4, 1/2, -1/2*E(5)^2-1/2*E(5)^3, 0 ], 
      [ -1/2, 1/2*E(5)+1/2*E(5)^4, 0, -1/2*E(5)^2-1/2*E(5)^3 ], 
      [ 1/2*E(5)^2+1/2*E(5)^3, 0, 1/2*E(5)+1/2*E(5)^4, -1/2 ], 
      [ 0, 1/2*E(5)^2+1/2*E(5)^3, 1/2, 1/2*E(5)+1/2*E(5)^4 ] ]
\endexample


%%%%%%%%%%%%%%%%%%%%%%%%%%%%%%%%%%%%%%%%%%%%%%%%%%%%%%%%%%%%%%%%%%%%%%%%%
%%
%E  algvspc.tex . . . . . . . . . . . . . . . . . . . . . . . . ends here
