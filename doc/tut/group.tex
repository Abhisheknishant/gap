\Chapter{Groups and Homomorphisms}

In this  chapter we will  show  some easy  computations with groups.  The
example uses permutation groups,  because they are  the most easy ones to
input.   The   functions  mentioned   here,   like   `Group',  `Size'  or
`SylowSubgroup', however, are the same for  all kinds of groups, although
the algorithms which compute the information of  course will be different
in most cases.

%%%%%%%%%%%%%%%%%%%%%%%%%%%%%%%%%%%%%%%%%%%%%%%%%%%%%%%%%%%%%%%%%%%%%%%%%
\Section{Permutation groups}

Permutation  groups  are so easy to  input  because their elements, i.e.,
permutations,  are  so easy to  type: they  are  entered and displayed in
disjoint cycle notation. So let's construct a permutation group:
\beginexample
    gap> s8 := Group( (1,2), (1,2,3,4,5,6,7,8) );
    Group( [ (1,2), (1,2,3,4,5,6,7,8) ], ... )
\endexample
We   formed  the  group  generated  by   the  permutations  `(1,2)'   and
`(1,2,3,4,5,6,7,8)',  which is well known as the symmetric group on eight
points,  and  assigned  it  to the  identifier `s8'.   `s8' contains  the
alternating group on eight points which can be described in several ways,
e.g., as group of  all even permutations in  `s8',  or  as its commutator
subgroup.
\beginexample
    gap> a8 := CommutatorSubgroup( s8, s8 );
    Group( [ (1,3,2), (2,4,3), (3,5,4), (4,6,5), (5,7,6), (6,8,7) ], ... )
    gap> Size( a8 ); IsAbelian( a8 ); IsPerfect( a8 );
    20160
    false
    true
\endexample
Once information about a group like `a8'  has been computed, it is stored
in the group so that it can simply be looked  up if it is required again.
This   holds for the  last three  pieces  of information  in the previous
example, but  not for the commutator   subgroup of `s8' and  `s8' because
this was computed as  a function of *two* arguments,  and hence cannot be
attributed to just of them. (Had we computed `a8' as `DerivedSubgroup( s8
)', it would have been stored.) The  situation is bit different for Sylow
<p>-subgroups: The  function `SylowSubgroup' also requires two arguments,
namely a group and a prime <p>, but the result is stored in the group ---
namely  in position <p> of   a list called `ComputedSylowSubgroups', from
where it can be looked up later.
\beginexample
    gap> Set( Factors( Size( a8 ) ) );
    [ 2, 3, 5, 7 ]
    gap> for p  in last  do
    >      SylowSubgroup( a8, p );
    >    od;
    gap> ComputedSylowSubgroups( a8 );
    [ , 
      Group( [ (1,3)(5,8), (2,7)(5,8), (4,6)(5,8), (2,4)(6,7), (1,5)(3,8),
	  (1,2)(3,7)(4,5)(6,8) ], ... ), 
      Group( [ (4,7,8), (3,5,6) ], ... ),, Group( [ (4,7,6,8,5) ], ... ),
      , Group( [ (2,4,5,8,3,6,7) ], ... ) ]
\endexample
Observe   that the   list `ComputedSylowSubgroups(   a8  )'  has holes in
positions 1, 4  and 6. A call of  `SylowSubgroup' for the cyclic group of
order 65521 and for the prime 65521 would cause {\GAP} to store the group
at the end  of a list  of length 65521,  so there are special  situations
where it is possible  to bring {\GAP}  and yourself into trouble.  We now
can investigate the Sylow 2-subgroup of~`a8'.
\beginexample
    gap> syl2 := last[2];; Size( syl2 );
    64
    gap> Normalizer( a8, syl2 ) = syl2;
    true
    gap> cent := Centralizer( a8, Centre( syl2 ) );; Size( cent );
    192
    gap> DerivedSeries( cent );; List( last, Size );
    [ 192, 96, 32, 2, 1 ]
\endexample
One  goal  in this  section   is to  construct  a  subgroup of  structure
$2^3:L_3(2)$ in `a8'. One idea is to look for  an appropriate $2^3$ which
is specified by the  fact that all  its involutions are fixed point free,
and  then  compute its normalizer   in  `a8'. We  also  use the  function
`SetName'  to  assign a name  to  a  group, which will   then  be used in
printing that group.
\beginexample
    gap> elab := Group( (1,2)(3,4)(5,6)(7,8), (1,3)(2,4)(5,7)(6,8),
    >                   (1,5)(2,6)(3,7)(4,8) );; Size( elab );
    8
    gap> IsElementaryAbelian( elab );
    true
    gap> SetName( elab, "2^3" );
    gap> norm := Normalizer( a8, elab );; Size( norm );
    1344
\endexample
Now that we  have the  subgroup  `norm' of  order  1344 and its  subgroup
`elab', we  want to look  at its factor group.  But since we also want to
find preimages of factor group elements in `norm', we really want to look
at the    *natural homomorphism*
\index{homomorphism!natural}%
defined on `norm' with kernel `elab' and whose image is the factor group.
\beginexample
    gap> hom := NaturalHomomorphismByNormalSubgroup( norm, elab );
    <operation homomorphism>
    gap> f := Image( hom );
    Group( [ (), (), (), (4,5)(6,7), (4,6)(5,7), (2,3)(6,7), (2,4)(3,5), 
      (1,2)(5,6) ], ... )
    gap> Size( f );
    168
\endexample
The  factor group  is again   represented  as a permutation  group, whose
operation domain, however,  has nothing to do  with the one of `norm'. It
only happens that both  are subsets  of the natural  numbers. We  can now
form  images  and preimages under  the  natural  homomorphism. The set of
preimages of an element under `hom' is a coset  modulo `elab'. We use the
function `PreImages' here because `hom' is not a bijection, so an element
of the range can have several preimages or none at all.
\beginexample
    gap> Kernel( hom );
    2^3
    gap> x := (1,8,3,5,7,6,2);; Image( hom, x );
    ( 1, 7, 5, 6, 2, 3, 4)
    gap> coset := PreImages( hom, last );
    RightCoset(2^3,( 2, 8, 6, 7, 3, 4, 5))
    gap> x in coset;
    true
\endexample
We  wanted to construct  `norm' and `elab' in such  a way that the factor
group `f'  is the simple linear group  $L_3(2)$. We  can now test whether
this is the case.
\beginexample
    gap> IsSimple( f ); IsomorphismTypeFiniteSimpleGroup( f );
    true
    "A(1,7) = L(2,7) ~ B(1,7) = O(3,7) ~ C(1,7) = S(2,7) ~ 2A(1,7) = U(2,7\
    ) ~ A(2,2) = L(3,2)"
    gap> SetName( f, "L_3(2)" );
\endexample
(The simple group $L_3(2)$ also has a lot of other names.) `norm' acts on
the eight elements of its normal subgroup `elab' by conjugation, yielding
a representation of $L_3(2)$ in `s8' which leaves one point fixed (namely
point~`1').  The image of this  representation  is even contained in  the
subgroup `norm'  of `s8' and we  can show that  `norm' is indeed  a split
extension of   the  elementary abelian group   $2^3$ with  this image  of
$L_3(2)$.
\beginexample
    gap> op := Operation( norm, elab );
    Group( [ (), (), (), (5,6)(7,8), (5,7)(6,8), (3,4)(7,8), (3,5)(4,6), 
      (2,3)(6,7) ], ... )
    gap> IsSubgroup( a8, op ); IsSubgroup( norm, op );
    true
    true
    gap> IsTrivial( Intersection( elab, op ) );
    true
    gap> SetName( norm, "2^3:L_3(2)" );
\endexample
By the way, you should not try the operator `\<'  instead of the function
`IsSubgroup'. Something like
\beginexample
    gap> elab < a8;
    false
\endexample
will not cause an error, but the result does not  tell anything about the
inclusion of one group in another; `\<' tests  which of the two groups is
less is some total order. On the other hand, the equality operator `=' in
fact does test the equality of groups.

%%%%%%%%%%%%%%%%%%%%%%%%%%%%%%%%%%%%%%%%%%%%%%%%%%%%%%%%%%%%%%%%%%%%%%%%%
\Section{Operations!of groups}

In  order to get  another  representation  of  `a8', we consider  another
action, namely  that on  the elements   of a  certain conjugacy  class by
conjugation.
\beginexample
    gap> ccl := ConjugacyClasses( a8 );; Length( ccl );
    14
    gap> List( ccl, c -> Order( Representative( c ) ) );
    [ 1, 7, 7, 5, 15, 15, 3, 3, 6, 6, 2, 4, 2, 4 ]
    gap> List( ccl, Size );
    [ 1, 2880, 2880, 1344, 1344, 1344, 112, 1120, 1680, 3360, 105, 1260, 
      210, 2520 ]
\endexample
Note the  difference between  `Order'  (which means  the element  order),
`Size' (which means the size of the conjugacy  class) and `Length' (which
means the length of a  list).
\beginexample
    gap> class := First( ccl, c -> Size(c) = 112 );;
    gap> op := Operation( a8, AsList( class ) );;
\endexample
We use `AsList' here  to convert the conjugacy class  into a list  of its
elements whereas we   wrote  `Operation( norm,  elab )'   directly in the
previous section. The reason is that  the elementary abelian group `elab'
can  be quickly enumerated by   {\GAP}  whereas the standard  enumeration
method for conjugacy classes is  slower than just explicit calculation of
the elements. However, {\GAP} is reluctant  to construct explicit element
lists, because for really large groups this direct method is infeasible.

We got a permutation representation `op' on 112 points, which we now test
for primitivity.
\beginexample
    gap> IsPrimitive( op, [ 1 .. 112 ] );
    false
    gap> blocks := Blocks( op, [ 1 .. 112 ] );;
    gap> op2 := Operation( op, blocks, OnSets );;
    gap> IsPrimitive( op2, [ 1 .. 56 ] );
    true
\endexample
The action of `op' on the given block system gave us a new representation
on 56 points which is primitive, i.e.,  the point stabilizer is a maximal
subgroup. We compute its preimage  in the representation on eight  points
using homomorphisms (which of course are monomorphisms). We construct the
composition  of  two   homomorphisms   with the  `*'  operator,   reading
left-to-right.
\beginexample
    gap> ophom := OperationHomomorphism( a8, op );;
    gap> ophom2 := OperationHomomorphism( op, op2 );;
    gap> composition := ophom * ophom2;;
    gap> stab := Stabilizer( op2, 1 );;
    gap> preim := PreImages( composition, stab );
    Group( [ ( 1, 3, 2), ( 4, 8, 7), ( 1, 2)( 7, 8), ( 2, 3, 5), 
      ( 1, 2)( 5, 6) ], ... )
\endexample
The normalizer of an element in the conjugacy class `class' is a group of
order 360, too. In fact, it is a conjugate of the maximal subgroup we had
found before, and a conjugating element in `a8'  is found by the function
`RepresentativeOperation'.
\beginexample
    gap> sgp := Normalizer( a8, Subgroup(a8,[Representative(class)]) );;
    gap> Size( sgp );
    360
    gap> RepresentativeOperation( a8, sgp, preim );
    (5,6)(7,8)
\endexample
The scalar product  of permutation characters  of two subgroups <U>, <V>,
say, equals  the number  of $(<U>,<V>)$-double  cosets. For example,  the
norm of  the natural permutation character of  degree  eight is two since
the action of `a8' on the cosets of a point stabilizer is at least doubly
transitive. We   also   compute  the   numbers  of    $(`sgp',`sgp')$ and
$(`sgp',`stab')$ double cosets.
\beginexample
    gap> stab := Stabilizer( a8, 1 );;
    gap> Length( DoubleCosets( a8, stab, stab ) );
    2
    gap> Length( DoubleCosets( a8, sgp, sgp ) );
    4
    gap> Length( DoubleCosets( a8, sgp, stab ) );
    2
\endexample

So far we have  seen a few  applications of the functions `Operation' and
`OperationHomomorphism'.  But perhaps even more   interesting is the fact
that   the  natural homomorphism `hom'    constructed  above is  also  an
*operation homomorphism*;
\index{homomorphism!operation}%
this is also the  reason why its  image  is represented as  a permutation
group: it is the natural representation for  operations. We will now look
at   this natural  homomorphism again to   find  out on  what objects  it
operates. These objects form a so-called *external set*
\index{external set}%
which is stored in the homomorphism similar to a record component.
\beginexample
    gap> t := hom!.externalSet;
    RightTransversal( 2^3:L_3(2), Group( 
    [ (1,2)(3,4)(5,6)(7,8), (1,3)(2,4)(5,7)(6,8), (1,5)(2,6)(3,7)(4,8), 
      (5,6)(7,8), (5,7)(6,8), (3,4)(7,8), (3,5)(4,6) ], ... ) )
\endexample
The external  set is a *right  transversal*
\index{right transversal}%
of a subgroup <U> in `norm', such that the kernel of the operation is the
desired  normal subgroup  `elab'.  This  right  transversal contains  one
element from each coset of <U>. But  an external set  is more than just a
list  of its points: Altogether three  things are necessary to specify an
operation: a group~<G>, a  set~<D>, and a function $<opr>\colon <D>\times
<G>\to  <D>$. We   can access   these   ingredients with   the  following
functions:
\beginexample
    gap> ActingDomain(t); Enumerator(t); FunctionOperation(t)=OnRight;
    2^3:L_3(2)
    RightTransversal( 2^3:L_3(2), Group( 
    [ (1,2)(3,4)(5,6)(7,8), (1,3)(2,4)(5,7)(6,8), (1,5)(2,6)(3,7)(4,8), 
      (5,6)(7,8), (5,7)(6,8), (3,4)(7,8), (3,5)(4,6) ], ... ) )
    true
\endexample
The function `OnRight' means multiplication from the  right, which is the
usual way to  operate on a right transversal.  `OnRight(  <d>, <g> )'  is
defined as `<d> * <g>'.

\exercise In analogy to `OnRight'  you might imagine a function `OnLeft',
but  actually  there      are   two   functions     `OnLeftInverse'   and
`OnLeftAntiOperation'. How are they defined and why?

\answer   `OnLeftInverse(  <d>,  <g>  )'  means   `<g>^-1 *  <d>',  i.e.,
multiplication  with the  inverse   from   the  left, which defines    an
operation. In constrast,  the mapping $<left>\colon (<d>,<g>) \mapsto <g>
* <d>$ does not yield an operation, but an anti-operation, i.e., $<left>(
<left>( <d>,  <g>_1 ), <g>_2  ) = <left>( <d>, <g>_2  * <g>_1 )$, whereas
for an  operation, the  right hand side  is required  to have the product
$<g>_1 * <g>_2$.  The {\GAP} function which  performs this mapping <left>
is therefore called  `OnLeftAntiOperation', and, unlike  `OnLeftInverse',
this cannot be used as last argument to an operation function.

Observe that the  external set `t' and  its `Enumerator' are  printed the
same way,  but be aware that an  external  set also comprises  the acting
domain and the  operation function.  The  `Enumerator' itself, i.e.,  the
right transversal in turn comprises  knowledge about the group `norm' and
the  subgroup <U>. Yet it   behaves as a  list and  you  can ask for  the
position of an element in it.
\beginexample
    gap> elm := (1,4)(2,7)(3,6)(5,8);;
    gap> Position( Enumerator(t), elm );
    fail
    gap> PositionCanonical( Enumerator(t), elm );
    5
\endexample
The result `fail'  means  that the element was   not found at all  in the
list: it is not among the chosen representatives. The difference
\atindex{Position vs. PositionCanonical}{@\noexpand `Position' vs.
\noexpand `PositionCanonical'}%
between the functions   `Position'  and `PositionCanonical' is   that the
first  simply looks whether  `elm' is contained among the representatives
which together form the right  transversal `t', whereas the second really
looks  for  the position of  the  coset  described  by the representative
`elm'.  In  other   words,  it  first    replaces `elm'  by  a  canonical
representative of  the same  coset (which must  be contained  in `t') and
then looks for its position, hence the name.

\exercise   What   would     be  the  result    of   the  function   call
`PositionCanonical( AsList(t), elm )' (where `AsList(t)' is equivalent to
`AsList(Enumerator(t))')?

\answer `fail', because `AsList(t)'  is simply a list of  representatives
without  any knowledge about the  cosets  they represent. In other words,
{\GAP}    does not  know  that  they   are   representatives, and so  the
``canonical representative'' associated with `elm' is `elm' itself, which
is not among the elements of the list.

The  image  of the natural  homomorphism   is the permutation  group that
results from the operation of `norm' on the right  transversal. It can be
calculcated by either of the following commands. The second of them shows
that the external set `t' contains all information  that is necessary for
`Operation' to do its work.
\beginexample
    gap> Operation( norm, Enumerator(t), OnRight );
    L_3(2)
    gap> Operation( t );
    L_3(2)
\endexample

\exercise What difference does  it make whether something  is stored in a
record  component   like `hom!.externalSet'   or   as  an attribute  like
`Enumerator(t)'?

\answer  A record component  like `hom!.externalSet' must be entered when
the object `hom'  is constructed, which makes  sense because an operation
homomorphism   like  `hom' is  constructed for     a given external  set.
Attributes like `Enumerator(t)' need not  be set upon construction of the
object, they can be computed  and stored when  they are required for  the
first time and subsequently be looked up. This  mechanism is explained in
section~"Attributes".

%%%%%%%%%%%%%%%%%%%%%%%%%%%%%%%%%%%%%%%%%%%%%%%%%%%%%%%%%%%%%%%%%%%%%%%%%
\Section{Subgroups!as stabilizers}

Operation functions can also be used  without constructing external sets.
We will  try to find  several  subgroups in `a8'   as stabilizers of such
operations. One subgroup  is immediately available, namely the stabilizer
of one point. The index of the stabilizer must  of course be equal to the
length of the orbit, i.e., 8. Here the operation function is omitted, but
we could also write it down, it is called  `OnPoints' and also denoted by
the caret operator~`^'. `OnPoints' and its relatives are simply functions
that  take an  element of  the  domain  and an  element  of the  group as
arguments  and return the  image of the element   of the domain under the
group element. In  our example, `OnPoints(  <pnt>, <elm> )' is equivalent
to `<pnt> ^ <elm>'.
\beginexample
    gap> u8 := Stabilizer( a8, 1 );
    Group( [ (2,4,3), (3,5,4), (4,6,5), (5,7,6), (6,8,7) ], ... )
    gap> Index( a8, u8 );
    8
    gap> Orbit( a8, 1 ); Length( last );
    [ 2, 3, 1, 4, 5, 6, 7, 8 ]
    8
\endexample
This gives us a hint how to find further  subgroups. Each subgroup is the
stabilizer of a point of an appropriate  transitive operation (namely the
operation  on  the cosets of that  subgroup  or another operation that is
equivalent to  this operation).  So the question   is how  to find  other
operations. The obvious thing is to operate  on pairs of points. So using
the function `Tuples' we first generate a list of all pairs.
\beginexample
    gap> pairs := Tuples( [1..8], 2 );;
\endexample
Now we would like to have `a8' operate on this  domain. But we cannot use
the  default operation `OnPoints'  because    `<list> ^ <perm>' is    not
defined. So we  must tell the functions  from the operations package  how
the group elements operate on the elements of  the domain. In our example
we can do this by simply passing `OnPairs' as optional last argument. All
functions  from the operations package  accept  such an optional argument
that describes the operation. One example is `IsTransitive'.
\beginexample
    gap> IsTransitive( a8, pairs, OnPairs );
    false 
\endexample
The operation is not transitive,  so we want to  find out what the orbits
are. The function `Orbits' does that for us. It returns a list of all the
orbits.
\beginexample
    gap> orbs := Orbits( a8, pairs, OnPairs );; Length( orbs );
    2
    gap> List( orbs, o -> o[1] );
    [ [ 1, 1 ], [ 1, 2 ] ]
\endexample
The operation of `a8' on the first orbit  (the one containing `[1,1]') is
of course equivalent to the original operation,  so we ignore it and work
with the second orbit.
\beginexample
    gap> u56 := Stabilizer( a8, [1,2], OnPairs );; Index( a8, u56 );
    56 
\endexample
So   now   we have  found   a  second subgroup.   To   make the following
computations a little bit easier and more efficient  we would now like to
work on the points `[1..56]'  instead of the list  of pairs. The function
`OperationHomomorphism' does what  we need.   It creates a   homomorphism
defined on `a8' whose image is a new group  that operates on `[1..56]' in
the same way that `a8' operates on the second orbit.
\beginexample
    gap> h56 := OperationHomomorphism( a8, orbs[2], OnPairs );;
    gap> a8_56 := Image( h56 );;
\endexample
We would now like to know if the subgroup `u56' of index 56 that we found
is  maximal or  not.   Again  we can make  use  of  a function  from  the
operations package.  Namely a subgroup is maximal if the operation on the
cosets of  this subgroup is primitive.
\beginexample
    gap> IsPrimitive( a8_56, [1..56] );
    false 
\endexample
Note that we  must specify the domain  of the operation. You might  think
that  in the last example   `IsPrimitive' could use `[1..56]' as  default
domain  if no domain was given.  But this is   not so simple, for example
would the  default domain of `Group(  (2,3,4) )' be `[1..4]' or `[2..4]'?
To  avoid confusion,  all  operations package  functions require that you
specify the  domain of operation. If  you  specify `[1..4]' in  the above
example, point~`1' will be a fixpoint.

\exercise What happens if  you call a function  like `Operation( <G>, <D>
)',  where the group  <G>  moves points  that are   not contained in  the
operation domain~<D>?

\answer If the  operation domain <D> is  closed  under the action  of the
group <G>, i.e., if  it is a union of  orbits, `Operation' will construct
the operation only on that domain, i.e.,  it will compute the restriction
of <G> to~<D>. If  the operation domain  is  not closed, however,  {\GAP}
will   fail to construct  the     generating permutations and return    a
`MagmaWithInverses( [ fail,  \dots,  fail  ] )',  which  is no  good  for
further computations. An exception to this rule is the function `Orbits':
if  <D> is not   closed under <G>, then    `Orbits( <G>, <D> )'  silently
replaces  <D> by its  closure under <G>  and then computes the orbits. In
other words, it  computes all <G>-orbits that contain  at least one point
from the original~<D>.

We see that `a8_56'   is not primitive. This  means  of course   that the
operation  of `a8'  on  `orb[2]'  is not  primitive,  because those   two
operations are equivalent. So the stabilizer `u56' is not maximal. Let us
try to find its supergroups. We use the function `Blocks' to find a block
system. The  (optional) third  argument  in the following   example tells
`Blocks' that we want a block system where 1 and 10 lie in one block.
\beginexample
    gap> blocks := Blocks( a8_56, [1..56], [1,10] );
    [ [ 1, 10, 17, 20, 25, 33, 38 ], [ 2, 5, 6, 12, 14, 24, 36 ], 
      [ 3, 7, 15, 32, 37, 44, 48 ], [ 4, 8, 13, 23, 26, 35, 47 ], 
      [ 9, 16, 27, 28, 40, 52, 54 ], [ 11, 18, 21, 29, 39, 46, 53 ], 
      [ 19, 30, 41, 42, 49, 50, 55 ], [ 22, 31, 34, 43, 45, 51, 56 ] ]
\endexample
The result is a list  of sets, such that  `a8_56' operates on those sets.
Now we would like  the stabilizer of this  operation on the sets. Because
we  wanted to operate on   the sets we  have  to  pass `OnSets' as  third
argument.
\beginexample
    gap> u8_56 := Stabilizer( a8_56, blocks[1], OnSets );;
    gap> Index( a8_56, u8_56 );
    8
    gap> u8b := PreImages( h56, u8_56 );; Index( a8, u8b );
    8
    gap> u8 = u8b;
    false
\endexample
Here  is a *warning*:   If you specify `OnSets'  as  third argument  to a
function like  `Stabilizer', you have to  make sure that the  point (i.e.
the second argument) is  indeed a set. Otherwise you  will get a puzzling
error message or  even  wrong results! In  the above  example, the second
argument  `blocks[1]'  came from the  function  `Blocks', which returns a
list of sets, so everything was OK.

Actually there  is a third  block system of `a8_56'  that gives rise to a
third subgroup.
\beginexample
    gap> blocks := Blocks( a8_56, [1..56], [1,6] );;
    gap> u28_56 := Stabilizer( a8_56, [1,6], OnSets );;
    gap> u28 := PreImages( h56, u28_56 );;
    gap> Index( a8, u28 );
    28 
\endexample
We know that  the subgroup `u28' of index  28 is maximal, because we know
that  `a8' has no  subgroups  of index 2,  4,  or 7.  However we can also
quickly verify this by checking  that `a8_56' operates primitively on the
28 blocks.
\beginexample
    gap> IsPrimitive( a8_56, blocks, OnSets );
    true 
\endexample

`Stabilizer' is not only applicable to groups like `a8' but also to their
subgroups like  `u56'. So another  method  to find  a  new subgroup is to
compute the stabilizer of another point in `u56'. Note that `u56' already
leaves 1 and 2 fixed.
\beginexample
    gap> u336 := Stabilizer( u56, 3 );;
    gap> Index( a8, u336 );
    336 
\endexample
Other functions  are also  applicable to subgroups.  In  the following we
show that  `u336' operates regularly on the  60 triples of `[4..8]' which
contain no  element twice, which  means that this operation is equivalent
to the operations of `u336' on its 60 elements from the right. `OnTuples'
is a generalization of `OnPairs'.
\beginexample
    gap> IsRegular( u336, Orbit( u336, [4,5,6], OnTuples ), OnTuples );
    true 
\endexample
Just as we did in the  case of the  operation on the  pairs above, we now
construct a new permutation group that operates on `[1..336]' in the same
way that `a8' operates on the cosets of `u336'. But this time we let `a8'
operate  on a right   transversal, just like  `norm'  did in  the natural
homomorphism above.
\beginexample
    gap> t := RightTransversal( a8, u336 );;
    gap> a8_336 := Operation( a8, t, OnRight );;
\endexample
To find  subgroups above  `u336'  we  again  check if  the  operation  is
primitive.
\beginexample
    gap> blocks := Blocks( a8_336, [1..336], [1,43] );; blocks[1];
    [ 1, 43, 85 ]
\endexample
The subgroup of index  112 in `a8_336'  is  the union of `u336'  with its
43rd and its  85th coset.   It  is thus   the  closure of `u336' with   a
representative of the  43rd coset, which  is obtained as the 43rd element
of the list~`t'.
\beginexample
    gap> u112 := ClosureGroup( u336, t[43] );;
    gap> Index( a8, u112 );
    112 
\endexample
Above this subgroup of index 112 lies a  subgroup  of index 56, which  is
not conjugate to `u56'.  In fact, unlike `u56' it is  maximal.  We obtain
this subgroup in  the same way that we obtained `u112', this time forcing
two points, namely 39 and 43 into the first block.
\beginexample
    gap> blocks := Blocks( a8_336, [1..336], [1,39,43] );;
    gap> Length( blocks );
    56
    gap> u56b := ClosureGroup( u112, t[39] );; Index( a8, u56b );
    56
    gap> IsPrimitive( a8_336, blocks, OnSets );
    true 
\endexample
We already   mentioned in the beginning  that  there is  another standard
operation of permutations,  namely the conjugation.  E.g., since no other
operation is specified   in the following  example, `OrbitLength'  simply
operates via `OnPoints', and because `$<perm>_1$ ^ $<perm>_2$' is defined
as  the  conjugation of $perm_2$  on  $perm_1$ we effectively compute the
length of  the  conjugacy   class of  `(1,2)(3,4)(5,6)(7,8)'.   (In  fact
`$<elm>_1$ ^ $<elm>_2$' is always defined as the conjugation if $<elm>_1$
and $<elm>_2$ are group elements  of the same  type. So  the length of  a
conjugacy  class of any  element <elm> in an   arbitrary group <G> can be
computed as `OrbitLength( <G>, <elm> )'.  In general however this may not
be a good  idea, `Size( ConjugacyClass( <G>,  <elm> ) )' is probably more
efficient.)
\beginexample
    gap> OrbitLength( a8, (1,2)(3,4)(5,6)(7,8) );
    105
    gap> orb := Orbit( a8, (1,2)(3,4)(5,6)(7,8) );;
    gap> u105 := Stabilizer( a8, (1,2)(3,4)(5,6)(7,8) );; Index( a8, u105 );
    105 
\endexample
Of course the  last stabilizer is in  fact the centralizer of the element
`(1,2)(3,4)(5,6)(7,8)'.  `Stabilizer' notices    that and computes    the
stabilizer using the centralizer algorithm for permutation groups. In the
usual way we now look for the subgroups above `u105'.
\beginexample
    gap> blocks := Blocks( a8, orb );; Length( blocks );
    15

    gap> blocks[1];
    [ (1,2)(3,4)(5,6)(7,8), (1,3)(2,4)(5,8)(6,7), (1,4)(2,3)(5,7)(6,8),
      (1,5)(2,6)(3,8)(4,7), (1,6)(2,5)(3,7)(4,8), (1,7)(2,8)(3,6)(4,5),
      (1,8)(2,7)(3,5)(4,6) ]
\endexample
To find the subgroup of index 15 we  again use closure. Now  we must be a
little bit  careful to avoid    confusion. `u105' is the  stabilizer   of
`(1,2)(3,4)(5,6)(7,8)'. We  know  that there is  a correspondence between
the  points  of  the   orbit and  the   cosets  of  `u105'.   The   point
`(1,2)(3,4)(5,6)(7,8)' corresponds   to `u105'.  To get  the  subgroup of
index 15 we  must add to `u105' an  element of the coset that corresponds
to the   point `(1,3)(2,4)(5,7)(6,8)' (or any other   point  in the first
block).  That means   that  we must  use  an  element of `a8'  that  maps
`(1,2)(3,4)(5,6)(7,8)' to  `(1,3)(2,4)(5,7)(6,8)'.        The    function
`RepresentativeOperation'   does what we need. It   takes a group and two
points and returns an  element of the group that  maps the first point to
the   second. In fact it   also allows you  to   specify the operation as
optional fourth argument as usual,  but we do  not need this here. If  no
such element exists in the  group, i.e., if the  two points do not lie in
one orbit under the group, `RepresentativeOperation' returns `fail'.
\beginexample
    gap> rep := RepresentativeOperation( a8, (1,2)(3,4)(5,6)(7,8),
    >                                        (1,3)(2,4)(5,7)(6,8) );
    (2,3)(6,7)
    gap> u15 := ClosureGroup( u105, rep );; Index( a8, u15 );
    15 
\endexample
`u15' is of course a maximal  subgroup, because `a8'  has no subgroups of
index 3 or~5.  There is in fact  another  class of subgroups  of index 15
above `u105' that we get by adding `(2,3)(6,8)' to `u105'.
\beginexample
    gap> u15b := ClosureGroup( u105, (2,3)(6,8) );; Index( a8, u15b );
    15 
    gap> RepresentativeOperation( a8, u15, u15b );
    false 
\endexample
`RepresentativeOperation' tells us that  there is no  element <g> in `a8'
such that `u15 ^ <g> = u15b'. Because `^' also denotes the conjugation of
subgroups this tells us  that  `u15' and  `u15b' are not  conjugate.

This  concludes  our  example.   In  this  section  we demonstrated  some
functions  from  the operations  package.   There  is  a  whole class  of
functions that  we  did  not mention, namely  those that  take  a  single
element instead  of a  whole  group  as first argument, e.g., `Cycle' and
`Permutation'.

%%%%%%%%%%%%%%%%%%%%%%%%%%%%%%%%%%%%%%%%%%%%%%%%%%%%%%%%%%%%%%%%%%%%%%%%%
\Section{Group homomorphisms!by images}

We  have already    seen examples of  group  homomorphisms   in the  last
sections, namely  natural homomorphisms and  operation homomorphisms.  In
this section we will show how to construct  a group homomorphism $G\to H$
by specifying a generating set for $G$ and the images of these generators
in~$H$.
\beginexample
    gap> f := FreeGroup( 2 ); g := Group( (1,2,3,4), (1,2) );
    Group( [ f.1, f.2 ], ... )
    Group( [ (1,2,3,4), (1,2) ], ... )
    gap> hom := GroupHomomorphismByImages( f, g,
    >           GeneratorsOfGroup( f ), GeneratorsOfGroup( g ) );
    [ f.1, f.2 ] -> [ (1,2,3,4), (1,2) ]
\endexample
Since `f'  is a  free  group, its free  generators  can be mapped  to any
generating set of the same length in any group; so `hom' really defines a
group homomorphism.  On the other  hand, calculation in subgroups of free
groups is fairly impossible, and  a call of  `Kernel( hom )' would result
in an infinite calculation. Our goal  in this section is  to find out why
this is so and how far we can get.

First, the kernel of the homomorphism `hom' is calculated as the cokernel
of the inverse  of `hom'. What   does that mean?  The inverse  mapping of
`hom' goes from  `g'  to  `f' and  maps  the  generators  of `g'   to the
generators of `f'.
\beginexample
    gap> inv := Inverse( hom );
    fail
    gap> inv := InverseGeneralMapping( hom );
    [ (1,2,3,4), (1,2) ] -> [ f.1, f.2 ]
\endexample
Simply `Inverse' does not work, because `hom' is not a bijection. We have
to use the more cautious   `InverseGeneralMapping': since `f' is a   free
group, so are all its subgroups, and since  `g' is not free, `inv' cannot
define a group homomorphism  $`g' \to `f'$.  What exists is a multivalued
mapping, or relation, i.e., a subset $R\subset  `g' \times `f'$, with the
property that  if  two pairs $(g_1,f_1)$  and $(g_2,f_2)$  belong to~$R$,
then also $(g_1\cdot g_2,\,f_1\cdot f_2)\in R$. Such a relation is called
a *group general mapping* in  {\GAP} and it  can be constructed with  the
command    `GroupGeneralMappingByImages'.     For   example,   `inv     =
GroupGeneralMappingByImages(    g,      f,    GeneratorsOfGroup(   g   ),
GeneratorsOfGroup( f ) )'.

The  image $I$ of $R$ is  the set of all   elements that occur as images,
i.e.,  as right hand sides of   pairs from $R$.   The cokernel of a group
general mapping with  image $I$ is the set  of all images of the identity
element, they form a normal subgroup of~$I$. The set of  all images of an
arbitrary  element  is then  a coset  of the cokernel.  For group general
mappings defined  on a permutation group  (like  `inv') {\GAP} calculates
the cokernel   as the normal   closure in $I$ of  certain  images  of the
identity,  which  are obtained as  Schreier  generators (in  a stabilizer
chain  for  the permutation group).  The  only problem  in our particular
example is the determination of the normal closure in~$I$, because $I$ is
a free group. We can,  however, look at  the Schreier generators with the
following command:
\beginexample
    gap> cok := CoKernelGensPermHom( inv );
    [ f.1*f.2*f.1*f.2*f.1*f.2, f.2^2, f.1^-1*f.2*f.1^2*f.2*f.1*f.2*f.1^
	2*f.2*f.1^2, f.2^-1*f.1^4*f.2, f.1^-2*f.2*f.1^4*f.2*f.1^2, 
      f.1^2*f.2*f.1^4*f.2*f.1^2, f.2^-1*f.1*f.2*f.1^3*f.2*f.1*f.2*f.1^
	2*f.2*f.1^2, f.2^-1*f.1^-1*f.2^-1*f.1*f.2*f.1^3*f.2*f.1^2*f.2*f.1^
	2 ]
\endexample
A  group general mapping like  `inv' is a genuine  mapping if and only if
this  list of Schreier generators contains  at most the identity element,
and this what {\GAP}  tests if you ask  `IsMapping(  inv )'. In  general,
these generators form  a set of defining  relators for a presentation  of
the factor  $I/\hbox{cokernel}$,    which is isomorphic  to   the  factor
preimage/kernel.  Since in our   case   `inv' is  injective,  this is   a
presentation for  `g'. We will  use it  to  build a  *finitely  presented
group* `p' and set up an isomorphism between `p' and~`g'.
\beginexample
    gap> p := f / cok;
    Group( ... )
    gap> iso := GroupHomomorphismByImages( g, p,
    >           GeneratorsOfGroup( g ), GeneratorsOfGroup( p ) );
    [ (1,2,3,4), (1,2) ] -> [ f.1, f.2 ]
\endexample
We       used      `GroupHomomorphismByImages'        here,           not
`GroupGeneralMappingByImages', because we  wanted to  promise {\GAP} that
`iso' actually defines a group homomorphism.  (This is so because we have
factored out  the cokernel to obtain `p'.)  To see that this really gives
an  isomorphism, we take  the  cruellest approach by  looking  at all the
pairs   (preimage,image) that `iso' contains  when  viewed as a relation,
i.e., as a subset  of~$`g'\times `p'$. To get the  kernel, we filter  out
all pairs whose second entry is `One( p )'.
\beginexample
    gap> pairs := AsList( UnderlyingRelation( iso ) );; Length( pairs );
    24
    gap> Filtered( pairs, pair -> pair[2] = One( p ) );
    [ () ]
\endexample

%%%%%%%%%%%%%%%%%%%%%%%%%%%%%%%%%%%%%%%%%%%%%%%%%%%%%%%%%%%%%%%%%%%%%%%%%
\Section{Nice monomorphisms}

In this section we will construct an ugly group, namely a matrix group of
degree~2 over the Gaussian rationals. Because the group of all invertible
$2\times 2$-matrices over this  field is infinite,  {\GAP} cannot be sure
from the beginning that our group is finite, even though  it turns out to
be the quaternion  group of order~8, hence  finite. To investigate it, we
will construct  a  monomorphism of  the matrix  group into a  permutation
group, which  is much   nicer  to work with.  This   image of this  *nice
monomorphism* is then isomorphic to our  matrix group, and we can perform
calculations in the nice  group and lift the  results back to the  matrix
group with the nice monomorphism.
\beginexample
    gap> i := E(4);; grp := Group([[i,0],[0,-i]],[[0,1],[-1,0]]);;
    gap> SetName( grp, "Q8" );
    gap> orb := Union( Orbits( grp, One( grp ) ) );
    [ [ 1, 0 ], [ E(4), 0 ], [ 0, 1 ], [ -1, 0 ], [ 0, E(4) ],
        [ 0, -E(4) ], [ -E(4), 0 ], [ 0, -1 ] ]
\endexample
The list `orb' is the  union of orbits under `grp'  which contain a basis
for the  two-dimensional row  vector space over   the rationals (we which
conveniently obtain as `One( grp )',  i.e., the identity matrix). We have
made use here of the possibility to call `Orbits' with a list that is not
closed under the operation:  it will be replaced by  the closure on which
the orbits  are then determined (in our  case, there  is only one orbit).
The finiteness of `grp' implies the finiteness of `orb',  and we obtain a
faithful permutation representation on 8~points.
\beginexample
    gap> hom := OperationHomomorphism( grp, orb );; IsInjective( hom );
    true
    gap> p := Image( hom );
    Group( [ (1,7,6,8)(2,5,3,4), (1,2,6,3)(4,8,5,7) ], ... )
\endexample
To   demonstrate  the technique of  nice    monomorphisms, we compute the
conjugacy classes of  the permutation group and  lift them back  into the
matrix group with the monomorphism `hom'.  Lifting back a conjugacy class
means finding the preimage of  the representative and of the centralizer;
the latter is also called `StabilizerOfExternalSet' in {\GAP}.
\beginexample
    gap> pcls := ConjugacyClasses( p );; gcls := [  ];;
    gap> for pc  in pcls  do
    >      gc := ConjugacyClass(grp,PreImageElm(hom,Representative(pc)));
    >      SetStabilizerOfExternalSet(gc,PreImage(hom,Centralizer(pc)));
    >      Add( gcls, gc );
    >    od; List( gcls, Size );
    [ 1, 1, 2, 2, 2 ]
\endexample
Since the  mapping `hom' is injective, but  not surjective, we cannot use
the function `PreImage' for  preimages of elements  (they need not have a
preimage). Instead,  we have to use `PreImageElm'  to denote the (unique)
preimage of an element, but can use `PreImage'  for the preimage of a set
(which always exists, even if it is empty).

All the steps we have made above are automatically performed by {\GAP} if
you  simply   ask for `ConjugacyClasses(  grp  )',   provided that {\GAP}
already knows that  `grp' is finite,  e.g., because you asked  `IsFinite(
grp )'  before. The reason  for this is that a   finite matrix group like
`grp' is ``handled by a nice monomorphism'',  like `hom' is one. For such
groups,  {\GAP} uses  the command   `NiceMonomorphism'   to construct   a
monomorphism like `hom' and then proceeds as we have done above.
\beginexample
    gap> grp := Group([[i,0],[0,-i]],[[0,1],[-1,0]]);;
    gap> IsHandledByNiceMonomorphism( grp );
    false

    gap> IsFinite( grp );
    true
    gap> IsHandledByNiceMonomorphism( grp );
    true

    gap> hom := NiceMonomorphism( grp );
    <operation homomorphism>
    gap> hom!.externalSet; p := Image( hom );
    [ [ 1, 0 ], [ 0, 1 ], [ E(4), 0 ], [ 0, -E(4) ], [ -1, 0 ],
      [ 0, E(4) ], [ 0, -1 ], [ -E(4), 0 ] ]
    Group( [ (1,3,5,8)(2,4,7,6), (1,2,5,7)(3,6,8,4) ], ... )

    gap> ConjugacyClasses( grp ) = gcls;
    true
\endexample

Nice monomorphisms are   not only used  for matrix  groups, but  also for
other kinds of  groups  in which  one cannot calculate  easily enough. As
another   example, let  us    show that the   automorphism  group  of the
quaternionic  group of  order~8 is isomorphic  to  the symmetric group of
degree~4   by   examining  the ``nice   object''    associated with  that
automorphism group.
\beginexample
    gap> aut := AutomorphismGroup( p );; niceaut := NiceObject( aut );
    Group( [ (1,5)(3,4), (2,6)(3,4), (1,3,5,4), (1,2)(3,4)(5,6) ], ... )
    gap> IsomorphismGroups( niceaut, SymmetricGroup( 4 ) );
    [ (1,5)(3,4), (2,6)(3,4), (1,3,5,4), (1,2)(3,4)(5,6) ] -> 
    [ (1,4)(2,3), (1,2)(3,4), (1,3,4,2), (2,4) ]
\endexample

\exercise The nice monomorphism associated with the automorphism group of
`p' is an operation homomorphism. What is its underlying external set?

\answer `NiceMonomorphism( aut )!.externalSet' gives the list
\begintt|nobreak
    [ (1,2,5,7)(3,6,8,4), (1,3,5,8)(2,4,7,6), (1,4,5,6)(2,8,7,3), 
      (1,6,5,4)(2,3,7,8), (1,7,5,2)(3,4,8,6), (1,8,5,3)(2,6,7,4) ]
\endtt
which contains the six elements of order~4 in `p'. The automorphism group
must permute  these elements,  and the  action  is faithful because  they
generate~`p'.

The range  of a nice  monomorphism is in most  cases a permutation group,
because nice  monomorphisms are  mostly operation homomorphisms.  In some
cases, like  in our  last example,  the group is   solvable and you might
prefer a pc group as nice object. You cannot change the nice monomorphism
of the automorphism  group, but  you  can compose it  with an isomorphism
from the permutation  group to a pc  group to obtain your personal  nicer
monomorphism. If you  reconstruct the automorphism   group, you can  even
prescribe it this nicer monomorphism as its `NiceMonomorphism', because a
newly-constructed group will not yet have a `NiceMonomorphism' set.
\beginexample
    gap> nicer := NiceMonomorphism(aut) * IsomorphismPcGroup(niceaut);;
    gap> aut2 := GroupByGenerators( GeneratorsOfGroup( aut ) );;
    gap> SetNiceMonomorphism( aut2, nicer );
\endexample
The star `*' denotes composition of mappings  from the left to the right,
as we  have  seen in  "Operations!of  groups" above.   Reconstructing the
automorphism group may of course result in the  loss of other information
{\GAP} had already gathered, besides the (not-so-)nice monomorphism.

\exercise In  analogy to `IsomorphismPcGroup', there  is also the command
`IsomorphismPermGroup'. Continuing  the example of  this section, what is
the    difference   between      `IsomorphismPermGroup(   grp  )'     and
`NiceMonomorphism( grp )'?

\answer `IsomorphismPermGroup( grp  )'  returns a bijective   mapping, in
particular its  `Range' is a permutation  group of size~8, whereas a nice
monomorphism  which  is an operation homomorphism   has as `Range' a full
symmetric group. Also a  nice monomorphism could  be defined on a  larger
group. This is   not   the case in   the   example `grp', but   the  nice
monomorphism of a $d$-dimensional matrix group over the finite field with
$q$ elements is defined on the general linear group~$GL(d,q)$.

%%%%%%%%%%%%%%%%%%%%%%%%%%%%%%%%%%%%%%%%%%%%%%%%%%%%%%%%%%%%%%%%%%%%%%%%%%%%%
% Local Variables:
% mode:               text
% mode:               outline-minor
% outline-regexp:     "\\\\Chapter\\|\\\\Section"
% paragraph-start:    "\\\\begin\\|\\\\end\\|\\$\\$\\|.*%\\|^$"
% paragraph-separate: "\\\\begin\\|\\\\end\\|\\$\\$\\|.*%\\|^$"
% fill-column:        73
% End:
%%%%%%%%%%%%%%%%%%%%%%%%%%%%%%%%%%%%%%%%%%%%%%%%%%%%%%%%%%%%%%%%%%%%%%%%%%%%%
