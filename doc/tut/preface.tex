%%%%%%%%%%%%%%%%%%%%%%%%%%%%%%%%%%%%%%%%%%%%%%%%%%%%%%%%%%%%%%%%%%%%%%%%%
%%
%W  preface.tex               GAP documentation         Joachim Neubueser
%%
%H  @(#)$Id$
%%
%Y  Copyright (C) 1997, Lehrstuhl D fuer Mathematik, RWTH Aachen, Germany
%%
%%  This file contains the preface of the GAP manual.
%%
\Chapter{Preface}


%%%%%%%%%%%%%%%%%%%%%%%%%%%%%%%%%%%%%%%%%%%%%%%%%%%%%%%%%%%%%%%%%%%%%%%%%
\Section{From the Preface for GAP 3.4, June 1994}

{\GAP} stands for *Groups,  Algorithms  and  Programming*.  The name  was
chosen to reflect the  aim of the  system,  which is  introduced in  this
manual.

Until  well into the  eighties  the  interest  of pure mathematicians  in
computational  group theory  was  stirred  by,  but in  most  cases  also
confined to  the  information  that  was  produced  by  group theoretical
software  for  their special research  problems  --  and  hampered by the
uneasy  feeling  that  one  was   using  black  boxes  of  uncontrollable
reliability.  However the last years have seen a rapid spread of interest
in the understanding, design and even implementation of group theoretical
algorithms.  These are gradually becoming accepted both as standard tools
for a working group theoretician,  like certain  methods of proof, and as
worthwhile  objects of study, like  connections between notions expressed
in theorems.

{\GAP} was  started as  an attempt to meet  this  interest.   Therefore a
primary design goal has  been to give its user full access  to algorithms
and the data  structures used  by them, thus  allowing  critical study as
well as  modification of existing methods.  We also intend to relieve the
user from unwanted technical chores and to assist him in the programming,
thus supporting invention and implementation of new algorithms as well as
experimentation with them.

We have tried  to achieve these goals by a design which in addition makes
{\GAP} easily portable, even to computers such as Atari ST and Amiga, and
at the same  time facilitates the maintenance of {\GAP} with  the limited
resources of an academic environment.

While I had felt for some time rather strongly  the wish for such a truly
*open* system for computational group theory, the concrete idea of {\GAP}
was born when, together with a larger group of  students, among whom were
Johannes   Meier,    Werner   Nickel,   Alice     Niemeyer,   and  Martin
Sch\accent127onert who eventually wrote the first  version of {\GAP}, I
had my first contact   with the Maple system   at the EUROCAL  meeting in
Linz/Austria  in  1985.  Maple demonstrated   to us the feasibility  of a
strong  and efficient computer algebra system  built from a small kernel,
with an  interpreted library of   routines written in  a  problem-adapted
language.  The discussion of the plan of a system for computational group
theory organized  in    a similar  way  started  in  the  fall  of  1985,
programming only in the second half  of 1986.  A  first version of {\GAP}
was operational by  the end of 1986.  The  system was first  presented at
the Oberwolfach meeting    on computational group   theory  in May  1988.
Version  2.4  was  the first  officially  to  be  given  away from Aachen
starting in December 1988.  The strong interest in this version, in spite
of its  still rather small  collection of group theoretical  routines, as
well  as constructive criticism  by many colleagues, confirmed our belief
in the general design principles of the  system.  Nevertheless over three
years had passed until in April 1992  version 3.1 was released, which was
followed in February 1993 by version 3.2, in November 1993 by version 3.3
and is now in June 1994 followed by version 3.4.

A main reason for the long time between versions 2.4 and 3.1 and the fact
that there had not been  intermediate releases was that  we had found  it
advisable to make a number of changes to basic data structures until with
version 3.1 we  hoped  to have reached a   state where we could  maintain
upward compatibility over further  releases, which were planned to follow
much more frequently.  Both  goals have been  achieved over the last  two
years. Of course the time has  also been used to extend  the scope of the
methods implemented in {\GAP}.   A rough estimate   puts the size  of the
program library of version 3.4 at about sixteen times the size of that of
version 2.4, while for version 3.1 the factor  was about eight.  Compared
to {\GAP}~3.2,  which  was the  last version  with  major  additions, new
features of {\GAP}~3.4 include the following:

$\ldots$

Work on the  extension of {\GAP}  is going on in Aachen  as well as in an
increasing number of  other places. We  hope to be  able to have the next
release of {\GAP} after about  9 months again,  that is in the first half
of 1995.

The system that you are getting now consists of four parts.
\beginlist
  \item{1.}
    A comparatively small *kernel*, written in C, which provides the user
    with
    \itemitem{-}
      automatic dynamic storage management, which the user needn't bother
      about in his programming;
    \itemitem{-}
      a   set of  time-critical basic   functions, e.g.   ``arithmetic'',
      operations for integers, finite fields,  permutations and words, as
      well as natural operations for lists and records;
    \itemitem{-}
      an interpreter for the {\GAP} language, which belongs to the Pascal
      family, but, while  allowing additional types for group theoretical
      objects, does not require type declarations;
    \itemitem{-}
      a   set  of programming tools  for   testing, debugging, and timing
      algorithms.

  \item{2.}
    A   much larger *library of    {\GAP} functions* that implement group
    theoretical and other algorithms.  Since  this is written entirely in
    the  {\GAP} language, in contrast  to  the situation  in older  group
    theoretical   software, the  {\GAP}    language   is both   the  main
    implementation language   and  the  user  language of   the   system.
    Therefore  the  user can    as easily  as  the  original  programmers
    investigate  and vary algorithms of  the library and  add new ones to
    it, first for own  use and eventually for  the benefit of  all {\GAP}
    users.  We hope  that moreover the  structuring of the  library using
    the concept of  *domains* and the techniques  used for their handling
    that have   been   introduced     into   {\GAP}~3.1    by      Martin
    Sch\accent127onert will be further helpful in this respect.

  \item{3.}
    A *library of group  theoretical data* which already contains various
    libraries of groups  (cf. chapter "Group Libraries"), large libraries
    of  ordinary character tables, including all  of the Cambridge *Atlas
    of  Finite Groups* and modular tables  (cf.  chapter "Character Table
    Libraries"), and a *library  of tables of  marks*. We hope to  extend
    this  collection further  with  the   help  of colleagues  who   have
    undertaken larger classifications of groups.

  \item{4.}
    The *documentation*.  This is available as  a file that can either be
    used for on-line  help or be printed  out to form  this manual.  Some
    advice for  using this manual   may  be helpful.   The first  chapter
    *About GAP* is   really an introduction  to  the use  of the  system,
    starting from  scratch and, for the  beginning, assuming neither much
    knowledge  about group  theory   nor  much versatility  in   using  a
    computer.   Some  of  the later sections  of  chapter  1 assume more,
    however.  For   instance section `About Character  Tables' definitely
    assumes  familiarity  with representation   theory of  finite groups,
    while in particular sections `About the Implementation of Domains' to
    `About  Defining New Group Elements'  address more advanced users who
    want to extend  the system to meet their  special needs.  The further
    chapters of the manual give then a  full description of the functions
    presently available in {\GAP}.
\endlist

Together with the  system we distribute *GAP  share libraries*, which are
separate packages which have been written by various groups of people and
remain   under their responsibility.  Some  of these packages are written
completely in the  {\GAP} language, others totally or  in parts in C  (or
even  other languages). However  the  functions in these  packages can be
called  directly from   {\GAP}  and results are   returned  to {\GAP}. At
present   there   are  10  such  share    libraries  (cf. chapter  "Share
Libraries").

The policy for the further development of {\GAP} is to keep the kernel as
small  as possible,  extending  the set  of basic functions  only by very
selected   ones that  have  proved  to   be  time-critical and,  wherever
feasible,  of  general use.    In  the interest  of the    possibility of
exchanging functions written in the {\GAP} language the  kernel has to be
maintained in  a  single place  which in  the foreseeable  future will be
Aachen.  On the other hand we hoped from the beginning that the design of
{\GAP} would  allow the library  of {\GAP} functions   and the library of
data to grow not only by continued work in Aachen  but, as does any other
part of mathematics, by contributions  from  many sides, and these  hopes
have been fulfilled very well.

There are some other points to make on further policy.

\beginlist
  \item{-}
    When we began work on {\GAP} the typical user that we had in mind was
    the  one wanting to  implement his own  algorithmic  ideas.  While we
    certainly  hope  that we still serve  such  users well it  has become
    clear from the experience of the last  years that there are even more
    users  of  two different species,  on the  one   hand the established
    theorist,  sometimes with little experience  in the use of computers,
    who wants   an  easily understandable tool, on   the   other hand the
    student,   often quite  familiar with  computers,   who wants to  get
    assistance  in learning the theory   by  being able to do  nontrivial
    examples.  We think that in fact {\GAP} can well be used by both, but
    we realize  that for each a  special introduction would be desirable.
    We apologize that we have not had the time yet to write such, however
    have learned (through the  {\GAP} forum) that in  a couple  of places
    work on the  development of Laboratory Manuals for  the use of {\GAP}
    alongside with standard Algebra texts is undertaken.

  \item{-}
    When we began  work on {\GAP}, we designed  it as a system for  doing
    *group theory*.  It has already turned out that in fact the design of
    the system is general  enough,  and some  of  its functions are  also
    useful,  for  doing work in  other  neighbouring areas.  For instance
    Leonard  Soicher has used {\GAP} to  develop a system {\sf GRAPE} for
    working with graphs, which meanwhile is available as a share library.
    We certainly enjoy seeing this happen, but we  want to emphasize that
    in Aachen our  primary interest is the development  of a group theory
    system and that    we do not plan  to   try to extend it  beyond  our
    abilities into a general computer algebra system.

  \item{-}
    Rather  we hope to provide tools  for linking {\GAP} to other systems
    that  represent years  of  work  and  experience   in areas  such  as
    commutative algebra, or to very efficient special purpose stand-alone
    programs.  A link of this kind exists e.g.  to the MOC system for the
    work with modular characters.

  \item{-}
    We invite you  to further extend   {\GAP}.  We are willing either  to
    include such extensions into {\GAP} or to make them available through
    the same channels as {\GAP} in the form of the above mentioned *share
    libraries*.  Of course, we will do this  only if the extension can be
    distributed free of charge like {\GAP}.  The copyright for such share
    libraries shall remain with you.

  \item{-}
    Finally to answer an often asked question: The  {\GAP} language is in
    principle  designed to be compilable.  Work  on a compiler  is on the
    way, but this is not yet ready for inclusion with this release.
\endlist

{\GAP} is given  away under the  conditions that have always  been in use
between  mathematicians, i.e.  in particular *completely  in source*  and
*free  of  charge*.  We  hope  that  the  possibility  offered  by modern
technology of  depositing {\GAP} on a number of  computers  to be fetched
from them by `ftp', will assist us in this policy.  We want to emphasize,
however, two points.  {\GAP} is  *not* public domain software; we want to
maintain  a *copyright* that  in particular forbids  commercialization of
{\GAP}.  Further we ask that use of {\GAP} be quoted in publications like
the use of any  other mathematical work, and  we would be grateful if  we
could keep track of where {\GAP} is implemented.  Therefore we ask you to
notify us if you have got {\GAP}, e.g., by sending a short e-mail message
$\ldots$ .
The simple reason,  on top of our
curiosity, is that  as anybody  else in  an academic  environment we have
from time to time to prove that we are doing meaningful work.

We  have established a {\GAP} forum, where interested  users  can discuss
{\GAP}  related  topics  by  e-mail.  In particular  this  forum  is  for
questions about  {\GAP}, general  comments, bug  reports,  and  maybe bug
fixes.   We will read this forum and answer questions  and  comments, and
distribute  bug  fixes.  Of course  others  are  also  invited to  answer
questions, etc.  We will  also announce future releases of {\GAP} in this
forum.

$\ldots$

The reliability of  large systems  of  computer programs is  a well known
general problem and, although over the past year the  record of {\GAP} in
this respect has not been too  bad, of  course  {\GAP} is not exempt from
this problem.  We therefore feel that it is mandatory  that  we, but also
other users, are warned of bugs that  have been encountered in  {\GAP} or
when doubts have arisen.  We  ask all users of {\GAP} to  use  the {\GAP}
forum for issuing such warnings.

We   have  also established  an e-mail    address  `gap-trouble' to which
technical  problems   of a  more local   character  such as  installation
problems can be sent. Together  with some experienced {\GAP} users abroad
we try to give advice on such problems.

{\GAP} was started as a joint Diplom project of four students whose names
have  already  been  mentioned.   Since then many   more finished  Diplom
projects have contributed to {\GAP} as well as other members of Lehrstuhl
D  and colleagues from other  institutes.  Their individual contributions
to the programs and to the manual are documented in the respective files.
To all of   them as well    as to all who  have   helped proofreading and
improving this manual  I want to express  my thanks for their  engagement
and enthusiasm as well as to many users  of {\GAP} who  have helped us by
pointing out   deficiencies and  suggesting improvements.   Very  special
thanks however go to  Martin Sch\accent127onert.  Not only  does {\GAP}
owe many  of  its  basic design  features  to his  profound  knowledge of
computer languages  and the techniques for  their  implementation, but in
many long discussions he has in the name of  future users always been the
strongest defender of clarity of the design against my impatience and the
temptation for ``quick and dirty'', solutions.

Since  1992 the development of  {\GAP}  has been financially supported by
the Deutsche     Forschungsgemeinschaft    in  the   context      of  the
Forschungsschwerpunkt  ``Algorithmische Zahlentheorie   und  Algebra''.
This very important help is gratefully acknowledged.

As with the previous versions we send this version out hoping for further
feedback of constructive   criticism.  Of course  we ask  to be  notified
about bugs,  but moreover  we shall  appreciate   any suggestion  for the
improvement of the  basic  system as  well  as of  the algorithms  in the
library.  Most of all,  however, we hope that in  spite of such criticism
you will enjoy working with {\GAP}.

Aachen, June 1.,1994, \hfill Joachim Neub\accent127user.

%%%%%%%%%%%%%%%%%%%%%%%%%%%%%%%%%%%%%%%%%%%%%%%%%%%%%%%%%%%%%%%%%%%%%%%%%
\Section{Preface for the first beta release of GAP 4}

The transition from {\GAP}~3.4 which got its presumably last update 3.4.4
in April  this year to this  present first beta  release {\GAP}~4.B.1  of
{\GAP}~4  marks a  major step in the system  design of {\GAP}, similar in
importance to the  step from {\GAP}~2.4 to   {\GAP}~3.1 in April 1992  on
which I comment in my preface  to {\GAP}~3.4 of  June 1994, most of which
is preceding this  preface. However in  contrast to the situation in 1992
we hope that the  changes will be much less  bothering to the majority of
the  {\GAP} users this time.   Let me first talk   about some reasons and
background  for developing {\GAP}~4  and then briefly sketch what remains
and what changes.

The  planning of {\GAP}~4 started  already at the time  of the release of
{\GAP}~3.4 (Summer 1994) and its development has been  a major reason for
the fact that  since then  only updates  (up to {\GAP}~3.4.4)  but no new
releases of {\GAP}~3 have come out. Also a number  of new algorithms have
been implemented   in  Aachen anticipating {\GAP}~4   and  hence have not
become generally available yet.

There were three major reasons for the development of {\GAP}~4:

\beginlist
  \item{-}
    There has  been a  growing   demand  to implement  new   mathematical
    structures in  {\GAP} (Lie algebras are just  one  example).  However
    {\GAP}~3 was not really designed for such tasks.

  \item{-}
    The number and diversity of (sometimes competing) algorithmic methods
    is  growing rapidly.   We  definitely want  to maintain the principle
    that the  user  should  be  able to control   what  methods are used.
    However, the growing complexity  of  the interrelation of  algorithms
    makes it mandatory to have  also a 'method selection' mechanism which
    controls the choice between different possibilities to proceed within
    a  {\GAP} function,  that  is  at least  partially  guided by already
    computed knowledge about the objects under investigation.

  \item{-}
    While the two  points mentioned above  have  caused 'visible' changes
    from {\GAP}~3 to {\GAP}~4, in  this transition also important changes
    have taken place 'behind the scene'.  There are e.g.  improvements of
    the  storage  management and function  calls,   and last  not least a
    compiler from {\GAP}~4 to C is part of this beta release.
\endlist

So regarding system aspects let's briefly sketch:

What is left unchanged?

\beginlist
  \item{-}
    The syntax of that part  of the {\GAP} language  that most users need
    investigating mathematical problems.

  \item{-}
    The great majority of function names.

  \item{-}
    Data libraries and the access to them.
\enditems

What has changed? 

\beginlist
  \item{-}
    Some function names that need finer specifications now that there are
    more structures available in {\GAP}.

  \item{-}
    The access to  information   already obtained about  a   mathematical
    structure.  E.g.  in {\GAP}~3 such information about a group could be
    looked up by directly inspecting the  group record, whereas in {\GAP}
    4 functions must be used to access such information.
\endlist

What is new?

\beginlist
  \item{-}
    A whole machinery for the definition of new structures. 

  \item{-}
    A  hopefully  clearer separation of  aspects  of knowledge  about the
    mathematical objects that {\GAP} handles  by the introduction of  the
    concepts of attributes, families, categories, and representations.

  \item{-}
    A number of new structures, such as  Lie algebras.
\enditems

Then to the mathematical functionality provided by {\GAP}~4 in comparison
to {\GAP}~3:

\beginlist
  \item{-}
    Almost all of the program library of {\GAP}~3 has been transferred to
    {\GAP}~4.   Some of this had  just to be adapted to  the new features
    (which in  itself has been  a huge  job in view   of the size  of the
    program library), but for quite a  few tasks the opportunity has been
    used to implement new and more efficient algorithms - notably so e.g.
    for permutation groups and polycyclic groups.

  \item{-}
    A number    of new algorithms  have  been  implemented for  which the
    features of {\GAP}~4 proved more adequate or even necessary and which
    are now made public in {\GAP} for the first time.

  \item{-}
    One main deficiency of the present beta release is that the meanwhile
    large   library of share  packages  of   {\GAP}~3  has not  yet  been
    transferred.
\endlist

The other main deficiency is that there is not yet  a complete manual for
{\GAP}~4.

It is intended to provide eventually at least four  books as parts of the
manual. The  first and second  are intended for  people  who want  to use
{\GAP} ``as is''.  Books 3 and 4  on the other hand  are meant for people
who want to extend {\GAP}~4 by introducing new structures.  Books 1 and 3
are  tutorials for the respective   purpose while Books  2 and  4 are the
corresponding reference manuals.

Of these four books a  good deal  of Books 1   and 3, i.e.  the  tutorial
parts, are provided with  this release, while  there are only rudimentary
parts  of books  2  and 4  available.   For people  already familiar with
{\GAP}~3 the chapter ``Migrating  to {\GAP}~4'' in the  first book may be
particularly helpful.

There have been (weekly  changing) alpha test versions  of {\GAP}~4 since
December  1996, and a number   of specially experienced  {\GAP} users  in
addition to  the {\GAP} teams at Aachen  and St.  Andrews have used these
and   provided helpful criticism and  suggestions.   It is envisaged that
there will be new beta releases from now on about  every couple of months
until an  official version  {\GAP}~4.1 can be released  next year.  It is
hoped that these further beta releases will gradually provide the missing
parts mentioned above as well as further enhancements. However, since the
date of the official  handover of {\GAP} from  Aachen to St.  Andrews has
now been fixed to be July 21, 1997, this  further development will happen
under the responsibility of St. Andrews.

It remains to me to thank all those who have done the huge amount of work
that was needed to  bring {\GAP}~4 on its way.   Many basic ideas for the
new concepts as well as  most of the  new kernel implementation are still
due to Martin  Sch\accent127onert before and  even in parts after he left
Lehrstuhl D  fuer Mathematik.  However  together with   him while he  was
still working here and continuing after he left,  Thomas Breuer and Frank
Celler have in  long discussions found the  way to the  concepts and done
crucial  parts  of  the new  implementations.   Many   others have worked
adapting and rewriting    the library, of   whom I  want   to mention  in
particular  Bettina Eick, Alexander Hulpke   and Heiko Theissen from  the
Aachen team  but also acknowledge the  help lended already for  some time
from St.  Andrews, in particular by Steve Linton.

To these and all others,  whom I did  not  mention explicitly, I want  to
express my  thanks for a yearlong  cooperation in a spirit of enthusiasm,
dedication and perseverance.     I wish   the team  at    St.  Andrews  a
successful continuation  of the development and  maintenance of {\GAP} in
that same spirit and all users fun and success in using {\GAP}.

Aachen, July 18, 1997 \hfill Joachim Neub\accent127user

%%%%%%%%%%%%%%%%%%%%%%%%%%%%%%%%%%%%%%%%%%%%%%%%%%%%%%%%%%%%%%%%%%%%%%%%%
\Section{Further Information about GAP}

In  the preface for {\GAP}~3.4 we  have i. a.   omitted information about
availability of {\GAP}. In fact  this and further information is nowadays
best obtained from the {\GAP} Web pages that you find on:

\begintt
    http://www-groups.dcs.st-and.ac.uk/~gap/
\endtt

and its mirrors at:

\begintt
    http://www.math.rwth-aachen.de/LDFM/GAP/
    http://www.ccs.neu.edu/mirrors/GAP/
    http://wwwmaths.anu.edu.au/research.groups/algebra/GAP/www/
\endtt

%%%%%%%%%%%%%%%%%%%%%%%%%%%%%%%%%%%%%%%%%%%%%%%%%%%%%%%%%%%%%%%%%%%%%%%%%
%%
%E  preface.tex . . . . . . . . . . . . . . . . . . . . . . . . ends here
