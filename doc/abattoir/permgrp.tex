%%%%%%%%%%%%%%%%%%%%%%%%%%%%%%%%%%%%%%%%%%%%%%%%%%%%%%%%%%%%%%%%%%%%%%%%%
\Chapter{Permutations and Permutation Groups}

\def\labelto#1{\buildrel\lower0.8ex\hbox{$\scriptstyle #1$}\over\to}%

%%%%%%%%%%%%%%%%%%%%%%%%%%%%%%%%%%%%%%%%%%%%%%%%%%%%%%%%%%%%%%%%%%%%%%%%%
\Section{Permutations}

Before we come to one of the basic types of  groups {\GAP} can deal with,
namely permutation  groups,  we will   dedicate  a few  sections  to  the
elements these groups consist  of,  namely permutations. The reason   for
defering the treatment  of permutations groups is that  you can work with
permutations without defining  a permutation  group that contains   them.
This is just like it is with integers, with which you can compute without
caring about the domain `Integers' that contains them. It also means that
you can multiply any two permutations.

The points on  which permutations   in  {\GAP} operate  are the  positive
integers less than  $2^{28}-1$,  and the image   of a point <i> under   a
permutation <p> is written $i^p$,  which is expressed  as `<i> ^ <p>'  in
{\GAP}. If `<i> ^ $p\ne i$', we say that <i> is *moved* by~<p>, otherwise
it is _fixed_. Permutations in {\GAP} are  entered and displayed in cycle
notation, such as `(1,2,3)(4,5)'.

In the  names of the  {\GAP} functions  that deal with  permutations, the
word  `Permutation' is  always abbreviated  to `Perm', to  save  you some
typing. For  example,   the category  test  function  for permutations is
called `IsPerm'.

Internally, {\GAP}  stores a permutation as a  list of the  <d> images of
the  integers  $1,\ldots, d$,  where the ``internal  degree'' <d>  is the
largest integer moved by the permutation or bigger. When a permutation is
read  in  in  cycle  notation, <d> is  always  set  to  the largest moved
integer,   but a bigger   <d> can  result  from  a multiplication of  two
permutations, because the product is  not shortened if it fixes~<d>.  The
images are either all stored as 16-bit integers or all as 32-bit integers
(actually as {\GAP} immediate integers less  than $2^{28}$), depending on
whether  $d\le 65536$  or not. This  means that  the identity permutation
`()'  takes <4m> bytes  if it was calculated as   `(1, \dots, <m>) \* (1,
\dots, <m>) ^ -1'. It  can take even more  because the internal list  has
sometimes room for more than <d> images.  For example, the maximal degree
of   any permutation in  {\GAP}  is  $m  = 2^{22}-1024 =  4{,}193{,}280$,
because  bigger permutations  would have  an  internal list with room for
more than $2^{22}$ images, requiring  more than $2^{24}$~bytes. $2^{24}$,
however, is  the  largest possible size   of  an object that  the  {\GAP}
storage manager can deal with.

%%%%%%%%%%%%%%%%%%%%%%%%%%%%%%%%%%%%%%%%%%%%%%%%%%%%%%%%%%%%%%%%%%%%%%%%%
\Section{Comparison!of permutations}

\>`<p_1> = <p_2>'{equality test}!{for permutations}
\>`<p_1> \< <p_2>'{precedence test}!{for permutations}

Two permutations $p_1$  and  $p_2$ of internal degrees  $d_1$ resp.~$d_2$
are equal   if $i^{p_1} =  i^{p_2}$   for all $i\in\{1,\ldots,\allowbreak
\max(d_1,d_2)\}$. $p_1\<p_2$ if $i^{p_1}  \<  i^{p_2}$ where <i> is   the
first integer with $i^{p_1} \ne i^{p_2}$.  This implies that the identity
permutation is the least permutation.

\>SmallestGeneratorPerm( <p> )

`SmallestGeneratorPerm' returns  the smallest  permutation that generates
the  same cyclic group  as the permutation   <p>. This is very efficient,
even when <p> has huge order.

%%%%%%%%%%%%%%%%%%%%%%%%%%%%%%%%%%%%%%%%%%%%%%%%%%%%%%%%%%%%%%%%%%%%%%%%%
\Section{Moved points!of permutations}

\>SmallestMovedPointPerm( <perm> )
\>SmallestMovedPoint( <G> )
\>LargestMovedPointPerm( <perm> )
\>LargestMovedPoint( <G> )
\>MovedPointsPerm( <perm> )
\>MovedPoints( <G> )
\>NrMovedPointsPerm( <perm> )
\>NrMovedPoints( <G> )

`MovedPointsPerm(   <perm> )'  returns the   set of points  moved by  the
permutation <perm>,  i.e., the set   of positive integers  <i> such  that
`$i^{perm}\ne i$'.  `NrMovedPointsPerm'  returns the cardinality  of that
set, `SmallestMovedPointPerm'    and `LargestMovedPointPerm' its smallest
resp.\ largest  element. If <perm> is trivial,  the set will be empty, in
this case  the last two functions  return `infinity' resp.~`0'. The other
functions do the corresponding computations for permutation groups.

%%%%%%%%%%%%%%%%%%%%%%%%%%%%%%%%%%%%%%%%%%%%%%%%%%%%%%%%%%%%%%%%%%%%%%%%%
\Section{Sign homomorphism}\null

\>SignPerm( <perm> )

This function calculates  the image of  the permutation <perm> under  the
sign homomorphism as  $\prod_{c} (-1)^{\ell(c)-1}$ where <c>  ranges over
the  cycles  in the cycle  notation of  <perm> and $\ell(c)$  denotes the
number of points in the cycle~<c>.

%%%%%%%%%%%%%%%%%%%%%%%%%%%%%%%%%%%%%%%%%%%%%%%%%%%%%%%%%%%%%%%%%%%%%%%%%
\Section{Permutations and lists}\null

\>ListPerm( <perm> )
\>ListPerm( <perm>, <length> )

`ListPerm' returns a list <list> that contains the images of the positive
integers  under the permutation   <perm>. That means that  `<list>[<i>] =
<i>^<perm>',  where <i>  lies between 1   and the largest  point moved by
<perm> (see "LargestMovedPointPerm").

An optional second argument specifies the length of the desired list.

\>PermList( <list> )

`PermList' returns the permutation <perm>  that moves points as describes
by the list <list>.  That means that  `<i>^<perm>  = <list>[<i>]' if  <i>
lies between 1 and the length of <list>, and `<i>^<perm> = <i>' if <i> is
larger than  the length of  the list <list>. It will  signal an  error if
<list> does  not define a permutation,  i.e., if <list> is  not a list of
integers  without holes, or  if <list> contains  an  integer twice, or if
<list> contains an integer not in the range `[1..Length(<list>)]'.

\>MappingPermListList( <list1>, <list2> )

`MappingPermListList'    returns   a   permutation    <perm>  such   that
`<list1>[<i>]  ^ <perm> = <list2>[<i>]'.  <perm> fixes  all points larger
then the maximum  of the  entries in <list1>   and <list2>. If  there are
several     such    permutations,  it      is   not     specified   which
`MappingPermListList' returns.   <list1> and  <list2>  must  be  lists of
positive integers of the same length, and neither  may contain an element
twice.

\>RestrictedPerm( <perm>, <list> )

`RestrictedPerm' returns  the new permutation <new>  that operates on the
points in the list <list> in the same  way as the permutation <perm>, and
that fixes those points that are not in <list>.  <list> must be a list of
positive integers  such that for each <i>  in <list> the image $i^{perm}$
is also in <list>, i.e., it must be the union of cycles of <perm>.

\>Permuted( <list>, <perm> )

`Permuted' returns  a new list  which contains in position $i^{perm}$ the
element    $list[i]$.  In other words,       it returns a  (shallow,  see
"ShallowCopy") copy of <list>, permuted by <perm>.

%%%%%%%%%%%%%%%%%%%%%%%%%%%%%%%%%%%%%%%%%%%%%%%%%%%%%%%%%%%%%%%%%%%%%%%%%
\Section{Permutation groups}

A permutation group is a  group of permutations on  a finite set $\Omega$
of  positive integers. {\GAP} does *not*  require the user to specify the
operation domain  $\Omega$ when a permutation  group is defined. Instead,
there  are   some functions  that  determine  a   set  on  which a  given
permutation group acts, although they  might not always guess the  domain
the user had  in mind. These functions  were already described  in "Moved
points!of  permutations"   after  this introduction.  Then  come  several
sections  that describe the concept of  stabilizer chains, which are used
by most functions for permutation groups (see "Stabilizer chains").

Then  come functions that     perform  group-theoretic calculations    in
permutation groups, using  stabilizer chains and  the so-called partition
backtrack  method (see "Backtrack  searching"). These functions partially
overlay the generic group functions that  are of course all applicable to
permutation groups (see "Group  functions!for permutation  groups"). More
efficient functions that   avoid backtracking are available for  solvable
permutation groups (see   "Group   functions!for   solvable   permutation
groups").

Finally  each permutation    group  operates naturally    on the positive
integers,  so all  operations  functions can  be applied (see  "Operation
functions!for  groups" and "Operation functions!for permutation groups").
These operations  also give  rise to  special kinds  of  homomorphisms of
permutation groups (see "Homomorphisms!for permutation groups").

Remember that     in   the  names   of  {\GAP}    functions,  the  phrase
`PermutationGroup' is always abbreviated to `PermGroup'. For example, the
category test function is called `IsPermGroup'.

%%%%%%%%%%%%%%%%%%%%%%%%%%%%%%%%%%%%%%%%%%%%%%%%%%%%%%%%%%%%%%%%%%%%%%%%%
\Section{Group constructor functions!for permutation groups}

\>DirectProduct( ... )!{for permutation groups}

\>WreathProduct( <G>, <H> )!{for permutation groups}
\>WreathProductProductAction( <G>, <H> )

These functions  take  a permutation group~<G>  on $\Omega$,  a group~<H>
acting on $\Delta$  (via  a homomorphism $\alpha\colon  <H>\to S_\Delta$)
and   construct the wreath product   $<G>\wr_\alpha <H>$ as a permutation
group
\beginlist
  \item{$\bullet$} acting   on `[ 1 ..   $|\Omega\times\Delta|$ ]' in the
    imprimitive action in the case of `WreathProduct',
  \item{$\bullet$} acting on `[ 1  .. $|\Omega^\Delta|$ ]' in the product
    action in the case of `WreathProductProductAction'.
\endlist
Both  arguments <G>  and <H>  may be  permutation groups or homomorphisms
with permutation groups as range. If <G> is given as a group, $\Omega$ is
set  to `MovedPoints(  <G> )',  but if <G>   is given as a homomorphisms,
$\Omega$ is set to  `MovedPoints( Range( <G> )  )', whereas <G> will then
be replaced by  `Image( <G> )'.  (This allows $\Omega$  to include points
not moved  by <G>,  e.g.,  in  the case  of `OperationHomomorphism(  <G>,
$\Omega$ )', whose `Range' is $S_\Omega$.) Similarly,  if <H> is given as
a group, `$\alpha$ = IdentityMapping( <H> )', otherwise `$\alpha$ = <H>';
and again `$\Delta$ =  MovedPoints( Range( $\alpha$  ) )' and <H> will be
set to `Image( $\alpha$ )'.
\beginexample
    gap> s4 := Group((2,3,4,5),(2,3));;
    gap> WreathProduct(s4,s4);
    Group( [ (1,2,3,4), (1,2), (5,6,7,8), (5,6), ( 9,10,11,12), ( 9,10), 
      (13,14,15,16), (13,14), ( 1, 5, 9,13)( 2, 6,10,14)( 3, 7,11,15)
        ( 4, 8,12,16), ( 1, 5)( 2, 6)( 3, 7)( 4, 8) ], ... )  # on [1..16]
|_
    gap> w := WreathProduct( OperationHomomorphism( s4, [1..5] ), s4 );;
    gap> Orbits( w, MovedPoints(w) );
    [ [ 1, 6, 11, 16 ], 
      [ 2, 3, 7, 4, 8, 12, 5, 9, 13, 17, 10, 14, 18, 15, 19, 20 ] ]
|_
    gap> WreathProduct( s4,
    >        OperationHomomorphism( TrivialSubgroup(s4), [2..5] ) );
    Group( [ (1,2,3,4), (1,2), (5,6,7,8), (5,6), ( 9,10,11,12), ( 9,10), 
      (13,14,15,16), (13,14) ], ... )  # the base group |$S|sb4|sp4|$
\endexample
There is one exception to the `MovedPoints' rule: If the top group <H> is
the  trivial group, e.g.,  constructed  as `SymmetricGroup( 1  )',  it is
considered to be of degree~1, because otherwise  the wreath product would
be trivial.

%%%%%%%%%%%%%%%%%%%%%%%%%%%%%%%%%%%%%%%%%%%%%%%%%%%%%%%%%%%%%%%%%%%%%%%%%
\Section{Stabilizer chains}

The more complex algorithms  for permutation groups that  are implemented
in  {\GAP} rely on the concept  of base and  stabilizer  chain, which was
introduced by C.~Sims in~\cite{Sims70}. If   $B=(b_1, \ldots, b_n)$ is  a
base    for   <G>,   we    write    $G^{(1)}=G$   and     $G^{(i+1)}={\rm
Stab}_{G^{(i)}}(b_i)$   for $i>0$  for the   members   of the  associated
stabilizer chain.

\>Base( <G> )!{for permutation groups}

`Base( <G> )' returns a base for the permutation  group <G>. To compute a
base is  actually equivalent to computing a  stabilizer chain (as we will
see below), and if this   has already been   done for <G>, the result   a
simply  returned. Otherwise  a  shortcut is tried  that works   if <G> is
solvable (see "Group  functions!for solvable permutation groups"). If <G>
is  not  solvable, a  reduced stabilizer  chain for the lexicographically
smallest base is computed as follows.

\danger The structure  of  <G> is sufficiently described  by transversals
$R_i$ for the cosets of $G^{(i+1)}$ in $G^{(i)}$ (for $i=1,\ldots,n$). By
the  *orbit-stabilizer theorem*,  any such $R_i$  is  in bijection to the
*basic orbit*  $O_i =  b_iG^{(i)}$, with  $R_i[p]\in G^{(i)}$  an element
taking $b_i$ to $p\in O_i$. Then the order  of <G> is  the product of the
indices  $[G^{(i)}:G^{(i+1)}]$,  i.e.,  the product   of the  transversal
lengths $|R_i|$. If <g> is a permutation mapping $b_i$ to $c_i$, then
$$
  \vbox{\halign{\hfil$#$& $\iff #$\hfil & \quad #\hfil \cr
  g\in G & gr_1^{-1}\in G^{(2)} & (with $r_1=R_1[c_1]$) \cr
         & gr_1^{-1}r_2^{-1}\in G^{(3)} &
                                  (with $r_2=R_1[c_2r_1^{-1}]$) \cr
  \ldots & gr_1^{-1}\ldots r_n^{-1}\in G^{(n+1)}=\{{\rm id}\} &
           (with $r_n=R_n[c_nr_1^{-1}\ldots r_{n-1}^{-1}]$).\cr}}$$
This   method  of  dividing  <g> by  transversal  elements   is known  as
_reduction of a permutation   along a stabilizer  chain_\index{stabilizer
chain!reduction of permutation along}, and it gives a membership test for
whether  $g\in G$.   Moreover,  each  element    of  <G> has   a   unique
representation  $r_n\ldots  r_1$ with  $r_i\in  R_i$, and if $g\notin G$,
then the remainder $gr_1^{-1}\ldots r_n^{-1}$ is also uniquely determined
by the stabilizer chain.

In  order  to save   memory,  it is  advisable   to store  not the  whole
permutation $R_i[p]$ for each orbit point <p>, but instead only a pointer
to one of the generators of $G^{(i)}$, say <g>, where <g> takes <p> to an
earlier point in  the orbit $O_i$ (note that  we consider $O_i$ as a list
not as  a set). If we assume  inductively that we know  an element $r \in
G^{(i)}$ that  takes $b_i$  to  <pg>,  then  $rg^{-1}$  is an element  in
$G^{(i)}$ that takes $b_i$ to <p>. {\GAP} orders the  orbit $O_i$ in such
a way that the base point $b_i$ is first. By  tracing back a point <p> to
the base  point and multiplying the  the generators  that we encounter on
this way, we thus find the inverse of the transversal element $R_i[p]$. A
transversal stored in this  way is called  a *factorized inverse Schreier
transversal*.

The generators that  must be assigned  to every  point in the  factorized
transversal are  automatically  found  during the   orbit algorithm  (see
"Orbit"), because every new point (that will be placed later in $O_i$) is
found as the preimage of an earlier point under  one of the generators. \
(The orbit algorithm is carried out with the inverses of the generators.)
\ In this way we obtain  a tree-like structure in  which the vertices are
the  orbit points and where  there  is an edge  descending  from an orbit
point <p> to  $q=pg^{-1}$ for the inverse of  some generator <g>. Such an
edge is labelled with   the corresponding factorized inverse  transversal
element, i.e., with <g> and we will occasionally write $p\labelto g q$ to
denote this labelled edge. The root of the  tree is the base point $b_i$.
This tree is called the *Schreier tree* of the orbit.

So far we understand how the orbit and the transversal are calculated. In
order  to  complete  our  (rough) description  of  the  calculation  of a
stabilizer chain, we only need to know how  we can find the stabilizer of
the current base point, because we can then  iterate the whole process if
the  stabilizer is not yet trivial.  We make use  of the subgroup theorem
due to Schreier which states in our case that  the stabilizer of $b_i$ is
generated by the set $\{  r_mg_n.  R_i[b_i{r_mg_n}]^{-1} \}$ where  $g_n$
runs over generator list of $G^{(i)}$ and $r_m$ runs over the transversal
$R_i$.  Using   the combined efforts of   Sims  and Schreier we  have now
described the so-called *Schreier-Sims algorithm* which {\GAP} applies to
calculate stabilizer chains. Observe  that the Schreier subgroup  theorem
together with the reduction   process mentioned above implies that  every
element in <G> has a canonical representation $w(g_1,  \ldots, g_n)$ as a
word in the  generators $g_j$ of <G>.  We will use  this fact when we are
dealing    with          permutation    group       homomorphisms    (see
"GroupGeneralMappingByImages!for permutation groups").

\>StabChain( <G> \[, <options> \| <list> \] )

`StabChain( <G> )' performs  the same calculations  as `Base( <G> )', but
returns the whole stabilizer  chain, not just  the base.
If no stabilizer chain has been entered into <G> so far, the result is
stored as value of the attribute `StabChainImmutable'.
See "ext:Stabilizer chain records" in the extension manual for detailed
information on how stabilizer chains look like.
Note that the base can be reconstructed from the stabilizer chain by
looping over the Schreier trees and collecting the root points.
This is what the function `Base' does when a stabilizer chain has already
been calculated.

A second argument <options>  may be given to `StabChain',   it must be  a
record that may contain the components listed  below. If they are absent,
in  <options>, but present in the   record `StabChainOptions( <G> )', the
values  there are used.  Also some components  have default values, which
you   can  inspect    and       change   in   the    global      variable
`DefaultStabChainOptions'. If there is already a  stabilizer chain in the
group record  of <G>, it is neither  overwritten nor modified, even  if a
different chain is returned by `StabChain' (e.g., with a different base).
Also, if there is already a stabilizer chain  in the group record of <G>,
`StabChain( <G> )'  without <options> simply  returns it, even if it does
not meet the specifications   of the defaults; whereas `StabChain(   <G>,
rec() )'  would respect  the  defaults  (and, e.g.,  return the   reduced
version of the present chain, if it is not already reduced). Here are the
available options:
\beginitems
`base' &
        If this is  present, it must be  a list of  points, through which
        the resulting stabilizer   chain  will run. This  means  that the
        sequence  of base points will start  off with `base' if the chain
        is not  reduced; if it is  reduced,  however, points  from `base'
        that would lead to trivial basic orbits will be skipped. Repeated
        occurrences  of  points in `base'  are  ignored.  If a stabilizer
        chain for <G> is already known a base change is performed instead
        as described  in~\cite{Butler82b}. If `base'   is the only option
        you want to  pass to `StabChain', you  can type  `StabChain( <G>,
        <list> )' instead of `StabChain( <G>, rec( base := <list> ) )'.

`knownBase' &
        If a base for <G> is known in advance, it can be  given as a list
        in the component  `knownBase'. A  known  base makes it easier  to
        test whether a   product   of certain permutations  (namely,    a
        Schreier generator) is the identity, because  it is enough to map
        the known base  with   each  factor consecutively, rather    than
        multiplying the   whole permutations (which means   mapping every
        point). This trick will speed up the Schreier-Sims algorithm if a
        new  stabilizer chain has to  be constructed (it  will not affect
        the  base change, however). This   component bears no relation to
        the  `base' component, you  may specify a  known base `knownBase'
        and a desired base `base' independently.

`reduced' (default `true') &
        \index{stabilizer chain!reduced}If this  is `true' the  resulting
        stabilizer chain   will  be reduced, i.e.,   the case  $G^{(i)} =
        G^{(i+1)}$  will  not occur. To set   `reduced'  to `false' makes
        sense only  if  `base' is also  set;  in this case  all points of
        `base' will occur  in the base,  even if they have  trivial basic
        orbits. However, if `base' is  not a complete base  but has to be
        prolonged, the prolongation will be reduced.

`tryPcgs' (default `true') &
        If this is `true', {\GAP} will first try to  construct a pcgs for
        <G> with  the method described  in "Pcgs!for solvable permutation
        groups".  If this is  successful, i.e.,  if  <G> is  solvable, it
        automatically yields a stabilizer chain for <G>  and this will be
        returned. If <G> turns out non-solvable, one of the other methods
        will be used. This solvability  check is comparatively fast, even
        if the result is ``non-solvable'', and it  can save a lot of time
        if <G> is solvable.

`random' (default `1000') &
        If this  is  less than~1000,  the  computation of the  stabilizer
        chain  will  be   done  non-deterministically, as  described   in
        \cite{Seress97}. The  probability  that the  resulting  chain  is
        correct will be at least~$`random'/1000$.

`size' (default `Size( <G> )' if this is present, i.e., if `HasSize( <G>
)')&
        If this is present, it is assumed to be the size of the group <G>
        This      information    can  be    used    to   prove    that  a
        non-deterministically constructed stabilizer chain is correct. In
        this case, {\GAP} does a non-deterministic construction until the
        size is correct.

`limit' (default `Size( Parent( <G> ) )' or `StabChainOptions( Parent(
<G> ) ).limit' if this is present) &
        If this is present, it must be greater than or  equal to the size
        of~<G>.   {\GAP}   then    persues   the   usual   strategy   for
        non-deterministic construction,  but finishes if the size reaches
        `limit'.
\enditems

\>StabChainBaseStrongGenerators( <base>, <sgs> )

If a base <base> for a group and a strong generating  set with respect to
<base> are given, this function constructs a stabilizer chain without the
need to find Schreier  generators; so this  is much faster than the other
algorithms.

\>StrongGenerators( <G> )

This function  returns a  list of  strong generators for  the permutation
group  <G>.    If a    stabilizer chain   for    <G>  is  already  known,
`StrongGenerators' returns a strong generating  set corresponding to this
stabilizer chain. Otherwise a stabilizer chain with the lexicographically
smallest   reduced base   is    computed and  a  strong  generating   set
corresponding to this chain is returned.

After execution  of a Schreier-Sims   algorithm, one can obtain a  strong
generating  set by joining together the  generators  at all levels of the
stabilizer chain. Such strong  generating sets, however, are rather large
and also    rather uninteresting   because   they  are just     unions of
uncorrelated generating sets for all the members of the stabilizer chain.
However, this  is just what the preceding  function does to find a strong
generating set.     \ (More interesting  strong   generating  sets can be
produced   by `Pcgs'  for  solvable  permutation  groups,  see  "Pcgs!for
solvable permutation groups" below.)

%%%%%%%%%%%%%%%%%%%%%%%%%%%%%%%%%%%%%%%%%%%%%%%%%%%%%%%%%%%%%%%%%%%%%%%%%
\Section{Set functions!for permutation groups}

See "Set functions!for domains" for general  remarks on set functions and
also for other  set functions   that  are not  overlaid for   permutation
groups.

\>`<g> in <G>'{membership test}!{for permutation groups}
\>Elements( <G> )!{for permutation groups}
\>Size( <G> )!{for permutation groups}
\>Random( <G> )!{for permutation groups}

These calculations are done with the help of a stabilizer chain using the
methods described in "Stabilizer chains". A random element is constructed
by choosing at each level a random point from  the basic orbit. All these
points together uniquely determine a permutation  contained in the group,
and every element is chosen with the same probability.

\)\fmark Intersection( <G1>, <G2> )

This is really  a  group function,  so we describe  it in  section "Group
functions!for permutation groups".

%%%%%%%%%%%%%%%%%%%%%%%%%%%%%%%%%%%%%%%%%%%%%%%%%%%%%%%%%%%%%%%%%%%%%%%%%
\Section{Backtrack searching}

\>ElementProperty( <G>, <prop> \[, <L> \[, <R> \] \] )

`ElementProperty' returns an element from the permutation group <G> which
fulfills the  property <prop>,  or `false'  if   no such  element exists.
<prop> must be a {\GAP} function that takes as argument an element <g> of
<G>  and returns `true'  if <g> has  the property, and `false' otherwise.
The optional third and fourth arguments must be subgroups of~<G> which do
not alter  the property~<prop> upon multiplication   from the left resp.\
from the right, they will speed up the computation when given.

This function  can  be pretty  slow,  especially  if  no  element of  <G>
satisfies  <prop> and  no  subgroups <L> and  <R> are  given, because all
elements of <G> must be tried in this case.

\>SubgroupProperty( <G>, <prop> \[, <S> \] )

If the fulfilling set of a property~<prop> is a  subgroup <U> of <G> (not
just a subset), `SubgroupProperty' calculates this subgroup together with
base and stabilizer chain. The optional third argument must be a subgroup
of <U> known in advance, which will speed up the computation when given.

\danger The method which  is used to determine the  subgroup <U> is based
on the concept  of an *R-base*, which generalizes  the concept  of a base
(see  \cite{Leon91}).   An  R-base  consists  of a   decreasing  sequence
$(\Omega) \ge \Pi_0 \ge \Pi_1 \ge \ldots \ge \Pi_r$ of ordered partitions
and a sequence $(a_1,\ldots,a_r)$ of  points from $\Omega$ such that, for
$i\ge 1$,
\beginlist
  \item{(1)} $\Pi_i$ contains the singleton cell~$\{a_i\}$;
  \item{(2)} if $g\in  U$ fixes the partition  $\Pi_{i-1}$ and the  point
    $a_i$, then it also fixes the partition~$\Pi_i$;
  \item{(3)} the points in singleton cells of $\Pi_r$ form a base of~<G>.
\endlist

It  then  follows that  $(a_1,\ldots,a_r)$  is  a base  for  <U> (but not
necessarily  for <G>).  To enumerate all  elements of  <U>, the backtrack
algorithm constructs, for each possible  base image $(b_1,\ldots,b_r)$, a
decreasing sequence  $(\Omega) \ge \Sigma_0  \ge  \Sigma_1 \ge \ldots \ge
\Sigma_r$ of ordered partitions such that, for $i\ge 1$,
\beginlist
  \item{(1')} $\Sigma_i$ contains the singleton cell~$\{b_i\}$;
  \item{(2')}    if  $g\in U$     maps    the partition $\Pi_{i-1}$    to
    $\Sigma_{i-1}$  and the point $a_i$ to  $b_i$, then it  also maps the
    partition $\Pi_i$ to~$\Sigma_i$.
\endlist

Because of property~(3), there is then a  unique $g\in G$ mapping $\Pi_r$
to $\Sigma_r$,  and it can be constructed   with the aid of  a stabilizer
chain.  If $g\in U$, it  is added to the  generator list, otherwise it is
discarded.

The descent from  $\Pi_{i-1}\wedge (\{a_i\},\Omega-\{a\})$ to $\Pi_i$ and
from $\Sigma_{i-1}\wedge (\{b_i\}, \Omega-\{b\})$ to $\Sigma_i$ according
to rules~(2)  and~(2') is known as  *refinement*. The partition backtrack
method is efficient if strong refinements are known  that are tailored to
the property  in question, which is  the case in  the applications of the
algorithm, e.g., to centralizer and normalizer calculations.

%%%%%%%%%%%%%%%%%%%%%%%%%%%%%%%%%%%%%%%%%%%%%%%%%%%%%%%%%%%%%%%%%%%%%%%%%
\Section{Group functions!for permutation groups}

\>ConjugateGroup( <G>, <p> )!{for permutation groups}

This  function returns the  conjugate of the  permutation group <G> under
the permutation <p>. <p> must be  an element of  the parent group of <G>.
If a stabilizer  chain for <G> is already  known, it is also  conjugated.
The   method    for this  and  some    generalizations   are explained in
"ext:Generalized conjugation technique" in the extension manual.

\>Centralizer( <G>, <U> \| <u> )!{for permutation groups}
\>Normalizer( <G>, <U> )!{for permutation groups}
\>TwoClosure( <G> )!{for permutation groups}
\>Intersection( <G>, <H> )!{for permutation groups}

These functions    are    applications  of  the   method   described   in
"SubgroupProperty". Below the refinements  used  in each particular  case
are described.

\medskip
*Centralizer:* $\Pi \mapsto \Pi  \wedge \Pi u$.  Here, $\wedge$ denotes a
``meet'' of partitions which obeys $(\Psi \wedge  \Psi')h = \Psi h \wedge
\Psi'h$   for arbitrary partitions $\Psi,\Psi'$   and for any permutation
<h>. If <U> is a subgroup, this refinement is used for each generator <u>
of~<U>.

\medskip
*Normalizer:*  First, every $g\in N_G(U)$ must  permute the <U>-orbits of
equal  length among   themselves  (and if  <g>   is  in a  multiple-point
stabilizer $U^{(i)}$,  also the   $U^{(i)}$-orbits). Furthermore, if  the
image <ag> of a  point <a> is prescribed, <g>  must map the orbit <aU> to
the orbit $aUg  = agU$, so  the refinement $\Pi  \mapsto \Pi  \wedge (aU,
{\rm complement}\allowbreak\ {\rm of\ }aU)$ can be used.

Second, $\Pi \mapsto \Pi  \wedge \Delta$  is  used where $\Delta$  is the
partition into distance zones of an orbital graph of <U> (w.r.t.~a ``base
point''). The orbital graph $(a,b)U$ is a directed graph whose vertex set
is a common  operation domain of <G> and  <U> and whose  arrow set is the
orbit $(a,b)U$ of <U> on pairs  of points. We say  that the point <w> has
distance  <d> in $(a,b)U$ (w.r.t.~<a>)  if   there is  a sequence of  <d>
arrows in  that  graph  leading from <a>  to  <w>,  but no   shorter such
sequence.  This yields a  refinement (see~\cite{Theissen97}) because such
every $g\in   N_G(U)$  maps an  orbital  graph  $(a,b)U$  to  $(a,b)Ug  =
(ag,bg)U$, which is another orbital graph of~<U>.

If <U> is regular,  then as soon  as we  know  the images of points  <a>,
$a.g_1$, \dots,  $a.g_n$ we know  the image of  every  point in the orbit
$a.\langle  g_1,\ldots,g_n \rangle$. This  gives rise to a different kind
of   refinement, this is    used for regular   subgroups. Also  if <U> is
primitive of affine type, the normalizer must also normalize the socle of
<U>  (which can be     calculated as described  in   "Socle!for primitive
permutation  groups"),  this socle is   regular (and elementary abelian);
then the refinements   for the regular   socle are also  employed in  the
calculation of the normalizer of~<U>.

% \medskip
% *Two-closure:* By definition, an element of the full symmetric group lies
% in the two-closure of <G> if it preserves all the orbital graphs $(a,b)G$
% of~<G>,   and the   refinements  derived   from  this  condition  for the
% normalizer  calculation (where the  orbital graphs may  more generally be
% permuted) can be strengthened for the computation  of the two-closure. To
% check  whether a   given  element  belongs to   the  two-closure is  more
% difficult, but for a simple primitive group~<G>  (with few exceptions) it
% is shown  in~\cite{LPS88}  that an element    belongs to the  two-closure
% of~<G> if and only if it normalizes~<G> and preserves all orbital graphs.

\medskip
*Intersection:*      If     $\Delta_i$   is  the       partition     into
$H_{a_1,\ldots,a_{i-1}}$-orbits, then the  refinement  used to  construct
$\Pi_i$    resp.~$\Sigma_i$    is   $\Pi\mapsto      \Pi\wedge  \Delta_i$
resp.~$\Sigma\mapsto \Sigma\wedge   \Delta_i  t$  where  $t\in  H$  is  a
permutation  mapping  the sequence  of  fixpoints of  $\Pi_{i-1}$ to  the
sequence of  fixpoints $\Sigma_{i-1}$ (such a <t>  must exist if there is
an element in <H> mapping the $a_i$ to the~$b_i$).

\>SylowSubgroup( <G>, <p> )!{for permutation groups}

If <G> is not transitive, its <p>-Sylow subgroup  is computed by starting
with  $P=G$, and repeatedly   replacing <P> by  `PreImage( SylowSubgroup(
Image( <hom>, <P>   ),  <p> ) )'   as  <hom> loops  over  the  transitive
constituent   homomorphisms  for <G>   (see    "OperationHomomorphism!for
constituents").

If  <G>  is  transitive but  not   primitive,  its <p>-Sylow subgroup  is
computed as `SylowSubgroup( PreImage( SylowSubgroup( Image( <hom>, <G> ),
<p> ) ),  <p> )' where  <hom> is the  associated block homomorphism  (see
"OperationHomomorphism!for blocks").

If <G> is primitive, `SylowSubgroup'  takes random elements in <G>, until
it finds a  <p>-element <g>, whose centralizer in  <G> contains the whole
<p>-Sylow subgroup. Such an element must exist, because a <p>-group has a
nontrivial  centre. Then the <p>-Sylow  subgroup   of the centralizer  is
computed and returned. Note  that the centralizer  operates imprimitively
on the cycles of <g>, and hence must be a proper subgroup of <G>.

\>ConjugacyClasses( <G> )!{for permutation groups}
\>RationalClasses( <G> )!{for permutation groups}

The default method for computing conjugacy classes in a permutation group
is to compute the  rational classes and decompose   them as shows in  the
example in "Rational classes!for groups".  But this method is replaced by
more elementary methods in the following cases:
\beginlist
  \item{$\bullet$} if  <G> has order at most  1000 or `HasAsList( <G> )',
    an orbit algorithm is performed on the element list,
  \item{$\bullet$}   if <G> is  simple,   classes  are  found by   random
    searching and conjugacy tests,
  \item{$\bullet$} if  <G> is  solvable (this  is tested),   the solvable
    method from "ConjugacyClasses!for solvable groups" is used.
\endlist
The  determination of the rational  classes follows  the method described
in~\cite{Butler94}:  The rational  classes  of  the Sylow subgroups   are
determined by the special method for $p$-groups (see "RationalClasses!for
p-groups"), they  are then fused in the  full group. The rational classes
of elements of non  prime power order are  found as homomorphic images in
the   centralizers of  the  $p$-class representatives  (among whose roots
representatives  of all  classes can be  found).  If  <G>  is a $p$-group
itself, only the first step of this computation is needed.

\>Socle( <G> )!{for primitive permutation groups}
\>Earns( <G>, <D> \[, OnPoints \] )!{for permutation groups}

At   present,   {\GAP} can  only  calculate the    socle   of a primitive
permutation group. If  a primitive group is affine,  its socle  is called
the *earns*, and  it can then  be calculated as `Earns( <G>, MovedPoints(
<G> ) )' (remember that this implies that the earns is stored in <G>, see
"Operation functions!for groups"). The algorithms for `Earns' and `Socle'
are   based on  the O'Nan-Scott theorem    on  primitive groups and  were
developed  by  P.~Neumann. The theorem and   the algorithms are described
in~\cite{Neumann87}. `Earns'  returns  `false'  if <G>  is  not primitive
on~<D>  or not  of  affine type,  whereas `Socle'  signals  an error  for
non-primitive~<G>.

\>Omega( <G>, <p> \[, <n> \] )!{for abelian permutation p-groups}

If <G> is an abelian <p>-group in  normal form $G=C_{p^n_1} \times \ldots
\times  C_{p^n_r}$ where the  direct factor  $C_{p^n_i}$ is generated  by
$\zeta_i\in   G$,   then  `Omega(   <G>,      <p>,  <n>  )   =   $\langle
\zeta_1^{p^{n_1-n}}\rangle      \times              \ldots         \times
\langle\zeta_r^{p^{n_r-n}}\rangle$', where negative powers of $p$ must be
replaced     by~1  (e.g.,  if   $\zeta$    has  order~<p>, then   `Omega(
$\langle\zeta\rangle$,  <p>, 2 )' would be $\langle\zeta^{1/{<p>}}\rangle
:= \langle\zeta\rangle$, i.e., the whole group). The generators~$\zeta_i$
are   determined by  the  function `IndependentGeneratorsOfAbelianGroup',
which has an efficient method for permutation groups.

\>RightTransversal( <G>, <U> )!{for permutation groups} 

\danger Suppose we want to enumerate the  (right) cosets in a permutation
group <G>  of a     subgroup <U>,  i.e.,    we want   a list  of    coset
representatives.  By induction we  assume that we  know such a list $S_1$
for the cosets of $U_1$ in $G_1$. If $T_1$ is a  transversal for $G_1$ in
<G>, i.e., a  transversal constructed during  the Schreier-Sims algorithm
for <G>, then $S_1.T_1$ is a transversal for $U_1$ in  <G>, so contains a
transversal  <S>  for <U>  in <G>.  Since  different  elements   of $S_1$
represent  different cosets in  <G> modulo  <U>,  we have $S_1  \subset S
\subset S_1.T_1$. The basic method is as follows:

Start with the  list $\{(s,1)\mid s\in S_1\}$  and let <t> run over $T_1$
(in an  outer loop) and let <s>  run over $S_1$  (in an  inner loop). For
each pair $(s,t)$,   test whether $Ust   =  Us't'$ for  a pair  $(s',t')$
already in the list, and  if not, add $(s,t)$  to the list. Then finally,
the products of the pairs in the list will form  a transversal for <U> in
<G>. The pairs  in the final list are  ordered first with respect to  <t>
(this is  ordered like $T_1$,  i.e., like `<G>.orbit'),  and for  a fixed
value of <t>, the pair $(s,t)$ precedes the pair $(s',t)$ if <s> precedes
$s'$ in $S_1$.  Note, however, that in general  not all elements of $S_1$
will be used for  each value of  <t>, it is  therefore necessary to store
with  each <t> the subset  $S_1^{(t)}\subset  S_1$ that is actually used.
The transversal constructed by this method looks like
$$ S = \bigcup_{t\in T_1}^\bullet S_1^{(t)}.t\. $$
A transversal for <U> in <G> can also be written as $S'.T'$ where $S'$ is
a transversal  for   <U> in $V=\langle  U,G_1   \rangle$  and  $T'$  is a
transversal for <V> in <G>, i.e., a transversal for the  action of <G> on
the block system with block $1.V\subset  1.G$. $S'$ is then determined by
the method of  the last paragraph,  but where <t> has  to  loop only over
$1.V$, not  over $1.G$. Also the  final transversal is  described by only
$|S'| + |T'|$ elements,  not by $|S'|.|T'|$ elements   as with the  other
method.

With the aid of such block systems, the number of a coset <Ur> is defined
as follows:  The number of the trivial  coset $U.1$ is <1>. Otherwise the
block system  determined by $1.V$ contains one  block <B>  which contains
$1.r$. Let <b> be the number of this block and $g\in T'$ be the canonical
representative which maps  $1.V$   to <B>.  The  quotient  $rg^{-1}$  now
stabilizes  $1.V$,  i.e.,  $rg^{-1}\in   V$.  We   intend   to  sort  the
representatives of <U> in <V> in the order  of $S_1.T_1'$ where $T_1'$ is
sorted like $1.V$ and $S_1$ is sorted with respect to  the numbers of the
cosets of $U_1$  in $G_1$   (which are  determined  recursively with  the
method we are just  describing, unless $G_1$  is trivial). Let $1.t$ with
$t\in T_1'$ be  the  first point in  $1.V$  such that $Ur=Ustg$ for  some
$s\in S_1^{(t)}$,   where <g> is  as above.  Then  the number of  <Ur> is
defined recursively as
$$ (b-1).[V\mathbin:U]\, + \, \sum_{t'\<t}|S_1^{(t')}|\, + \,
   \hbox{`Position(  $S_1^{(t)}$, number of  $U_1s$ in $G_1$ )'.} $$
This  method of enumerating a  right transversal is  slightly modified by
{\GAP}. If both the  transversal  $S'$ (for <U>  in the  block stabilizer
$\langle  U,G_1\rangle$) and the transversal $T'$  (for <G> acting on the
block   system)  have    size   at  most   `MAX_SIZE_TRANSVERSAL'\atindex
{MAX\noexpand   _SIZE\noexpand  _TRANSVERSAL}{@\noexpand    `MAX\noexpand
_SIZE\noexpand _TRANSVERSAL'}  (where size  means length times  degree of
the  permutations), they are stored explicitly  as lists of permutations,
which makes it faster to  construct the transversal elements in  $S'.T'$.
Otherwise, only the sets $S_1^{(t)}$ are stored, so that more computation
is required to construct a transversal element from its number. This rule
applies not only  to the ``top'' transversal for  <U> in <G>, but also to
the transversals  of the stabilizers (e.g., for  $U_1$ in $G_1$) that are
needed in the recursive structure  of the object constructed. {\GAP} sets
`MAX_SIZE_TRANSVERSAL'  to~100{,}000,  but you  can  assign other values,
depending on your memory configuration.

\>PermutationCharacter( <P> )!{for permutation groups}

Computes the character of the  natural permutation representation of <P>,
i.e. it does the same as `PermutationCharacter( <P>, Stabilizer( <P>, <1>
) )' but works much faster.

\>CompositionSeries( <G> )!{for permutation groups}

A composition  series for <G> is  computed with an algorithm by A.~Seress
and R.~Beals. In this case the factor group of each  element $G_i$ in the
composition  series modulo the next   one $G_{i+1}$  is represented as  a
primitive    permutation group.   The  natural   homomorphism   $G_i  \to
G_i/G_{i+1}$   onto   this   factor    group   can   be    obtained    as
`NaturalHomomorphismByNormalSubgroupInParent(   $G_{i+1}$    )'      (see
"NaturalHomomorphismByNormalSubgroup"), because $G_i$ is the parent group
of~$G_{i+1}$.

\>IsSolvable( <G> )!{for permutation groups}
\>IsNilpotent( <G> )!{for permutation groups}

To test solvability of a permutation group <G>, {\GAP} tries to construct
a pcgs as described in "Group functions!for solvable permutation groups".
<G> is solvable if and  only if this is  successful, i.e. if the bound on
the  derived length mentioned in the   description of `DerivedSeries' for
solvable   permutation groups is    not  exceeded. Nilpotency is   tested
analogously with  a pcgs whose series has  central factors (and  with the
bound   on  the   nilpotency  class  mentioned  in    the description  of
`LowerCentralSeries' for nilpotent permutation groups).

%%%%%%%%%%%%%%%%%%%%%%%%%%%%%%%%%%%%%%%%%%%%%%%%%%%%%%%%%%%%%%%%%%%%%%%%%
\Section{Group functions!for solvable permutation groups}

\>StabChain( <G> \[, <base> \] )!{for solvable permutation groups}

The Schreier-Sims algorithm for  arbitrary permutation groups builds up a
stabilizer chain ``top down'' by using Schreier's subgroup theorem to get
down to the next stabilizer.   If a pcgs  (see  "Pcgs")  is known for   a
solvable permutation group <G>, it is, however, faster to build the chain
``bottom  up'' by starting  with  the  trivial  group and inserting   the
elements of the pcgs one by  one into the chain.  The method was proposed
by C.~Sims (see \cite{Sims90b}) and  consists of extending the stabilizer
chain of a group <H> with an element <g> which normalizes <H>.

\danger In this case, if the first base point <b> of <H> is moved by <g>,
the group $\langle H,g\rangle$ has a block system consisting of the basic
orbit <bH>   of <H> and  its <p>  images under the  cyclic action  of <g>
(where <p> is the relative order of <g> modulo <H>).  This means that the
basic orbit  $b\langle    H,g\rangle$  of the  extended     group can  be
constructed with a   simplified  orbit algorithm.  Moreover,  since $g^p$
fixes the block <bH>,  there must be a transversal  element $t\in H$ such
that $g^p.t^{-1}$ fixes <b>, and  then the stabilizer  of <b> in $\langle
H,g\rangle$  is the extension of the  stabilizer $H_b$ by the normalizing
element $g^p.t^{-1}$, so  the whole method  can be repeated to extend the
next layer. It may be necessary to add a new base point at the end of the
chain.

But what  if we do not  know a pcgs for  <G> in advance? A  normal series
with abelian factors for  <G> can be constructed  from the trivial series
for the trivial subgroup by adding  in the generators  of <G> one by one.
If $M_0 \ge M_1 \ge \ldots$ is such a series for  a normal subgroup $M_0$
of <G>, and <s> in the next generator, we can  form the normal closure of
$M_0$ and <s> by   adding to $M_0$   the whole conjugacy class $s^G$.  To
guarantee abelian   factors, it  is  necessary  to form   all commutators
$[s^g,s^h]$ and  test whether they  already lie in  $M_1$. If this is so,
then  whenever an element  added to the growing  $M_0$, it normalizes the
group it is added  to. This allows updating  of  the stabilizer chain  of
$M_0$ in a far more  efficient way than  usual. If some commutator <w> is
not in $M_1$, then $M_1$ is extended by  <w> in the  same manner as $M_0$
is extended by <s>. This  means that at the  same time the partial series
$M_1  \ge M_2  \ge \ldots$   is updated.  This  recursion cannot   become
arbitrarily deep since the weight of such a non-trivial <w> is bounded by
the derived  length of <G>. After all  generators of <G> have been added,
$M_0 \ge M_1 \ge \ldots$ will be the derived series of <G>.

One advantage of this  method is that, whenever an  element must be added
to a group, it normalizes that group, so Sims' method described above can
be applied to extend the stabilizer chain. The elements added in this way
form a pcgs for <G>, so this need not be known in advance.

If  the group <G> is  nilpotent, a completely  analogous procedure can be
used to construct the lower central series. In  this case it is necessary
to form and test not only  the commutators among  the conjugates of a new
generator  <s> but  all   commutators  with `<G>.generators'.  Again  the
recursion  cannot become  arbitrarily   deep  since the  weight   of such
non-trivial commutators  is bounded by  the nilpotency  class  of <G>.  A
series  with    elementary abelian   factors   is achieved   if not  only
commutators by also <p>th powers are tested to lie in~$M_1$.

\>DerivedSeries( <G> )!{for solvable permutation groups}
\>LowerCentralSeries( <G> )!{for nilpotent permutation groups}
\>ElementaryAbelianSeries( <G> )!{for solvable permutation groups}
\>PCentralSeries( <G>, <p> )!{for permutation $p$-groups}
\>CompositionSeries( <G> )!{for solvable permutation groups}

`DerivedSeries'  uses the algorithm    just  described to  construct  the
derived series of <G>.  If <G> is  not solvable, this  would result in an
infinite recursion. For permutation groups,  however, there is a bound on
the derived length  that  depends only on   the degree <d> of  the group.
According   to   \cite{Dix68} this  is  $(5    \log_3(d))/2$.  So if  the
commutators  get  deeper than this  bound  the  algorithm stops  and sets
`<G>.isSolvable' to `false'. In this case, the  default method for groups
is called.

The function `LowerCentralSeries'  steps up  and  down the lower  central
series rather  than the derived  series, and  a  bound on  the nilpotency
class based on the degree <d> is given by $\max_p (p^{\log_p(d)}-1)/(p-1)
- 1$  where <p>  runs over  the  primes dividing  <d>. With   this bound,
`LowerCentralSeries' pursues a strategy similar to `DerivedSeries'.

`ElementaryAbelianSeries' first tries to  construct a central series with
elementary abelian factors. If this is not successful, i.e., if the bound
from the last paragraph is exceeded, `<G>.isNilpotent'  is set to `false'
and  an ordinary elementary abelian  series is constructed, which must be
successful since <G> is assumed to be solvable. `ElementaryAbelianSeries'
can also be given a decreasing series of normal subgroups of <G> (i.e., a
list) as argument. In this case  it does not  try any central factors but
returns an  elementary abelian series   running through that  series. The
given series need not start with <G> nor end in the trivial subgroup.

If <G> is a  <p>-group, `PCentralSeries( <G>, <p>  )' does the same thing
as  `LowerCentralSeries(   <G> )'  and  `ElementaryAbelianSeries( <G>  )'
combined, otherwise the default method for groups is called.

A composition  series for the  solvable group  <G> is calculated  from an
elementary abelian   series  by  inserting  intermediate subgroups  which
correspond to  subspaces of the vector spaces  in the  elementary abelian
series.

\>NormalClosureSolvablePermGroup( <G>, <U> )

If a subgroup <U>  has a solvable normal  closure <C> in~<G>, one can use
the  normal series algorithm to  add the generators of <U>  one by one to
the trivial  subgroup of <G>;   then  one eventually arrives  at~<C>.  In
addition to this, the algorithm yields  a pcgs for  <C>, even if the size
of <G> and <U> were (and remain) unknown.

%%%%%%%%%%%%%%%%%%%%%%%%%%%%%%%%%%%%%%%%%%%%%%%%%%%%%%%%%%%%%%%%%%%%%%%%%
\Section{Polycyclic generating systems!for solvable permutation groups}

\>Pcgs( <G> )!{for solvable permutation groups}

The generating  system which the  normal series algorithm assigns  to the
first group  $M_0$ in the series of   a solvable <G>  is a *p*oly*c*yclic
*g*enerating *s*ystem for~<G>. It can be obtained by the function `Pcgs'.
The concept of a pcgs is explained in section "Pcgs", where we borrow the
notation from.

\danger In a permutation group <G>, a pcgs  is calculated together with a
stabilizer chain (which is stored  in `Pcgs( <G> )!.stabChain') with  the
following properties:

(1) $b_jG_i^{(j)}$ is a block for $G_{i-1}^{(j)}$ because $G_i$ is normal
in $G_{i-1}$. If  $g_i$  fixes this block,   then it also fixes the  base
point    $b_j$, i.e.,  $g_i   \in  G^{(j+1)}$.  This can  be  achieved by
translating $g_i$ by elements from $G_i^{(j)}$, which  does not alter the
characteristic properties of a pcgs.

(2)  The  orbit $b_jG_{i-1}^{(j)}$ is   the  concatenation of the  mapped
orbits $b_jG_i^{(j)}g_i^k$ for  $k=0,\ldots,  p-1$ with <p>  the relative
order of~$g_i$.

(3) The `labels'  components   (as  explained in  "ext:Stabilizer   chain
records" in the  extension manual) are  identical on  every level  of the
stabilizer  chain and  this common  `labels'  list (except  for the first
entry, which is the  identity) is equal to the  list <pcgs>. In fact, the
calculation of the generating sequence is a side-effect of the stabilizer
chain extensions performed according to rules (1) and~(2).

\>ExponentsOfPcElement( <pcgs>, <g> )!{for solvable permutation groups}
\>DepthOfPcElement( <pcgs>, <g> )!{for solvable permutation groups}
\>LeadingExponentOfPcElement( <pcgs>, <g> )%
  !{for solvable permutation groups}

The algorithms  for determining the  (leading) exponents and  depth of an
element of a solvable permutation group make use  of the stabilizer chain
that is calculated along with <pcgs> and stored in `<pcgs>!.stabChain' by
the function `Pcgs'.

\danger We continue the notation of the  last section. Let $b_jG_1^{(j)}$
be the first basic orbit of $G_1$ (i.e., the one with smallest <j>) which
is moved setwise by $g_1$ (or, equivalently by rule~(1) above, whose base
point $b_j$ is moved    by $g_1$). This   basic  orbit  is a  block   for
$G_0^{(j)}$, and it  is stored as an  initial piece of  $b_jG_0^{(j)}$ in
the   record component  `$G_0^{(j)}$.orbit'  (as  explained  in  rule~(2)
above). Now  the cyclic group  $G_0^{(j)}/G_1^{(j)}$ (of order, say, <p>)
operates regularly  on the  <p> images  of $b_jG_1^{(j)}$ which  together
form  $b_jG_0^{(j)}$. Since <g> and $g_1^{e[1]}$   have the same image in
this factor group, $e[1]$ is  determined by the  condition that $b_jg \in
b_jG_1^{(j)} g_1^{e[1]}$. Thus $e[1]$  is  equal to  the number  of times
$g_1$ appears as label  in the Schreier   tree of $G_0^{(j)}$ on the  way
from $b_jg$ back  to the base  point $b_j$ (see "Stabilizer chains"). Now
<g> is replaced by $g_1^{-e[1]}g$ and $G_0$ by $G_1$, then the whole step
is repeated to determine $e[2]$, and so on.

\>RelativeOrderOfPcElement( <pcgs>, <g> )%
  !{for solvable permutation groups}

The  relative  orders  of  <g>  is  found during   the   extension of the
stabilizer chain as the ratio  of the orbit  lengths before and after the
extension with <g>. The corresponding values  for the elements which make
up <pcgs> are stored as a side-effect of the pcgs calculation.

Functions like `Pcgs' and `ExponentsOfPcElement' were originally designed
for solvable groups given by  a power-commutator presentation rather than
by a strong generating  set of permutations.  With the extension of these
functions  to  solvable  permutation groups, {\GAP}   allows  you to  use
certain  functions that were written   for  ag groups  also for  solvable
permutation  groups.  Such functions  are  described in chapter~"Solvable
Groups".

\>NormalPcSeries( <pcgs> )!{for solvable permutation groups}
\>NormalPcFirst( <pcgs> )!{for solvable permutation groups}

The pcgs  which {\GAP} constructs for  a permutation group runs through a
normal  series with abelian  factors. ``Runs through'' means that certain
end-parts  `<pcgs>[ $i$ ..  Length( <pcgs> ) ]'  of it form  pcgs for the
subgroups in this series. The list of these certain values  of $i$ is the
result of `NormalPcFirst( <pcgs>  )', whereas `NormalPcSeries( <pcgs>  )'
returns the list of subgroups.

\>IsomorphismPcGroup( <G> )!{for solvable permutation groups}

This function calculates an  isomorphism onto a  polycyclically presented
group whose source is the given solvable permutation  group. First a pcgs
is  calculated whose  associated  series  refines an  elementary  abelian
series. If the  group is  nilpotent, the  factors of this  series will be
central. Then  the relators belonging  to this  prime-step generating set
are determined by the function  `ExponentsOfPcElement' (see above) and an
pc group <A> is constructed.

\>NaturalHomomorphismByNormalSubgroup( <G>, <N> )%
  !{for solvable factor groups!of permutation groups}

The  construction of  a pcgs of  a  permutation group  by repeated cyclic
extension applies   not  only to  solvable permutation   groups  but more
generally to a solvable  factor group $G/N$  of a permutation group  <G>,
which may itself  be unsolvable. Is this case,  the  resulting pcgs looks
very    much like a  modulo    pcgs  (see "Modulo polycyclic   generating
systems"),   but  it   cannot  have   a    `NumeratorOfModuloPcgs' nor  a
`DenominatorOfModuloPcgs'.

The image of `NaturalHomomorphismByNormalSubgroup( <G>,  <N> )' will then
be represented as a pc group, which is built  from the pcgs of $G/N$ just
like  in the case of `IsomorphismPcGroup'.   This is possible because the
function `ExponentsOfPcElement' and its relatives  are also applicable to
a pcgs of such a factor group.

In the description of the pcgs algorithm in "Group functions!for solvable
permutation groups" we started  with a normal  subgroup $M_0$  (which you
were  told  to imagine  as the trivial   subgroup), but  we  did not need
solvability  of $M_0$  to  show that the   algorithm  terminates, we only
needed solvability  of the  factor group modulo  $M_0$.  If we start with
$M_0=N$ and add one by  one the generators of <G>,   we will construct  a
pcgs for the factor group~$G/N$.

%%%%%%%%%%%%%%%%%%%%%%%%%%%%%%%%%%%%%%%%%%%%%%%%%%%%%%%%%%%%%%%%%%%%%%%%%
\Section{Operation functions!for permutation groups}

\>RepresentativeOperation( <G>, \[ <D>, \] <d>, <e> \[, <gens>, <oprs> %
                           \] \[, <opr> \] )!{for permutation groups}
\>Stabilizer( <G>, \[ <D>, \] <d> \[, <gens>, <oprs> \] \[, <opr> \] )%
  !{for permutation groups}

These two  functions are related  to each other just as `ElementProperty'
and `SubgroupProperty' are (see "Backtrack searching").

If the operation is `OnPoints' and <d> (and <e>) is a positive integer, a
stabilizer   chain  with  first  base  point  <d>   is  computed  and the
representative is found  from the factorized inverse  transversal whereas
the stabilizer  can   directly be  found  in the   stabilizer chain  (see
"Stabilizer chains"). If the operation is `OnSets' and <d> (and <e>) is a
set of points, or if  the operation is `OnPoints' and  <d> (and <e>) is a
permutation or a permutation group, then a backtrack search is performed.

If the operation  is `OnPairs' or `OnTuples'  and <d> and  <e> are lists,
{\GAP} starts with  $r = 1$ and  loops over the points of  <d> and <e> as
follows: <r> is replaced by `RepresentativeOperation( <G>, <d>[1], <e>[1]
^ (<r>^-1), OnPoints ) \* <r>', then <G> is replaced by the stabilizer in
<G> of <d>[1], the first elements of <d>  and <e> are  taken off, and the
process  is repeated until <d>  and <e> are empty.  Then <r> is returned.
Similarly, the stabilizer can be  calculated by repeating the calculation
for the elements in the list~<d>.

In all other cases, the default methods are used.

\>Blocks( <G>, <D> \[, <seed> \] \[, OnPoints \] )%
  !{for permutation groups}

In the case of a permutation group acting naturally on a set of integers,
a   minimal  block   system   is   determined  by   a  method   described
in~\cite{SS94}.

\>IsSemiRegular( <G>, <D> \[, OnPoints \] )!{for permutation groups}

The   action of  permutation  group <G>   on  a  set <D> of   integers is
semiregular if  and  only if  ($S_D$ denotes the  symmetric  group of all
permutations on <D>)
\beginlist
  \item{$\bullet$} all orbits of <G> on <D> have the same length,
  \item{$\bullet$}  for every point $p\in D$  and for every generator <g>
    of <G> there  is a permutation  $z_g\in C_{S_D}(G)$ such that $pz_g =
    pg$, and
  \item{$\bullet$} there is a permutation $z\in C_{S_D}(G)$ that permutes
    the orbits of <G> on <D> setwise.
\endlist
This can  be tested in time  proportional to $o n^2 +  d n$, where <o> is
the size of a single  orbit, <n> is the number  of generators of <G>, and
<d> is the size of~<D>.

%%%%%%%%%%%%%%%%%%%%%%%%%%%%%%%%%%%%%%%%%%%%%%%%%%%%%%%%%%%%%%%%%%%%%%%%%
\Section{Homomorphisms!for permutation groups}

\>GroupGeneralMappingByImages( <P>, <H>, <gens>, <imgs> )%
  !{for permutation groups}

A group homomorphism  of a permutation group  <P> into  another group <H>
(which need not be  a permutation group)  may be specified by prescribing
the images in <H>  of the elements of  a  generating system for  <G> (see
"GroupGeneralMappingByImages"). To calculate with <ghom>, {\GAP} computes
a stabilizer chain for  the source of <ghom> based  on the generating set
<gens> and  stores it   in  a  stabilizer chain `StabChain(   <ghom>  )'.
Reduction along this stabilizer    chain  (as described  in   "Stabilizer
chains") allows to  express an element  of <P>  as  a word  $w(gens)$. An
image of  this  element  is then given   by  $w(imgs)$, and  this can  be
calculated if for each `label'  in the chain  (which is a word in <gens>)
an image under  <ghom> is known. Such  images are stored in the following
additional components at each level of the chain:
\beginitems
`idimage' &
        the identity element of `<ghom>.range'.

`labelimages' &
        a list whose <i>th entry is an image of `labels[<i>]'.

`genimages' &
        a list whose <i>th entry is an image of `generators[<i>]'.

`transimages' &
        a list whose <j>th entry is an image of `transversal[<j>]', i.e.,
        `labelimages[ translabels[<j>] ]'.
\enditems
The stabilizer  chain is computed with  a random Schreier-Sims algorithm,
which uses the size of the source to know when to stop.

\>CoKernel( <ghom> )%
  !{for `GroupGeneralMappingByImages'!for permutation groups}

The  cokernel   of  <ghom>  is  determined by   the   method described in
"CoKernel!for  `GroupGeneralMappingByImages'", where  we  take the set of
all Schreier generators as a defining set of relations  for the source of
<ghom>. More precisely, for  each  level <S>  of the stabilizer  chain of
<ghom>, for each `$p\in S$.orbit'  and for each `$g\in S$.generators', we
form the Schreier generator  `<s> := InverseRepresentative(  <S>, <p> ) ^
-1 \* <g> \* InverseRepresentative( <S>, <p> ^ <g>  )'. This is a word in
the labels of <S> (and so in  <gens>) which lies in `<S>.stabilizer' (see
"ext:Stabilizer chain records" in the  extension manual), and we consider
as  relator  the  quotient  `<s>  /  $w(gens)$' where  $w(gens)$   is the
factorization  of   <s>   by   reduction  along  the   stabilizer   chain
`<S>.stabilizer'   as    described    in "Stabilizer     chains".     The
Reidemeister-Schreier rewriting method shows that this gives indeed a set
of defining   relations. Each relator  is  a word $w_j$   in <gens> which
yields the identity in the source  of <ghom>, because  it is a relator in
the free group. The  normal closure  in the  range of  <ghom> of the  set
$\{w_j(imgs)\}$ of the values of  these words with the images substituted
is then the subgroup of images of the identity  element of the source. In
particular, a total <ghom> is a mapping if and only if $w_j(imgs)$ is the
identity for all~<j>.

\>ImagesRepresentative( <ghom>, <elm> )%
  !{for `GroupGeneralMappingByImages'!for permutation groups}

If $elm=w(gens)$  by reduction along  the stabilizer chain of <ghom>, its
image is `<elm>^$ghom=w(imgs)$'.

\>CompositionMapping( <hom>, <ghom> )%
  !{for `GroupGeneralMappingByImages'!for permutation groups}

The  composition  of  <ghom>  with   an arbitrary homomorphism   <hom> is
constructed  by copying the  stabilizer chain of  <ghom> and replacing in
the `labelimages' components every element with its image under <hom>.

%                                *  *  *
\stars

\>OperationHomomorphism( <P>, <list> )!{for constituents}

For the   construction  of  a  *constituent  homomorphism* <chom>,  every
generator of  <P> is restricted to  <list> (see "RestrictedPerm") and the
resulting permutation  is conjugated with a permutation `<chom>!.conperm'
which maps <list> to `[1..Length(<list>)]'.

\>ImagesRepresentative( <chom>, <elm> )!{for constituent homomorphisms}
\>Image( <chom>, <H> )!{for constituent homomorphisms}

The image  of an element is  computed by restricting  <elm> to <list> and
conjugating the restricted permutation with `<chom>!.conperm'.

For  the image of  a subgroup <H>, first  a  stabilizer chain  for <H> is
computed whose base starts off with <list>. Then the images of the strong
generators of <H> form a strong generating set of the image. A stabilizer
chain for  the  image   is constructed   by the generalized   conjugation
technique described  in   "ConjugateGroup!for permutation  groups"  (with
`$map(pnt) =  pnt$ ^ chom!.conperm' and $hom  = chom$ in the  notation of
that section).

\>PreImage( <chom>, <H> )!{for constituent homomorphisms}

For  the preimage of a subgroup  <H>,  first a  stabilizer  chain for the
`<chom>.source' is computed whose base starts off with <list>. The kernel
of <chom> is  a stabilizer in  this chain.  The  preimages of the  strong
generators for    <H> (i.e.,  their  images  under  `<chom>!.conperm^-1')
together  with  the strong  generators   for  the  kernel form  a  strong
generating set of the preimage of <H>. Again,  a stabilizer chain for the
preimage  is constructed by  the  generalized conjugation technique (this
time   with  `$map(pnt) =   pnt$   ^  chom!.conperm'   and `$hom(h)    =$
PreImagesRepresentative(   <chom>,   <h>    )'  in   the     notation  of
"ConjugateGroup!for  permutation   groups";   the   latter  is  not     a
homomorphism, so the kernel of <chom> is added to the resulting chain).

%                                *  *  *
\stars

\>OperationHomomorphism( <P>, <blocks>, OnSets )!{for blocks}

A  *block   homomorphism* <bhom>  in   {\GAP}    has a record   component
`<bhom>!.reps', which is a list whose  <i>th position the block number of
the point~<i>.

\>ImagesRepresentative( <bhom>, <elm> )!{for block homomorphisms}
\>Image( <bhom> \[, <H> \] )!{for block homomorphisms}
\>PreImagesRepresentative( <bhom>, <elm> )!{for block homomorphisms}
\>PreImage( <bhom>, <H> )!{for block homomorphisms}
\>Kernel( <bhom> )!{for block homomorphisms}

(1) The image of an element  <elm> under <bhom>  is the permutation which
maps the point <i> to `<bhom>!.reps[ <b>^<elm> ]', where <b> is any point
from the <i>th block. So  the image  can  be constructed with  `PermList'
(see "PermList").

(2) To find the image of a subgroup  <H> (default is `Source( <bhom> )'),
{\GAP} constructs a ``blockwise  stabilizer chain''~<S> of~<H>, which  is
internally represented like an ordinary  stabilizer chain; indeed, at the
beginning <S> is just a copy of the stabilizer chain of <H>. Then for the
first block <B> from <blocks> the current base point of <S> is changed to
`<B>[1]',  and one   level of  the  stabilizer  chain   of  the image  is
constructed  by      the    generalized conjugation     technique    (see
"ConjugateGroup!for  permutation     groups"   with      `$map(pnt)     =
hom$!.reps[<pnt>]' and <hom> the block  homomorphism  in the notation  of
that section; here the next  stabilizer must be computed explicitly). The
next level `<S>.stabilizer' is built by enlarging the point stabilizer of
`<B>[1]' to the  block stabilizer of <B>  by adding  transversal elements
from~<S> which  map `<B>[1]' to the  other points of~<B>. This process is
repeated  with <S>  replaced by  `<S>.stabilizer'  and <B> the next block
from <blocks>. After all blocks have been processed in this way, <S> is a
stabilizer chain for the subgroup of <H> acting trivially on <blocks>.

(3) Thus, if  <H> is `Source(  <hom>  )', <S>  also yields  the kernel of
<hom> after the image has been calculated.

(4) Using the transversals at all levels of <S>, it is then also possible
to construct a preimage  of a given permutation  of <blocks>, i.e., of an
element of `Range( <hom> )'.

(5) The preimage   of a subgroup <I>  of  `Range( <bhom> )'  (default  is
`Range( <bhom> )') is constructed by walking  the stabilizer chain of <I>
upwards from the trivial subgroup (whose preimage  is the kernel of <hom>
with stabilizer chain <S>) and adding for every level <T> a level to <S>:
First a base change is performed to get as new base point the first point
in the ``base block'' of <T>, even if this leads to a trivial basic orbit
of  <S>. Then the new top  level of <S> is extended  with  a preimage for
every element of `<T>.generators'. Finally, when  <T> has reached the top
level of the stabilizer  chain of <I>,  the modified <S> is a  stabilizer
chain for the preimage.

%                                *  *  *
\stars

\>NaturalHomomorphismByNormalSubgroup( <G>, <N> )%
  !{for permutation groups}

If  the factor  group  $G/N$  is solvable,   the  method is described  in
"NaturalHomomorphismByNormalSubgroup!for   solvable   factor    groups!of
permutation groups".

In all other cases, ask Alexander.

