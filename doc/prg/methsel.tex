%%%%%%%%%%%%%%%%%%%%%%%%%%%%%%%%%%%%%%%%%%%%%%%%%%%%%%%%%%%%%%%%%%%%%%%%%
%%
%W  methsel.tex               GAP manual                    Thomas Breuer
%W                                                       Martin Schoenert
%%
%H  @(#)$Id$
%%

%%%%%%%%%%%%%%%%%%%%%%%%%%%%%%%%%%%%%%%%%%%%%%%%%%%%%%%%%%%%%%%%%%%%%%%%%
\Chapter{Method Selection}

An *operation* is a special {\GAP} function that bundles a set of
functions, its *methods*.

All methods of an operation compute the same result.
But each method is installed for specific types of arguments.

If an operation is called with a tuple of arguments,
one of the applicable methods is selected and called.

Special cases of methods are partial methods, immediate methods,
and logical implications.

%%%%%%%%%%%%%%%%%%%%%%%%%%%%%%%%%%%%%%%%%%%%%%%%%%%%%%%%%%%%%%%%%%%%%%%%%
\Section{Operations and Methods}

Operations are functions in the category `IsOperation'.

So on one hand, *operations* are {\GAP} functions, that is,
they can be applied to arguments and return a result or cause a
side-effect.

On the other hand, operations are more.
Namely, an operation corresponds to a set of {\GAP} functions,
called the *methods* of the operation.

Each call of an operation causes a suitable method to be selected
and then called.
The choice of which method to select is made according to the types
of the arguments,
the underlying mechanism is described in the following sections.

Examples of operations are the binary infix operators `=', `+' etc.,
and `PrintObj' is the operation that is called for each argument of
`Print'.

Also all attributes and properties are operations.
Each attribute has a special method which is called
if the attribute value is already stored;
this method of course simply returns this value.

As stated above, the setter of an attribute is called automatically
if an attribute value has been computed.
Attribute setters are operations, too.
They have a default method that ignores the request to store the value.
Depending on the type of the object,
there may be another method to store the value in a suitable way,
and then set the attribute tester for the object to `true'.


%%%%%%%%%%%%%%%%%%%%%%%%%%%%%%%%%%%%%%%%%%%%%%%%%%%%%%%%%%%%%%%%%%%%%%%%%
\Section{Method Installation}

In order to describe what it means to select a method of an operation,
we must describe how the methods are connected to their operations.

A function <method> is installed as a method for the operation <opr> by

\>InstallMethod( <opr>, <text>, <famp>, <args-filts>, <val>, <method> )

where <text> is a short but informative string that describes for what
situation the method is installed,
<famp> is a function to be applied to the families of the arguments,
<args-filts> is a list of requirements for the arguments,
each entry being a filter,
and <val> is an integer that measures the priority of the method.

The exact meaning of the arguments <famp>, <args-filts>, and <val>
is explained below.

<opr> expects its methods to require certain filters for their arguments.
For example, the argument of a method for the operation `Zero' must be
in the category `IsAdditiveElementWithZero'.
It is not possible to use `InstallMethod' to install a method for which
the entries of <args-filt> do not imply the respective requirements of
the operation <opr>.
If one wants to override this restriction,
one has to use `InstallOtherMethod' instead.


%%%%%%%%%%%%%%%%%%%%%%%%%%%%%%%%%%%%%%%%%%%%%%%%%%%%%%%%%%%%%%%%%%%%%%%%%
\Section{Applicable Methods and Method Selection}

A method installed as above is *applicable* for an arguments tuple
if the following conditions are satisfied.

The number of arguments equals the length of the list <args-filts>,
the $i$-th argument lies in the filter `<args-filts>[$i$]',
and <famp> returns `true' when applied to the families of the arguments.

So <args-filt> describes conditions for each argument,
and <famp> describes a relation between the arguments.

For unary operations such as attributes and properties,
there is no such relation to postulate,
<famp> is `ReturnTrue' for these operations,
a function that does always return `true'.
For binary operations, the usual value of <famp> is `IsIdentical',
which means that both arguments must lie in the same family.

If no method is applicable,
the error message `no method found' is signaled.

Otherwise, the applicable method with highest *rank* is selected and then
called.
This rank is given by the sum of the ranks of the filters in the list
<args-filt>,
*including involved filters*,
plus the number <val> used in the call of `InstallMethod'.
So the argument <val> can be used to raise the priority of a method
relative to other methods for <opr>.

Note that from the applicable methods,
an efficient one shall be selected.
This is a method that needs only little time and storage for the
computations.

It seems to be impossible to achieve that {\GAP} will select an optimal
method in all cases.
The present ranking of methods is based on the assumption
that a method installed for a special situation shall be preferred
to a method installed for a more general situation.

For example, a method for computing a Sylow subgroup of a nilpotent
group is expected to be more efficient than a method for arbitrary
groups.
So the more specific method will be selected if {\GAP} knows that the
group given as argument is nilpotent.

Of course there is no obvious way to decide between the efficiency of
incommensurable methods.
For example, take an operation with one method for permutation groups,
another method for nilpotent groups,
but no method for nilpotent permutation groups,
and call this operation with a permutation group known to be
nilpotent.

%%%%%%%%%%%%%%%%%%%%%%%%%%%%%%%%%%%%%%%%%%%%%%%%%%%%%%%%%%%%%%%%%%%%%%%%%
\Section{Partial Methods}

After a method has been selected and called,
this method may recognize that it cannot compute the desired result,
and give up by calling `TryNextMethod()'.

In effect, the execution of the method is terminated,
and the method selection calls the next method that is applicable w.r.t.
the original arguments.
In other words, the applicable method is called that is subsequent to the
one that called `TryNextMethod',
according to decreasing rank of the methods.

For example, since every finite group of odd order is solvable,
one may install a method for the property `IsSolvableGroup' that checks
whether the size of the argument is an odd integer,
returns `true' if yes,
and gives up otherwise.

%%%%%%%%%%%%%%%%%%%%%%%%%%%%%%%%%%%%%%%%%%%%%%%%%%%%%%%%%%%%%%%%%%%%%%%%%
\Section{Immediate Methods}

Usually a method is called only if its operation has been called
and if this method has been selected.

For attributes and properties, one can install also *immediate methods*.
An immediate method is called automatically as soon as it is applicable
to an object, provided that the value is not yet known.
Afterwards the attribute setter is called in order to store the value.

Note that in such a case {\GAP} executes a computation for which
it was not explicitly asked by the user.
So one should install only those methods as immediate methods
that are *extremely cheap*.
To emphasize this, immediate methods are also called *zero cost methods*.
The time for their execution should really be approximately zero.

An immediate method <method> for the attribute or property <attr>
with requirement <req> is installed via

\>InstallImmediateMethod( <attr>, <req>, <val>, <method> )

where <val> is an integer value that measures the priority of <method>
among the immediate methods for <attr>.

Note the difference to `InstallMethod' that no family predicate occurs
because <attr> expects only one argument,
and that <req> is not a list of requirements but the argument requirement
itself.

For example, the size of a permutation group can be computed very cheaply
if a stabilizer chain of the group is known.
So it is reasonable to install an immediate method for `Size' with
requirement `IsGroup and Tester( <stab> )',
where <stab> is the attribute corresponding to the stabilizer chain.

Another example would be the implementation of the conclusion that
every finite group of prime power order is nilpotent.
This could be done by installing an immediate method for the attribute
`IsNilpotentGroup' with requirement `IsGroup and Tester( Size )'.
This method would then check whether the size is a finite prime power,
return `true' in this case and otherwise call `TryNextMethod()'.
But this requires factoring of an integer,
which cannot be guaranteed to be very cheap,
so one should not install this method as an immediate method.

Immediate methods are thought of as a possibility for objects to gain
useful knowledge.
They must not be used to force the storing of ``defining information''
in an object.
In other words, {\GAP} should work even if all immediate methods are
invalidated.

%%%%%%%%%%%%%%%%%%%%%%%%%%%%%%%%%%%%%%%%%%%%%%%%%%%%%%%%%%%%%%%%%%%%%%%%%
\Section{Logical Implications}

It may happen that a property <prop> shall be implied by another filter
<filt>, which is usually a meet of other properties,
or the meet of some properties and some categories.
Such a logical implication can be installed as an immediate method for
<prop> that requires <filt> and that always returns `true'.
It should be installed via

\>InstallTrueMethod( <prop>, <filt> )

This has the effect that <prop> becomes an implied filter of <filt>,
see~"Filters".

For example, each cyclic group is abelian,
each finite vector space is finite dimensional,
and each division ring is integral.
The first of these implications is installed as follows.

\begintt
    InstallTrueMethod( IsCommutative, IsGroup and IsCyclic )
\endtt

Contrary to other immediate methods,
logical implications cannot be switched off.
This means that after the above implication has been installed,
one can rely on the fact that every object in the filter
`IsGroup and IsCyclic' will also be in the filter `IsCommutative'.

%%%%%%%%%%%%%%%%%%%%%%%%%%%%%%%%%%%%%%%%%%%%%%%%%%%%%%%%%%%%%%%%%%%%%%%%%
%%
%E  methsel.tex

