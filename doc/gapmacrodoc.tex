%%%%%%%%%%%%%%%%%%%%%%%%%%%%%%%%%%%%%%%%%%%%%%%%%%%%%%%%%%%%%%%%%%%%%%%%%
%%
%W  gapmacrodoc.tex           GAP documentation             Thomas Breuer
%W                                                         & Frank Celler
%W                                                     & Martin Schoenert
%W                                                       & Heiko Theissen
%%
%H  @(#)$Id$
%%
%Y  Copyright 1997,  Lehrstuhl D fuer Mathematik,  RWTH Aachen,   Germany
%%
%%  self-contained document with the former chapter
%%  ``The gapmacro.tex Manual Format'' of ``Extending GAP''
%%
\input gapmacro
\BeginningOfBook{gapmacro}
\TitlePage{
  \centerline{\titlefont GAP}\bigskip
  \centerline{\secfont Release 4.5}\medskip
  \centerline{\secfont DD MM 2008}\bigskip\bigskip
  \centerline{\titlefont The gapmacro.tex Manual Format}\vfill
  \centerline{\secfont The GAP Group}\bigskip
  \centerline{\secfont http://www.gap-system.org}
}
\TableOfContents
\FrontMatter
\input ref/copyrigh
\Chapters

\Chapter{The gapmacro.tex Manual Format}

This document describes a restricted {\TeX} format,
which is defined by the of macros in the file `GAPPATH/doc/gapmacro.tex',
and how to create the final documents (which can be printed or used by
{\GAP}'s online help) from it.
Some {\GAP}~4 package documentation is written in this format.
Up to version 4.4, the same was true for the main {\GAP} manuals.

See "TeX Macros" and "Examples, Lists, and Verbatim" for details  on  the
restricted set of available {\TeX} commands.

The first sections "The main file" and
"gapmacro:Structuring the text: Chapters and Sections" describe
the  general  layout of the  files  in case  you need to   write your own
package documentation.

If you are planning to write new documentation for a {\GAP} package,
one alternative to using the format described in this document 
is to use the \package{GAPDoc} package,
see Chapter ``Introduction and Example''
in the \package{GAPDoc} manual,
for example type

%notest
\beginexample
gap> ?GAPDoc:chapters
\endexample

in {\GAP}'s online help for a  table  of  contents,  or  (if  it  is  not
available in your installation) see:
\URL{http://www.math.rwth-aachen.de/~Frank.Luebeck/GAPDoc/}

If you want to use yet another document format you must  provide  certain
information to the interface of {\GAP}'s online help. This  is  described
in Chapter ``Interface to the GAP Help System'' of the
{\GAP} Reference Manual.

%%%%%%%%%%%%%%%%%%%%%%%%%%%%%%%%%%%%%%%%%%%%%%%%%%%%%%%%%%%%%%%%%%%%%%%%%
\Section{The Main File}

\indextt{gapmacro.tex}\indextt{\\BeginningOfBook}\indextt{\\UseReferences}
\indextt{\\TitlePage}\indextt{\\Colophon}\indextt{\\TableOfContents}
\indextt{\\OneColumnTableOfContents}
\indextt{\\FrontMatter}\indextt{\\Chapters}\indextt{\\Appendices}
\indextt{\\Bibliography}\indextt{\\Index}\indextt{\\EndOfBook}
\indextt{\\Package}\indextt{\\package}
The main {\TeX} file is called `manual.tex'.
This file should contain the following commands:

%\begintt
\){\kernttindent\\input ../gapmacro}
\){\kernttindent\\Package\{<package-name>\}}
\){\kernttindent\\BeginningOfBook\{<name-of-book>\}}
\){\kernttindent\ \ \\UseReferences\{<book1>\}}
\){\kernttindent\ \ ...}
\){\kernttindent\ \ \\UseReferences\{<bookn>\}}
\){\kernttindent\ \ \\TitlePage\{<title>\}}
\){\kernttindent\ \ \\Colophon\{<text>\}}
\){\kernttindent\ \ \\TableOfContents}
\){\kernttindent\ \ \\FrontMatter}
\){\kernttindent\ \ \ \ \\immediate\\write\\citeout\{\\bs bibdata\{<mybibliography>\}\}}
\){\kernttindent\ \ \ \ \\Input\{<file1>\}}
\){\kernttindent\ \ \ \ ...}
\){\kernttindent\ \ \ \ \\Input\{<filen>\}}
\){\kernttindent\ \ \\Chapters}
\){\kernttindent\ \ \ \ \\Input\{<file1>\}}
\){\kernttindent\ \ \ \ ...}
\){\kernttindent\ \ \ \ \\Input\{<filen>\}}
\){\kernttindent\ \ \\Appendices}
\){\kernttindent\ \ \ \ \\Input\{<file1>\}}
\){\kernttindent\ \ \ \ ...}
\){\kernttindent\ \ \ \ \\Input\{<filen>\}}
\){\kernttindent\ \ \ \ \\Bibliography}
\){\kernttindent\ \ \ \ \\Index}
\){\kernttindent\\EndOfBook}
%\endtt

Now we describe what these commands do:

\beginitems

`\\input <path>/gapmacro.tex'&
inputs the {\GAP} ``style'' and macros file `gapmacro.tex'.  If  you  are
writing a {\GAP} package either copy this file or use  a  relative  path.
The former method will always work but requires  you  to  keep  the  file
consistent with the system while the latter forces users  to  change  the
`manual.tex' file if they are installing a package in a private location.
See also Section ``GAP Root Directory'' in the {\GAP} Reference Manual.

\indextt{\\Package}
`\\Package\{<package-name>\}'&
defines   a   macro   `\\<package-name>'   so   that   when   you    type
`\{\\<package-name>\}'  (please  include  the  curly  braces)  the   text
<package-name> is typeset in the right way for {\GAP}  packages,  e.g.~if
you are writing a package \package{MyPackage} then you should include the
line

\){\kernttindent\\Package\{MyPackage\}}

  &
in  your  `manual.tex'  file  and  then  in  your   chapter   files   use
`\{\\MyPackage\}' when you refer to \package{MyPackage} by name. There is
also the command `\\package\{<pkg>\}' when you wish  to  refer  to  other
{\GAP} packages; don't confuse the two i.e.~`\\Package\{<package-name>\}'
defines  a  macro  `\\<package-name>'   but   produces   no   text,   and
`\\package\{<pkg>\}' produces <pkg> set in the font  that  is  right  for
{\GAP} packages.

\indextt{\\BeginningOfBook}
`\\BeginningOfBook\{<name-of-book>\}'&
starts the book <name-of-book>. It  is  used  for  cross-references,  see
"Labels and References". If you are writing a {\GAP} package use the name
of your package here.

\indextt{\\UseReferences}\indextt{\\UseGapDocReferences}
`\\UseReferences\{<booki>\}'&
If your manual cross-refers to another manual, `\\UseReferences'  can  be
used to load the labels of  the  other  books  in  case  cross-references
occur. <booki> should be the path of the directory  containing  the  book
whose references you want to load. If you are writing  a  {\GAP}  package
and you need to reference the main {\GAP} manual,  use  `\\UseReferences'
for each book you want to reference. However, as said above this requires
changes to the `manual.tex' file if the package is not installed  in  the
standard location.

  &
You  may  also  cross-refer to other `gapmacro.tex' format manuals and to
\package{GapDoc}-produced manuals. Just ensure you get the  path  to  the
other manual's directory correct *relative* to  the  directory  in  which
your manual resides, and if it's a \package{GapDoc}-produced manual  that
you  are  cross-referring  to,  use  `\\UseGapDocReferences'  instead  of
`\\UseReferences'.

  &
If your `manual.tex' file lives in `pkg/qwer/doc' and  you  want  to  use
references to the {\GAP} Tutorial use

\begintt
\UseGapDocReferences{../../../doc/tut}
\endtt

\indextt{\\TitlePage}
`\\TitlePage'&
produces a page containing the *title*. Please see the example.

\indextt{\\Colophon}
`\\Colophon'&
`\\Colophon' produces a page following the title that  can  be  used  for
more  explicit  author  information,  acknowledgements,  dedications   or
whatsoever.

\indextt{\\TableOfContents}\indextt{\\OneColumnTableOfContents}
`\\TableOfContents'&
produces a table of contents in double-column format. For short  manuals,
the  double-column  format  may  be  inappropriate;  in  this  case,  use
`\\OneColumnTableOfContents' instead.

\indextt{\\FrontMatter}
`\\FrontMatter'&
starts the front matter chapters such as a copyright notice or a preface.

  &
The line

%\begintt
\){\kernttindent\\immediate\\write\\citeout\{\\bs bibdata\{<mybibliography>\}\}}
%\endtt

  &
is for users of Bib{\TeX}. It will use the file `<mybibliography>.bib' to
fetch bibliography information.

\indextt{\\Chapters}
`\\Chapters'&
starts the chapters of the manual,  which  are  included  via  `\\Input'.
`\\Input\{<filei>\}' inputs the file `<filei>.tex',  i.e.~<filei>  should
be the name of the file *without* the `.tex' extension. For  the  chapter
format, see Section~"Structuring the text: Chapters and Sections".

\indextt{\\Appendices}
`\\Appendices'&
starts the appendices, i.e.~it modifies the `\\Chapter'  command  to  use
uppercase letters to number chapters.

%`\\Answers'  produces an answers chapter, see  "Exercises  and  Answers".
\indextt{\\Bibliography}
`\\Bibliography'&
produces a bibliography, i.e. it reads and typesets the `manual.bbl' file
produced by Bib{\TeX}.

\indextt{\\Index}
`\\Index'&
produces an index, i.e. it  reads  and  typesets  the  `manual.ind'  file
produced by the external `manualindex' program.

\indextt{\\EndOfBook}
`\\EndOfBook'&
Finally `\\EndOfBook' closes the book.

\enditems

*Example*

Assume you have a {\GAP} package `qwert' with two  chapters  `Qwert'  and
`Extending Qwert', a copyright notice, and a preface, then your
`manual.tex' would basically look like:

\begintt
\input ../../../doc/gapmacro         % The right path from pkg/qwert/doc
\Package{Qwert}                      % Defines macro {\Qwert}
\BeginningOfBook{qwert}
  \TitlePage{
    \centerline{\titlefont Qwert}\medskip           % Package name
    \centerline{\titlefont ---}\medskip
    \centerline{\titlefont A GAP4 Package}\bigskip\bigskip
    \centerline{\secfont Version 1.0}\medskip
    % If the package interfaces with an external program ...
    \centerline{\secfont Based on qwert Standalone Version 3.14}\vfill
    \centerline{\secfont by}\vfill
    \centerline{\secfont Q. Mustermensch}\medskip   % Author
    \centerline{Department of Mathematics}\medskip  % Affiliation
    \centerline{University of Erewhon}\medskip
    \centerline{\secfont email: qmuster@erewhon.uxyz.edu.ut} % Email address
    \vfill
    \centerline{\secfont{\Month} \Year}
  }
  \TableOfContents
  \FrontMatter
    \Input{copyright}
    \Input{preface}
  \Chapters
    \Input{qwert}
    \Input{extend}
  \Appendices
    \Index
\EndOfBook
\endtt

\indextt{\\PseudoInput}\indextt{\\Day}\indextt{\\Month}\indextt{\\Year}
\indextt{\\Today}
Occasionally there will be the need  for  additional  commands  over  and
above those shown above. The ones described below should  be  the  *only*
exceptions.
\beginlist%unordered

\item{--}
There may be other packages that are referred to  a  lot,  so  that  it's
worthwhile to add more `\\Package'  commands.  (There's  nothing  special
about `\\Package', you can use it to define  macros  for  other  packages
besides the package being documented.)

\item{--}
Besides the macros `\{\\Month\}'  and  `\{\\Year\}',  which  typeset  the
current month (as an English  word)  and  the  year  (all  four  digits),
respectively, there are also  `\{\\Day\}'  and  `\{\\Today\}'  which  are
mainly intended for drafts. `\{\\Day\}' typesets the day of the month  as
a number and  `\{\\Today\}'  is  equivalent  to:  `\{\\Day\}  \{\\Month\}
\{\\Year\}'.

\item{--}
Sometimes one desires a chapter to be unnumbered in  the  {\TeX}-produced
manuals, e.g.~the Tutorial manual has {\GAP}'s  Copyright  Notice  as  an
unnumbered chapter. To achieve this one inputs the  file  containing  the
chapter via {\TeX}'s `\\input' command rather  than  `\\Input'.  However,
neither the on-line help browser nor the  HTML  converter  ``sees''  such
chapters. Thus if it is desired that the on-line  help  browser  and  the
HTML manuals should also have such chapters, they must be ``input'' again
via the `\\PseudoInput' command (not necessarily in the same manual).

\item{--}
For chapters that should only appear via the on-line help browser  or  in
the  HTML  manuals,  one  may  use  the  `\\PseudoInput'   command.   Any
`\\PseudoInput'  commands should come  *after*  all  `\\Input'  commands;
failure to do this  will  result  in  different  numbering  of  `\\Input'
chapters for {\TeX}-produced and HTML manuals. The syntax of this command
is as follows:

\)\kernttindent\\PseudoInput\{<filename>\}\{<six-entry>\}\{<chaptername>\}

%display{tex}
\item{}
%enddisplay
where <filename> is the name of the file containing the  chapter  without
the `.tex' extension, as for the `\\Input' command,  <six-entry>  is  the
section-index-entry for the chapter (written to  the  `manual.six'  file)
and <chaptername> is the *actual* argument  of  the  `\\Chapter'  command
that  appears  at  the  beginning  of  `<filename>.tex'.   The   argument
<six-entry> enables the on-line text browser to reference the chapter  by
a name other than <chaptername>. Thus a copyright chapter  for  the  book
with name <name-of-book> might have <chaptername> ```Copyright  Notice'''
but <six-entry> ```Copyright''', which would enable  one  to  access  the
chapter ```Copyright Notice''' via  `?<name-of-book>:copyright'  via  the
on-line browser.  The  HTML  converter  adds  an  index  entry  for  both
<six-entry> and <chaptername>.

\endlist

*Note*

Usage of the commands `\\input' and `\\PseudoInput' in the way  described
above will necessitate special treatment of references to such  chapters.
For such  purposes,  there  is  a  special  variant  of  the  `\%display'
environment (see "Catering for Plain  Text  and  HTML  Formats"),  e.g.~a
copyright  notice  appearing  via  `\\input'  at  the  beginning   of   a
{\TeX}-produced manual and appearing in the  non-{\TeX}  manuals  --  the
on-line help browser or HTML manual -- via a `\\PseudoInput'  command  as
described above, may be referenced via

\begintt
%display{tex}
See the copyright notice at the beginning of this book.
%display{nontex}
%See "Copyright".
%enddisplay
\endtt

%%%%%%%%%%%%%%%%%%%%%%%%%%%%%%%%%%%%%%%%%%%%%%%%%%%%%%%%%%%%%%%%%%%%%%%%%
\Section{Structuring the text: Chapters and Sections}

\indextt{\\Chapter}\indextt{\\Section}
The contents of each chapter must  be  in  its  *own*  `.tex'  file.  The
command  `\\Chapter\{<chaptername>\}'  starts   a   new   chapter   named
<chaptername>; it should constitute the first non-comment (and non-blank)
line of  the  file  containing  a  chapter.  A  chapter  begins  with  an
introduction to the chapter and is followed by sections created with  the
`\\Section\{<secname>\}' command. The strings <chaptername> and <secname>
are  automatically  available  as  references  (see  Section~"Labels  and
References").

There must be *no further commands* on the same line as  the  `\\Chapter'
or  `\\Section'  line,  and  there  *must*  be  an  empty  line  after  a
`\\Chapter' or `\\Section' command. This means  that  `\\index'  commands
referring to the chapter or section can be placed only after  this  empty
line.

Finally, the HTML  converter  requires  that  each  `\\Section'  line  is
preceded  by  a  line  starting  with  at  least  16   percentage   signs
(conventionally, one actually types a full line of percentage signs). The
HTML converter stops converting a section whenever it hits such  a  line;
therefore do not add lines starting with 16 or more \%  signs  which  are
*not* just before a `\\Section' command. Failure to include the  line  of
percentage signs before a `\\Section' line will cause  the  converter  to
crash, due to the discovery of what it sees as two  `\\Section'  commands
within the one section.

%%%%%%%%%%%%%%%%%%%%%%%%%%%%%%%%%%%%%%%%%%%%%%%%%%%%%%%%%%%%%%%%%%%%%%%%%
\Section{Suppressing Indexing and Labelling of a Section and Resolving 
Label Clashes}

\indextt{\\null!use in index suppression}
Sometimes one does not wish a section to  be  indexed.  To  suppress  the
indexing  of  a  section,  simply  add  the  macro  `\\null'  after   the
`\\Section' command, e.g.

\)\kernttindent\\Section\{<section-name>\}\\null

and then <section-name> will still generate a  label  (so  that  you  can
still refer to it via `Section~\"<section-name>\"'),  but  <section-name>
will not appear in the index.

\indextt{\\nolabel!use in index and label suppression}
Occasionally, one has a dedicated section for the description of a single
function. If the label generated for the section coincides with the label
for a subsection (generated by a `\\>' command) a multiply defined  label
results. In these cases, one would generally rather that the section  did
not generate a label or an index entry. To  suppress  the  generation  of
both the label and the index entry of such  a  section,  simply  add  the
macro `\\nolabel' immediately after the `\\Section' command,  e.g.~for  a
section dedicated to the function <func>:

\)\kernttindent\\Section\{<func>\}\\nolabel

*Note:* Labels are generated by  converting  to  lowercase  and  removing
whitespace. So coincidences can occur when you might  not  have  expected
it. An alternative to index suppression to resolve label  clashes  is  to
include  a  sub-label  for  the  function  in  the  `\\>'  command   (see
Section~"TeX Macros").

%%%%%%%%%%%%%%%%%%%%%%%%%%%%%%%%%%%%%%%%%%%%%%%%%%%%%%%%%%%%%%%%%%%%%%%%%
\Section{Labels and References}

Each `\\Chapter', `\\Section' and `\\>' command generates a (short) label
<label>,  which  is  extended  by   <name-of-book>   (the   argument   of
`\\BeginningOfBook' mentioned earlier in  Section~"The  Main  File"),  to
create a ``long label'' <long-label>, and emitted to a file `manual.lab'.
The construction of <long-label> is `<name-of-book>:<label>',  where  the
<label> generated by either of the commands `\\Chapter' or `\\Section' is
just its <chaptername> or <secname> argument. For `\\>', there are a  few
cases to consider, and we'll consider them in Section~"TeX Macros", where
we meet the various forms of the `\\>' command. To  see  how  to  resolve
problems  with  label  clashes  see  Section~"Suppressing  Indexing   and
Labelling of a Section and Resolving Label Clashes".

\index{reference to a label}
A reference to a label <any-label> (long or short) is made  by  enclosing
<any-label> in a pair of double quotation marks: `\"<any-label>\"'; it is
replaced by the number of the `\\Chapter', `\\Section' or  `\\>'  command
that generated <any-label> in the first place. Generally, one only  needs
to make references to long labels when referring to  other  manuals.  For
references within the same manual, short labels  are  sufficient,  except
when the short label itself contains a colon.

*Example*

Since  the  `\\BeginningOfBook'   command   for   this   manual   defines
<name-of-book> to be `gapmacro',
the long label for  the  current  section  is
`gapmacro:Labels and References'
and so a reference  to  this  section  within
this manual might be:  `Section  \"Labels  and  References\"'  (which  is
typeset as: Section "Labels and References"). From another manual, a long
label reference is required.

*Another example*

A section of this document has the title
``Structuring the text: Chapters and Sections'',
which contains a colon. Hence, to refer to that section, one *must* use a
long label: 
\begintt
Section "gapmacro:Structuring the text: Chapters and Sections"
\endtt
produces: Section "gapmacro:Structuring the text: Chapters and Sections".

*Note*

In actual fact long labels are first sanitised by conversion to lower case
and removal of superfluous white space (multiple blanks and new lines are
converted to a single space). The same sanitisation process is applied to
references. Thus, 
\begintt
Section "gapmacro:Structuring the text:                          Chapters
and Sections"
\endtt
also produces: 
Section "gapmacro:Structuring the text:                          Chapters
and Sections".
So, don't worry about references to labels being broken over  lines
and think of  them  as  being  case-insensitive,  except  that  the  HTML
converter  requires  that  one  respects  case  for  the   <name-of-book>
component of a long label.

%%%%%%%%%%%%%%%%%%%%%%%%%%%%%%%%%%%%%%%%%%%%%%%%%%%%%%%%%%%%%%%%%%%%%%%%%
\Section{TeX Macros}

\indextt{\\lq}\indextt{\\pif}\indextt{\\rq}\indextt{\\accent127}
\indextt{\\>}\indextt{\\.}\indextt{.}\indextt{\\index}\indextt{\\indextt}
\indextt{\\atindex}\index{indexing commands}\index{command mark-up}
\index{subsection mark-up}\indextt{\\fmark}\indextt{\\kernttindent}
\indextt{\\URL}\indextt{\\Mailto}
\indextt{A!Attribute mark-up}\indextt{F!Function mark-up}
\indextt{P!Property mark-up}\indextt{O!Operation mark-up}
\indextt{C!Category mark-up}\indextt{R!Representation mark-up}
\indextt{V!(global) Variable mark-up}
\indextt{\\Package}\indextt{\\package}
As the manual pages are also used as on-line help, and are  automatically
converted to HTML, the use of special {\TeX} commands should be  avoided.
The following macros can be used to structure  the  text,  the  mentioned
fonts are used when printing the manual, however  the  on-line  help  and
HTML are free to use other fonts or even colour. Since,  the  plain  text
on-line help, doesn't have special  fonts,  it  leaves  in  much  of  the
markup, including the left  and  right  quotes  that  surround  something
intended to be displayed in typewriter  type,  the  angle  brackets  that
surround something intended to appear in italics,  and  the  dollar-signs
enclosing mathematics; you will need to keep that in  mind  when  reading
the following section.

\beginitems

`\lq<text>\pif' &
    sets <text> in `typewriter style'.
    This is typically used to denote {\GAP} keywords such  as  `for'  and
    `false' or variables that are not  arguments  to  a  function,  e.g.,
    `{\lq}for\pif' produces `for'. 
    See also `\<<text>>'. 
    Use `\\\<' to get a ``less than'' sign.

`\lq\lq<text>\pif\pif' &
    encloses <text> in double quotes, 
    e.g., `\lq{\lq}double-quoted text\pif\pif' produces 
    ``double-quoted text''.
    In particular, `\lq\lq<text>\pif\pif' does *not* set  `\lq<text>\pif'
    in typewriter  style;  use  `{\lq\{{\lq}text\pif\}\pif}'  to  produce
    `{`text'}'. Double quotes are mainly used to mark a phrase which will
    be defined later or is used in an uncommon way.

`\\lq' &
    sets a single left quote: `\lq'. 
    For a phrase <text> that is to be defined later  or  is  used  in  an
    uncommon way, please use `\lq\lq<text>\pif\pif' (which encloses <text>
    in double quotes rather than single quotes).

`\\rq', `\\pif' &
    each set a single apostrophe (right quote): `\pif'.  For the HTML and 
    on-line manuals `\\accent19\{\}' also sets an apostrophe; however the
    {\TeX}-derived manuals produce an acute-d blankspace (what it in fact
    is).

`\\accent127' &
    sets an umlaut, e.g. `\\accent127a' produces  `\accent127a'.  Do  not
    use the shorthand `\\\"'  (otherwise  the  HTML  converter  will  not
    translate it properly).

`\<<text>>' &
    sets <text> in italics. This can also be used  inside  `\$...\$'  and
    `{`...'}'. Use `\\\<' to get a ``less than'' sign. `\<...>'  is  used
    to denote a variable which is an argument of a  function;  a  typical
    application is the description of a function:

\begintt
\>Group( <gens> ) F
The function `Group' constructs a group generated by <gens>.
\endtt

  & 
    The `F' at the end of the first line in the above  example  indicates
    that `Group' is a function (see the `\\>' entry, below).

`*<text>*' &
    sets <text> in *emphasized style*.

`\$a.b\$' &
    Inside math mode, you can use `.'  instead  of  `\\cdot'  (a  centred
    multiplication dot). Use `\\.' for a full stop inside math mode.  For
    example, `\$a.b\$' produces $a.b$ while `\$a\\.b\$' produces $a\.b$.

`\\cite\{...\}' &
    produces  a     reference     to    a    bibliography   entry    (the
    `\\cite[...]\{...\}' option of La{\TeX} is *not* supported).

`"<label>"' &
    produces a reference to <label>. Labels are generated by the commands
    `\\Chapter', `\\Section' (see~"Labels  and  References"),  and  `\\>'
    commands (see below).

`\\index\{<index-entry>\}' &
    defines an index entry <index-entry>. Besides appearing in the index,
    <index-entry> is also written to the section index file  `manual.six'
    used by the on-line help. An exclamation mark (`!'), if  present,  is
    used to partition <index-entry>  into  main  entry  (left  part)  and
    subentry (right part)  components,  in  the  index.  {\TeX}  converts
    <index-entry> to lowercase and sets it in roman type, in  the  index.
    The HTML converter respects  case  and  uses  the  default  font,  in
    producing the HTML manual index.  <index-entry>  must  be  completely
    free of special characters and font changing commands;  if  you  need
    special fonts, characters or  commands  use  one  of  `\\indextt'  or
    `\\atindex'.

  & Note that for  the  HTML  converter  to  process  indexing   commands
    (`\\index', `\\indextt' and `\\atindex') correctly they *must* be  on
    lines of their own. There can be several  indexing  commands  on  the
    same line, but there should be no horizontal whitespace  before  each
    indexing command, and if an indexing command needs to be broken  over
    lines place a `\%' at the point of the break at the end of  the  line
    to mark a ``continuation''.

  & For the HTML converter it works best to  put  indexing  commands  all
    together at the beginning of a paragraph, rather than strewn  between
    lines of a paragraph. However, for the {\TeX}-produced manuals  after
    a maths display one gets a rogue space if you  do  this  (this  is  a
    bug); you can work around the bug by putting at least one word of the
    paragraph followed by your line(s) of indexing commands.

  & *Note* also that  indexing  commands  do  *not*  produce  labels  for
    cross-references; they *only* produce entries for the  index.  Labels
    are  *only*  produced   by   the   chapter   (`\\Chapter'),   section
    (`\\Section') and subsection (`\\>') commands.

`\\indextt\{<index-entry>\}' &
    is the same as `\\index\{<index-entry>\}', except that  <index-entry>
    is set by {\TeX} in  typewriter  style,  respecting  case;  the  HTML
    converter  sets   <index-entry>   in   the   default   font.   Again,
    <index-entry> should be completely free  of  special  characters  and
    font changing commands, and `!' may be used for  sub-entries  in  the
    same way as for  `\\index'.  Note  that  a  sub-entry  component,  if
    present, is *not* set in typewriter  style  for  the  {\TeX}-produced
    manuals; if you want that it is, use `\\atindex'.

`\\atindex\{<sort-entry>\}\{|indexit\}' &
    is simply a special form of the `\\index' command that  tells  {\TeX}
    to typeset the page number in italics.

`\\atindex\{<sort-entry>\}\{@<index-entry>\}' &
    The   HTML   converter   treats   this   command   as   if   it   was
    `\\index\{<index-entry>\}',  except  that  it  strips  out  any  font
    information and  sets  it  in  the  default  font,  but  nevertheless
    respects case. <index-entry> may have `|indexit' at the end which is
    ignored by the HTML converter.

  & The {\TeX}-produced manuals set  the  index  entry  as  <index-entry>
    respecting font and case, and list it according to <sort-entry>. If a
    sub-entry is required then it should  be  present  behind  a  `!'  in
    *both*  the  <sort-entry>  and  <index-entry>;  the  only  difference
    between the sub-entry in <sort-entry> and that in  <index-entry>,  is
    that the <sort-entry> sub-entry should be  stripped  of  mark-up  and
    font changing command. The <index-entry> component  is  ignored  when
    constructing the `manual.six' files, and is also ignored by the  HTML
    converter. Anything after an `!'  in  <sort-entry>  is  ignored  when
    constructing the `manual.idx' file that is  processed  by  MakeIndex.
    Macros like `\{\\GAP\}' are allowed in  <index-entry>.  However,  any
    `\lq'  that  appears  in  <index-entry>   *must*   be   preceded   by
    `\\noexpand';  <sort-entry>  must  be  completely  free  of   special
    characters and font changing commands.

  & In general, one should make <sort-entry> the  same  as  <index-entry>
    modulo fonts and other mark-up, e.g.,

\begintt
\atindex{Fred!Nerk}{@\noexpand`Fred'!\noexpand`Nerk'}
\endtt

`\{\\GAP\}' &
    typesets {\GAP}.

`\\package\{<pkg>\}' &
    typesets <pkg> in the font correct  for {\GAP}  packages  (respecting
    case). This is intended for cross-referencing  other {\GAP} packages.
    There  is  also  the  command  `\\Package\{<mypkg>\}'  command  which
    defines a macro `\\<mypkg>' so that  when  you  type  `\{\\<mypkg>\}'
    (please include the curly braces) the text <mypkg> is typeset in  the
    right way for {\GAP} packages. The `\\Package' command should normally
    be included in one's `manual.tex' file (see~"The Main File") and just
    allows  one  to  type  `\{\\<mypkg>\}'   rather   than   the   longer
    `\\Package\{<mypkg>\}'  as  one  is  frequently  likely  to  do  when
    formulating one's own {\GAP} package documentation.  So, just  to  be
    clear about  the  difference  between  `\\Package'  and  `\\package',
    `\\Package\{<mypkg>\}' defines a macro `\\<mypkg>'  but  produces  no
    text, and `\\package\{<pkg>\}' produces <pkg> set in the font that is
    right for {\GAP} packages.

`\\>' &
    produces a subsection. 
    The line following the `\\>' entry must either contain another `\\>'
    entry (in which case the further entries are assumed to be variants
    and do not start a new subsection) or must be empty.
    The description text will follow this empty line.

  & There are several forms of the `\\>' command.
    In all forms, a label and index entry are generated; the HTML 
    converter uses the label to form an index entry, respecting case
    and setting in roman type.
    If the next non-space character is not a left quote (`\lq') it is 
    assumed that the subsection is for a ``function''; we exhibit 
    these forms first. 

`\\><func>' &
    While this form is supported; it  is  discouraged.  If  <func>  is  a
    0-argument function, <func> should be followed by an  empty  pair  of
    brackets (see `\\><func>(<args>)' below). If  <func>  is  actually  a
    global variable  then  `\\>\lq<global-var>{\rq}  V'  should  be  used
    instead (see below). In order for this form to  be  parsed  correctly
    the remainder of the line to the right of <func> must  be  empty.  It
    generates <func> as both a label and index entry; <func>  appears  as
    is, in typewriter type in the {\TeX}-derived manual index.

`\\><func>(<args>)' &
    The macro uses the brackets  after  <func>  to  parse  the  arguments
    <args>. Thus, it is necessary for the function to  use  brackets  and
    for the arguments to have none. (We use the term ``function'' loosely
    here to mean ``a {\GAP} command with arguments''; we really  mean  an
    object  that  {\GAP}  knows   as   a:   ``Function'',   ``Property'',
    ``Operation'',  ``Category'',  or  ``Representation''  ---  but   not
    ``Variable'', since a ``Variable''  does  not  have  arguments.)  The
    label and index entry generated consists of the text between the  `>'
    and opening bracket. The index entry is  set  as  is  (i.e.\  without
    conversion to lowercase) in typewriter  type  in  the  {\TeX}-derived
    manual index. Here is an example of how to use `\\>'; the index entry
    is ```Size'''  (in  typewriter  type,  with  mixed  case  preserved).

\begintt 
\>Size( <obj> ) A
\endtt

  & The `A' indicates that `Size' is an  ``Attribute''.  Instead  of  `A'
    there can be `F', `P', `O', `C', or `R' which indicate that a command
    is  a  ``Function''  (probably  the   most   common),   ``Property'',
    ``Operation'', ``Category'', or ``Representation'', respectively. For
    the forms of  the  `\\>'  command  followed  by  a  left  quote,  `V'
    indicating  ``Variable''  (an  object  without  arguments),  is  also
    possible.  (See Section ``Manual Conventions'' and
    Chapter ``Types of Objects'' in the {\GAP} Reference Manual).

`\\><func>(<args>)!\{<sub-entry>\}' &
    This is a special form of the previous command, that  forms  a  label
    `<func>!<sub-entry>' and an index entry with main entry  <func>  (set
    in typewriter type and respecting  case)  and  sub-entry  <sub-entry>
    (set in roman type but also respecting case).

  & The remaining forms of the command `\\>' expect to be followed by a
    `\lq'.

`\\>\lq<command>\rq\{<label>\}' &
    works  as  `\\>'  without  `\lq...\rq',  but  will  not  use  bracket
    matching; it simply displays <command> as a header, which appears  in
    typewriter type. It will use <label> as  both  the  label  and  index
    entry, and the index entry is set in  roman  type.  Whenever  <label>
    contains a `!', it is used to partition <label> into main entry (left
    part) and subentry (right part) components, in the index.

\begintt
\>`<a> + <b>'{addition}
\>`Size( <obj> )'{size} A
\endtt

  & In the first of the examples immediately above,  the  first  form  of
    `\\>' cannot be used because no brackets occur.  Also,  observe  that
    there is no command type (it's not appropriate here); {\TeX} does not
    need it to correctly parse a `\\>'  entry,  in  general.  The  second
    example differs from our previous `Size' example, in that  the  index
    entry will be typeset  as  ``size''  (in  roman  type),  rather  than
    ``{`Size'}''. Also, the index entry is always converted to lowercase,
    no matter what case or mixed case was used.

`\\>\lq<command>\rq\{<label>\}!\{<sub-entry>\}' &
    is equivalent to: `\\>\lq<command>\rq\{<label>!<sub-entry>\}'.

`\\>\lq<command>\rq\{<label>\}@\{<index-entry>\}' &
    works as `\\>\lq<command>\rq\{<label>\}', except that it uses <label>
    for sorting the index entry and the index entry itself is printed  as
    <index-entry>. References to the subsection use <label>.  (Note  that
    the HTML converter ignores everything after an `@'  symbol  in  these
    commands,  essentially  treating  the   command   as   if   it   were
    `\\>\lq<command>\rq\{<label>\}'. However,  the  HTML  converter  also
    always preserves the case in a label. ) Here are two examples.

\begintt
\>`Size( <obj> )'{size}@{`Size'} A
\>`Size( GL( <n>, <q> ) )'{Size!GL( n, q )}@{`Size'! `GL'( \noexpand<n>, \noexpand<q> )} A
\endtt

  & The first of these examples is equivalent to ``{`\\>Size( \<obj> )'}''.
    For the second example, it was necessary  to  use  `\lq'  and  `\rq',
    since the argument contained  brackets.  Note  that  `\\noexpand'  is
    needed  before  `\<'  here,  but  not  needed  before  `\lq'  in  the
    <index-entry> argument. Otherwise, the rules for sub-entries are  the
    same as for `\\atindex'.

`\\>\lq<global-var>{\rq} V' &
    This is actually a short-hand for: 
    ``{`\\>\lq<global-var>\rq\{<global-var>\}@\{\lq<global-var>\rq\} V'}''
    to  save  you  some  typing  when  creating  subsections  for  global
    variables, i.e., <global-var>  is  the  label  and  the  index  entry
    appears in typewriter type, with mixed case preserved.

`\\)\{\\fmark ...\}' &
    is like `\\>' except that it produces no label and index entry. It is
    `\\fmark' that produces  the  filled  in  right  arrow.  Omitting  it
    produces a line in typewriter type.

`\\)\{\\kernttindent ...\}' &
    is useful for producing a line in typewriter  type,  that  you  might
    otherwise have typed between `\\begintt' and `\\endtt', but where you
    actually  want  the  {\TeX}  macros  and  variables  `\<...>'  to  be
    interpreted.

`\\URL\{<url>\}' &
    prints the WWW URL <url>. In the HTML version this will be a HREF link.

`\\Mailto\{<email>\}' &
    prints the email address <email>. In the HTML version this will be  a
    `mailto' link.

\enditems

*Note:*
When a {\TeX} macro is followed by a space, {\TeX} generally swallows  up
the space; one way, and it is the {\GAP}-preferred way, of preventing the
space being swallowed up, is by enclosing the macro in `\{...\}'. When  a
macro is often followed by a space, it's a good  habit  to  get  into  to
*always* enclose that macro in `\{...\}' (the braces do nothing when  the
macro is not followed by a space, and prevent {\TeX} from  swallowing  up
the space, otherwise). Thus the macro for {\GAP} should *always* be typed
`\{\\GAP\}'. Similarly, macros like `\\lq',  `\\rq'  and  `\\pif'  should
probably always  appear  in  braces;  moreover  the  word  ``don{\pif}t''
typeset via ``{`don\{\\pif\}t'}'' will actually be interpreted  correctly
by the on-line browser.

%%%%%%%%%%%%%%%%%%%%%%%%%%%%%%%%%%%%%%%%%%%%%%%%%%%%%%%%%%%%%%%%%%%%%%%%%
\Section{TeX Macros for Domains}

\indextt{\\N}\indextt{\\Z}\indextt{\\Q}\indextt{\\R}
\indextt{\\C}\indextt{\\F}\indextt{\\calR}
The following macros are required for the following common domains:

\beginlist

\item{`\\N'}
the natural  numbers  (you  should  probably  indicate  whether  by  your
convention $\N$ includes zero or not, when using this);

\item{`\\Z'}
the integers;

\item{`\\Q'}
the rational numbers;

\item{`\\R'}
the real numbers;

\item{`\\C'}
the complex numbers;

\item{`\\F'}
a field; and

\item{`\\calR'}
a general domain e.g.~a ring.

\endlist

%%%%%%%%%%%%%%%%%%%%%%%%%%%%%%%%%%%%%%%%%%%%%%%%%%%%%%%%%%%%%%%%%%%%%%%%%
\Section{Examples, Lists, and Verbatim}

\index{list environment!description}
\indextt{\\beginitems}\indextt{\\enditems}
% produce itemized texts with 3pc hanging indentation
In order  to   produce  a  list   of  items with   descriptions  use  the
`\\beginitems', `\\enditems' environment, i.e. this is a ``description''
environment in the parlance of La{\TeX} and HTML.

For example, the   following  list describes   `base', `knownBase',  and
`reduced'.
The different item/description pairs must be separated by blank lines.

\begintt
\beginitems
`base' &
    must be a  list of points ...

`knownBase' &
    If a base for <G> is known in advance ...

`reduced' (default `true') &
    If this is `true' the resulting stabilizer chain will be ...
\enditems
\endtt

This will be printed as
\beginitems
`base' &
    must be a  list of points ...

`knownBase' &
    If a base for <G> is known in advance ...

`reduced' (default `true') &
    If this is `true' the resulting stabilizer chain will be ...
\enditems

\index{list environment!compact description}
\indextt{\\beginlist}\indextt{\\endlist}\indextt{\\item}\indextt{\\itemitem}
In order to produce a list in a more compact format,
use the `\\beginlist', `\\endlist' environment.

An example is the following list.

\begintt
\beginlist
\item{(a)}
    first entry
\item{(b)}
    second entry
\itemitem{--}
    a sub-item of the second entry
\itemitem{--}
    another sub-item of the second entry
\item{(c)}
    third entry
\endlist
\endtt

It is printed as follows.
\beginlist
\item{(a)}
    first entry
\item{(b)}
    second entry
\itemitem{--}
    a sub-item of the second entry
\itemitem{--}
    another sub-item of the second entry
\item{(c)}
    third entry
\endlist

\index{list environment!unordered}\index{list environment!ordered}
The above example will take advantage of the ordered and unordered list
environments in the HTML version, with the addition of slightly more mark-up.
First, we present the example again with that additional mark-up, and then
we explain how it works.

\begintt
\beginlist%ordered{a}
\item{(a)}
    first entry
\item{(b)}
    second entry
\itemitem{--}%unordered
    a sub-item of the second entry
\itemitem{--}
    another sub-item of the second entry
\item{(c)}
    third entry
\endlist
\endtt

It is printed as follows (of course, you should see no difference in the
{\TeX}-produced and on-line versions of this manual).
\beginlist%ordered{a}
\item{(a)}
    first entry
\item{(b)}
    second entry
\itemitem{--}%unordered
    a sub-item of the second entry
\itemitem{--}
    another sub-item of the second entry
\item{(c)}
    third entry
\endlist

In the HTML version the above example is interpreted as a nested list. The
outer list is interpreted as an *ordered* list. The HTML standard provides
5 different types of ordered list, and these mirror the types provided by the
`enumerate' La{\TeX} package. To signify that the outer list was *ordered*
the comment `\%ordered' was added after `\\beginlist'. If there is no further
markup the list is numbered in the default manner, namely with integers.
Otherwise, following `\%ordered' there should be one of the following:

\beginlist
\item{`\{1\}'}
indicates the list should be numbered with integers (the default obtained
when there is nothing following `\%ordered');

\item{`\{a\}'}
indicates the list should be numbered with lowercase letters (`a', `b',
\dots);

\item{`\{A\}'}
indicates the list should be numbered with uppercase letters (`A', `B',
\dots);

\item{`\{i\}'}
indicates the list should be numbered with lowercase roman numerals (`i', `ii',
\dots); and finally

\item{`\{I\}'}
indicates the list should be numbered with uppercase roman numerals (`I', `II',
\dots).

\endlist

The `\\beginlist' of the above example was followed by `\%ordered\{a\}' and
so the list is numbered using lowercase letters in the HTML version and using
the ordered list environment (rather than the description environment). 

Occasionally, it is necessary to break from a list, add some explanatory
text and then restart the list, and resume numbering the items from where
you left off. To do this follow the comment mark-up already mentioned by an
*integer* in curly braces, i.e.~if the outer list should actually start at
`c' then you would need to have `\%ordered\{a\}\{3\}' after `\\beginlist'
because `c' is the 3rd letter of our alphabet. Note that, for an
integer-numbered list not starting at 1, you must have the full markup; you
cannot omit the `\{1\}' after `\%ordered' in this case.

The inner list of the above example is an *unordered* list (this corresponds
to a plain `itemize' environment in La{\TeX}). To indicate this the first
`\\itemitem' above was followed by `\%unordered'.

Of course, to get an unordered outer list, one would start the list with
`\\beginlist\%ordered', and to get an ordered inner list numbered with
lowercase letters the first `\\itemitem' would need to be followed by
`\%ordered\{a\}', i.e.~the same syntax is used for the comment added after
a `\\beginlist' and after the first `\\itemitem' in a sequence of
`\\itemitem's.

*Notes*

\beginlist%ordered
\item{1.} Only lists to a maximum depth of two are supported.

\item{2.} You cannot change the type of a sublist halfway through. Only
the comment after the first `\\itemitem' in a sequence is interpreted.

\endlist

\index{verbatim environments}
\indextt{\\begintt}\indextt{\\endtt}
\indextt{\\beginexample}\indextt{\\endexample}
% verbatim text in typewriter style
There are two types of *verbatim* environments.
Example {\GAP} sessions are typeset in typewriter style
using the `\\beginexample', `\\endexample' environment.

\begintt
\beginexample
gap> 1+2;
3
\endexample
\endtt

typesets the example
\beginexample
gap> 1+2;
3
\endexample

\indextt{\\beginexample!indicating unstable output}
Examples whose output may vary should be introduced with `\%notest', e.g.

\begintt
%notest
\beginexample
gap> Exec("date");
Sun Oct  7 16:23:45 CEST 2001
\endexample
\endtt

typesets in all manual versions in the same way:

%notest
\beginexample
gap> Exec("date");
Sun Oct  7 16:23:45 CEST 2001
\endexample

but the automatic manual checker knows to treat the example differently.

Non-{\GAP} examples are typeset in typewriter style
using the `\\begintt', `\\endtt' environment.

*Notes*

\beginlist%ordered

\item{1.}
The manual style will automatically indent examples. It also will break
examples which become too long to fit on one page. If you want to encourage
breaks at specific points in an example, end the example with `\\endexample'
and immediately start a new example environment with `\\beginexample' on
the next line.

\item{2.}
To typeset a pipe symbol `|' in the `\\begintt', `\\endtt' environment or
`\\beginexample', `\\endexample' you need to actually type `||'.

\endlist

%%%%%%%%%%%%%%%%%%%%%%%%%%%%%%%%%%%%%%%%%%%%%%%%%%%%%%%%%%%%%%%%%%%%%%%%%
\Section{Tables, Displayed Mathematics and Mathematics Alignments}

\index{tables}\index{mathematics displays}\index{mathematics alignments}
\indextt{\\matrix}
Tables  should  normally  be  set  using   the   `\\begintt',   `\\endtt'
environment. This means that one should enter the appropriate white space
so that columns line up. Note that to get a  vertical  line  `|'  in  the
`\\begintt', `\\endtt' environment  one  must  actually  type  `||'.  The
reason for setting tables this way is so that both the HTML converter and
{\GAP}'s built-in  text  browser  have  no  trouble  in  displaying  them
correctly.

The HTML converter when used with its `-t' option (which causes it to use
TtH to translate mathematics) usually does a reasonable job of converting
mathematics  displays  and  mathematics  alignments.  To  help   {\GAP}'s
built-in text browser, however, one should follow a few rules:

\beginlist%unordered

\item{--} Place the `\$\$'s that begin and end the mathematics display on
lines of their own. (If you don't do this it will  be  displayed  in  the
same way as ordinary in-line mathematics.)

\item{--} Use only the `\\matrix\{ ..  \}'  environment  for  mathematics
alignments. The `\\matrix\{' starting the alignment should be on  a  line
on its own, (flush left and no trailing whitespace). The `\}' closing the
environment should also be on a line of its  own.  The  built-in  browser
doesn't do anything special to  line  things  up;  you  must  insert  the
whitespace where it's needed. Any `\\hfill' macros you add  to  help  the
line things up in the {\TeX} and HTML formats is ignored by the  {\GAP}'s
built-in text browser. The `\\matrix\{ .. \}' environment should be  used
even when one might like to use {\TeX}'s `\\cases\{ .. \}' environment.

\endlist

The following example shows a typical usage of  the  `\\matrix\{  ..  \}'
environment (in particular, it shows how one can use it  to  avoid  using
the `\\cases\{ .. \}' environment). Observe,  how  sufficient  whitespace
has been added in order that alignment is maintained by {\GAP}'s built-in
text browser. (Recall that `\\right.' which produces nothing is  required
to match `\\left\\\{'.)

\begintt
From a theorem of Gauss we know that
$$
b_N = \left\{
\matrix{
\frac{1}{2}(-1+\sqrt{N})    &{\rm if} &N \equiv 1  &\pmod 4\cr
\frac{1}{2}(-1+i \sqrt{N})  &{\rm if} &N \equiv -1 &\pmod 4\cr
}
\right.
$$
\endtt

The example produces \dots

From a theorem of Gauss we know that
$$
b_N = \left\{
\matrix{
\frac{1}{2}(-1+\sqrt{N})    &{\rm if} &N \equiv 1  &\pmod 4\cr
\frac{1}{2}(-1+i \sqrt{N})  &{\rm if} &N \equiv -1 &\pmod 4\cr
}
\right.
$$

%%%%%%%%%%%%%%%%%%%%%%%%%%%%%%%%%%%%%%%%%%%%%%%%%%%%%%%%%%%%%%%%%%%%%%%%%
\Section{Testing the Examples}

For purposes of automatically checking the manual,
the {\GAP} examples in one chapter (the text between `\\beginexample'
and `\\endexample') should produce the same output, up to line breaks
and whitespace, whenever they are run in the same order
immediately after starting {\GAP}
(this will ensure that the global random number generator is initialized
to the *same* values).
For more details,
see the last paragraph of ``Starting and Leaving GAP''
in the {\GAP} Tutorial.

To permit this automatic running,
examples that shall produce error messages should be put between
`\\begintt' and `\\endtt'
such that they will not be seen by this automatic test.

The automatic test also requires that examples are not indented
in the files;
in the printed manual,
the lines between `\\beginexample' and `\\endexample'
and the lines between `\\begintt' and `\\endtt' are automatically
indented.

% mention how this test can be executed!

%%%%%%%%%%%%%%%%%%%%%%%%%%%%%%%%%%%%%%%%%%%%%%%%%%%%%%%%%%%%%%%%%%%%%%%%%
\Section{Usage of the Percent Symbol}

\indextt{\%}
The `\%' symbol has a number of very specific uses. Take  care  that  you
use it correctly. These uses are:
\beginlist%ordered
\item{1.}
A line *beginning* with 16 (or more) `\%' symbols marks the  *end*  of  a
section, or the *end* of a chapter introduction  (which  may  be  empty).
Such  a  line  *must*  precede  *every*  `\\Section'  (see~"Chapters  and
Sections").

\item{2.}
A `\%' at the beginning of a line tells {\TeX} that the line is a comment
and is to be ignored by {\TeX}, *except* in  the  verbatim  environments:
`\\begintt..\\endtt'   and    `\\beginexample..\\endexample'.    However,
`\%display' or `\%enddisplay'  commands  have  special  meaning  for  the
on-line text help browser and for the HTML converter and may  temporarily
alter the meaning of an initial `\%' for these (see~"Catering  for  Plain
Text and HTML Formats" for details); otherwise the meaning of an  initial
`\%' is the same as for {\TeX}.

\index{continuation}
\item{3.}
A `\%' at the end of a line marks a  ``continuation'',  *except*  in  the
situation mentioned in item 4. A ``continuation'' may be needed for lines
of  indexing  commands  (`\\index',  `\\indextt'  or  `\\atindex').  Such
commands *must* occur on lines of their  own  (see~"TeX  Macros"),  *not*
mixed with text, and there must not be any superfluous whitespace (modulo
the next statement). Occasionally an indexing  command  is  too  long  to
easily fit on a line; this is where a continuation is desirable;  a  `\%'
at the end of such a line indicates that the line is to  be  joined  with
the next line after removal of the `\%' symbol and any initial whitespace
on the next line (this is what  {\TeX}  does!  \dots~and  we  mimic  this
behaviour for both the on-line text help browser and the HTML manuals).

A ``continuation'' may also  be  necessary  for  subsections,  i.e.~lines
beginning with `\\>' or `\\)' (again see~"TeX Macros"); the usage  is  as
for indexing line continuations.

\item{4.}
A line ending with a `\%' that is not  an  indexing  command  line  or  a
subsection line that after any  initial  whitespace  is  removed  matches
*exactly* `\{\%' or `\}\%', or begins with `\{\\' or `\\' and is followed
by a letter, is  ignored  by  both  the  on-line  browser  and  the  HTML
converter. This is intended  to  screen  the  on-line  browser  and  HTML
converter from {\TeX} commands  such  as  `\\obeylines',  `\\begingroup',
`\\def'   etc.,   without    having    to    resort    to    using    the
`\%display\{tex\}..\%enddisplay' environment.

\endlist

*Warning.* In view of items 3.~and 4.~above, avoid using a  `\%'  at  the
end of a line  unless  you  really  need  it,  and  it  fits  into  those
categories. In particular, do *not* put a `\%' at the end of an  indexing
command line that is immediately followed by a line of  text;  otherwise,
the text line will not appear in the HTML manual or on-line via the  text
help browser. Similarly, do not put a `\%' line at the end of a text line
that is immediately followed by an indexing command line; this causes the
indexing command line to be ignored by the HTML converter. For  the  HTML
converter it works best to put indexing  commands  all  together  at  the
beginning  of  a  paragraph,  rather  than  strewn  between  lines  of  a
paragraph. However, for the {\TeX}-produced manuals after a maths display
one gets a rogue space if you do this (this  is  a  bug);  you  can  work
around the bug by putting at least one word of the paragraph followed  by
your lines(s) of indexing commands.

%%%%%%%%%%%%%%%%%%%%%%%%%%%%%%%%%%%%%%%%%%%%%%%%%%%%%%%%%%%%%%%%%%%%%%%%%
\Section{Catering for Plain Text and HTML Formats}

\indextt{\%display}\indextt{\%enddisplay}
As described in~"TeX Macros", the use of macros should be  restricted  to
the ones given in the previous sections. By doing  so,  you  should  find
that the documentation you write will still look ok in  {\GAP}'s  on-line
help (plain text format) and in the translated  HTML.  However,  in  rare
situations one might be forced to use other {\TeX} macros, for example in
order  to  typeset  a  lattice.  In  this  case  you  should  provide  an
alternative for the on-line help, and possibly also for the HTML version.
This can be done by putting in guiding commands as {\TeX} comments:

\begintt
%display{tex}
TeX version (only used by TeX manual)
%display{html}
%HTML version (only used by HTML manual)
%display{text}
%Text version (only used by the built-in manual browser)
%enddisplay
\endtt

Observe that the lines that should appear  only  in  the  {\TeX}-produced
manuals do not begin with a `\%'. For the HTML (resp.~text)  version  the
lines  begin  with  a  `\%';   each   line   of   a   `\%display\{html\}'
(resp.~`\%display\{text\}')  environment  is  printed   verbatim,   after
removing the initial `\%' symbol. The above example produces:

%display{tex}
TeX version (only used by TeX manual)
%display{html}
%HTML version (only used by HTML manual)
%display{text}
%Text version (only used by the built-in manual browser)
%enddisplay

(Note the above example will vary according to whether you are viewing it
as a {\TeX}-produced manual,  or as an HTML manual,  or via the  built-in
manual browser --- as it should!)

Sometimes one needs a `\%display' environment to be not seen  by  {\TeX},
but still interpreted normally (i.e.~not printed verbatim). The following
variant of the above provides this capability.

\begintt
%display{tex}
TeX version (only used by TeX manual)
%display{nontex}
%HTML and Text version (interpreted normally, after removing the \% symbol)
%enddisplay
\endtt

The above example produces:

%display{tex}
TeX version (only used by TeX manual)
%display{nontex}
%HTML and Text version (interpreted normally, after removing the \% symbol)
%enddisplay

It  is  permissible  to  abbreviate  any  of  the   above   by   omitting
`\%display\{tex\}', `\%display\{html\}', or `\%display\{text\}'  if  that
portion of the environment would be empty.

There are yet two more variants of conditional display. Firstly,

\begintt
%display{nonhtml}
%Text version (interpreted normally by built-in browser, after removing the
%\% symbol)
%enddisplay
\endtt

is normally used to  ensure  text  only  appears  via  the  on-line  help
browser.  If  there  is  no  initial  `\%'  it  also   appears   in   the
{\TeX}-produced manuals. The above example produces:

%display{nonhtml}
%Text version (interpreted normally by built-in browser, after removing the
%\% symbol)
%enddisplay

Finally, there is

\begintt
%display{nontext}
%HTML version (interpreted normally by HTML converter, after removing the 
%\% symbol)
%enddisplay
\endtt

which  excludes  text  from  the   on-line   help   browser.   Like   the
`\%display\{nonhtml\}' environment, if there is no initial `\%'  it  also
appears in the {\TeX}-produced manuals. The example produces:

%display{nontext}
%HTML version (interpreted normally by HTML converter, after removing the
%\% symbol)
%enddisplay

However, the use of these special environments should be avoided as  much
as  possible,  since  it  is  much  more  difficult  to   maintain   such
pseudo-duplicated documentation.

%%%%%%%%%%%%%%%%%%%%%%%%%%%%%%%%%%%%%%%%%%%%%%%%%%%%%%%%%%%%%%%%%%%%%%%%%
\Section{Umlauts}

To produce umlauts,  use  `\\accent127'  and  not  the  shorthand  `\\\"'
(otherwise the HTML converter will not translate it properly).

%%%%%%%%%%%%%%%%%%%%%%%%%%%%%%%%%%%%%%%%%%%%%%%%%%%%%%%%%%%%%%%%%%%%%%%%%
\Section{Producing a Manual}

\indextt{manual.tex}\indextt{manual.mst}\indextt{manualindex}
\indextt{manual.six}\indextt{manual.dvi}\indextt{makeindex}\indextt{bibtex}
\indextt{manual.lab}\indextt{manual.bib}\indextt{manual.bbl}
To produce a manual you will need the following files:

\beginitems
`manual.tex'&
    contains the body of the manual
    (as described in Section~"The Main File")
    and an `\\Input' command for each chapter/appendix file.

`<file1>.tex', `<file2>.tex', \dots &
    the chapter/appendix files. There must be one file for each chapter
    or appendix, and each such file should have a `\\Chapter' or
    `\\PreliminaryChapter' command. Alternatively, one can write
    `.msk' files and use `buildman.pe' to generate the corresponding
    `.tex' files (see~"Using buildman.pe").

`gapmacro.tex'&
    contains the macros for the manual.
    It must be input by an `\\input' statement (*not* and `\\Input'
    statement, which creates a Table of Contents entry) in `manual.tex'.
    You can either use the version in the `doc' directory of {\GAP}
    (use a relative path then) or make a copy.

`manual.mst'&
    is a ``configure'' file used by `makeindex' when processing
    index information in a {\TeX}-generated and `manualindex'-preprocessed
    `manual.idx' file.
    It must reside in your manual directory.

`GAPDOCPATH/manualindex'&
    is used to call `makeindex'.
    `GAPDOCPATH' is the path of the `doc' directory of your {\GAP}
    distribution.
\enditems

For bibliography information you will need a file `manual.bbl'. If you
intend to create it with Bib{\TeX}, you will need to indicate the
appropriate `.bib' file (as described in section "The main file"). Then
after running {\TeX} once over the manual, run Bib{\TeX} to create the
`manual.bbl' file.

Assuming that all necessary files are there (a `manual.lab' file for each
<book> argument of a `\\UseReferences' command, `mrabbrev.bib' and
`manualindex' in the {\GAP} `doc' directory),
on a Unix system the following calls will then produce a file `manual.dvi'
as well as a file `manual.six' which is used by the {\GAP} help functions.
If you are missing some of the needed files and don't have `CVS' access to
{\GAP}, just send an email request for them to 
\Mailto{support@gap-system.org}.

Go to the directory holding the manual. Call
\begintt
tex manual
\endtt
to produce bibliography information. Unless you provide a `manual.bbl' file
which is not produced by Bib{\TeX}, call
\begintt
bibtex manual
\endtt
to produce the `manual.bbl' file. Then run {\TeX} twice over the manual to
fill all references and produce a stable table of contents:
\begintt
tex manual
tex manual
\endtt
If you have sections which are named like commands, you may get messages
about redefined labels. At this point you can ignore these.

Now it is time to produce the index. Call
\begintt
GAPDOCPATH/manualindex manual
\endtt
which preprocesses the `manual.idx' file and then
runs `makeindex'. Provided that `manual.mst' exists, this produces a file
`manual.ind'. Finally, once again run
\begintt
tex manual
\endtt
to incorporate the index. The manual is ready.

%%%%%%%%%%%%%%%%%%%%%%%%%%%%%%%%%%%%%%%%%%%%%%%%%%%%%%%%%%%%%%%%%%%%%%%%%
\Section{Using buildman.pe}

\indextt{buildman.pe}\indextt{\\Declaration}\indextt{\\FileHeader}
Rather than write the chapter/appendix `.tex'  files  directly,  one  may
incorporate one's documentation in comments in one's {\GAP} code.  To  do
it this way, there are four ingredients:

\beginitems

`.gd' files &
{\GAP} files with `.gd' suffixes that have the documentation in  comments
(actually files with `.g' or  `.gi'  or  any  other  extension  are  also
possible, but files with extension `.gd' are the default);

`.msk' files &
which are just like the `.tex' files, and must obey all the  rules  given
for `.tex' files previously, but additionally may have `\\FileHeader'  or
`\\Declaration' commands at places where text should be inserted  from  a
`.gd' file, and with `\{\{<variable>\}\}' patterns which are replaced  by
<replacement> when written to the `.tex' file, if the configuration  file
<configfile> has a line of form: `<variable>=<replacement>';

<configfile> &
a file which defines `msfiles' (the list of `.msk' files), `gdfiles' (the
list of `.gd' files), `LIB' (the directory containing the  `.gd'  files),
`DIR' (the directory in which to put the constructed  `.tex'  files,  one
`.tex' file for each `.msk' file), and optionally  a  line  `check'  (see
below) and `<variable>=<replacement>' lines; and

`buildman.pe' &
a perl program (in the `etc' directory for those  with  `CVS'  access  to
{\GAP}), which strips the comments from the `.gd' files according to  the
`\\FileHeader'  or  `\\Declaration'  commands  in   the   `.msk'   files,
translates  any  `\{\{<variable>\}\}'  patterns  defined  by   the   file
<configfile> and constructs the `.tex' files.

\enditems

If you don't have `CVS' access and want to use `buildman.pe', just  email
\Mailto{support@gap-system.org} and ask for it.  Please  note  that
there is no obligation for package authors to `buildman.pe'; nor does  it
attract the same level of support as the rest of {\GAP}; in general, bugs
can be expected to be fixed (eventually), but no  new  features  will  be
added. Also, note that the \package{GAPDoc} package  provides  a  similar
facility.

The perl program `buildman.pe' is called as follows:

\){\kernttindent}buildman.pe -f <configfile>

*The form of <configfile>*

There is no restriction on how to name <configfile>, but by convention it
is of form `config.<something>' or `buildman.config'; <configfile> should
contain lines of form:

\){\kernttindent}msfiles=<msfile1>,<msfile2>,...,<msfilem>;
\){\kernttindent}gdfiles=<gdfile1>,<gdfile2>,...,<gdfilen>;
\){\kernttindent}LIB=<gdfile_dir>;
\){\kernttindent}DIR=<TeX_dir>;

Optionally, as mentioned above, one may also have:

\){\kernttindent}check;

which says to construct a `notfound' file  that  lists  missing  expected
data, and any number of lines of form

\){\kernttindent}<variable>=<replacement>

\goodbreak%
The file <configfile> should obey the following syntactic rules:

\beginlist%unordered

\item{--}
After   `msfiles='   there    should    be    a    comma-separated    and
semicolon-terminated list of `.msk'  files  with  the  `.msk'  extensions
removed; `buildman.pe' assumes that the `.msk' files are all  in,  or  at
least have path relative to, the  directory  in  which  `buildman.pe'  is
called.

\item{--}
Similar to the `msfiles' definition, after `gdfiles=' there should  be  a
comma-separated and semicolon-terminated list of ```.gd'''  files.  If  a
```.gd''' file really does have a  `.gd'  extension,  it  may  be  listed
without extension; otherwise the extension *must* be  included.  All  the
```.gd''' files must be  listed  with  path  relative  to  the  directory
defined by `LIB'.

\item{--}
For both the `msfiles' and `gdfiles' definitions, the lists following the
`=' may continue over several lines if  necessary,  and  any  whitespace,
parentheses (round brackets) or double-quotes characters are ignored.

\item{--}
The paths after `LIB=' and `DIR=' are assumed relative to  the  ``current
directory'', i.e. the directory in which `buildman.pe' is  executed.  For
each  <msfilei>  listed  after  the  `msfiles'   keyword,   `buildman.pe'
constructs  from  `<msfilei>.msk'  a  corresponding  `<msfilei>.tex'   in
<TeX_dir>. The `LIB' and `DIR' definitions must be on a single line.

\item{--}
The terminating `;' is optional on  the  lines  containing  the  keywords
`LIB', `DIR' or `check'.

\item{--}
Superfluous characters around any of the keywords  `msfiles',  `gdfiles',
`LIB', `DIR' or `check', but before the `=' on the  lines  where  `='  is
required,  are  ignored.  Whitespace  and  double-quotes  characters  are
ignored, everywhere.

\item{--}
The `<variable>=<replacement>' lines (if there are any)  should  have  no
other punctuation or whitespace.  These  lines  direct  `buildman.pe'  to
replace any string of form `\{\{<variable>\}\}' in  a  `.msk'  file  with
<replacement>.

\endlist

*Special `.msk' file commands*

Now  we  describe  the  special   (non-{\TeX})   commands   that   direct
`buildman.pe' to extract text from ```.gd''' files.

\beginitems

`\\FileHeader[<n>]\{<gdfile>\}' &
This command is replaced by  the  text  following  a  `\#<n>'  line  (for
positive integer <n>) in file  `<gdfile>.gd'  (or  <gdfile>  if  <gdfile>
already contains a suffix). The argument  `[<n>]'  of  `\\FileHeader'  is
optional; if it is omitted <n> is taken  to  be  1.  See  below  for  the
typical form of a fileheader extracted by the `\\FileHeader' command; the
comments in the example describe its required format.

`\\Declaration\{<func>\}[<gdfile>]\{<label>\}!\{<sub-entry>\}@\{<index-entry>\}' &
This command is replaced by a `\\>' subsection declaration  or  block  of
`\\>' declarations, and their description extracted from  a  block  in  a
```.gd''' file that starts with a line matching `\#<X> <func>', for  some
letter <X> in `F', `M',  `A',  `P',  `O',  `C',  `R'  or  `V'.  The  line
``matches'' if there is a  `(',  space,  or  newline  after  <func>.  The
argument <func> (in `\{..\}') is the only mandatory argument.

&
If present, `[<gdfile>]', says that <func> is to be  found  in  the  file
`<gdfile>.gd' (or <gdfile> if <gdfile> already contains a suffix); it  is
required only if <func> appears in more than one of the  ```.gd'''  files
listed in the file  <configfile>.  The  <gdfile>  argument  is  typically
required for distinguishing methods of operations.

&
The remaining arguments (if present) have exactly the purpose  that  they
have  in subsection   declarations, i.e.~lines of the following forms:

\)\kernttindent\\><func>!\{<sub-entry>\}
\)\kernttindent\\>\lq<command>\rq\{<label>\}
\)\kernttindent\\>\lq<command>\rq!\{<sub-entry>\}
\)\kernttindent\\>\lq<command>\rq@\{<index-entry>\}

&
(see Section~"TeX Macros"), and are used to build subsection  declaration
lines  of  these  forms.  Note  that   the   <label>,   <sub-entry>   and
<index-entry> arguments, if needed,  should  follow  the  `\\Declaration'
command (and *not* be in the  ```.gd'''  file  `\#<X>  <func>...'  lines,
where they will be indistinguishable from comments). If in the  ```.gd'''
file the `\#<X> <func>' line is followed by other `\#<Xi> <funci>' lines,
then each `\\>' subsection  declaration  formed  has  the  same  <label>,
<sub-entry> and <index-entry> arguments appended.

\enditems

Corresponding to `\\FileHeader[<n>]\{<gdfile>\}', in the  ```.gd'''  file
denoted by <gdfile>, there should be:

\)\kernttindent\#<n>
\)\kernttindent\#\#\ \ Text for \\FileHeader[<n>]\{<gdfile>\}. Each line
\)\kernttindent\#\#\ \ should have two \# characters followed by 2 blank
\)\kernttindent\#\#\ \ space characters at the left margin. The text
\)\kernttindent\#\#\ \ can and should include any necessary \{\\TeX\}
\)\kernttindent\#\#\ \ mark-up and indexing commands. 
\)\kernttindent\#\#
\)\kernttindent\#\#\ \ A fileheader may consist of any number of paragraphs.
\)\kernttindent\#\#\ \ It is terminated by a totally empty line (i.e.~a 
\)\kernttindent\#\#\ \ line devoid even of \# characters).
\)\kernttindent\#\#

Corresponding to each `\\Declaration\{<func>\}...' line of a `.msk'  file
there should be in one of the ```.gd''' files, a block of form:

\)\kernttindent\#<X>\ \ <func>( <args> ) <comment>
\)\kernttindent\#<Y>\ \ <func2>( <args2> ) <comment2>
\)\kernttindent.
\)\kernttindent.
\)\kernttindent\#<Z>\ \ <funcn>( <argsn> ) <commentn>
\)\kernttindent\#\#
\)\kernttindent\#\#\ \ description of <func>, <func2>, ..., <funcn>.
\)\kernttindent\#\#
\){\kernttindent}Declare...( "<func>" ...);
\){\kernttindent}Declare...( "<func2>" ...);
\)\kernttindent.
\)\kernttindent.
\){\kernttindent}Declare...( "<funcn>" ...);

The above block should comply with the following syntactic  rules.  Below
we use the term ``function'' in a  general  sense  to  mean  any  one  of
function  (in   the   strict   sense),   attribute,   category,   method,
representation, operation, property or variable.

\beginlist%unordered

\item{--}
$<X>, <Y>, \dots, <Z> \in \{ `A', `C', `F', `M', `R', `O', `P', `V'  \}$.
If the letter is `V' then no parentheses or arguments should  follow  the
``function name'' <funci>.

\item{--} 
The letters, <X>, <Y>, \dots, <Z> are printed in the manual. If a  letter
is `A' or `P', then also the letters `S'  and  `T'  are  printed  if  the
setter and the tester are available. If  the  letter  is  `A',  then  the
letter `M' is printed if the attribute is mutable.

\item{--}
The comments <comment>,  <comment2>,  \dots,  <commentn>  (by  convention
starting with spaced dots)  which  do  not  appear  in  the  manual,  are
optional.

\item{--}
The  <X>,  <Y>,  \dots,  <Z>  ``function  name''  lines  must  appear  on
consecutive lines, i.e.~not intermingled with text lines.

\item{--}
After the ``function name'' lines there should be text  lines  describing
the ``functions''. As  with  fileheader  text  these  text  lines  should
contain any {\TeX} mark-up and indexing commands that are necessary,  and
there should be two blank space characters between  the  `\#\#'  and  the
text. Lines starting with `\#T' (or  some  other  non-`\#'  character  in
place of `T') are ignored.

\item{--} 
It is assumed that for each ``function  name''  <func>,  <func2>,  \dots,
<funcn> there is a corresponding {\GAP} declaration (which  need  not  be
via a `Declare...' command, e.g.~it  might  be  `BindGlobal')  after  the
`\#\#' text lines (and comment lines), *and*  that  they  appear  in  the
*same* order.

\endlist

\goodbreak%

*Example*

Suppose we have a manual whose `.msk' files are in the directory `doc/build',
whose `.tex' files are created in the directory `doc/ref',
and whose {\GAP} code files are in the directory `lib',
one of them being the file `lib/algebra.gd',
which contains the following declaration:

\begintt
#############################################################################
##
#O  DirectSumOfAlgebras( <A1>, <A2> )
#O  DirectSumOfAlgebras( <list> )
##
##  is the direct sum of the two algebras <A1> and <A2> respectively of the
##  algebras in the list <list>.
##
##  If all involved algebras are associative algebras then the result is also
##  known to be associative.
##  If all involved algebras are Lie algebras then the result is also known
##  to be a Lie algebra.
##
##  All involved algebras must have the same left acting domain.
##
##  The default case is that the result is a structure constants algebra.
##  If all involved algebras are matrix algebras, and either both are Lie
##  algebras or both are associative then the result is again a
##  matrix algebra of the appropriate type.
##
DeclareOperation( "DirectSumOfAlgebras", [ IsDenseList ] );
\endtt

Further suppose that the file `doc/build/algebra.msk' contains the line:

\begintt
\Declaration{DirectSumOfAlgebras}
\endtt

The ``config'' file `doc/build/config.alg':

\begintt
@msfiles = ("algebra","algfp","alglie","mgmring");
@gdfiles = ("algebra","alghom","alglie","object","liefam","mgmring","algrep",
            "lierep");
DIR = "../ref";
LIB = "../../lib";
\endtt

specifies  `algebra.msk'  via  the   first   entry   of   `msfiles'   and
`lib/algebra.gd' via the first entry of `gdfiles' and (its directory  by)
the definition of `LIB'. Observe that there are `@' and `"'  symbols,  as
well as parentheses and whitespace, in the above ``config'' file;  *none*
of these is necessary, but they don't do any harm either. Generally,  one
calls `buildman.pe' in the same directory  that  contains  the  `msfiles'
(which is why one doesn't need to specify the  directory  containing  the
`msfiles') and the ``config'' file. Since `DIR = "../ref"', `buildman.pe'
constructs `algebra.tex' from `algebra.msk' in directory  `doc/ref'.  The
subsection  generated  in  `algebra.tex'  by  the  above  `\\Declaration'
command starts with the header:

\begintt
\>DirectSumOfAlgebras( <A1>, <A2> ) O
\>DirectSumOfAlgebras( <list> ) O
\endtt

and is followed by its description, i.e.~the  lines  beginning  with  two
hashes and two blanks, but with the hashes and blanks stripped  away,  so
that when it is processed the resulting subsection appears as:

\){\fmark}DirectSumOfAlgebras( <A1>, <A2> )\hfill O
\){\fmark}DirectSumOfAlgebras( <list> )\hfill O

is the direct sum of the two algebras <A1> and <A2> respectively of the
algebras in the list <list>.

If all involved algebras are associative algebras then the result is also
known to be associative.
If all involved algebras are Lie algebras then the result is also known
to be a Lie algebra.

All involved algebras must have the same left acting domain.

The default case is that the result is a structure constants algebra.
If all involved algebras are matrix algebras, and either both are Lie
algebras or both are associative then the result is again a
matrix algebra of the appropriate type.

*Variable replacement*

As mentioned above the ``config'' file may also contain lines that assign
variables, e.g.

\begintt
versionnumber=4.3
versionsuffix=4r3
\endtt

Occurrences of these variables in double curly braces will be replaced by
their value. For example the lines

\begintt
When `unzoo -x' is applied to {\GAP}~{{versionnumber}}'s `zoo' file
`gap{{versionsuffix}}.zoo' a directory `gap{{versionsuffix}}' is formed.
\endtt

in a `.msk' file will be replaced by:

\begintt
When `unzoo -x' is applied to {\GAP}~4.3's `zoo' file
`gap4r3.zoo' a directory `gap4r3' is formed.
\endtt

in the corresponding `.tex' file.
This feature is very handy for information that changes over time.

*Final note*

There is a document for version `0.0'  of  `buildman.pe'  that  describes
features that have either never been used or have  since  been  disabled.
Only the features described in this section can be relied  upon  to  have
currency.

\EndOfBook

