\Chapter{Group Libraries}

When you start it, {\GAP}  already knows several groups.  Currently {\GAP}
initially knows the  following groups:
{\parindent\manindent
\item{$\bullet$} some basic groups,  such as cyclic groups or symmetric
 groups (see "The Basic Groups Library"),
\item{$\bullet$} a library of primitive groups (see "Primitive permutation
groups"),
\item{$\bullet$} the transitive permutation groups  of degree  at most 23
  (see "Transitive permutation groups"),
\item{$\bullet$} the small groups of size at most 1000, excluding sizes 512
and 768 (see "The Small Groups Library"),
\item{$\bullet$} the  finite perfect groups  of size  at most  $10^6$
  (excluding 11 sizes) (see "Finite Perfect Groups"),
%\item{$\bullet$} the irreducible solvable subgroups of  $GL(n,p)$  for  $n>1$ 
%  and $p^n  \<\  256$   (see  "The  Irreducible  Solvable  Linear  Groups
%      Library"),
%\item{$\bullet$} the irreducible maximal finite integral matrix groups 
%  of dimension at most  24  (see  "Irreducible Maximal Finite Integral
%  Matrix Groups"),
%\item{$\bullet$} the  crystallographic  groups  of  dimension  at most  4 
%  (see "The Crystallographic Groups Library").
\par}

There is usually no relation between the groups in the different libraries.

Several of the libraries  are accessed in  a uniform manner.  For each of
these libraries there is a so called *selection function* that allows you
to select the list of groups that satisfy given criterias from a library.
The  *example  function* allows you to  select  one  group that satisfies
given   criteria from the library.   The  low-level *extraction function*
allows you  to extract a  single  group from a   library, using a  simple
indexing scheme. These functions are described in the sections "Selection
Functions", "Example Functions", and "Extraction Functions".

Note  that  a system administrator may  choose  to install  all, or  only
a few, or even none of the libraries.  So some of the libraries mentioned
below may not be available on your installation.

%%%%%%%%%%%%%%%%%%%%%%%%%%%%%%%%%%%%%%%%%%%%%%%%%%%%%%%%%%%%%%%%%%%%%%%%%
\Section{Primitive permutation groups}

\>PrimitiveGroup( <deg>, <nr> )

{\GAP} contains a library of primitive permutation groups which includes,
up to  isomorphism of permutation groups   (i.e., up to conjugacy  in the
corresponding symmetric group)
{\parindent\manindent
\item{$\bullet$} all   primitive permutation  groups  of  degree $\<256$,
  namely
\itemitem{$\circ$}  the  primitive permutation  groups   up to degree~50,
  calculated by C.~Sims,
\itemitem{$\circ$}  the  solvable (hence  affine)   primitive permutation
  groups of degree $\<256$, calculated by M.~Short \cite{Short92},
\itemitem{$\circ$} the insolvable affine  primitive permutation groups of
  degree $\<256$, calculated in \cite{Theissen97},
\item{$\bullet$} the non-affine primitive   permutation groups of  degree
  $\<1000$,  described   in   \cite{DixonMortimer88},  with    generators
  calculated in \cite{Theissen97}.\par}
Note that the affine primitive permutation groups of degrees 256--999 are
not included.

\>PrimitiveGroupsSims( <deg>, <nr> )
\>AllPrimitiveGroups( $<attr>_1$, $<val>_1$, \dots\ )
\>OnePrimitiveGroup( $<attr>_1$, $<val>_1$, \dots\ )

For compatibility  with earlier versions  of {\GAP}, the original list of
Sims,  with  the same numbers  and  the  names  given by  Buekenhout  and
Leemans \cite{BuekenhoutLeemans96},
is     also  included.   It is    accessed     by  the function
`PrimitiveGroupSims'.   You   can   also use  the    well-known functions
`AllPrimitiveGroups' and `OnePrimitiveGroup' with this old list.

%%%%%%%%%%%%%%%%%%%%%%%%%%%%%%%%%%%%%%%%%%%%%%%%%%%%%%%%%%%%%%%%%%%%%%%%%
\Section{Transitive permutation groups}

The transitive groups library contains representatives for all transitive
permutation groups of degree at most  23.  Two permutations groups of the
same degree are considered to be equivalent, if there is a renumbering of
points, which maps one group into the other one.  In other words, if they
lie in the save  conjugacy class  under operation  of the  full symmetric
group by conjugation.

\>TransitiveGroup( <deg>, <nr> )

'TransitiveGroup' returns the  <nr>-th transitive group of  degree <deg>.
Both <deg> and <nr> must be positive integers.   The transitive groups of
equal  degree are  sorted with  respect  to  their  size, so for  example
'TransitiveGroup( <deg>, 1 )' is a transitive group of degree and size
<deg>, e.g, the cyclic group  of size <deg>, if <deg>   is a prime.  
The arrangement of the groups, the generators and their names correspond
to the lists in \cite{ConwayHulpkeMcKay97}

\>'AllTransitiveGroups( <fun1>, <val1>, <fun2>, <val2>, ... )'

'AllTransitiveGroups' returns   a list containing   all transitive groups
that have the properties given as arguments.   Each property is specified
by  passing a pair   of arguments, the   first being a function, and  the
second being a   value or a  list of  values.  'AllTransitiveGroups' will
return  all groups  from the  transitive  groups  library, for which  all
specified functions have the specified values.

If the degree is not restricted to 23 at most, 'AllTransitiveGroups' will
print a warning.

\>'OneTransitiveGroup( <fun1>, <val1>, <fun2>, <val2>, ... )'

'OneTransitiveGroup' returns one transitive group that has the properties
given  as  argument.  Each  property  is specified by   passing a pair of
arguments, the first being a function, and the second  being a value or a
list of values.   'OneTransitiveGroup'  will return one groups   from the
transitive  groups library, for which  all  specified functions have  the
specified values.  If no such group exists, 'false' is returned.

If the degree is not restricted  to 23 at most, 'OneTransitiveGroup' will
print a warning.

'AllTransitiveGroups'  and 'OneTransitiveGroup'   recognize the following
functions and get the corresponding properties from a precomputed list to
speed up  processing:

'NrMovedPoints', 'Size', 'Transitivity', and
'IsPrimitive'.  You do not  need to pass   those functions first,  as the
selection function picks the these properties first.

This library was computed by Gregory Butler, John McKay, Gordon Royle and
Alexander Hulpke. The  list    of transitive groups   up   to degree  11
was  published in \cite{ButlerMcKay83}, the list of degree  12 was published
in \cite{Royle87}, degree 14 and 15 were published in \cite{Butler93} and
degrees 16-22 in \cite{Hulpke96}.

The library was  brought into {\GAP} format by  Alexander  Hulpke, who is
responsible for all mistakes.

\beginexample
gap> TransitiveGroup(10,22);
S(5)[x]2
gap> l:=AllTransitiveGroups(NrMovedPoint,12,Size,1440,IsSolvable,false);
[ S(6)[x]2, M_10.2(12) = A_6.E_4(12) = [S_6[1/720]{M_10}S_6]2 ]
gap> List(l,IsSolvable);
[ false, false ] |
\endexample

\>'TransitiveIdentification( <G> )'

Let <G> be a permutation group, acting transitively on a set  of up to 23
points.  Then 'TransitiveIdentification' will return the position of this
group in the transitive  groups library.  This means,  if <G> operates on
$m$ points and    'TransitiveIdentification'  returns $n$,  then <G>   is
permutation isomorphic to the group 'TransitiveGroup(m,n)'.

\beginexample
gap> TransitiveIdentification(Group((1,2),(1,2,3)));
2
\endexample

%%%%%%%%%%%%%%%%%%%%%%%%%%%%%%%%%%%%%%%%%%%%%%%%%%%%%%%%%%%%%%%%%%%%%%%%%
\Section{Finite Perfect Groups}%
\index{perfect groups}

The {\GAP} library of finite  perfect groups provides, up to isomorphism,
a list of all perfect groups whose sizes are less than  $10^6$  excluding
the following sizes:
{\parindent\manindent
\item{$\bullet$}
      For $n = 61440$, 122880, 172032, 245760, 344064, 491520, 688128, or
      983040,  the perfect groups  of size  $n$  have not completely been
      determined yet.  The library  neither provides  the number of these
      groups nor the groups themselves.
\item{$\bullet$}
      For  $n = 86016$,  368640,  or  737280,  the library  does not  yet
      contain  the perfect groups  of size  $n$,  it  only provides their
      their numbers which are 52, 46, or 54, respectively.
\par}

Except for these eleven sizes, the list of altogether 1096 perfect groups
in the library is  complete.  It relies  on results of Derek~F. Holt  and
Wilhelm  Plesken which are published  in  their book {\it Perfect Groups}
\cite{HP89}.  Moreover, they have supplied us with files with presentations
of  488 of the groups.  In terms of  these, the  remaining 607 nontrivial
groups in  the  library can  be   described as 276 direct   products, 107
central  products, and  224  subdirect    products.  They are    computed
automatically by suitable {\GAP} functions whenever they are needed.

We are grateful to Derek Holt and Wilhelm Plesken for making their groups
available to the {\GAP} community  by contributing their files. It should
be noted that  their book contains a  lot of further information for many
of the library groups.  So we would like  to recommend  it to any  {\GAP}
user who is interested in the groups.

The library has been brought into {\GAP} format by Volkmar Felsch.

\>'PerfectGroup( <selector>, <size> )'
\>'PerfectGroup( <selector>, <size>, <n> )'
\>'PerfectGroup( <selector>, [ <size>, <n> ] )'

'PerfectGroup' is the  essential extraction function  of the library.  It
returns a group, $G$  say, which is isomorphic to the
library  group specified  by the size  number  [<size>,<n>] or by the two
separate arguments  $size$ and $n$, assuming a default value of $n = 1$.
The <selector> defines the representation, in which the group is returned.
Possible selectors so far are 'IsPermGroup' and 'IsSubgroupFpGroup'

\beginexample
gap> G := PerfectGroup(IsPermGroup,6048,1);  
U3(3)
\endexample

The  generators and  relators used coincide  with  those  given  in
\cite{HP89}.

As all groups are stored by presentations, a permutation representation is
obtained by coset enumeration. Note that some of the  library groups do not
have a faithful permutation representation of  small  degree. Computations
in  these groups  may  be rather time consuming.

\beginexample
gap> G:=PerfectGroup(IsPermGroup,129024,2);
L2(8) N ( 2^6 E 2^1 A ) C 2^1
gap> NrMovedPoints(G);
14336
\endexample

\>'PerfectIdentification(<G>)'

Besides the standard attributes for a perfect group in the respective
representation, each groups obtained from the library has set the attribute
'PerfectIdentification' to $[<size>,<nr>]$.

\>'NumberPerfectGroups( <size> )'%

'NumberPerfectGroups' returns the number of non-isomorphic perfect groups
of size $size$ for  each positive integer  $size$ up to $10^6$ except for
the eight  sizes listed at the beginning  of  this section for  which the
number is not yet known. For these values as well as for any argument out
of range it returns the value $-1$.

\>'NumberPerfectLibraryGroups( <size> )'%

'NumberPerfectLibraryGroups' returns the number of perfect groups of size
$size$ which are available in the library of finite perfect groups.
(The purpose  of the  function  is  to  provide  a simple  way  to
formulate a loop over all library groups of a given size.)

\>'SizeNumbersPerfectGroups( <factor1>, <factor2> ... )'%

'SizeNumbersPerfectGroups' returns a list of  the size and index  of
all  library groups that contain the specified factors  among
their composition  factors.  Each argument must either  be  the name of a
simple group or  an integer expression which  is the product of the sizes
of one or  more cyclic factors.  (In fact, the function replaces any list
of more than one integer expression among the arguments by their product.)

The following text strings are accepted as simple group names.
{\parindent\manindent
\item{} ''A<n>'' or ''A(n)'' for the alternating groups  $A_n$,
        $5\leq n\leq9$, for example ''A5'' or ''A(6)''.
\item{} ''L<n>(<q>)''  or ''L(<n>,<q>)'' for PSL(<n>,<q>), where
$n\in\{2,3\}$ and $q$ a prime power, ranging
\itemitem for $n=2$ from 4 to 125
\itemitem for $n=3$ from 2 to 5
\itemitem for $n=2$ from 4 to 125
\item{} ''U<n>(<q>)''  or ''U(<n>,<q>)'' for PSU(<n>,<q>), where
$n\in\{3,4\}$ and $q$ a prime power, ranging
\itemitem for $n=3$ from 3 to 5
\itemitem for $n=4$ from 2 to 2
\item{} ''Sp4(4)'' or ''S(4,4)'' for the symplectic group S(4,4),
\item{} ''Sz(8)'' for the Suzuki group Sz(8),
\item{} ''M<n>'' or ''M(<n>)'' for the Matthieu groups 
        $M_{11}$, $M_{12}$, and $M_{22}$, and
\item[] ''J<n>'' or  ''J(<n>)'' for the Janko groups $J_1$
      and $J_2$.
\par}

Note that,    for most of  the  groups,  the preceding   list  offers two
different names in order  to  be consistent  with  the notation  used  in
\cite{HP89}       as  well  as   with    the    notation    used in   the
'DisplayCompositionSeries' command of {\GAP}. However,  as the names  are
compared as text strings,  you are restricted to  the above  choice. Even
expressions like ''L2(32)'' or ''L2(2\^5)'' are not accepted.

As the use of the  term $PSU(n,q)$ is  not  unique in the literature,  we
state that here  it denotes the factor  group of $SU(n,q)$ by its centre,
where $SU(n,q)$ is  the group of all $n  \times n$ unitary matrices  with
entries in $GF(q^2)$ and determinant 1.

The purpose  of the function is  to provide a  simple way to  formulate a
loop over all library groups which contain certain composition factors.

\>'DisplayInformationPerfectGroups( <size> )'%
\>'DisplayInformationPerfectGroups( <size>, <n> )' 
\>'DisplayInformationPerfectGroups( [ <size>, <n> ] )'

'DisplayInformationPerfectGroups' displays  some  information about   the
library group   $G$,  say,  which  is  specified    by the   size  number
[<size>,<n>] or by the two  arguments $size$ and $n$.   If, in the second
case, <n> is omitted,  the function will  loop over all library groups of
size <size>.

The information provided for $G$ includes the following items:
{\parindent\manindent
\item{$\bullet$} a headline containing the size number $[<size>,<n>]$ of $G$
  in the form $<size>.<n>$ (the suffix $.<n>$ will be suppressed if, up to
  isomorphism, $G$ is the only perfect group of size $size$),
\item{$\bullet$} a message  if $G$ is  simple or quasisimple,  id est, if the
  factor group of $G$ by its centre is simple,
\item{$\bullet$} the ''description''  of the  structure of $G$  as it is
  given by Holt and Plesken in \cite{HP89} (see below),
\item{$\bullet$} the size of the centre of  $G$  (suppressed, if  $G$  is
  simple),
\item{$\bullet$} the prime decomposition of the size of $G$,
\item{$\bullet$} orbit sizes for a faithful permutation representation of $G$ 
  which is provided by the library (see below),
\item{$\bullet$} a reference to each occurrence of $G$  in the tables of
  section 5.3 of \cite{HP89}. Each of these references consists of a class
  number and an internal number  $(i,j)$  under which $G$  is listed in that
  class.  For some groups,  there is more than one reference  because
  these groups  belong to more than one  of the classes  in the book.
\par}

\beginexample
gap> DisplayInformationPerfectGroups( 30720, 3 );
#I Perfect group 30720:  A5 ( 2^4 E N 2^1 E 2^4 ) A
#I   size = 2^11*3*5  orbit size = 240
#I   Holt-Plesken class 1 (9,3)
gap> DisplayInformationPerfectGroups( 30720, 6 );
#I Perfect group 30720:  A5 ( 2^4 x 2^4 ) C N 2^1
#I   centre = 2  size = 2^11*3*5  orbit size = 384
#I   Holt-Plesken class 1 (9,6)
gap> DisplayInformationPerfectGroups( Factorial( 8 ) / 2 );
#I Perfect group 20160.1:  A5 x L3(2) 2^1
#I   centre = 2  size = 2^6*3^2*5*7  orbit sizes = 5 + 16
#I   Holt-Plesken class 31 (1,1) (occurs also in class 32)
#I Perfect group 20160.2:  A5 2^1 x L3(2)
#I   centre = 2  size = 2^6*3^2*5*7  orbit sizes = 7 + 24
#I   Holt-Plesken class 31 (1,2) (occurs also in class 32)
#I Perfect group 20160.3:  ( A5 x L3(2) ) 2^1
#I   centre = 2  size = 2^6*3^2*5*7  orbit size = 192
#I   Holt-Plesken class 31 (1,3)
#I Perfect group 20160.4:  simple group  A8
#I   size = 2^6*3^2*5*7  orbit size = 8
#I   Holt-Plesken class 26 (0,1)
#I Perfect group 20160.5:  simple group  L3(4)
#I   size = 2^6*3^2*5*7  orbit size = 21
#I   Holt-Plesken class 27 (0,1)
\endexample

For any library  group  $G$, the  library  files  do not  only  provide a
presentation, but,  in addition, a  list of  one or  more subgroups $S_1,
\ldots,  S_r$   of  $G$  such   that  there  is   a faithful  permutation
representation of $G$ of degree $\sum_{i=1}^{r}  [G:S_i]$ on the set
$\{  S_i g \mid 1 \leq  i \leq r,  \, g  \in G  \}$  of the cosets of the
$S_i$. This allows to construct the groups as permutation groups, see below.
The 'DisplayInformationPerfectGroups' function
displays only the available degree. The message

\begintt
    orbit size = 8
\endtt

in the above example means  that the available permutation representation
is transitive and of degree 8, whereas the message

\begintt
    orbit sizes = 5 + 16 |
\endtt

means that a nontransitive  permutation representation is available which
acts on two orbits of size 5 and 16 respectively.

The notation used in the  ''description'' of a  group is explained  in
section 5.1.2 of \cite{HP89}. We quote the respective page from there:

{\sl ''Within a class $Q\,\#\,p$, an  isomorphism type of groups will be
denoted by an ordered pair of integers $(r,n)$, where $r \geq 0$ and $n >
0$. More precisely, the isomorphism  types in $Q$  \#  $p$ of order  $p^r
\!\! \mid \!\! Q \!\! \mid$ will be denoted by $(r,1)$, $(r,2)$, $(r,3)$,
$\ldots\,$. Thus $Q$ will always get the size number $(0,1)$.

In  addition to the symbol $(r,n)$,  the groups in $Q  \,  {\sl \#} \, p$
will  also be given a  more descriptive name.  The purpose of  this is to
provide a very rough idea of  the structure of  the group.  The names are
derived in the following manner. First of all, the isomorphism classes of
irreducible  $F_pQ$-modules $M$ with $\mid \!\!  Q  \!\! \mid \mid \!\! M
\!\! \mid \,  \leq 10^6$,  where $F_p$ is   the field of  order $p$,  are
assigned symbols. These  will either  be simply $p^x$,  where $x$  is the
dimension of the module, or, if there is more  than one isomorphism class
of irreducible modules having the same dimension, they will be denoted by
$p^x$, $p^{x^\prime}$,  etc.   The one-dimensional  module   with trivial
$Q$-action  will therefore be denoted  by  $p^1$. These  symbols will  be
listed under the description of $Q$.  The group name consists essentially
of a list  of the composition factors  working from the  top of the group
downwards; hence  it always  starts with the  name  of $Q$ itself.  (This
convention is the most  convenient in our  context,  but it is  different
from that adopted  in the ATLAS (Conway  {\it et al}.~1985), for example,
where composition factors  are listed in the  reverse order. For example,
we denote a  group isomorphic to $SL(2,5)$  by  $A_5 2^1$ rather  than $2
\cdot A_5$.)

Some  other symbols are used  in the name, in order  to give some idea of
the   relationship  between these    composition  factors, and  splitting
properties. We shall now list these additional symbols.
{\parindent\manindent
\item{$\times$}  between  two  factors   denotes   a  direct  product  of
      $F_pQ$-modules or groups.
\item{C} (for ''commutator''  between two factors  means  that the second
      lies in the commutator subgroup of the first.  Similarly, a segment
      of the form  $(f_1 \! \times \! f_2) {\sl C} f_3$  would mean  that
      the factors $f_1$ and $f_2$  commute modulo $f_3$ and $f_3$ lies in
      $[f_1,f_2]$.
\item{A} (for ''abelian''  between two factors  indicates that the second
      is  in the  $p$th  power  (but not the commutator subgroup)  of the
      first.   ''A''   may  also   follow  the  factors,   if  bracketed.
\item{E} (for ''elementary abelian''  between two factors  indicates that
      together   they  generate  an  elementary  abelian  group   (modulo
      subsequent factors), but that the resulting $F_pQ$-module extension
      does not split.
\item{N}  (for ''nonsplit''  before  a  factor  indicates  that  $Q$  (or
      possibly its covering group)  splits down as far at this factor but
      not over the factor itself.  So  ''$Q f_1 {\sl N} f_2$'' means that
      the normal subgroup  $f_1f_2$  of the group  has no complement but,
      modulo $f_2$, $f_1$, does have a complement.
\par}
Brackets have their obvious meaning. Summarizing, we have:
{\parindent\manindent
\item{$\times$} = dirext product;
\item{C} = commutator subgroup;
\item{A} = abelian;
\item{E} = elementary abelian; and
\item{N} = nonsplit.
\par}
Here are some examples.
{\parindent\manindent
\item[(i)]  $A_5 (2^4 {\sl E} 2^1 {\sl E} 2^4) {\sl A}$  means  that  the
      pairs  $2^4 {\sl E} 2^1$  and $2^1 {\sl E} 2^4$ are both elementary
      abelian of exponent 4.
\item[(ii)]   $A_5 (2^4 {\sl E} 2^1 {\sl A}) {\sl C} 2^1$    means   that
      $O_2(G)$  is of  symplectic type  $2^{1+5}$,  with  Frattini factor
      group  of type   $2^4 {\sl E} 2^1$.   The   ''A''  after the  $2^1$
      indicates that $G$ has a  central cyclic subgroup $2^1 {\sl A} 2^1$
      of order 4.
\item[(iii)]     $L_3(2) ((2^1 {\sl E}) \! \times \! ({\sl N} 2^3 {\sl E}
      2^{3^\prime} {\sl A}) {\sl C}) 2^{3^\prime}$    means   that    the
      $2^{3^\prime}$ factor at the bottom lies in the commutator subgroup
      of the pair $2^3 {\sl E} 2^{3^\prime}$ in the middle, but the lower
      pair $2^{3^\prime} {\sl A} 2^{3^\prime}$  is abelian of exponent 4.
      There  is  also  a submodule  $2^1 {\sl E} 2^{3^\prime}$,  and  the
      covering group  $L_3(2) 2^1$  of  $L_3(2)$  does not split over the
      $2^3$  factor.  (Since $G$ is perfect,  it goes without saying that
      the extension $L_3(2) 2^1$ cannot split itself.)
\par}

We  must  stress  that this  notation does   not  always succeed in being
precise  or even unambiguous, and the  reader is free to  ignore it if it
does not seem helpful.'' }

If such a group description has  been given in  the book for $G$ (and, in
fact, this is the case for  most of the library  groups), it is displayed
by the 'DisplayInformationPerfectGroups' function. Otherwise the function
provides a less explicit  description  of  the  (in these cases   unique)
Holt-Plesken class to which $G$ belongs, together with a serial number if
this is necessary to make it unique.

%%%%%%%%%%%%%%%%%%%%%%%%%%%%%%%%%%%%%%%%%%%%%%%%%%%%%%%%%%%%%%%%%%%%%%%%%%%%%
% Local Variables:
% mode:               text
% mode:               outline-minor
% outline-regexp:     "\\\\Chapter\\|\\\\Section"
% paragraph-start:    "\\\\begin\\|\\\\end\\|\\$\\$\\|.*%\\|^$"
% paragraph-separate: "\\\\begin\\|\\\\end\\|\\$\\$\\|.*%\\|^$"
% fill-column:        73
% End:

