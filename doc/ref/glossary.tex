\Chapter{Glossary}

\begindoublecolumns
\def*#1*{\hangindent\smallmanindent{\bf #1}}
\parskip 1ex plus 0.5ex minus 0.5ex
\parindent 0pt
\emergencystretch 2em

%%%%%%%%%%%%%%%%%%%%%%%%%%%%%%%%%%%%%%%%%%%%%%%%%%%%%%%%%%%%%%%%%%%%%%%%%
\letter A

*affine type*, the  primitive permutation groups  of affine  type are the
affine groups~$V.M$ (i.e. the semidirect product of $V$ with $M\le GL(V)$
acting in the natural way) acting on a vector space~$V$ via $v.(\psi,w) =
v\psi + w$, for an irreducible subgroup~$M\le GL(V)$.

%%%%%%%%%%%%%%%%%%%%%%%%%%%%%%%%%%%%%%%%%%%%%%%%%%%%%%%%%%%%%%%%%%%%%%%%%
\letter B

*base  for a permutation   group*, a sequence  of $B$  of points from the
operation domain of a permutation group $G$ is  called a base if the only
element of $G$ that fixes $B$ elementwise is the identity.

*basic orbit*, if $(b_1,\ldots,b_n)$ is  a base for a permutation  group,
the orbits $b_iG^{(i)}$ of the corresponding stabilizers (see "Stabilizer
chains") are called the basic orbits.

*block system*, if a permutation group $G$ operates on a set $\Omega$, an
(unordered) partition of $\Omega$ into cells is called a block system for
$G$ if $G$ permutes these cells setwise.

*blocks   homomorphism*,   the operation   homomorphism   of the  setwise
operation on a block system.

%%%%%%%%%%%%%%%%%%%%%%%%%%%%%%%%%%%%%%%%%%%%%%%%%%%%%%%%%%%%%%%%%%%%%%%%%
\letter C

*central subgroup*,  a  subgroup is central in  $G$  if  its elements are
invariant under  conjugation by $G$, i.e., if  its centralizer is $G$, or
equivalently, if it is contained in the centre.

*centralizer*, the stabilizer in a group $G$  of an element or a sequence
of elements  under the pointwise action  of $G$ on itself by conjugation,
i.e., $C_G(U) = \{g\in G\mid u^g=u\,\forall u\in U\}$.

*centre*, the centralizer of a group in itself.

*commutator subgroup*, the commutator  subgroup $[U,V]$ of  two subgroups
$U$ and $V$ of  a group $G$  is the  group  generated by all  commutators
$[u,v]=u^{-1}v^{-1}uv$ with $u\in U$ and $v\in V$.

*composition series*, a maximally  refined  subnormal series,  i.e.,  one
with simple factors.

*constituent homomorphism*,  the operation homomorphism of  the pointwise
operation on a union of orbits.

*core*, the core of a subgroup $U$ in a group  $G$ is the intersection of
all conjugates of $U$  under $G$, i.e., the biggest  subgroup of $U$ that
is normal in $G$.

*coset*, a left coset of a subgroup $U$ in its supergroup $G$ is a subset
$gU = \{gu\mid u\in U\}\subset G$ for some $g\in G$, similarly, $Ug$ is a
right coset.

%%%%%%%%%%%%%%%%%%%%%%%%%%%%%%%%%%%%%%%%%%%%%%%%%%%%%%%%%%%%%%%%%%%%%%%%%
\letter D

*degree  of transitivity  of an  operation*, the  highest number $k$ such
that the operation is $k$-transitive (see transitive).

*depth of a commutator*, every element of  a group is  said to have depth
at least~0, and a commutator $[a,b]$ is said to have depth at least $i+1$
if both $a$ and $b$ have depth at least~$i$.

*depth w.r.t.~pcgs*,  the least number  $i$  such that  $\nu_i\ne 0$ (see
exponents,  w.r.t.~pcgs). By  convention, the identity  element has depth
one greater than the length of the exponent tuple.

*derived length  of a solvable  group*, one  less than the  length of the
derived series.

*derived series*,  the derived  series  of a  group  $G$ is  the strictly
decreasing  series $G  = G_0   > G_1 >   \ldots$ defined   by $G_{i+1}  =
[G_i,G_i]$ for   $i\ge 0$. It  is  finite if and  only  if some  $G_n$ is
perfect; $G_n=\{1\}$ if and only if $G$ is solvable.

*derived subgroup*, the derived subgroup  of a group  $G$ is $[G,G]$ (see
commutator subgroup).

%%%%%%%%%%%%%%%%%%%%%%%%%%%%%%%%%%%%%%%%%%%%%%%%%%%%%%%%%%%%%%%%%%%%%%%%%
\letter E

*earns*, the socle of a primitive permutation group  of affine type is an
*e*lementary *a*belian *r*egular *n*ormal *s*ubgroup, abbreviated earns.

*elementary abelian  series*,   a normal series  with elementary  abelian
factors. Exists for solvable groups and only for these.

*exponent of  a group*, the least  common  multiple of the orders  of all
elements.

*exponents w.r.t.~pcgs*, if a  group $G$  has a pcgs  $(g_1,\ldots,g_n)$,
every   element of $G$  has  a  unique factorization  ${g_1}^{\nu_1}\cdot
\ldots \cdot {g_n}^{\nu_n}$  with $0\le \nu_i \< r_i$  where $r_i$ is the
relative order  of $g_i$ (see relative  order, in  a pcgs). The $n$-tuple
$(\nu_1,\ldots,\nu_n)$ is called the associated exponent tuple.

*external orbit*, a {\GAP} object  describing a transitive operation of a
group $G$ on an orbit  $pnt.G = \{pnt.g;\,g\in  G\}$ of some point  $pnt$
(see operation).

*external set*, a {\GAP} object describing an operation (see there).

%%%%%%%%%%%%%%%%%%%%%%%%%%%%%%%%%%%%%%%%%%%%%%%%%%%%%%%%%%%%%%%%%%%%%%%%%
\letter F

*factorized   inverse     transversal*,    see     the   explanation   in
section~"Stabilizer chains".

*Fitting subgroup*, the largest nilpotent normal subgroup of $G$.

*Frattini subgroup*, the intersection of all maximal subgroups of $G$, or
$G$ itself if there are no maximal subgroups.

%%%%%%%%%%%%%%%%%%%%%%%%%%%%%%%%%%%%%%%%%%%%%%%%%%%%%%%%%%%%%%%%%%%%%%%%%
\letter I

*imprimitive action  of a wreath product*,  if $G$ is a permutation group
on   $\Omega$   and $\sigma\colon H\to  S_n$,   then  the  wreath product
$H\wr_\sigma G$ acts imprimitively on $\Omega \times \{1,\ldots,n\}$ with
the $i$th direct factor of $G^n$ acting on the $i$th copy of~$\Omega$ and
$H$   permuting   these  $n$~copies  by  $(\omega,i).(h;g_1,\ldots,g_n) =
(\omega.g_i,i\sigma_h^{-1})$.

*index*, the index of a subgroup $U$ in  its supergroup $G$ is the number
of left (or equivalently, of right) cosets of $U$ in $G$.

%%%%%%%%%%%%%%%%%%%%%%%%%%%%%%%%%%%%%%%%%%%%%%%%%%%%%%%%%%%%%%%%%%%%%%%%%
\letter J

*Jennings series* of a $p$-group $G$, the series $G = G_0 > G_1 > \ldots$
defined by setting for $i>0$, $G_i = [ G_{i-1}, G ] {G_j}^p$ where $j$ is
the smallest integer greater than or equal to~$i / p$.

%%%%%%%%%%%%%%%%%%%%%%%%%%%%%%%%%%%%%%%%%%%%%%%%%%%%%%%%%%%%%%%%%%%%%%%%%
\letter L

*leading exponent w.r.t.~pcgs*, the earliest exponent different from zero
(see exponents, w.r.t.~pcgs). By   convention, the identity   element has
leading exponent~1.

*Loewy series* of a $p$-group $G$, the series $F_pG = J^0  > J^1 > \ldots
> J^{l+1}  = \{0\}$ of ideals  in the group ring $F_pG$  where $J$ is the
Jacobson radical (i.e.,  the intersection of all  maximal left ideals or,
equivalently, the largest nilpotent ideal).

*lower central  series*, the lower central  series of a  group $G$ is the
strictly decreasing series $G = G_0 > G_1 > \ldots$ defined by $G_{i+1} =
[G_i,G]$ for $i\ge  0$. It is  finite if and  only if  some  $G_n$ has no
$G$-central factor group; $G_n=\{1\}$ if and only if $G$ is nilpotent.

%%%%%%%%%%%%%%%%%%%%%%%%%%%%%%%%%%%%%%%%%%%%%%%%%%%%%%%%%%%%%%%%%%%%%%%%%
\letter N

*nilpotent*, a group $G$   is nilpotent if it   has a normal series  with
central factors $G_{i-1}/G_i \le Z(G/G_i)$.

*nilpotency class of a nilpotent group*, one  less than the length of the
lower central series.

*normal series*, a finite series of  subgroups $G = G_0 >  \ldots > G_n =
\{1\}$  (with  strict inclusions) such that   $G_i$ is normal  in $G$ for
$1\leq i\leq n$.

*normal subgroup*, a subgroup is normal  in $G$ if  it is invariant under
conjugation by $G$, i.e., if its normalizer is~$G$.

*normalizer*, the stabilizer in a group $G$ of a subgroup or subset under
the setwise action   of $G$ on  itself by  conjugation,  i.e., $N_G(U)  =
\{g\in G\mid u^g\in U\,\forall u\in U\}$.

%%%%%%%%%%%%%%%%%%%%%%%%%%%%%%%%%%%%%%%%%%%%%%%%%%%%%%%%%%%%%%%%%%%%%%%%%
\letter O

*operation*,   a group $G$  operates on   a  set $\Omega$  via a  mapping
$\varphi\colon         \Omega\times            G\to     \Omega$        if
(1)~$\varphi(\varphi(\omega,g),h)      =      \varphi(\omega,gh)$     and
(2)~$\varphi(\omega,1)=\omega$ for all~$\omega\in \Omega$ and~$g,h\in G$.
We  often write $\omega g$  for  $\varphi(\omega,g)$, the $\varphi$ being
understood.

*operation homomorphism*, if  $G$ operates on~$\Omega$ via~$\varphi$, the
mapping defined by $g  \mapsto  (\omega \mapsto \varphi(\omega,g))$  is a
homomorphism of $G$  into the symmetric group on  $\Omega$. It  is called
the associated operation homomorphism.

*orbit-stabilizer theorem*,  if   $G$ operates on  $\Omega$,  the mapping
$\omega  g \to G_\omega  g$ is a  bijection between the orbit of $\omega$
under $G$ and the cosets of the point stabilizer  $G_\omega$ in $G$. This
statement is called the orbit-stabilizer theorem.

*ordered partition*, an ordered partition of a set $\Omega$ is a sequence
of   non-empty, mutually disjoint  sets  (so-called  *cells*) whose union
is~$\Omega$.

%%%%%%%%%%%%%%%%%%%%%%%%%%%%%%%%%%%%%%%%%%%%%%%%%%%%%%%%%%%%%%%%%%%%%%%%%
\letter P

*$p$-central series* of a group $G$, the strictly  decreasing series $G =
G_0 > G_1 > \ldots$ defined by $G_{i+1} = [G,G_i]G_i^p$ for $i\ge0$. This
series reaches $\{1\}$ if and only if $G$ is a nilpotent (not necessarily
finite) $p$-group.

*$p$-core*, the core of a $p$-Sylow subgroup.

*pcgs, polycyclic generating  system*, see the explanation in "Polycyclic
generating systems".

*perfect*, a group $G$ is perfect if $G=[G,G]$.

*polycyclic generating system*, see pcgs.

*primitive*, an operation on a domain~$\Omega$ is primitive  if it has no
block  systems other  than   $\{\Omega\}$  and  $\{\{\omega\};\,\omega\in
\Omega\}$.

*product  action of a wreath  product*, if $G$  is a permutation group on
$\Omega$  and    $\sigma\colon H\to    S_n$,   then  the  wreath  product
$H\wr_\sigma G$ acts   on  $\Omega^n$ with  $G^n$  acting on   $\Omega^n$
componentwise     and      $H$  permuting     the    $n$~coordinates   by
$(\omega_1,\ldots,\omega_n).               (h;g_1,\ldots,g_n)           =
(\omega_{1\sigma_h^{-1}}.             g_{1\sigma_h^{-1}},         \ldots,
\omega_{n\sigma_h^{-1}}. g_{n\sigma_h^{-1}})$. This is called the product
action; it is primitive if  and only if  $G$ is primitive but not regular
and $H$ is transitive.

%%%%%%%%%%%%%%%%%%%%%%%%%%%%%%%%%%%%%%%%%%%%%%%%%%%%%%%%%%%%%%%%%%%%%%%%%
\letter R

*radical of a group*, the largest normal solvable subgroup.

*reduced  base  for a permutation group*,  a  base  $B$ for a permutation
group  $G$ is called reduced  if no proper subsequence of  $B$  is also a
base for~$G$.

*refinement*, see the explanation in section~"Backtrack searching".

*regular*, an  operation is regular if  it is transitive and semiregular,
i.e., if a fixed point is mapped to any other point  by exactly one group
element.

*relative order in a pcgs*, if $(g_1,\ldots,g_n)$ is a pcgs, the relative
order    of $g_i$   is   its    order     modulo the  normal     subgroup
$\langle g_{i+1},\ldots,g_n\rangle$.

%%%%%%%%%%%%%%%%%%%%%%%%%%%%%%%%%%%%%%%%%%%%%%%%%%%%%%%%%%%%%%%%%%%%%%%%%
\letter S

*Schreier generators, Schreier's subgroup theorem*,  if $U$ is a subgroup
of      $G=\langle    g_1,\ldots,g_n\rangle$   with  transversal  $G=Ut_1
\mathbin{\dot\cup}  \ldots \mathbin{\dot\cup}  Ut_m$,  then  $U = \langle
t_ig_j (\overline{t_ig_j})^{-1};\, i=1,\ldots,m, \, j=1,\ldots,n \rangle$
where $\overline g$  denotes the transversal element with  $Ug=U\overline
g$.   This statement  is   called  Schreier's subgroup  theorem  and  the
generators for $U$ are called Schreier generators.

*Schreier tree of an  orbit*, if $O$  is an orbit  of a permutation group
$G$, a Schreier tree for $O$  is a directed tree with  vertex set $O$ and
where there is an edge $i \labelto g j$ labelled with an element $g\in G$
such that $i=jg$.

*semidirect   product*, given   groups    $G$, $H$   and  a  homomorphism
$\varphi\colon H\to {\rm Aut}(G)$ (automorphisms  to be applied from  the
right),  the semidirect product  of  $G$  with~$H$ via~$\varphi$  is  the
cartesian  product $H\times  G$   with  multiplication  $(h,g).(h',g')  =
(hh',g\varphi_{h'}.g')$.

*semiregular*, an operation  is semiregular if   the stabilizer of  every
point is trivial.

*sign homomorphism*,  the sign of a  permutation $p\in S_n$ is defined to
be $1$ if $p$  is the product of an  even number of transpositions `(ij)'
with   $`i' \ne `j'$, and  $-1$  otherwise. The  sign  of an $n$-cycle is
$(-1)^{n-1}$.

*simple*, a group  is simple if it has  no normal subgroups except itself
and~$\{1\}$.

*socle*, the socle of a group is the product of its minimal (non-trivial)
normal subgroups.

*solvable*, a group $G$ is solvable if it has a subnormal series from $G$
to the trivial subgroup with abelian factors $G_{i-1}/G_i$.

*solvable residuum*,  the last member of the  derived series, which  is a
perfect normal subgroup with the largest possible solvable factor group.

*stabilizer*,  if a group $G$ operates  on a set $\Omega$, the stabilizer
in $G$ of a point $\omega\in\Omega$ is the subgroup ${\rm Stab}_G(\omega)
= \{g\in G\mid \omega g=\omega\}$.

*stabilizer  chain*, if $(b_1,\ldots,b_n)$   is a base  for a permutation
group $G$, we  associate with it the stabilizer  chain  $G = G^{(1)}  \ge
G^{(2)}  \ge  \ldots \ge G^{(r+1)}  =  \{{\rm id}\}$ with $G^{(i+1)}={\rm
Stab}_{G^{(i)}}(b_i)$ for $i>0$.

*strong generating  set*, a set  $S$ of generators for $G$  is said to be
strong relative to a base $B$ if contains generating sets for all members
of the stabilizer chain, in other words, if $\langle S\cap G^{(i)}\rangle
= G^{(i)}$ for all~$i$.

*subnormal*, a subgroup $U$  of $G$ is  subnormal if there is a subnormal
series from $G$ to $U$ whose last member is~$U$.

*subnormal series*, if $U$ is a subgroup of $G$,  a subnormal series from
$G$ to $U$ is a finite series of subgroups $G = G_0 > \ldots > G_n$ (with
strict inclusions) such that  $G_i$  is normal  in $G_{i-1}$ for   $1\leq
i\leq n$ and  such that there is no  subgroup above  $U$ which is  normal
in~$G_n$.

*Sylow $p$-subgroup*, a subgroup of $G$  whose order is the largest power
of $p$ dividing the order of $G$. Exists always by Sylow's theorem.

%%%%%%%%%%%%%%%%%%%%%%%%%%%%%%%%%%%%%%%%%%%%%%%%%%%%%%%%%%%%%%%%%%%%%%%%%
\letter T

*transitive*,   an operation  of a  group  on $\Omega$  is  transitive if
$\Omega$ forms one   orbit. The operation   is $k$-transitive,  if it  is
transitive   on    the non-repeating   $k$-tuples in   $\Omega^k$  (under
componentwise action).

*trivial subgroup*, the subgroup consisting only of the identity element.

*two-closure of a permutation  group*, the two-closure of $G\le S_\Omega$
is   the  supergroup      $\{g\in S_\Omega\mid  (\omega,\omega')Gg      =
(\omega,\omega')G\,\forall \omega,\omega' \in\Omega\}$ of~$G$.

%%%%%%%%%%%%%%%%%%%%%%%%%%%%%%%%%%%%%%%%%%%%%%%%%%%%%%%%%%%%%%%%%%%%%%%%%
\letter U

*upper central  series*, the upper  central series of a  group $G$ is the
series $\{1\} =  G_0 \< G_1 \< \ldots$  defined such that $G_{i+1} / G_i$
is the centre of $G / G_i$ for $i\ge0$. Unlike for the other series, this
definition works ``bottom up'', and the series reaches $G$ if and only if
$G$ is nilpotent.

%%%%%%%%%%%%%%%%%%%%%%%%%%%%%%%%%%%%%%%%%%%%%%%%%%%%%%%%%%%%%%%%%%%%%%%%%
\letter W

*wreath product*, given  groups  $G$, $H$  and an operation  homomorphism
$\sigma\colon H\to S_n$, the wreath product $H\wr_\sigma G$ is defined to
be the semidirect product of  $G^n$ with~$H$ where  $H$ acts on $G^n$  by
permuting  the  direct    factors via~$(g_1,    \ldots,  g_n)\sigma_h   =
(g_{1\sigma_h^{-1}},   \ldots,   g_{n\sigma_h^{-1}})$.    If  $G$  is   a
permutation group, there is an imprimitive action and a product action of
$H\wr_\sigma G$ (see there).

\enddoublecolumns

%%%%%%%%%%%%%%%%%%%%%%%%%%%%%%%%%%%%%%%%%%%%%%%%%%%%%%%%%%%%%%%%%%%%%%%%%%%%%
% Local Variables:
% mode:             text
% mode:             outline-minor
% outline-regexp:   "\\\\Chapter\\|\\\\letter"
% fill-column:      73
% End:
%%%%%%%%%%%%%%%%%%%%%%%%%%%%%%%%%%%%%%%%%%%%%%%%%%%%%%%%%%%%%%%%%%%%%%%%%%%%%
