%%%%%%%%%%%%%%%%%%%%%%%%%%%%%%%%%%%%%%%%%%%%%%%%%%%%%%%%%%%%%%%%%%%%%%%%%
%%
%W  process.tex               GAP documentation              Frank Celler
%W                                                     & Martin Schoenert
%%
%H  @(#)$Id$
%%
%Y  Copyright 1997,  Lehrstuhl D fuer Mathematik,  RWTH Aachen,   Germany
%%
%%  This file contains the description of process.
%%


%%%%%%%%%%%%%%%%%%%%%%%%%%%%%%%%%%%%%%%%%%%%%%%%%%%%%%%%%%%%%%%%%%%%%%%%%
\Chapter{Processes}

{\GAP}  can  call other programs,  such  programs are called *processes*.
There are  two  kinds of processes:  first there  are  processes that are
started, run and  return a result,  while  {\GAP} is suspended  until the
process  terminates.  Then there are  processes that will run in parallel
to {\GAP} as  subprocesses  and {\GAP} can  communicate and   control the
processes using streams.

Note that the latter kind of process is *not* yet implemented.

%%%%%%%%%%%%%%%%%%%%%%%%%%%%%%%%%%%%%%%%%%%%%%%%%%%%%%%%%%%%%%%%%%%%%%%%%
\Section{Process}

\>Process( <dir>, <prg>, <stream-in>, <stream-out>, <options> )

runs  a new process.  `Process' returns  when the process terminates.  It
returns the return value of the process (if the operating system supports
such a concept).

<dir> must be a directory  object (see "Directories") which specifies the
directory in which the process is run.

<prg> is the filename of the program to launch, for portability it should
be   the  result  of   `Filename'   (see   "Filename") and   should  pass
`IsExecutableFile'.  Note that  `Process'  does *no* searching through  a
list of directories, this is done by `Filename'.

<stream-in>  is the  input stream   that  delivers the characters  to the
process.   For portability it  should either  be `InputTextNone' (if  the
process reads  no characters), `InputTextUser' (*@not implemented@*), the
result  of a call to `InputTextFile'  from which  no characters have been
read, or the result of a call to `InputTextString'.

`Process' is  free to consume *all* the  input even if the program itself
does not require any input at all.

<stream-out> is the output stream  which receives the characters from the
process.  For portability it should   either be `OutputTextNone' (if  the
process writes  no  characters), `OutputTextUser' (*@not  implemented@*),
the result of a call to `OutputTextFile' to which no characters have been
written, or the result of a call to `OutputTextString'.

<options> is a list of strings which are passed to the process as command
line argument.  Note that no substitutions are  performed on the strings,
i.e., they are passed immediately to the process and are not processed by
a command interpreter (shell).   Further note that  each string is passed
as one  argument,  even if it  contains  <space>  characters.  Note  that
input/output redirection commands are *not* allowed as <options>.

*Examples*

In   order to  find   a  system program  use  `DirectoriesSystemPrograms'
together with `Filename'.

\begintt
   gap> path := DirectoriesSystemPrograms();;
   gap> date := Filename( path, "date" );
   "/bin/date"
\endtt

Now execute `date' with no argument and no input, collect the output into
a string stream.

\begintt
    gap> str := "";; a := OutputTextString(str,true);;
    gap> Process( DirectoryCurrent(), date, InputTextNone(), a, [] );
    0
    gap> CloseStream(a);
    gap> Print(str);   
    Fri Jul 11 09:04:23 MET DST 1997
\endtt

%%%%%%%%%%%%%%%%%%%%%%%%%%%%%%%%%%%%%%%%%%%%%%%%%%%%%%%%%%%%%%%%%%%%%%%%%
\Section{Exec}

\>Exec( <cmd> )
\>Exec( <cmd>, <shell> )

runs a shell in the current directory to execute  the command <cmd>.  The
first form runs a  Bourne shell while the  second command runs  the shell
<shell>.  <shell> needs  to understand the  option  <-c> followed by  the
command to be executed as a string.  Any  output of <cmd>  is sent to the
standard output.

\begintt
    gap> Exec( "date" )
    Thu Jul 24 10:04:13 BST 1997
\endtt

<cmd> is  interpreted by the shell  and therefore we  can make use of the
various features that a shell offers as in following example.

\beginexample
   gap> Exec( "echo \"GAP is great!\" > foo" );    
   gap> Exec( "cat foo" );
   GAP is great!
   gap> Exec( "rm foo" );
\endexample

%%%%%%%%%%%%%%%%%%%%%%%%%%%%%%%%%%%%%%%%%%%%%%%%%%%%%%%%%%%%%%%%%%%%%%%%%
%%
%E  streams.tex . . . . . . . . . . . . . . . . . . . . . . . . ends here
