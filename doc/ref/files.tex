%%%%%%%%%%%%%%%%%%%%%%%%%%%%%%%%%%%%%%%%%%%%%%%%%%%%%%%%%%%%%%%%%%%%%%%%%
%%
%W  files.tex                 GAP documentation              Frank Celler
%W                                                     & Martin Schoenert
%%
%H  @(#)$Id$
%%
%Y  Copyright 1997,  Lehrstuhl D fuer Mathematik,  RWTH Aachen,   Germany
%%
%%  This file    contains the  description of   the   file, filename  and
%%  directory functions.
%%


%%%%%%%%%%%%%%%%%%%%%%%%%%%%%%%%%%%%%%%%%%%%%%%%%%%%%%%%%%%%%%%%%%%%%%%%%
\Chapter{Files and Filenames}

Files are identified by filenames, which are represented in {\GAP} as
strings.  Filenames can be created directly by the user or a program, but
of course this is operating system dependent.

Filenames for some files can  be constructed in  a system independent way
using the following functions.  This is done by first getting a directory
object for the directory the file shall  reside in, and then constructing
the filename.  However, it is  sometimes necessary to construct filenames
of files in subdirectories relative to a given directory object.  In this
case  the  directory separator is  *always*  `{'/'}' even under  `DOS' or
`MacOS'.

Section "Directories" describes how  to  construct directory objects
for the common GAP and system  directories.  Using the command `Filename'
described in  section "Filename" it  is possible  to construct a filename
pointing to a  file in  these  directories.  There are  also functions to
test for accessibility of files, see "File Access".

%%%%%%%%%%%%%%%%%%%%%%%%%%%%%%%%%%%%%%%%%%%%%%%%%%%%%%%%%%%%%%%%%%%%%%%%%
\Section{Portability}

For portability filenames and directory  names should be restricted to at
most   8 alphanumerical characters  optionally followed  by a dot `{'.'}'
and between 1 and 3 alphanumerical characters.  Upper case letters should
be  avoided because some  operating systems  do  not make any distinction
between case,  so that `NaMe', `Name'  and  `name' all  refer to the same
file  whereas  some   operating  systems are case   sensitive.   To avoid
problems only lower case characters should be used.

\>LastSystemError()

returns  a  record describing the   last system error that  has occurred.
This record   contains  at least   the  component  `message' which   is a
string. This message is, however,  highly operating system dependent  and
should only be used as informational message for the user.

%%%%%%%%%%%%%%%%%%%%%%%%%%%%%%%%%%%%%%%%%%%%%%%%%%%%%%%%%%%%%%%%%%%%%%%%%
\Section{GAP Root Directory}

When   starting {\GAP} it is possible   to specify various directories as
root directories.  In  {\GAP}'s view of  the world these  directories are
merged  into one meta-directory, this  directory  is called *{\GAP}  root
directory* in the following.

For example,  if  ``<root1>;<root2>;...''    is  passed as   argument  to
`{`-l'}' when   {\GAP} is started   and  {\GAP} wants  to locate  a  file
``group.gd'' in the {\GAP} root directory it will first check if the file
exists in `<root1>', if not, it checks `<root2>', and so on.

This layout makes  it possible to  have one  system-wide  installation of
{\GAP}  which is read-only  but still  allows users  to modify individual
files.  Therefore instead of constructing an absolute path name to a file
you should always use `DirectoriesLibrary' or `DirectoriesPackageLibrary'
together with `Filename' to construct a filename for a file in the {\GAP}
root directory.

*Example*

Assume that the system-wide installation lives in ``/usr/local/lib/gap4''
and you want  to modify the  file ``lib/files.gd'' without disturbing the
system installation.

In  this case create  a  new directory ``/home/myhome/gap'' containing  a
subdirectory ``lib'' which contains the modified ``lib/files.gd''.

The directory/file structure now looks like

\begintt
    /usr/local/lib/gap4/
    /usr/local/lib/gap4/lib/
    /usr/local/lib/gap4/lib/files.gd
    /home/myhome/gap/
    /home/myhome/gap/lib
    /home/myhome/gap/lib/files.gd
\endtt


If you start {\GAP} using (under UNIX)

\begintt
    you@unix> gap -l '/home/myhome/gap;/usr/local/lib/gap4'
\endtt

then the  file  ``/home/myhome/gap/lib/files.gd'' will  be  used whenever
{\GAP} references the  file with filename  ``lib/files.gd'' in the {\GAP}
root directory.

This setup also  allows to easily install  new share packages or bugfixes
even if no access to the system {\GAP} installation  is possible.  Simply
unpack the files into ``/home/myhome/gap''.

%%%%%%%%%%%%%%%%%%%%%%%%%%%%%%%%%%%%%%%%%%%%%%%%%%%%%%%%%%%%%%%%%%%%%%%%%
\Section{Directories}

\>DirectoryTemporary( <hint> )
\)DirectoryTemporary()

returns  a directory  object in the   category `IsDirectory' for a  *new*
temporary directory.   This is guaranteed to  be  newly created and empty
immediately  after the call to `DirectoryTemporary'.   {\GAP} will make a
reasonable effort   to *remove* this   directory  either  when a  garbage
collection  collects the directory   object  or upon termination  of  the
{\GAP}   job that   created  the  directory.     <hint> can  be  used  by
`DirectoryTemporary' to construct    the  name  of the    directory   but
`DirectoryTemporary' is free to use only a  part of <hint> or even ignore
it completely.

If `DirectoryTemporary' is  unable to create a  new  directory, `fail' is
returned.  In this case `LastSystemError' can be  used to get information
about the error.

\>DirectoryCurrent()

returns the directory object for the current directory.

\>DirectoriesLibrary( <lib> )
\)DirectoriesLibrary()

returns the  directory objects  for  the  {\GAP}  library <lib>  as list.
<lib> must be one of `"lib"' (the default), `"grp"', `"tbl"', and so on.

The directory <lib> must exist in at  least one of the root directories,
otherwise `fail' is returned.

As the files in the {\GAP} root  directory (see "GAP Root Directory") can
be  distributed  into different  directories in the  filespace  a list of
directories is returned.  In order to find an existing file in the {\GAP}
root directory you should pass that list to `Filename' as first argument.
In order to  create  a filename   for  new file  inside the  {\GAP}  root
directory you   should pass  the first  entry    of that list.   However,
creating files  inside the {\GAP} root  directory is not  encouraged, you
should use `DirectoryTemporary' instead.

\>DirectoriesPackageLibrary( <name-pkg>, <lib> )
\)DirectoriesPackageLibrary( <name-pkg> )

returns  the   directory objects  for   the   package library   <lib>  as
list. <lib> must be one of `"lib"' (the default), `"grp"', and so on, for
the contributed package <name-pkg>.

In order to  find a subdirectory hierarchy in  the package  directory you
can  pass subdirectories in <lib> separated  by `{'/'}'.  For example, to
find the subdirectory `m11' in the directory `data', use:

\begintt
    DirectoriesPackageLibrary( <name-pkg>, "data/m11" )
\endtt

However,   the directory <lib>  must  exist in at   least one of the root
directories, otherwise `fail' is returned.

\>DirectoriesPackagePrograms( <name-pkg> )

`DirectoriesPackagePrograms'   returns  the directory   objects   for the
directories where the  executable programs  for the contributed   package
<name-pkg> reside.

\>DirectoriesSystemPrograms()

`DirectoriesSystemPrograms' returns the directory objects for the list of
directories where  the system programs reside as  list.  Under  UNIX this
would usually represent `\$\{PATH\}'.

%%%%%%%%%%%%%%%%%%%%%%%%%%%%%%%%%%%%%%%%%%%%%%%%%%%%%%%%%%%%%%%%%%%%%%%%%
\Section{Filename}

\>Filename( <dir>, <name> )

returns the (system  dependent)  filename as  a string for  the file with
name   <name> in the  directory  object  <dir>.   `Filename' returns  the
filename regardless of  whether the directory contains  a file <name>  or
not.

\>Filename( <list-of-dirs>, <name> )

If <list-of-dirs> is a list (possibly  of length 1) of directory objects,
then `Filename' will search the directories in order, and will return the
filename for the file <name> in the first directory which contains a file
<name> or `fail' if no directory contains a file <name>.

*Examples*

In    order  to      locate   the    system   program    ``date''     use
`DirectoriesSystemPrograms' together with the second form of `Filename'.

\begintt
    gap> path := DirectoriesSystemPrograms();;
    gap> date := Filename( path, "date" );
    "/bin/date"
\endtt

In order to locate the library file ``files.gd'' use `DirectoriesLibrary'
together with the second form of `Filename'.

\begintt
    gap> path := DirectoriesLibrary();;
    gap> Filename( path, "files.gd" );
    "./lib/files.gd"
\endtt

In  order to construct filenames for  new  files in a temporary directory
use `DirectoryTemporary' together with the first form of `Filename'.

\begintt
    gap> tmpdir := DirectoryTemporary();;
    gap> Filename( [ tmpdir ], "file.new" );
    fail
    gap> Filename( tmpdir, "file.new" );    
    "/var/tmp/tmp.0.021738.0001/file.new"
\endtt

%%%%%%%%%%%%%%%%%%%%%%%%%%%%%%%%%%%%%%%%%%%%%%%%%%%%%%%%%%%%%%%%%%%%%%%%%
\Section{Special Filenames}

The special filename `"*stdin*"'   denotes the standard input, i.e.,  the
stream   through   which the  user  enters  commands  to  GAP.  The exact
behaviour of reading from `"*stdin*"' is  operating system dependent, but
usually   the  following happens.   If  GAP   was started with  no  input
redirection, statements are read from  the terminal stream until the user
enters the end of file character, which  is usually <ctr>-'D'.  Note that
terminal streams  are  special, in   that they may  yield ordinary  input
*after* an end of file.  Thus when control returns to  the main read eval
print loop  the user can continue with  GAP.  If GAP  was started with an
input redirection, statements are read  from the current position in  the
input file up to the  end of the file.   When control returns to the main
read eval print loop the input stream will still  return end of file, and
GAP will terminate.

The  special filename `"*errin*"' denotes   the stream connected  to  the
`stderr' output.  This stream is usually connected  to the terminal, even
if  the standard input was  redirected, unless the  standard error stream
was also redirected, in which case opening of `"*errin*"' fails.
    
The special filename `"*stdout*"'  can be used to  print to  the standard
output.

The special  filename `"*errout*"' can be  used to  print to the standard
error output file, which is usually  connected  to  the terminal, even if
the standard output was redirected.
    
%%%%%%%%%%%%%%%%%%%%%%%%%%%%%%%%%%%%%%%%%%%%%%%%%%%%%%%%%%%%%%%%%%%%%%%%%
\Section{File Access}

The  following functions  return `fail'  if a system  error occurred, for
example, if <name-file> does not exist (except for `IsExistingFile').  In
this case  the function `LastSystemError'  can be used to get information
about the error.

\>IsExistingFile( <name-file> )

returns `true'  if   a file  with the filename   <name-file>  exists, and
`false' otherwise.

\>IsReadableFile( <name-file> )

returns `true' if a  file with the  filename <name-file> exists *and* the
{\GAP} process has read  permissions for the  file, and `false' if it has
no such permissions.

\>IsWriteableFile( <name-file> )

returns `true'  if a file with the  filename <name-file> exists *and* the
{\GAP} process has write permissions for the  file, and `false' if it has
no such permissions.

\>IsExecuteableFile( <name-file> )

returns `true' if  a file with the  filename <name-file> exists *and* the
{\GAP} process has execute  permissions for the file,  and 'false' if  it
has no such permissions.  Note that execute permissions do not imply that
it is possible to execute the file, e.g., it could be an executable for a
different machine.

\>IsDirectoryPath( <name-file> )

returns `true' if the file with the  filename <name-file> exists *and* is
a directory.  It returns `false' if the file  exists but is a plain file.
Note  that this  functions does  not  check if   the  {\GAP} process  has
actually write or execute permissions for the directory.

*Examples*

\beginexample
    # the file ``/bin/date'' exists
    gap> IsExistingFile( "/bin/date" );    
    true

    # the file ``/bin/date.new'' does not exist
    gap> IsExistingFile( "/bin/date.new" );
    false

    # ``/bin/date'' is not a directory
    gap> IsExistingFile( "/bin/date/new" );
    fail
    gap> LastSystemError().message;
    "Not a directory"

    # the file ``/bin/date'' is readable
    gap> IsReadableFile( "/bin/date" );
    true

    # the file ``/bin/date.new'' does not exist
    gap> IsReadableFile( "/bin/date.new" );
    fail
    gap> LastSystemError().message;        
    "No such file or directory"

    # the file ``/bin/date'' is not writeable
    gap> IsWritableFile( "/bin/date" );
    false

    # but executable
    gap> IsExecutableFile( "/bin/date" );
    true
\endexample


%%%%%%%%%%%%%%%%%%%%%%%%%%%%%%%%%%%%%%%%%%%%%%%%%%%%%%%%%%%%%%%%%%%%%%%%%
\Section{File Operations}

\>Read( <name-file> )

reads   the input from  the  file  with  the  filename <name-file>,  this
filename must be given as a string.

`Read' first opens the file <name-file>.  If the file  does not exist, or
if GAP can not open it, e.g., because of access restrictions, an error is
signalled.

Then the contents of the file are read and evaluated, but the results are
not printed.  The reading and  printing happens exactly as described  for
the main loop (see "Main Loop").

If a statement in the file causes  an error a  break loop is entered (see
"Break  Loops").  The input   for this break loop  is  not taken from the
file, but from  the input connected to the  `stderr'  output of GAP.   If
`stderr' is not connected  to a terminal, no break  loop is  entered.  If
this break loop is left  with `quit' (or <ctr>-'D')  GAP does continue to
read from it.

Note that  a statement must  not begin in one  file  and end  in another,
i.e.,  <eof> (`end-of-file') is not    treated as whitespace,  but as   a
special symbol that must not appear inside any statement.

Note that one file may very well contain a read statement causing another
file to be read, before input is again taken from the  first file.  There
is an operating system dependent maximum on the  number of files that may
be open simultaneously, usually it is 15.

\>ReadAsFunction( <name-file> )

reads the  file  with filename <name-file> as  function  and returns this
function.

*Example*

Assume that the file ``/tmp/example.g'' contains the following

\begintt
    local a;

    a := 10;
    return a*10;
\endtt

Reading the files as function will not globber a global variable `a'.

\begintt
    gap> a := 1;
    1
    gap> ReadAsFunction("/tmp/example.g")();
    100
    gap> a;
    1
\endtt

\>ReadTest( <name-file> )

reads a test file, see "Test Files".

\>PrintTo( <name-file> )

works like `Print',  except that the output  is printed to the  file with
the name <name-file>  instead of the standard output.   This file must of
course be writable by  GAP, otherwise an  error is signalled.  Note  that
`PrintTo' will  *overwrite*  the  previous contents  of  this file  if it
already existed.  `AppendTo'  can   be used to   append to  a file   (see
"AppendTo").
    
There is an  operating system dependent maximum  to the number  of output
files that may be open simultaneously, usually this is 14.

\>AppendTo( <name-file> )

works like `PrintTo',  except  that the   output does  not overwrite  the
previous contents of the file, but is appended to the file.

\>LogTo( <name-file> )

causes the subsequent interaction to be logged  to the file with the name
<name-file>, i.e., everything you  see on your  terminal will also appear
in this file.  This file must of course be writable  by GAP, otherwise an
error  is signalled.  Note  that   `LogTo'  will overwrite the   previous
contents of this file if it already existed.

\)LogTo()

In this form `LogTo' stops logging again.

\>InputLogTo( <name-file> )

causes the  subsequent  input to be   logged to the   file with  the name
<name-file>, i.e., everything you type  on your terminal will also appear
in this file.  Note that `InputLogTo'  and `LogTo' cannot  be used at the
same time while `InputLogTo' and `OutputLogTo' can.

\)InputLogTo()

In this form `InputLogTo' stops logging again.

\>OutputLogTo( <name-file> )

causes   the subsequent output to   be logged to  the file  with the name
<name-file>,  i.e., everything {\GAP} prints  on  your terminal will also
appear in this file.  Note that  `OutputLogTo' and `LogTo' cannot be used
at the same time while `InputLogTo' and `OutputLogTo' can.

\)OutputLogTo()

In this form `OutputLogTo' stops logging again.

\>CrcFile( <name-file> )

computes  a checksum  value for the  file  with filename <name-file>  and
returns this value  as integer. The function return   `fail' if a  system
error occured, for example, if <name-file> does not  exist.  In this case
the function `LastSystemError' can  be used to  get information about the
error.

\>RemoveFile( <name-file> )

will remove the file with filename <name-file> and returns `true' in case
of  success.  The function returns `fail' if a  system error occured, for
example, if your permissions do not allow the removal of <name-file>.  In
this case the function  `LastSystemError' can be  used to get information
about the error.

%%%%%%%%%%%%%%%%%%%%%%%%%%%%%%%%%%%%%%%%%%%%%%%%%%%%%%%%%%%%%%%%%%%%%%%%%
%%
%E  files.tex . . . . . . . . . . . . . . . . . . . . . . . . . ends here
