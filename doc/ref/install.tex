%%%%%%%%%%%%%%%%%%%%%%%%%%%%%%%%%%%%%%%%%%%%%%%%%%%%%%%%%%%%%%%%%%%%%%%%%
%%
%W  install.tex               GAP documentation              Frank Celler
%W                                                     & Martin Schoenert
%%
%H  @(#)$Id$
%%
%Y  Copyright 1997,  Lehrstuhl D fuer Mathematik,  RWTH Aachen,   Germany
%%
%%  This file contains the  description of the installation procedure and
%%  command line options for various operating systems.
%%

%%%%%%%%%%%%%%%%%%%%%%%%%%%%%%%%%%%%%%%%%%%%%%%%%%%%%%%%%%%%%%%%%%%%%%%%%
\Chapter{Installing GAP}


%%%%%%%%%%%%%%%%%%%%%%%%%%%%%%%%%%%%%%%%%%%%%%%%%%%%%%%%%%%%%%%%%%%%%%%%%
\Section{Features of GAP for UNIX}

When you start {\GAP}  for UNIX, you may  specify a number of  options on
the command-line to change  the default behaviour  of {\GAP}.  All  these
options  start with a  hyphen `-', followed by a  single letter.  Options
must not be   grouped, e.g.,  `gap  -gq' is   illegal, use   `gap -g  -q'
instead.  Some options  require an argument, this  must follow the option
and  must be separated   by a <space>, e.g.,  `gap   -m 256k', it is  not
correct to say `gap -m256k' instead.

{\GAP} for UNIX will only accept lower case options.

As is  described in  the previous section  (see  "Installation of GAP for
UNIX") usually you  will not execute {\GAP}  directly.  Instead  you will
call a shell script, with the name `gap',  which in turn executes {\GAP}.
This shell script sets some options as  necessary to make {\GAP} work on
your  system.  This means that the   default settings mentioned below may
not be what you experience when you execute {\GAP} on your system.

`-h'

The  options  `-h' tells {\GAP}   to print  a   summary of all  available
options.  {\GAP} exits after printing the summary,  all other options are
ignored.

`-b'

The option `-b'  tells {\GAP} to  suppress the banner.  That  means  that
{\GAP} immediately prints the prompt.  This is useful  when you get tired
of the banner after a while.

`-q'

The option  `-q' tells {\GAP} to be  quiet.  This means that  {\GAP} does
not display the banner nor the prompt `gap>'.  This is useful if you want
to run {\GAP} as a filter with  input and output  redirection and want to
avoid the banner and the prompts clobbering the output file.

`-e'

The option `-e' tells {\GAP} not to quit when receiving a `<ctr>-D' on an
empty  input line.  This option  should not be  used when  the input is a
file or pipe.

`-f'

The option `-f' tells {\GAP} to enable the line  editing and history (see
"Line Editing").

In general  line editing will  be enabled if the input  is connected to a
terminal.  There are rare circumstances, for  example when using a remote
session with a broken telnet  implementation, when this detection  fails.
Try to use `-f' in this case to enable line editing.

`-n'

The option  `-n'  tells {\GAP}  to disable  the line editing  and history
(see "Line Editing").

You may want to do this if the command line  editing is incompatible with
another program that is used to run {\GAP}.  For example if {\GAP} is run
from inside a GNU Emacs shell window, `-n' should be used since otherwise
every input line will be echoed twice, once by Emacs and once by {\GAP}.

`-x <length>'

With this option  you can tell  {\GAP}  how long lines  are.  {\GAP} uses
this value to decide when to split long lines.

The default value is 80, which is the right value if you have  a standard
ASCII terminal.  If you have a larger monitor, or use a  smaller font, or
redirect the output to a printer, you may want to increase this value.

`-y <length>'

With this option   you can tell {\GAP} how   many lines your  screen has.
{\GAP} uses this value to decide after how many lines  of on-line help it
should wait.

The default value is 24, which is the right  value if you have a standard
ASCII terminal.  If you have a larger monitor, or use  a smaller font, or
redirect the output to a printer, you may want to increase this value.

`-g'

The option  `-g' tells {\GAP} to print a information message every time a
full garbage collection is performed.

\begintt
    #G  FULL 44580/2479kb live   57304/4392kb dead   734/4096kb free
\endtt

For example, this tells   you that  there   are 44580 live  objects  that
survived a  full garbage   collection,  that 57304  unused  objects  were
reclaimed by it, and that 734 KByte of totally  allocated 4096 KBytes are
available afterwards.

`-g -g'

If  you  give the option `-g'  twice  {\GAP} prints a information message
every time a partial or full garbage collection is performed.

\begintt
    #G  PART 9405/961kb+live   7525/1324kb+dead   2541/4096kb free
\endtt

For  example,  this  tells you  that  9405  objects survived  the partial
garbage collection and  7525 objects were  reclaimed, and that 2541 KByte
of totally allocated 4096 KByte are available afterwards.

`-m <memory>'

The option `-m' tells {\GAP} to allocate <memory>  bytes at startup time.
If the last character of <memory> is `k' or `K' it is taken in KBytes and
if the last character is `m' or `M' <memory> is taken in MBytes.

Under UNIX the default amount  of memory allocated  by {\GAP} is 8 MByte.
The amount of memory  should be large  enough so that computations do not
require  too  many  garbage  collections.  On the   other hand  if {\GAP}
allocates more virtual memory than  is physically available it will spend
most of  the  time  paging.  

`-o <memory>'

The option `-o' tells {\GAP} to allocate  at most <memory> bytes.  If the
last character of <memory> is `k' or `K' it is taken in KBytes and if the
last character is `m' or `M' <memory> is taken in MBytes.

Under UNIX the  default amount is 64 MByte.   If more than this amount is
required {\GAP} prints an error messages and quits.

`-l <pathname>'

The option `-l' tells  {\GAP}  that the  {\GAP} root directory  "GAP Root
Directory" is <pathname>.  Per default <pathname> is './', i.e., the root
directory is normally   expected  to be the   current  directory.  {\GAP}
searches for the library   files  which contain the functions   initially
known to  {\GAP},  in the   subdirectory `lib/'  of  the  root directory.
<pathname> should end with a   pathname separator, i.e., '/', but  {\GAP}
will silently   add one  if  it is  missing.  {\GAP}   will read the file
'<pathname>/lib/init.g' during startup.  If  {\GAP} cannot find this file
it will print the following warning

\begintt
    gap: hmm, I cannot find 'lib/init.g' maybe use option '-l <gaproot>'?
\endtt

It is not possible  to use {\GAP} without the  library files, so you must
not  ignore this warning.  You  should leave {\GAP}   and start it again,
specifying the correct root path using the '-l' option.

It  is  also possible  to  specify several  alternative library  paths by
separating them  with semicolons  ';'.  This is explained  in detail in
"GAP Root Directory".

`-r'

The option `-r'  tells {\GAP}  not to  read the user  supplied `~/.gaprc'
files.

`<filename> ...'

Further arguments are taken as filenames of files that are read by {\GAP}
during startup, after the system  and private init   files are read,  but
before the first prompt is  printed.  The files are read  in the order in
that they  appear on the command line.   {\GAP} only accepts 14 filenames
on the command  line.  If a  file cannot  be opened {\GAP}  will print an
error message and will abort.

%%%%%%%%%%%%%%%%%%%%%%%%%%%%%%%%%%%%%%%%%%%%%%%%%%%%%%%%%%%%%%%%%%%%%%%%%
\Section{Advanced Features of GAP}

The following options are in general not needed  for the normal operation
of {\GAP}.  They are mostly used for debugging.

`-a <memory>' 

GASMAN, the storage manager of {\GAP} uses `sbrk' to get blocks of memory
from (certain) operating systems and it is required that subsequent calls
to `sbrk' produce  adjacent blocks of memory  in this case because {\GAP}
only wants to  deal with one large  block  of memory.  If the  C function
`malloc' is called for whatever reason it is likely that `sbrk' no longer
produces adjacent blocks, therefore GAP does not use `malloc' itself.

However some operation systems insist on calling `malloc'  when a file is
open to fill a buffer.  In order to catch these  cases GAP preallocates a
block of memory with  `malloc' which is immediately  freed. The amount of
prealloc can be  controlled with the  `-a' option.  If the last character
of <memory> is `k' or `K' it is taken in KBytes and if the last character
is `m' or `M' <memory> is taken in MBytes.

`-D'

The `-D' options tells {\GAP} to print short  messages when it is reading
or completing files or loading modules.

\begintt
    |#I  READ_GAP_ROOT: loading 'lib/kernel.g' as GAP file
\endtt

Tells   you  that   {\GAP}  has  started    to  read the   library   file
``lib/kernel.g''.

\begintt
    |#I  READ_GAP_ROOT: loading 'lib/kernel.g' statically
\endtt

Tells you that {\GAP}  has used the compiled  version of the library file
``lib/kernel.g''.  This compiled    module was statically linked to   the
{\GAP} kernel at the time the kernel was created.

\begintt
    |#I  READ_GAP_ROOT: loading 'lib/kernel.g' dynamically
\endtt

Tells you that {\GAP} has loaded the compiled version of the library file
``lib/kernel.g''.   This compiled module  was  dynamically loaded to  the
{\GAP} kernel at runtime from a corresponding `.so' file.

\begintt
    |#I  completing 'lib/domain.gd'
\endtt

Tells you   that {\GAP} has   completed the file  ``lib/domain.gd''.  See
"Completion Files" for more information about completion of files.

`-M'

The `-M' option  tells  {\GAP}  not to check   for  nor to  use  compiled
versions of library files.

`-N'

The `-N' option tells  {\GAP}  not to check   for nor to use   completion
files, see "Completion Files".

`-X'

The `-X' option  tells {\GAP} to make  a consistency check of the library
file and the corresponding completion   file when reading the  completion
file.

`-Y'

The `-X' option tells {\GAP}  to make a  consistency check of the library
file and  the corresponding completion  file  when completing the library
file.

`-i <filename>'

The `-i' changes the name of the init file from the default ``init.g'' to
<filename>.

%%%%%%%%%%%%%%%%%%%%%%%%%%%%%%%%%%%%%%%%%%%%%%%%%%%%%%%%%%%%%%%%%%%%%%%%%
\Section{Completion Files}

The standard distribution of {\GAP} already  contains completion files so
in general *you do not need to create these files by yourself*.

When starting, {\GAP} reads  in the  whole  library.  As this takes  some
time library files are  normally condensed into  completion files.  These
completion files contain the  basic skeleton of  the library but  not the
function bodies.   When a function  body is required, for example because
you  want   to  execute the  corresponding    function, the  library file
containing the function body is completed.

Completion files reduce the startup time of {\GAP} drastically.  However,
it also means that the information stored in the completion files and the
library must  be  consistent.   If you   changes a library  file  without
recreating the completion files disaster is bound to happen.

Bugfixes distributed for {\GAP}  will  also update the completion  files,
therefore you only  need to update  them if  you  changed the  library by
yourself.

However, if you are modifying a library file  a more convenient way is to
use the `-X' option that allows you (in most cases) to use the completion
files   for the unchanged  parts of  library   files and avoids using the
completion files for the changed parts.  After you are finished modifying
the library files you can recreate the completion files using:

\>CreateCompletionFiles()
\)CreateCompletionFiles( <path> )

To create  completion files you must  have write permissions to `<path>',
which defaults to the  first root directory.   Start {\GAP} with the `-N'
option (to  suppress the reading  of any existing completion files), then
execute the command `CreateCompletionFiles( <path> );', where <path> is a
string giving a   path to the home   directory of  {\GAP} (the  directory
containing the `lib' directory).

This produces, among lots of informational output, the completion files.

\begintt
    $ gap4 -N
    gap> CreateCompletionFiles();
    |#I  converting "gap4/lib/read2.g" to "gap4/lib/read2.co"
    |#I    parsing "gap4/lib/process.gd"
    |#I    parsing "gap4/lib/listcoef.gi"
    ...
\endtt



