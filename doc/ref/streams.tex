%%%%%%%%%%%%%%%%%%%%%%%%%%%%%%%%%%%%%%%%%%%%%%%%%%%%%%%%%%%%%%%%%%%%%%%%%
%%
%W  streams.tex               GAP documentation              Frank Celler
%W                                                     & Martin Schoenert
%%
%H  @(#)$Id$
%%
%Y  Copyright 1997,  Lehrstuhl D fuer Mathematik,  RWTH Aachen,   Germany
%%
%%  This file contains the description of streams.
%%


%%%%%%%%%%%%%%%%%%%%%%%%%%%%%%%%%%%%%%%%%%%%%%%%%%%%%%%%%%%%%%%%%%%%%%%%%
\Chapter{Streams}

Files can be read and  written using `Read'  and `AppendTo', however, the
former allows only to read a complete file as {\GAP} input and the latter
has a high time penalty if a lot of small pieces of output are written to
large files.

In addition  to  `Read'  and `AppendTo'  files  can  also be  accessed as
*streams*.  An *input stream* delivers characters to {\GAP} which *reads*
them from the stream.  When an  input stream has delivered all characters
it is  at `end-of-stream'.  An  *output  stream* receives characters from
{\GAP} which *writes* them   to  the stream.    Streams differ in   their
treatment of `end-of-line' and  non-printable characters.  *Text streams*
translate the `end-of-line'     character (`{'\\n'}')  to  the    systems
representation   of   `end-of-line'   (e.g.,    <new-line>  under   UNIX,
<carriage-return> under  MacOS,  <carriage-return>-<new-line> under  DOS,
implicit under VMS), and  the behaviour  for nonprintable characters   is
undefined.  *Binary streams* do not translate the `end-of-line' character
and allow nonprintable characters.

Note that binary streams are *@not yet implemented@*.

Whereas it   is cheap to append  to  a stream, streams  do consume system
resources, only a limited amount can be open at any time, therefore it is
neccessary  to  close a stream   as soon as possible  using `CloseStream'
described in section "CloseStream".  If  creation of a stream failed then
`LastSystemError' (see "LastSystemError") can  be used to get information
about the failure.  However,  if an error occurred  in one of the read or
write functions `LastError' (see "LastError") should be used.

%%%%%%%%%%%%%%%%%%%%%%%%%%%%%%%%%%%%%%%%%%%%%%%%%%%%%%%%%%%%%%%%%%%%%%%%%
\Section{File Streams}

File streams  are  streams associated with  files.  An  input file stream
reads  the characters  it delivers from  a  file,  an output  file stream
prints the characters it receives to a file.  The following functions can
be used to create such streams.  They return `fail' if an error occurred,
in this case `LastSystemError' (see "LastSystemError") can be used to get
information about the error.

\>InputTextFile( <name-file> )

returns an input stream in the category `IsInputTextStream' that delivers
the characters from the file <name-file>.

\>OutputTextFile( <name-file>, <append> )

returns  an output stream  in the category `IsOutputTextFile' that writes
received characters  to the file   <name-file>.  If <append> is  `false',
then the  file is emptied first,  otherwise received characters are added
at the end of the list.

*Example*

\begintt
    # use a temporary directory
    gap> name := Filename( DirectoryTemporary(), "test" );;

    # create an output stream, append output, and close again
    gap> output := OutputTextFile( name, true );;
    gap> AppendTo( output, "Hallo\n", "You\n" );
    gap> CloseStream(output);

    # create an input, print complete contents of file, and close
    gap> input := InputTextFile(name);;
    gap> Print( ReadAll(input) );
    Hallo
    You
    gap> CloseStream(input);

    # append a single line
    gap> output := OutputTextFile( name, true );; 
    gap> AppendTo( output, "AppendLine\n" );

    # close output stream to flush the output
    gap> CloseStream(output);

    
    # create an input, print complete contents of file, and close
    gap> input := InputTextFile(name);;
    gap> Print( ReadAll(input) );
    Hallo
    You
    AppendLine
    gap> CloseStream(input);
\endtt

%%%%%%%%%%%%%%%%%%%%%%%%%%%%%%%%%%%%%%%%%%%%%%%%%%%%%%%%%%%%%%%%%%%%%%%%%
\Section{String Streams}

String streams   are streams associated  with   strings.  An input string
stream reads  the characters it delivers  from a string, an output string
stream  appends the characters  it receives  to  a string.  The following
functions can be used to create such streams.

\>InputTextString( <string> )

returns an  input stream that  delivers the  characters  from the  string
<string>.  The  <string> is not changed when  reading characters  from it
and  changing  the <string> after  the  call to `InputTextString'  has no
influence on the input stream.

\>OutputTextString( <list>, <append> )

returns an output stream that puts  all received characters into the list
<list>.   If <append>   is  `false',  then  the list   is  emptied first,
otherwise received characters are added at the end of the list.

*Example*

\begintt
    # read input from a string
    gap> input := InputTextString( "Hallo\nYou\n" );;
    gap> ReadLine(input);
    "Hallo\n"
    gap> ReadLine(input);
    "You\n"

    # print to a string
    gap> str := "";;
    gap> OutputTextString( str, true );;
    gap> str := "";;                    
    gap> out := OutputTextString( str, true );;
    gap> PrintTo( out, 1, "\n", (1,2,3,4)(5,6), "\n" );
    gap> CloseStream(out);
    gap> Print( str );
    1
    (1,2,3,4)(5,6)
\endtt

%%%%%%%%%%%%%%%%%%%%%%%%%%%%%%%%%%%%%%%%%%%%%%%%%%%%%%%%%%%%%%%%%%%%%%%%%
\Section{Dummy Streams}

The following  two commands create  dummy streams  which will consume all
characters and never deliver one.


\>InputTextNone()

returns a dummy input text stream, which delivers no characters, i.e., it
is always at end of stream.  Its main use  is for calls to `Process' (see
"Process") when the started program does not read anything.

\>OutputTextNone()

returns a dummy output   stream, which discards all received  characters.
Its main use is for calls to `Process' when  the started program does not
write anything.

%%%%%%%%%%%%%%%%%%%%%%%%%%%%%%%%%%%%%%%%%%%%%%%%%%%%%%%%%%%%%%%%%%%%%%%%%
\Section{CloseStream}

\>CloseStream( <stream> )

In order  to preserve system resources  and to flush output streams every
<stream> should  be  closed  as soon   as  it is   no longer   used using
`CloseStream'.

It is an error to  try to read  characters from or  write characters to a
closed  stream.   Closing a  stream tells  the {\GAP}   kernel and/or the
operating system kernel  that the file is  no longer needed.  This may be
necessary  because  the {\GAP} kernel  and/or  the  operating  system may
impose a limit on how many streams may be open simultaneously.

%%%%%%%%%%%%%%%%%%%%%%%%%%%%%%%%%%%%%%%%%%%%%%%%%%%%%%%%%%%%%%%%%%%%%%%%%
\Section{Input Stream Operations}

\>IsEndOfStream( <input-stream> )

returns `true' if  the input stream  is  at `end-of-stream', and  `false'
otherwise.  Note  that `IsEndOfStream' might return  `false' even  if the
next `ReadByte' fails.

\>Read( <input-text-stream> )!{for streams}

reads the input-text-stream as  input  until `end-of-stream' occurs.  See
"File Operations" for details.

\>ReadAsFunction( <input-text-stream> )!{for streams}

reads the input-text-stream as function and returns this function.

\>ReadTest( <input-text-stream> )!{for streams}

reads the input-text-stream as  test input until `end-of-stream' occurs.
See "File Operations" for details.

*Example*

\beginexample
    # as function with local `a' does not change the global one
    gap> a := 1;;
    gap> i := InputTextString( "local a; a := 10; return a*10;" );;
    gap> ReadAsFunction(i)();
    100
    gap> a;
    1

    # reading it via `Read' does
    gap> i := InputTextString( "a := 10;" );;                      
    gap> Read(i);
    gap> a;
    10
\endexample

\>ReadByte( <input-text-stream> )

returns the  next character as  integer  or `fail'  if <input-stream> has
reached the `end-of-stream'.

\>ReadLine( <input-text-stream> )

returns  the   next line as   string  *including*  the  newline character
`{'\\n'}' or `fail' if <input-stream> has reached `end-of-stream'.

Note that *no* newline is added if the last line is incomplete.

\>ReadAll( <input-text-stream> )

returns   the  whole available input as    string.  If <input-stream> has
reached `end-of-stream' an empty string is returned, *not* `fail'.

*Example*

\beginexample
    gap> i := InputTextString( "1Hallo\nYou\n1" );;
    gap> ReadByte(i);
    49
    gap> CHAR_INT(last);
    '1'
    gap> ReadLine(i);
    "Hallo\n"
    gap> ReadLine(i);
    "You\n"
    gap> ReadLine(i);
    "1"
    gap> ReadLine(i);
    fail
    gap> ReadAll(i);
    ""
    gap> RewindStream(i);;
    gap> ReadAll(i);     
    "1Hallo\nYou\n1"
\endexample

%%%%%%%%%%%%%%%%%%%%%%%%%%%%%%%%%%%%%%%%%%%%%%%%%%%%%%%%%%%%%%%%%%%%%%%%%
\Section{Output Stream Operations}

\>WriteByte( <output-stream>, <int> )

writes the  next  character  (given  as *integer*)  to the  output stream
<output-stream>.  The function  returns `true' if  the write succeeds and
`fail' otherwise.

\>WriteAll( <output-stream>, <string> )

appends  <string> to the <output-stream>.   No final  newline is written.
The function returns `true' if the write succeeds and `fail' otherwise.

\>WriteLine( <output-stream>, <string> )

appends  <string> to  the <output-stream>.   A  final newline is written.
The function returns `true' if the write succeeds and `fail' otherwise.

*Example*

\beginexample
    gap> str := "";; a := OutputTextString(str,true);;
    gap> WriteByte(a,INT_CHAR('H'));
    true
    gap> WriteLine(a,"allo");
    true
    gap> WriteAll(a,"You\n");
    true
    gap> CloseStream(a);
    gap> Print(str);
    Hallo
    You
\endexample

\>AppendTo( <output-stream>, <arg1>, ... )!{for streams}

works like  `Print',  except that the  output is  appended to the  output
stream <output-stream>.

*Example*

\beginexample
    gap> str := "";; a := OutputTextString(str,true);;
    gap> AppendTo( a, (1,2,3), ":", Z(3) );
    gap> CloseStream(a);
    gap> Print( str, "\n" );
    (1,2,3):Z(3)
\endexample


%%%%%%%%%%%%%%%%%%%%%%%%%%%%%%%%%%%%%%%%%%%%%%%%%%%%%%%%%%%%%%%%%%%%%%%%%
%%
%E  streams.tex . . . . . . . . . . . . . . . . . . . . . . . . ends here
