\input ../gapmacro

\FrontMatter
%\input copyrigh
%\input preface

\Chapters
%\input introduc
%\input inoutput (Chapter ``Communicating with GAP'')
\Input{lists}
%\input record   (Chapter ``Records'')
\Input{group}
\Input{solvgrp}
\Input{permgrp}
\Input{newkind}
%\input matgrp   (Chapter ``Matrix Groups'')
%\input finpres  (Chapter ``Finitely Presented Groups and Ag Groups'')
%\input arithmet (Chapter ``Arithmetic'' (incl. Integers, Rationals etc.))

\Input{grplib}

\Appendices
\input glossary

\Chapter Bibliography

\Bibliography

\Chapter Index

Entries in     `typewriter style' refer  to   {\GAP}  functions (e.g.,
`Size'); entries in roman type   refer to {\GAP}-specific concepts.  A
page number in  {\it italics} denotes a  whole section which is  named
like the   index  entry.  For  mathematical terms,    especially their
definitions, see   the glossary in  appendix~A. For  references to the
algorithms implemented in {\GAP}, see the bibliography in appendix~B.

Keywords   are   sorted  with    case  and   spaces    ignored,  e.g.,
```PermutationCharacter''' comes before ``permutation group''.

\Index

\null\vfill
\centerline{\titlefont Contributions to a forthcoming}
\centerline{\titlefont GAP 4 Manual}

\Chapter Contents
  
\TableOfContents

\bye
