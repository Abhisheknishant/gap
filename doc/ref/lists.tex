%%%%%%%%%%%%%%%%%%%%%%%%%%%%%%%%%%%%%%%%%%%%%%%%%%%%%%%%%%%%%%%%%%%%%%%%%
\Chapter{Lists and Sets}

Modern mathematics, especially algebra, is based on set theory. When sets
are represented in a computer, they inadvertently turn into lists. That's
why we start our  survey of the various  objects {\GAP} can handle with a
description of  lists  and their  manipulation. {\GAP}  regards sets as a
special kind of lists, namely  as lists without  holes or duplicates, and
whose entries are ordered with respect to the precedence relation~`\<'.

%%%%%%%%%%%%%%%%%%%%%%%%%%%%%%%%%%%%%%%%%%%%%%%%%%%%%%%%%%%%%%%%%%%%%%%%%
\Section{Lists and list objects}\null

%%%%%%%%%%%%%%%%%%%%%%%%%%%%%%%%%%%%%%%%%%%%%%%%%%%%%%%%%%%%%%%%%%%%%%%%%
\Section{Basic list operations}

\>Length( <list> )
\>IsBound( <list>[ <pos> ] )
\>`<list>[ <pos> ]'{list element!access}

What is a *list* in  {\GAP} may mathematically be  regarded as a function
<list> defined on  a subset of the  natural numbers which takes as values
{\GAP} objects. The highest number for which this  function is defined is
called the `Length' of the list. More  generally, `IsBound( <list>[ <pos>
] )' returns `true' if <list> is defined on the number  <pos>, and if so,
`<list>[ <pos> ]'  returns its value on this  number. These are the three
basic operations that must be implemented for any list. They allow access
to the entries of a list, but they do not allow changes to the list.

If you  want to  install your  own  methods for  the operation  `IsBound(
<list>[  <pos>  ] )', you   have to specify this   operation  as a {\GAP}
object, i.e., you  want to know  a  name for  it: It  is  assigned to the
variable `IsBound\\[\\]', which you can,  e.g., pass to  `InstallMethod'.
In other words,  `IsBound\\[\\]( <list>,  <pos>  )' is the function  call
equivalent  to the (more  elegant) syntactic expression `IsBound( <list>[
<pos> ]  )'.  Similarly, the operation   for `<list>[ <pos> ]'  is called
`\\[\\]'.

\>`<elm> in <list>'{membership test}!{for lists}
\>Position( <list>, <elm> \[, <after> \] )
\>`<list>\{ <poss> \}'{sublist!access}

`<elm> in <list>' returns `true' if <elm> is equal to  one of the entries
of  <list>, and `false'  otherwise. `Position( <list>,  <elm>, <after> )'
returns the first position $pos  > after$  such  that `<list>[ <pos> ]  =
<elm>', and `false' if  there is no  such  position. If a third  argument
<after> is  not given, it is  set to 0, i.e.  the whole list is searched.
Finally, in  the construct `<list>\{  <poss> \}',  <poss> must  be a list
without holes, this makes a  list as long as <poss>  whose <i>th entry is
`<list>[ <poss>[ <i> ] ]'.

These  functions  are automatically  manufactured    from the basic  list
operations   above.  The  operation  `<list>\{    <poss>  \}' is   called
`\\\{\\\}'.

\>Unbind( <list>[ <pos> ] )
\>`<list>[ <pos> ] := <elm>'{list element!assignment}
\>`<list>\{ <poss> \} := <sublist>'{sublist!assignment}

These   operations modify a  given  list.  `Unbind'  removes an entry and
decreases  the `Length' of the list  if it was  the  last entry. `<list>[
<pos> ]  := <elm>' sets an  entry  and `<list>\{ <poss> \}  := <sublist>'
sets a whole sublist  (of course <poss> and <sublist>  must have the same
length). This latter operation is automatically manufactured from the one
before  which sets a  single entry.  If  you  perform such an  assignment
operation and type  the left hand side of  the assignment afterwards, you
will get the right hand side as result.

These     operations      are    called   `Unbind\\[\\]',  `\\[\\]\\:\\='
resp.~`\\\{\\\}\\:\\='.

%%%%%%%%%%%%%%%%%%%%%%%%%%%%%%%%%%%%%%%%%%%%%%%%%%%%%%%%%%%%%%%%%%%%%%%%%
\Section{AsList and Enumerators}\null

\>AsList( <D> )

`AsList( <D> )' constructs a {\GAP} object that represents a domain as if
it were an  immutable list of its elements.  Note that we say ``a list'',
not ``the list''  because there are  many  ways of ordering  the elements
and, indeed, {\GAP} does  not  guarantee anything  about the  ordering it
chooses. If  you  say `$list_1$ := AsList(  $D_1$   );' and `$list_2$  :=
AsList( $D_2$ );' then even if `$D_1$ = $D_2$', the orderings of $list_1$
and $list_2$ may be different,  and consequently  you may have  `$list_1$
\<> $list_2$'. After all, $D_1$ may be represented differently than $D_2$
and the `AsList' may  be   constructed depending on  the  representation.
(Even if you say `$list_3$ := AsList( $D_1$ );' again later, you may have
`$list_3$ \<> $list_1$'.)

Objects constructed  by  `AsList'  behave like  immutable  lists,  so you
cannot modify them, neither by assigning to elements  or sublists, nor by
`Unbind'ing. But you  can use the three  basic  list operations `Length',
`IsBound' and `[ ]' with them.  The sublist operator `<list>\{ <poss> \}'
is  also  available, it  is  normally  just  manufactured from  the `[ ]'
operator. The  time needed by  these operations should  not depend on the
length of  the list, but the  memory needed should grow proportionally to
the length, as should  the time needed by `Position'.  This is just as it
is with ordinary lists, but it is not always what you want. To make these
operations  space  efficient  rather  than time  efficient,   and to make
`Position' more time  efficient, it is necessary  to  overlay the generic
methods by   more  special algorithms, which   will  generally  require a
different internal representation of the object <list>. To emphasize this
distinction,  such space   efficient    objects are not  constructed   by
`AsList', but instead by the `Enumerator'.

\>Enumerator( <D> )

An *enumerator* is a {\GAP} object that represents a  collection as if it
were an immutable list  of its elements;  just like an object constructed
by `AsList'. The remarks made at the beginning of that section also apply
to enumerators, and the   only difference is  that enumerators  are space
efficient   rather   than time  efficient. Again   you   can use the list
operations  like `<enum>[ <pos> ]'  and `Position', but you cannot assign
to single positions  or to sublists of enumerators,  nor can you `Unbind'
elements.

\>PositionCanonical( <enum>, <elm> )

`PositionCanonical( <enum>,  <elm> )'  works  basically  like  `Position(
<enum>, <elm>  )', but there  can be a  difference in the  behaviour when
<enum> is  not a plain list  but a specially-constructed enumerator. Such
enumerators  are   often used  to   enumerate domains  whose elements are
mathematically defined as equivalence classes   of other elements. As  an
example, take a projective space  (see "ProjectiveSpace"), whose elements
can be defined     as  equivalence  classes  of  vectors    under  scalar
transformations.  The     following   {\GAP}   commands    construct  the
three-dimensional projective space over the field with three elements.
\begintt
    gap> vec := GF(3)^4;;  pro := ProjectiveSpace( vec );
    ProjectiveSpace( ( GF(3)^4 ) )
    gap> enum := Enumerator( vec );
    <enumerator of ProjectiveSpace( ( GF(3)^4 ) )>
\endtt
Mathematically  the elements  of `pro'   are  equivalence classes  of two
non-zero vectors from `vec' each, but {\GAP} represents such a class by a
canonical  (i.e. here, left-normed)  representative.   This is what   the
enumerator returns when you ask for `enum[ <pos> ]'.
\begintt
    gap> enum{[3..4]};
    [ [ 0*Z(3), 0*Z(3), Z(3)^0, Z(3)^0 ], [ 0*Z(3), 0*Z(3), Z(3)^0, Z(3) ] ]
\endtt
On the other hand, if you want to  specify an element  of `pro', i.e., an
equivalence class,  you can give either  of its elements;  and if you ask
for  the   `PositionCanonical'  in  the   enumerator `enum',  {\GAP} will
silently replace the element by the  canonical representative. This does,
however, no  longer work, if you ask  for the `Position'  because in this
case {\GAP} replaces `pro'  by the plain  list  of its elements  and does
then no longer know that these elements form a projective space.
\begintt
    gap> v := [2,1,1,1]*Z(3)^0;;  PositionCanonical( enum, v );
    40
    gap> enum[40];
    [ Z(3)^0, Z(3), Z(3), Z(3) ]
    gap> Position( enum, v );
    fail
\endtt
This  manual always mentions explicitly when  such a silent conversion is
done by an enumerator.

%%%%%%%%%%%%%%%%%%%%%%%%%%%%%%%%%%%%%%%%%%%%%%%%%%%%%%%%%%%%%%%%%%%%%%%%%%%%%
% Local Variables:
% mode:             text
% mode:             outline-minor
% outline-regexp:   "\\\\Chapter\\|\\\\Section"
% fill-column:      73
% End:
%%%%%%%%%%%%%%%%%%%%%%%%%%%%%%%%%%%%%%%%%%%%%%%%%%%%%%%%%%%%%%%%%%%%%%%%%%%%%


