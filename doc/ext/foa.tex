\Chapter{Function-Operation-Attribute Triples}

{\GAP}  is eager  to maintain information  that  it has gathered about an
object, possibly by lengthy  calculcations. The most  important mechanism
for information maintenance  is the  automatic  storage and look-up  that
takes  place  for  *attributes*; and    this   was already mentioned   in
section~"tut:Attributes"  in the   tutorial. In   this  chapter  we  will
describe further mechanisms that  allow storage of  results that  are not
values of attributes.

\index{FOA triples}%
The  idea which is   common to all  sections  is that certain operations,
which are  not themselves attributes, have  an attribute  associated with
them. To automatically delegate tasks to the  attribute, {\GAP} knows, in
addition  to  the *operation*  and  the  *attributes*   also an  ordinary
*function*, which  is  ``wrapped around'' the  other  two. This ``wrapper
function''  is called by    the user and    decides whether to  call  the
operation    or  the    attribute     or  possibly     both.   The  whole
*f*unction-*o*peration-*a*ttribute triple (or *FOA triple*)  is set up by
a single   {\GAP} command which  writes the  wrapper function and already
installs  some methods,  e.g.,  for the attribute  to   fall back on  the
operation. The idea  is then that subsequent  methods, which  perform the
actual computation, are installed   only for the operation,   whereas the
wrapper function remains unaltered, and  in general no additional methods
for the attribute are required either.

%%%%%%%%%%%%%%%%%%%%%%%%%%%%%%%%%%%%%%%%%%%%%%%%%%%%%%%%%%%%%%%%%%%%%%%%%
\Section{Key-dependent operations}

There  are several functions  which  require as first  argument a domain,
e.g., a  group, and as second  argument  something much simpler,  e.g., a
prime. `SylowSubgroup'  is an  example. Since  its  value depends  on two
arguments, it cannot be  an attribute,  yet one would  like to  store the
Sylow subgroups once they have been  computed. The idea  is to store them
in a list,  which is then regarded as  an attribute of the  group, called
`ComputedSylowSubgroups'. The  name    implies that  the  value  of  this
attribute  changes in    the  course of   a  {\GAP}  session,  whenever a
newly-computed Sylow subgroup is put into the  list. Therefore, this is a
*mutable attribute*.

The   list    contains    primes     in    its   odd   positions      and
`ComputedSylowSubgroups(  <G> )[  <even>     ]'   holds the  value     of
`SylowSubgroup( <G>,  <p>  )' if  `<p> =  ComputedSylowSubgroups(  <G> )[
<even> - 1 ]'. The pairs are sorted in increasing order of <p>. This list
is automatically maintained by the  function `SylowSubgroup', which calls
the operation `SylowSubgroupOp(  <G>, <p> )' to do  the real work, if the
prime <p> cannot be found in the list.

The same mechanism  works for other functions as  well, e.g, for `PCore',
but also for `HallSubgroup', where the second argument is not a prime but
a set of primes.

\>KeyDependentFOA( <name>, <dom-req>, <key-req>, <key-test> )

To   set up  the   three objects  `SylowSubgroup',  `SylowSubgroupOp' and
`ComputedSylowSubgroups'  together,   the declaration file ``lib/grp.gd''
contains the following code:
\begintt
    tmp := KeyDependentFOA( "SylowSubgroup",
	       IsGroup, IsPosRat and IsInt, "prime" );
    SylowSubgroup          := tmp[1];
    SylowSubgroupOp        := tmp[2];
    ComputedSylowSubgroups := tmp[3];
\endtt
The second and third argument  to `KeyDependentFOA', namely <grp-req> and
<dom-req> specify   the  required filters  for   the first resp.\  second
argument of the operation  `SylowSubgroupOp', which are needed  when this
operation is created with  `NewOperation' (see "NewOperation"). <grp-req>
is   also    the required  filter    for  the   corresponding   attribute
`ComputedSylowSubgroups'. The fourth argument <key-test> is a function to
which  the second  argument <p> of  `SylowSubgroup(  <G>, <p> )' will  be
passed.  This function can  perform tests  on  <p> and  raise an error if
appropriate.  If, as in our example,  <key-test> has the value `"prime"',
it is silently  replaced  with a function that   tests whether  <p> is  a
prime.
\beginexample
    gap> s4 := Group((1,2,3,4),(1,2));;
    gap> SylowSubgroup( s4, 5 );;  ComputedSylowSubgroups( s4 );
    [ 5, Group( [  ], ... ) ]
    gap> SylowSubgroup( s4, 2 );;  ComputedSylowSubgroups( s4 );
    [ 2, Group( [ (3,4), (1,2), (1,3)(2,4) ], ... ), 5,
      Group( [  ], ... ) ]
|_
    gap> SylowSubgroup( s4, 6 );                                
    Error SylowSubgroup: <p> must be a prime at
    Error( name, ": <p> must be a prime" );
    Entering break read-eval-print loop, you can 'quit;' to quit to outer \
    loop,
    or you can return to continue
    brk> quit;
\endexample
Thus   the prime test need not    be repeated in  the   methods which are
installed  to  do   the   real  work.  They    are  installed  only    in
`SylowSubgroupOp', whereas the function `SylowSubgroup' and the attribute
`ComputedSylowSubgroups' remain as they were made by `KeyDependentFOA'.

%%%%%%%%%%%%%%%%%%%%%%%%%%%%%%%%%%%%%%%%%%%%%%%%%%%%%%%%%%%%%%%%%%%%%%%%%
\Section{In parent attributes}

This section described how you can add  new ``in parent attributes'' (see
"ref:Subdomains and parents" in the reference manual).  As an example, we
describe how `Index' and its related functions are implemented.

There  is  a function  `Index', an operation   `IndexOp' and an attribute
`IndexInParent'. They are created together as  shown below and after they
have been  created, methods need be  installed only for `IndexOp'. In the
creation  process,  `IndexInParent'  alread   gets  one method  installed
(besides the system getter), namely `D -> IndexOp( Parent( D ), D )'.

The function  `Index' takes   any number of  arguments and    proceeds as
follows:
\beginlist
\item{$\bullet$}
  If it is called  with exactly two arguments  <super> and <sub>,  and if
  `HasParent( <sub> )' and `IsIdentical( <super>, Parent( <sub> ) )', the
  tester `HasIndexInParent( <sub> )' is consulted.
  \itemitem{$\circ$}
    If  the  tester returns `true',  `IndexInParent(  <sub> )'  is called
    (which  will then hopefully choose  the system getter) and the result
    returned.
  \itemitem{$\circ$}
    If the tester returns `false', `IndexOp' is called in the obvious way
    and the result is passed to the setter `SetIndexInParent' (which will
    hopefully store it). The result is also returned.
\item{$\bullet$}
  Otherwise, `IndexOp' is called with the same arguments that `Index' was
  called with (which can be more  or less than two,  because there may be
  method    installed for     a  different  number   of   arguments  with
  `InstallOtherMethod',    see  "InstallOtherMethod").   The  result   is
  returned.
\endlist

\>InParentFOA( <name>, <super-req>, <sub-req>, NewAttribute \| %
  NewProperty )

To  set  up  these   three    objects  together, the  declaration    file
``lib/grp.gd'' contains the following code:
\begintt
    tmp:= InParentFOA( "Index", IsGroup, IsGroup, NewAttribute );
    Index         := tmp[1];
    IndexOp       := tmp[2];
    IndexInParent := tmp[3];
    SetIndexInParent := Setter( IndexInParent );
    HasIndexInParent := Tester( IndexInParent );
\endtt
The  second and third argument to  `InParentFOA',  namely <super-req> and
<sub-req> specify  the required   filters  for the first    resp.\ second
argument of the operation `IndexOp', which are needed when this operation
is  created with `NewOperation' (see  "NewOperation").  <sub-req> is also
the  required filter for the  corresponding attribute `IndexInParent( <U>
)' Note that `HasParent' is *not* required of <U>, because even without a
parent set, `Parent( <U> )' is legal, meaning <U> itself.

For property testing functions, such as `IsNormal', the `NewAttribute' is
replaced  by  `NewProperty'. After  these assignments,  methods  are only
installed for `IndexOp'. `Index'  must and `IndexInParent'  should remain
as it is.

%%%%%%%%%%%%%%%%%%%%%%%%%%%%%%%%%%%%%%%%%%%%%%%%%%%%%%%%%%%%%%%%%%%%%%%%%
\Section{Operation functions}

Section~"ref:Operation  functions!for  groups"  of  the  reference manual
explains that certain  operations like  `Orbits( <G>,  <D>, <opr> )'  can
also  be applied to external  sets, for which  they can be interpreted as
attributes. Moreover, they  can also be  interpreted  as attributes  of a
permutation group,  if it is  acting naturally  on the  set  of its moved
points. Thus  there is an  attribute  `OrbitsAttr' which is automatically
invoked by `Orbits'. Since there are so many options to invoke a function
like  `Orbits', there   is  a rather complicated   interplay between that
function and the attribute on  the one hand  and the operation `OrbitsOp'
on the other hand. The advantage is that methods which perform the actual
computations  are only installed  for  `OrbitsOp', and  they have  a very
restricted syntax.

The definition  of `OrbitsOp' says that its  methods should  be functions
`function( <G>, <D>,  <gens>,  <oprs>, <opr> )'  like  `ExternalSet' when
specified via <gens> and <oprs> (see "ref:External sets" in the reference
manual). All other  syntax variants allowed  for  `Orbits' (e.g., leaving
out <gens>  and <oprs>) are handled by  the wrapper function,  which does
the following:
\beginlist
  \item{1.}
    If the only  argument  is an  external  set <xset> and the  attribute
    tester `Tester( OrbitsAttr )( <xset> )' returns  `true', the value of
    that attribute is returned.
  \item{2.}
    If the  only argument is   an external set  <xset> and  the attribute
    value is  not  known,  the    arguments are  obtained as   `<G>    :=
    ActingDomain( <xset> )', `<D> := HomeEnumerator(  <xset> )' and <opr>
    either as `FunctionOperation( <xset> )' (if `IsExternalSetDefaultRep(
    <xset> )') or  as `<xset>!.funcOperation',  in  the latter case  also
    `<gens> := <xset>!.generators' and `<oprs> := <xset>!.operators'.
  \item{3.}
    If <gens> and <oprs>  are not specified, <gens> is  set to `Pcgs( <G>
    )'   if `IsPcgsComputable( <G> )'   and to `GeneratorsOfGroup( <G> )'
    otherwise, and <oprs> is set to <gens>.
  \item{4.}
    If <opr> is not specified, `OnPoints' is assigned.
  \item{5.}
    In the case of a natural operation of <G> on `MovedPoints( <G> )', if
    the  attribute tester `Tester( OrbitsAttr )(   <G> )' returns `true',
    the value of that attribute is returned.
  \item{6.}
    The  operation is called as `<result>  := OrbitsOp( <G>, <D>, <gens>,
    <oprs>, <opr> )'.
  \item{7.}
    If an  external set <xset> if  it was specified, the attribute setter
    is  called as `Setter( OrbitsAttr  )( <xset>, <result> )'. The setter
    is also called for <G> in the natural operation case.
  \item{8.}
    <result> is returned.
\endlist

\>OrbitsishFOA( <name>, <reqs>, <usetype>, NewAttribute \| NewProperty )

To set  up the FOA  triple, the declaration file ``lib/oprt.gd'' contains
the following code:
\begintt
    tmp := OrbitsishFOA( "Orbits", OrbitsishReq, false, NewAttribute );
    Orbits     := tmp[1];
    OrbitsOp   := tmp[2];
    OrbitsAttr := tmp[3];
\endtt
The second argument <reqs> specifies the list of required filters for the
list    of   arguments  to    the  operation   `OrbitsOp'.   The variable
`OrbitsishReq' contains the standard requirements
\begintt
    OrbitsishReq := [ IsGroup, IsList,
		      IsList,
		      IsList,
		      IsFunction ];
\endtt
If the third  argument <usetype> is   `true', the function  call `Orbits(
<xset> )' will --- if  the attribute value is  not yet known --- call the
operation as `OrbitsOp( <G>, <xset>, <gens>, <oprs>, <opr> )' rather than
as in step~6 above. This allows certain methods for the operation to make
use of `TypeObj( <xset> )', in which the types of the external subsets of
<xset> and of the external orbits in <xset> are stored.  (This is used to
avoid repeated calls   of  `NewType' in  functions  like `ExternalOrbits(
<xset> )', which call `ExternalOrbit( <xset>, <pnt> )' for several values
of~<pnt>.)

For  property testing functions  like `IsTransitive', the fourth argument
would be `NewProperty' instead of `NewAttribute'.

\>OrbitishFO( <name>, <reqs>, <famrel>, <usetype> )

A similar mechanism is provided for operations  like `Orbit', that do not
have  an associated  attribute, but   still need a  wrapper function   to
standardize  the arguments for the  operation  `OrbitOp'. The declaration
file ``lib/oprt.gd'' contains the following code:
\begintt
    OrbitishReq  := [ IsGroup, IsList, IsObject,
		      IsList,
		      IsList,
		      IsFunction ];
    
    tmp := OrbitishFO( "Orbit", OrbitishReq, IsCollsElms, false );
    Orbit   := tmp[1];
    OrbitOp := tmp[2];
\endtt
The  third argument <famrel> is used  to test the family relation between
the second and  third  argument  of  `Orbit( <G>,  <D>, <pnt>   )', where
`IsCollsElms' means that <pnt> must be an element of <D>. By contrast, in
`Blocks( <G>, <D>,  <seed> )', <seed> must  be  a subset of <D>,  and the
family relation  is `IsIdentical'. This  relation testing is necessary to
decide  whether  the argument <D>  was specified  at  all,  because it is
possible to  leave  it out in  `Orbit' (see  "ref:Operation functions!for
groups" in the reference   manual). Similarly, the third  argument <seed>
can be left out  in  `Blocks', but {\GAP} treats   both rules as one:  It
simply checks whether only one of the arguments <D> and <pnt> is present,
and if so,  a command like `Orbit( <G>,  <pnt> )' calls the  operation as
`OrbitOp( <G>, <pnt>, <gens>, <oprs>, <opr>  )' instead of `OrbitOp( <G>,
<D>, <pnt>, <gens>, <oprs>, <opr> )'. Respectively,  `Blocks( <G>, <D> )'
leads to a call of `BlocksOp( <G>, <D>, <gens>,  <oprs>, <opr> )' instead
of  `BlocksOp(  <G>, <operation domain>, <D>,   <gens>,  <oprs>, <opr> )'
because {\GAP}  thinks the argument <D> is   something like the  <pnt> in
`Orbit', but it is really the operation domain!

As a  consequence of  these rules, there  must be  not  only methods  for
`OrbitOp' with  the six arguments  promised  in `OrbitishReq',  but  also
``other methods'' for only five arguments (i.e., without <D>), the latter
are installed with `InstallOtherMethod'. Plenty of examples are contained
in the implementation file ``lib/oprt.gi''.

The fourth argument <usetype> serves the  same purpose as  in the case of
`OrbitsishFOA', but it has an additional meaning, which is currently used
for the functions `Blocks' and `MaximalBlocks'. Since they depend upon an
extra argument  <seed>, they cannot be  regarded as  attributes. However,
they are  most often called without  giving  <seed>, meaning ``choose any
minimal resp.\ maximal block system''. In this case,  their result can be
stored   as the value of an   attribute, and this can    be passed to the
function `OrbitishFO' as fourth argument. It will then be considered by a
wrapper function call `Blocks( <G>, <D>, <opr>  )' (i.e., without <seed>)
in the  same  way `Orbits' considers  `OrbitsAttr'. To  set  this up, the
declaration file ``lib/oprt.gd'' contains the following lines:
\begintt
    BlocksAttr := NewAttribute( "BlocksAttr", IsExternalSet );
    tmp := OrbitishFO( "Blocks",
	[ IsGroup, IsList, IsList,
	  IsList,
	  IsList,
	  IsFunction ], IsIdentical, BlocksAttr );
    Blocks   := tmp[1];
    BlocksOp := tmp[2];
\endtt
And this extraordinary FOA triple works as follows:
\beginexample
    gap> s4 := Group((1,2,3,4),(1,2));; Blocks( s4, MovedPoints(s4), [1,2] );
    [ [ 1, 2, 3, 4 ] ]
    gap> Tester( BlocksAttr )( s4 );
    false
|_
    gap> Blocks( s4, MovedPoints(s4) );       
    [ [ 1, 2, 3, 4 ] ]
    gap> Tester( BlocksAttr )( s4 );  BlocksAttr( s4 );
    true
    [ [ 1, 2, 3, 4 ] ]
\endexample

%%%%%%%%%%%%%%%%%%%%%%%%%%%%%%%%%%%%%%%%%%%%%%%%%%%%%%%%%%%%%%%%%%%%%%%%%%%%%
% Local Variables:
% mode:               text
% mode:               outline-minor
% outline-regexp:     "\\\\Chapter\\|\\\\Section"
% paragraph-start:    "\\\\begin\\|\\\\end\\|\\$\\$\\|.*%\\|^$"
% paragraph-separate: "\\\\begin\\|\\\\end\\|\\$\\$\\|.*%\\|^$"
% fill-column:        73
% End:
%%%%%%%%%%%%%%%%%%%%%%%%%%%%%%%%%%%%%%%%%%%%%%%%%%%%%%%%%%%%%%%%%%%%%%%%%%%%%
