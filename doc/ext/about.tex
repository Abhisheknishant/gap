%%%%%%%%%%%%%%%%%%%%%%%%%%%%%%%%%%%%%%%%%%%%%%%%%%%%%%%%%%%%%%%%%%%%%%%%%%%%
%%
%W about.tex           GAP documentation
%%
%H @(#)$Id$
%%
%Y Copyright 1990-1992, Lehrstuhl D fuer Mathematik, RWTH Aachen, Germany
%%
\Chapter{About: Extending GAP}

This is one of four parts of the {\GAP} documentation,
the others being
the *{\GAP} Tutorial*, a beginner's introduction to {\GAP},
the *{\GAP} Reference Manual*,
which contains the official definitions of {\GAP},
and *Programming in {\GAP}*
which also provides information for those who want to write their own
{\GAP} extensions.

*Extending {\GAP}* explains how to create files and functions that will work
together with mechanisms built in {\GAP}.

This manual is divided into chapters.
Each chapter is divided into sections,
and within each section, important definitions are numbered.
References therefore are triples.

The first chapters of this manual describe how to write documentation,
how to interface packages and components,
and roughly describes the style used for writing the library.
This is followed by chapters that explain advanced programming techniques in
{\GAP}.
Finally there are chapters (alas, at the moment there is only one due to a
lack of manpower) that describe how internal functions work and how to
interface ones own code to these internal functions.

Pages are numbered consecutively in each of the four manuals.
For manual conventions, see Section~"ref:Manual Conventions"
in the Reference Manual.


%%%%%%%%%%%%%%%%%%%%%%%%%%%%%%%%%%%%%%%%%%%%%%%%%%%%%%%%%%%%%%%%%%%%%%%%%
%%
%E

