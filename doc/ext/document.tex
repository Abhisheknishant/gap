%%%%%%%%%%%%%%%%%%%%%%%%%%%%%%%%%%%%%%%%%%%%%%%%%%%%%%%%%%%%%%%%%%%%%%%%%
%%
%W  document.tex              GAP documentation              Frank Celler
%%
%H  @(#)$Id$
%%
%Y  Copyright 1997,  Lehrstuhl D fuer Mathematik,  RWTH Aachen,   Germany
%%
%%  This file describes the manual format.
%%

%%%%%%%%%%%%%%%%%%%%%%%%%%%%%%%%%%%%%%%%%%%%%%%%%%%%%%%%%%%%%%%%%%%%%%%%%
\Chapter{Manual Format}

This  chapter describes  the  manual format  used to generate  the {\GAP}
manual and the on-line help. The {\TeX} source of the manual is also used
to generated the  on-line help. Therefore some  restriction  apply to the
{\TeX} macros one should use. These restriction are described in sections
"TeX Macros" and "Examples, Lists, and Verbatim".

The first sections "Manual  Files"  and "Chapters and Sections"  describe
the  general  layout of the  files  in case  you need to   write your own
package documentation.

%%%%%%%%%%%%%%%%%%%%%%%%%%%%%%%%%%%%%%%%%%%%%%%%%%%%%%%%%%%%%%%%%%%%%%%%%
\Section{Manual Files}

The main {\TeX} is called  ``manual.tex''.  This file should contain only
the following commands:

\begintt
  \input ../gapmacro.tex
  \BeginningOfBook{name-of-book}
    \UseReferences{book1}
    ...
    \UseReferences{bookn}
    \TitlePage{title}
    \TableOfContents
    \FrontMatter
      \Input{file1}
      ...
      \Input{filen}
    \Chapters
      \Input{file1}
      ...
      \Input{filen}
    \Appendices
      \Input{file1}
      ...
      \Input{filen}
      \Answers
      \Bibliography
      \Index
  \EndOfBook
\endtt

The first  line inputs the  file ``gapmacro.tex''.  If  you are writing a
share package  either copy this file or use a  relative path.  The former
method will always work but requires you to keep the file consistent with
the  system while  the latter  forces users to  change the ``manual.tex''
file  if they are installing  a package in  a private location.  See also
Section "ref:GAP Root Directory" in the reference manual.

`\\BeginningOfBook' starts   the   book, `name-of-book'   is  used    for
cross-references,  see "Labels  and References".   If  you are  writing a
share package use the name of your package here.

If   your manual   consists   of more    then    one book  the    command
`\\UseReferences' can  be used to  load the labels  of the other books in
case  cross-references  occur.  `booki'  is  the  path  of the  directory
containing  the book  whose  references you want  to   load.  If you  are
writing a share package and you need to reference the main {\GAP} manual,
use  `\\UseReferences' for each book  you want to reference.  However, as
said  above  this requires  changes to   the ``manual.tex'' file   if the
package is not installed in the standard location.

*Example*

If your  ``manual.tex''  lives in  ``pkg/qwer/doc''  and you want  to use
references to the tutorial use

\begintt
  \UseReferences{../../../doc/tut}
\endtt

`\\TitlePage' produces a page containing the `title'.

`\\TableOfContents' produces a table of contents.

`\\FrontMatter' starts the front matter  chapters like a copyright notice
or a preface.

`\\Chapters'  starts the chapters  of the manual,  which are included via
`\\Input'.  For the chapter format see Section "Chapters and Sections".

`\\Appendices'  starts the appendices.   `\\Answers'  produces an answers
chapter, see  "Excersises  and  Answers".  `\\Bibliography'  produces   a
bibliography, and `\\Index' an index.

Finally `\\EndOfBook' closes the book.

*Example*

Assume you have a share package ``qwert'' with two chapters ``Qwert'' and
``Extending  Qwert'', a copyright  notice,  a preface, no exercises, then
your ``manual.tex'' would basically like:

\begintt
  \input gapmacro.tex
  \BeginningOfBook{qwert}
    \TitlePage{
      \centerline{\titlefont The Share Package ``qwert''}
      \centerline{\secfont by}
      \centerline{\titlefont Q. Mustermensch}
    }
    \FrontMatter
      \TableOfContents
      \Input{copyrght}
      \Input{preface}
    \Chapters
      \Input{qwert}
      \Input{extend}
    \Appendices
      \Index
  \EndOfBook
\endtt

%%%%%%%%%%%%%%%%%%%%%%%%%%%%%%%%%%%%%%%%%%%%%%%%%%%%%%%%%%%%%%%%%%%%%%%%%
\Section{Chapters and Sections}

The  command `\\Chapter\{<chaptername>\}'  starts  a  new  chapter  named
<chaptername>,  a chapter begins with an  introduction to the chapter and
is  followed be sections     created with the    `\\Section\{<secname>\}'
command.  The   strings  <chaptername> and  <secname>   are automatically
available as references.

%%%%%%%%%%%%%%%%%%%%%%%%%%%%%%%%%%%%%%%%%%%%%%%%%%%%%%%%%%%%%%%%%%%%%%%%%
\Section{TeX Macros}

As the manual pages are  also used as on-line  help and are automatically
converted to HTML  the use of special {\TeX}  commands should be avoided.
The following macros can  be used to  structure  the text, the  mentioned
fonts are used  when printing the manual,  however  the on-line help  and
HTML are free to use other fonts or even colour.

\beginitems

`{`text'}' &
    sets the `text' in typewrite style. This is typically used to denoted
    {\GAP}  commands like `for' or  variables like `false'  which are not
    arguments to a function, see also `\<text>'.

`{``text''}' &
    encloses the text in doublequotes.  In particular this does *not* set
    `text'    in typewriter   style,   use  `{`\{`text'\}'}' to   produce
    `{`text'}'.  Doublequotes are mainly used to mark a phrase which will
    be defined later or is used in an uncommon way.
    
`\<text>' &
    sets the text in italics.  This can also be used inside `\$...\$' and
    `{`...'}'. Use `\\\<' to get a less than sign.  `\<text>' is  used to
    denote  a  variable which is  an argument of a  function,  a  typical
    application is the description of a function:
\begintt
        \>Group( <U> )
        The function `Group' constructs a group $G$ isomorphic to <U>.
\endtt

`*text*' &
    sets the text in emphasized style.

`\$a.b\$' &
    inside math mode, you can use `.'  instead of `\\cdot'. Use `\\.' for
    a  full  stop in `\$...\$'.   For   example, `\$a.b\$' produces $a.b$
    while `\$a\\.b\$' produces $a\.b$.

`\\cite\{...\}' &
    produces  a     reference     to    a    bibliography   entry    (the
    `\\cite[...]\{...\}' option of LaTeX is not supported).

`"ref"' &
    produces  a reference  to a  label.  References  are generated by the
    `\\Chapter', `\\Section', see "Labels and References".

`\\index\{...\}' &
    defines an index entry.  Index entries are  also used for the section
    index ``tutorial.six'' used by the on-line help.

`\{\\GAP\}' &
    typesets {\GAP}.

`\\>' &
    *@explaination missing@*

`\\)' &
    *@explaination missing@*

\enditems

%%%%%%%%%%%%%%%%%%%%%%%%%%%%%%%%%%%%%%%%%%%%%%%%%%%%%%%%%%%%%%%%%%%%%%%%%
\Section{Examples, Lists, and Verbatim}

In order  to   produce  a  list   of  items with   descriptions  use  the
`\\beginitems', `\\enditems' environment.

For examples, the   following  list describes   `base', `knownBase',  and
`reduced'.  The different item/description pairs are separated by a blank
line.

\begintt
  \beginitems
    `base' &
        must be a  list of points ...

    `knownBase' &
        If a base for <G> is known in advance ...

    `reduced' (default `true') &
        If this is `true' the resulting stabilizer chain will be ...
  \enditems
\endtt

typesets the list:

\beginitems
  `base' &
      must be a  list of points ...

  `knownBase' &
      If a base for <G> is known in advance ...

  `reduced' (default `true') &
      If this is `true' the resulting stabilizer chain will be ...
\enditems

Example  {\GAP}  sessions    are  typeset  using the    `\\beginexample',
`\\endexample' environment.  They should contain    the exact copy  of  a
{\GAP} session using a line width of 73 and indented  by 4 blanks because
the   manual checker     uses  the test   between  `\\beginexample'   and
`\\endexample' to generate a test file.

\begintt
  \beginexample
      gap> 1+2;
      3
  \endexample
\endtt

typesets the example

\beginexample
    gap> 1+2;
    3
\endexample

Other non-{\GAP} examples are typeset using the `\\begintt', `\\endtt'
environment.

%%%%%%%%%%%%%%%%%%%%%%%%%%%%%%%%%%%%%%%%%%%%%%%%%%%%%%%%%%%%%%%%%%%%%%%%%
\Section{Using TeX macros}

As  described in "TeX Macros"  the use of  macros should be restricted to
the ones given in the previous sections.  However, in rare situations one
might be forced  use  other {\TeX}, for  example  in order  to  typeset a
lattice.  In this case you should provide  an alternative for the on-line
help.

*@NOT SUPPORTED YET@*


%%%%%%%%%%%%%%%%%%%%%%%%%%%%%%%%%%%%%%%%%%%%%%%%%%%%%%%%%%%%%%%%%%%%%%%%%
%%
%E  document.tex  . . . . . . . . . . . . . . . . . . . . . . . ends here
