%%%%%%%%%%%%%%%%%%%%%%%%%%%%%%%%%%%%%%%%%%%%%%%%%%%%%%%%%%%%%%%%%%%%%%%%%%%%
%
%A about.tex           GAP documentation
%
%A @(#)$Id$
%
%Y Copyright 1990-1992, Lehrstuhl D fuer Mathematik, RWTH Aachen, Germany
%
\Chapter{About the New Features Manual}

This is a supplementary volume to the four main parts of the {\GAP}
documentation: the *{\GAP} Reference Manual*, which describes all the 
main features of {\GAP} for users, the *{\GAP} Tutorial*, a
beginner's introduction to {\GAP}, *Programming in {\GAP}* and
*Extending {\GAP}*, which provide information for those who want to
write their own {\GAP} extensions.

This manual, *New Features for Developers*, describes certain
features of {\GAP}, which meet the following conditions: 

\beginlist
\item{$\bullet$} They are *new*. Usually they were introduced at the
last major release of {\GAP}
\item{$\bullet$} They are likely to be of more interest to {\GAP}
programmers and package developers than to interactive users
\item{$\bullet$} We wish to retain the freedom to make some changes in 
them at the time of the next release
\endlist

We would encourage users to employ these features in their own
{\GAP} programs or packages, but ask them to let us know that they are
doing so. We will then invite feedback from them, and, as we approach
the next release, discuss with them any changes to the features that
might be desirable for inclusion in the next release. Unless
substantial problems are found, we would normally expect to move the
documentation into the reference manual at that time, and regard the
documented behaviour as fixed from that time onwards.



%%%%%%%%%%%%%%%%%%%%%%%%%%%%%%%%%%%%%%%%%%%%%%%%%%%%%%%%%%%%%%%%%%%%%%%%%
%%
%E

